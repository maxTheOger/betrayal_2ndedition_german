

\event{A Moment Of Hope}{Ein Moment der Hoffnung}{
    Irgendetwas in diesem Raum fühlt sich seltsam richtig an. Etwas widersteht dem Bösen des Hauses.
}{
    Plaziere einen \emph{Blessing}-Chip (Segnung) in diesem Zimmer.

    Jeder Held darf bei jedem Charakterwurf (Might, ...) in diesem Zimmer einen zusätzlichen Würfel verwenden.
}

\event{Angry Being}{Böses Wesen}{
    Es kommt aus dem Schleim an der Wand neben dir hervor.
}{
    Du musst einen \speedroll versuchen:
    \rolls
    \roll{5+}{Du kommst davon. Erhalte 1 \speed .}
    \roll{2-4}{Erhalte 1 mentalen Schaden.}
    \roll{0-1}{Erhalte 1 mentalen und 1 pysischen Schaden.}
    \erolls
}

\event{Bloody Vision}{Blutige Vision}{
    Die Wände in diesem Raum sind getränkt in  Blut. Blut tropft von der Decke, fließt die Wände herunter, über Schränke und Möbel und auf deine Schuhe. Im nächsten Augenblick ist es fort.
}{
    Du musst einen \sanityroll\ versuchen:
    \rolls
    \roll{4+}{Du festigst deinen Geist. Erhalte 1 \sanity.}
    \roll{2-3}{Verliere 1 \sanity.}
    \roll{0-1}{Wenn sich ein Entdecker oder Monster in deinem oder einem angrenzenden Zimmer befinden, musst du ihn/sie/es angreifen (wenn du kannst). Wenn möglich, wähle denjenigen mit der niedrigsten \might.}
    \erolls
}


\event{Burning Man}{Brennender Mann}{
    Ein brennender Mann rennt durch den Raum. Seine Haut schlägt Blasen und zerfällt. Ein glutroter Schädel verbleibt, schlägt auf dem Boden auf, rollt und verschwindet.
}{
    Du musst einen \sanityroll\ versuchen:
    \rolls
    \roll{4+}{Du spürst Wärme unter deiner Haut, bist aber ansonsten okay. Erhalte 1 \sanity.}
    \roll{2-3}{Raus, raus, du musst hier raus! Setze deinen Entdecker in die Eingangshalle.}
    \roll{0-1}{Du gehst in Flammen auf. Nehme 1 physischen Schaden. Dann nehme 1 mentalen Schaden als du die Flammen ausklopfst.}
    \erolls
}

\event{Closet Door}{Schranktür}{
    Die Schranktür dort ist offen... nur einen Spalt. Darin muss sich etwas befinden.
}{
    Lege den Closet-Chip (Schrank) in dieses Zimmer.

    Einmal während seines Zuges, kann ein Entdecker zwei Würfel werfen um den Schrank zu öffnen.

    \rolls
    \roll{4}{Ziehe eine \itemcardd .}
    \roll{2-3}{Ziehe eine \eventcard .}
    \roll{0-1}{Ziehe eine \eventcard\ und entferne den Closet-Chip.}
    \erolls

}

\event{Creepy Crawlies}{Gruselige Krabbler}{
    Eintausend Käfer stürzen sich aus deiner Haut, kommen unter deiner Kleidung und aus deinen Haaren hervor.
}{
    Du musst einen \sanityroll\ versuchen:
    \rolls
    \roll{5+}{Du blinzelst und sie sind weg. Erhalte 1 \sanity.}
    \roll{1-4}{Verliere 1 \sanity.}
    \roll{0}{Verliere 2 \sanity.}
    \erolls

}


\event{Creepy Puppet}{Gruselige Puppe}{
    Du siehst eine dieser Puppen, bei denen dir die Haare zu Berge stehen. Sie springt dich mit einem kleinen Speer in den Händen an.
}{
    Der Spieler zu deiner Rechten wirft einen \mightroll\ mit 4 Würfeln für die Puppe. Du verteidigst dich normal, indem du einen \mightroll\ ausführst.

    Wenn du Schaden davonträgst, erhält der Entdecker mit dem Speer (Spear) 2 \might\ (es sei denn du besitzt den Speer).
}


\event{Debris}{Schutt}{
    Mörtel fällt von den Wänden und der Decke.
}{
    Du musst einen \speedroll\ versuchen:
    \rolls
    \roll{+3}{Du weichst aus. Erhalte 1 \speed.}
    \roll{1-2}{Du bist unter Schutt vergraben. Ńehme 1 \emph{Würfel} physischen Schaden.}
    \roll{0}{Du bist unter Schutt vergraben. Ńehme 2 \emph{Würfel} physischen Schaden.}
    \erolls

    Wenn du unter Schutt begraben bist, behalte diese Karte. Du kannst nichts machen bis du befreit wurdest. Einmal pro Zug kann ein Entdecker einen \mightroll\ versuchen um dich zu befreien. (Du kannst diesen Wurf ebenfalls versuchen.) 4+ befreit. Nach drei erfolglosen Versuchen, befreist du dich in deinem folgenden Zug automatisch und kannst normal agieren.
}

\event{Disquieting Sounds}{Beunruhigende Geräusche}{
    Das Geschrei eines Babys, einsam und verlassen.

    Ein Entsetzensschrei.

    Das Kracken zerbrechenden Glases.

    Dann: Stille.
}{
    Würfle mit 6 Würfeln. Wenn du genauso viele oder mehr Augen würfelst, als Omen aufgedeckt wurden, erhälst du 1 \sanity.
    Falls nicht, nehme 1 \emph{Würfel} geistigen Schaden.
}

\event{Drip ... Drip ... Drip ...}{Tropf ... Tropf ... Tropf ...}{
    Ein rhythmisches Geräusch macht dich verrückt.
}{
    Lege einen Drip-Chip (Tröpfeln) in diesen Raum.

    Jeder Entdecker rollt bei jedem Charakterwurf (Might, ...) in diesem Raum mit einem Würfel weniger.
}