
\traitor{49}{You Wear it Well}{Er steht dir gut}{}

\introduction{
Als du 5 Jahre alt warst hast du einen Geist im Schlafzimmer deiner Mutter gesehen, der ihre Seele stehlen wollte. Du batest den Geist es nicht zu tun, und er tat es nicht… aber nicht umsonst. Du hast heute deine Freunde hierher gebracht um diesen Preis mit ihren Seelen zu bezahlen.
Jetzt fallen deine Freunde auf die Erde. Ein mächtiger Geist hat ihre Seelen aus ihren Körpern getrieben und sie in den Astralraum geschickt. Das Leben ist nicht fair… aber kleine Kinder brauchen ihre Mutter.

}

\Trightnow{

    \begin{itemize}
    \bitem Dein Charakter ist weiter im Spiel, wurde aber zum Verräter.
    \bitem Der pinke Astralgeist-marker kommt in den Raum in dem dein Charakter ist.
    \bitem Packe so viele dreieckige Gesundheits-Wurf-Marker(sanity roll) beiseite wie Spieler im Spiel sind.
    \bitem Der Astralgeist hat die Seelen aus den Körpern deiner Freunde getrieben, ihre Körper sind jetzt ohne Bewusstsein.
    \end{itemize}
}

\whatyouknowabouttheheros{
Sie werden versuchen den Astralgeist zu vernichten. Wenn sie das geschafft haben, können sie wieder zurück in ihre Körper.

}

\Tyouwinwhen{
entweder die Seelen aller Helden zerstört sind oder der Astralgeist den Körper eines Forscher in Besitz genommen hat.

}

\hauntsection{Wie der Geist einen Körper in Besitz nimmt}
Zerstörst du die Seele eines Helden, kann der Astralgeist ein Ritual beginnen um dessen leblosen Körper in Besitz zu nehmen.

  \begin{itemize}
    \bitem Einmal während jeder Spielrunde kann der Astralgeist einen Gesundheitswurf versuchen, um damit den Körper eines Forschers im gleichen Raum zu beeinflussen. Um Erfolg zu haben muss er eine höhere Zahl würfeln als die Start-Gesundheit des Helden dem dieser Körper gehört. Bei jedem Erfolg legt er einen Gesundheitswurf-marker neben den Körper.
    \bitem Liegen so viele Gesundheitswurf-marker neben einem Körper wie Spieler am Spiel teilnehmen, so nimmt der Astralgeist diesen Körper in Besitz und du gewinnst.
    \end{itemize}
\newpage
\monster{Astralgeist}{3}{}{6}{6}

  \begin{itemize}
    \bitem Der Astralgeist ist nicht betäubt wenn er besiegt wird.
    \bitem Der Astralgeist kann durch Wände ziehen aber nicht durch Decken und Böden.
    \bitem Der Astralgeist kann den physischen Körper eines Helden nicht angreifen. Er kann eine Seele angreifen, aber nur mit einem Gesundheits- oder Wissenswurf.
    \end{itemize}

\hauntsection{Wie man die Seelen der Forscher besiegt}

  \begin{itemize}
    \bitem Du kannst den Körper eines betäubtenForschers angreifen. Der Forscher kann sich gegen diesen Angriff nicht verteidigen. Wirf 2 Würfel und füge ihm soviel psychischen Schaden zu wie du würfelst.
    \bitem Wenn du die Seele eines Forschers mit einem Angriff auf seinen Körper zerstörst (indem du sein Wissen oder seine Gesundheit bis zum Totenkopf-Symbol reduzierst), kann der Astralgeist den entsprechenden Körper nicht mehr in Besitz nehmen.
    \end{itemize}



\Toutro{
Der Körper eines deiner Freunde richtet sich auf, als ob er gerade zum ersten Mal das Atmen erlernt. Es hat sich nicht wirklich etwas mit deinem Freund geändert… und doch ist alles anders. Dein Freund schaut auf und flüstert, „Ahhh… es ist so lange her dass ich einen Mantel aus Fleisch getragen haben.“
„Er steht dir gut“, sagst du. Und das stimmt, das stimmt wirklich.
}