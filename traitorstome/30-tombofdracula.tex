
\traitor{30}{Tomb of Dracula}{Draculas Grab}{}

\introduction{
Dieser Wein ist so gehaltvoll, so delikat und dabei gleichzeitig so kräftig, dass du fast in Ohnmacht fällst, wenn du nur an ihn denkst. Schon bald wirst du mehr tun als nur daran zu denken. Du wirst daran teilhaben. Für deinen neuen Gebieter, Dracula, der in seinem Grab erwacht. Gut dass du alle deine Freunde ins Haus gebracht hast.

Sie haben viel Blut.

}

\Trightnow{

    \begin{itemize}
        \bitem Dein Charakter ist noch im Spiel, wurde aber zum Verräter.
        \bitem Dein Charakter ist jetzt ein Vampir. Addiere 1 Punkt zu allen deinen Fähigkeiten.
        \bitem Lege den Dracula-marker in die Krypta oder auf den Friedhof (graveyard). Ist keiner der beiden Räume bisher entdeckt, lege ihn in einen leeren Raum mindestens 4 Räume vom nächsten Forscher entfernt (gibt es keinen der so weit entfernt ist, lege ihn so weit vom nächsten Forscher weg wie möglich). Wirf die Mädchen-Karte (girl) ab und lege den magenta Braut-marker (bride) in deinen Raum.
        \bitem Nimm die Turn/Damage-Karte mit einem Marker bei „0“ um den Verlauf der Zeit festzuhalten.
    \end{itemize}
}

\whatyouknowabouttheheros{
Die Sonne geht bald auf. Du musst die Forscher schnell töten, weil Vampire schwächer werden, je heller es wird. Wenn es heller wird, werden die Helden dir sagen, wie das Sonnenlicht die Vampire beeinflusst. Vermeide Räume, in die das Sonnenlicht eindringen kann.

}

\Tyouwinwhen{
alle Helden entweder tot oder zu Vampiren geworden sind.

}

\hauntsection{Während deines Zuges musst du}
Zu Beginn deines Zuges bewege den Marker auf der Turn/Damage-Karte 1 Feld weiter.


\monster{Dracula}{5}{8}{6}{}
\monster{Die Braut}{4}{4}{4}{}
\newpage

\hauntsection{Vampire}

  \begin{itemize}
        \bitem Dracula braucht Zeit, bis er wach ist. Er bewegt sich nicht und greift nicht an bis Runde 2 (kann sich aber verteidigen).
        \bitem  Jedes Mal wenn ein Vampir (Dracula, die Braut, der Verräter) versucht die Kapelle zu betreten oder einen Raum mit einem Forscher der das Heilige Symbol trägt, kann er einen Gesundheits-Wurf von 6+ versuchen. Schafft er den Wurf nicht, darf er den entsprechenden Raum nicht betreten.
        (Liegt das Heilige Symbol in der Kapelle würfelt der Vampir nur einmal.)
    \end{itemize}


\specialattackrules{

    \begin{itemize}
        \bitem Herrschaft: Greift ein Vampir (einschl. Dracula und Verräter) jemanden vom anderen Geschlecht an, kann er einen Gesundheits-Angriff versuchen. Besiegt er den Gegner, so erleidet dieser Schaden an der Geschwindigkeit und keinen mentalen Schaden. Verliert man, erleidet man selber keinen Schaden. Jeder Vampir kann diesen Spezialangriff auch in einen angrenzenden Raum mit Verbindungstür machen. Wenn er gewinnt, fügt er wieder Schaden an der Geschwindigkeit zu und der besiegte Held kann anschließend in den Raum mit dem Vampir ziehen, wenn er will.
        \bitem  Ist die Geschwindigkeit eines Forschers bis zum Totenkopf-Symbol gesunken, wird er zum Vampir und ist nicht länger ein Held (dieser Spieler liest sich dann den Abschnitt im Handbuch des Verräters durch (traitors tome)). Anschließend setzt er seine Geschwindigkeit wieder auf seinen Startwert. Der neu geschaffene Vampir addiert 1 Punkt zu allen 4 Eigenschaften und wird vom gleicher Spieler weiter geführt, agiert also während des Spielzuges dieser Spielers. Er kann keine neuen Räume auskundschaften.
        \bitem Wenn der Verräter gewinnt und ein Vampir hat einen Helden getötet, so gewinnt der Spieler der diesen Vampir führt ebenfalls. Vampire können ebenfalls Angreifen und Helden töten. Diese werden jedoch kein Vampir.

    \end{itemize}
}

\Toutro{
Der Biss in Fleisch ist wie der Biss in eine überreife Tomate – der Saft schießt mit solcher Wucht in deinen Mund. Aber mehr als jede Frucht entzückt dich dieses Lebenselixier wenn es durch deinen Körper strömt. Unvermeidlich jedoch, das passiert immer, versiegt die Quelle nach einiger Zeit. Du wirfst das verschrumpelte, ausgelaugte Ding weg. Dracula weiß, wo es mehr gibt.
}