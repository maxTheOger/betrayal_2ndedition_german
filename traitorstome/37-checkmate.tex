
\traitor{37}{Checkmate}{Schachmatt}{}

\introduction{
Du weißt was diese Narren in Onkel Ebenezers Aufzeichnungen lesen können: „Ich, Ebenezer Slocum, habe Mittel und Wege gefunden, den Tod leibhaftig vor mir erscheinen zu lassen. Ich bin bereit, den Tod herauszufordern, und ich werde ihn besiegen! Durch Studien habe ich meinen Geist aufs Höchste geschärft. Oh, der Tod wird heute Nacht nicht triumphieren!“
Gut, die Knochen dieses alten Narren liegen immer noch beim Schachspiel. Du zweifelst daran, dass diese Forscher es besser machen können. Den Tod besiegen? In der Tat!
Nur für den Fall schaust du, was du tun kannst, damit keiner von ihnen aus Zufall des Spiel gewinnen kann. Du glaubst niemand ist cleverer als du, kannst diesen Gedanken nicht ertragen, und es gibt keine Möglichkeit den Tod im Schach zu schlagen!

}

\Trightnow{

    \begin{itemize}
    \bitem Dein Charakter ist weiter im Spiel, wurde aber zum Verräter.
    \bitem Lege den pinken Tod-Counter (death) in einen Raum mit einem Helden deiner Wahl.
    \bitem Lege die 5 sechseckige heiligen Siegel-Counter (holy seal) in folgende Räume, entweder sofort, oder wenn diese entdeckt werden: die Gruft (vault), die Krypta, die beiden Laboratorien und das Spielzimmer (game room). Sage den Helden, dass es 5 heilige Siegel gibt, sage ihnen aber nicht in welchen noch nicht entdeckten Räumen diese liegen.
    \end{itemize}
}

\whatyouknowabouttheheros{
Eine von ihnen wird versuchen, den Tod im Schachspiel zu besiegen. Sie werden die heiligen Siegel dazu zu Hilfe nehmen.

}

\Tyouwinwhen{
alle Helden tot sind. Du gewinnst ebenso, wenn zu Beginn der Monsterrunde kein Spieler im Raum mit dem Tod ist (wenn niemand da ist, verlieren die Helden ihr Schachspiel).

}
\newpage



\monster{Tod}{}{}{}{8}

\hauntsection{Der Tod muss Folgendes während seines Spielzuges machen}

versuchen einen Wissenswurf aber der Tod betrügt. Nach seinem Wurf wirft er alle Würfel mit leerer Seite noch einmal.
Der Tod und sein Gegner vergleichen dann das Ergebnis. Ist es ein Unentschieden, passiert nichts. Hat der Tod ein höheres Ergebnis, schlägt er eine Schachfigur:
\begin{itemize}
    \bitem Gewinnt der Tod mit 1 und 2 Punkten Differenz, schlägt er einen Bauer und jeder Spieler verliert 1 Gesundheit.
    \bitem Gewinnt der Tod mit 3 oder 4 Punkten Differenz, schlägt er einer der anderen Figuren und jeder Spieler verliert 1 Stärke.
    \bitem Gewinnt der Tod mit 5 oder mehr Punkten Differenz sagt er „Schach!“ und jeder Held verliert 1 Gesundheit und 1 Stärke.
\end{itemize}

\specialattackrules{

    \begin{itemize}
    \bitem Der Tod kann nicht angegriffen werden, außer durch eine Partie Schach. Damit kann man ihn schlagen.
    \bitem Du kannst den Raum mit dem Tod nicht betreten (dein Meister lässt sich nicht gerne ablenken).
    \bitem Du kannst keine Gegenstände aufnehmen (die Helden können dies aber) aber sie von Ihnen stehlen.
    \end{itemize}
}

\Toutro{
Ha! Diese Narren. Du wusstest, dass sie den Tod niemals in seinem eigenen Spiel schlagen würden. Warum hättest du ihm also nicht helfen sollen. Das Wichtigste ist dass sie tot sind, und weg, und du bist noch hier. Schachmatt.
}