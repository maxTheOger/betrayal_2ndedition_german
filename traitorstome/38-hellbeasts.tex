
\traitor{38}{Hellbeasts}{Höllenviecher}{}

\introduction{
Feuerfledermäuse sind seltsame Kreaturen. Sie kommen nur bei Nacht hervor und sie können nicht brüten, ohne menschliches Blut zu trinken. Trotzdem, sie sind putzig, und sie sind deine Haustiere. Du hast ihnen die Menschen besorgt, die sie brauchen. Nun musst du nur noch dafür sorgen, dass sie an das Blut herankommen.

}

\Trightnow{

    \begin{itemize}
    \bitem Dein Charakter ist weiter im Spiel, wurde aber zum Verräter.
    \bitem Lege so viele rote Feuerfledermäuse (repräsentiert durch die Fledermaus-marker (bat)) wie Spieler am Spiel teilnehmen in den Raum, wo der Fluch begonnen hat.
    \end{itemize}
}

\whatyouknowabouttheheros{
Ihre Körper enthalten das Blut, das deine Fledermäuse zum Brüten benötigen. Die Helden werden versuchen einen Weg zu finden, dich zu stoppen.

}

\Tyouwinwhen{
alle Helden tot sind.


}

\hauntsection{In deiner Spielrunde tust du Folgendes}
Wenn du in der Monsterphase würfelst um festzustellen wie viele Felder weit die Fledermäuse ziehen können, legst du gleichzeitig genau so viele Fledermaus-marker in den Raum wo der Verrat begonnen hat.

\newpage

\monster{Feuerfledermäuse}{3}{}{}{}
Feuerfledermäuse können sich in der Runde in der sie erscheinen nicht bewegen.



\specialattackrules{

    \begin{itemize}
    \bitem Die Fledermäuse können nicht angreifen oder angegriffen werden.
    \bitem Wenn du an der Reihe bist die Monster zu bewegen, wirf 1 Würfel für jede Fledermaus, die sich im gleichen Raum mit einem Forscher aufhält. Der Forscher erleidet entsprechend physischen Schaden (die Rüstungskarte (armor) kann nur einen Schadenspunkt vermeiden.
    \bitem Feldermaus-marker beeinflussen die Bewegung der Forscher nicht.
    \end{itemize}
}

\Toutro{
Deine kleinen Fledermäuschen saugen genüsslich an den Menschen die überall im Haus verstreut herumliegen. In kürzester Zeit werden mehr Feuerfledermäuse das Haus mit ihrer brennenden Anwesenheit bevölkern. Der Kreislauf des Lebens… was könnte schöner sein?
}