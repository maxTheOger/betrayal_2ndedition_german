
\haunttitle{I Was a Teenage Lycanthrope}{Ich war ein Teenagerwerwolf}
Lycanthrope = Werwolf, Wandlung in einen Werwolf

\introduction{
 Mondlicht strömt in den Raum, streichelt deine Haut. Du stolperst zum Fenster und starrst hinauf zum Vollmond. Dein Verstand beginnt zu schreien, als würde er absterben, schon heult dein Körper vor Vergnügen ...und neu entdeckter Kraft.


}

\rightnow{

    \begin{itemize}
        \bitem Du bist noch im spiel aber der Verräter.
        \bitem Lege die rote Werwolfscheibe (Wolf Token) auf die Charakterkarte deines Entdeckers.
        \bitem Du bist jetzt ein Werwolf. Lege alle deine Gegenstände ab. Wenn du das Mädchen (Girl) oder den Verrückten (Madman) hast, verlierst du deren Schutz. Lege diese Karten bei Seite und stelle deine Eigenschaften dementsprechend ein. Lege für jede Karte ein Token in den Raum. Andere Spieler können diese Aufnehmen.
        \bitem Stelle jede deiner Eigenschaften eine Nummer unter die dortige Startnummer. Dann steigere eine deiner Eigenschaften um 1 Punkt für jeden anderen Mitspieler. Z.B. Wenn du 3 Mitspieler hast, bekommst du 3 Punkte um damit deine Eigenschaften zu steigern.
    \end{itemize}
}

\whatyouknowabouttheheros{
Du wusstest wahrscheinlich einiges über sie, aber du bist jetzt einfach zu sehr ein Wolf um dich daran zu erinnern.
}

\youwhinwhen{
...alle Helden entweder tot sind oder sich in Werwölfe verwandelt haben.

}

\hauntsection{Das musst du während deines Zuges machen}
Zu Beginn jedes deiner Züge gewinnst du 1 Punkt Kraft (Might) oder 1 Punkt Geschwindigkeit (Speed).


\monster{Der Hund}{4}{4}{3}{}

Der Hund ist jetzt ein Monster, welches von dir kontrolliert wird. Lege die orangene Hundescheibe (Dog token) in den Raum zu deinem Entdecker. Den Text auf der Hund-Karte kannst du ignorieren.




\specialattackrules{

    \begin{itemize}
        \bitem Immer wenn du oder der Hund einen Entdecker besiegst, erhält dieser Charakter den üblichen Schaden. Zu Beginn jedes folgenden Zuges des Entdeckers muss dieser Charakter einen Würfelwurf auf Gesundheit (Sanity) von 4+ bestreiten, um dem Fluch des Werwolfs zu wiederstehen. Schlägt der Würfelwurf des Entdeckers fehl, dann wird auch er zum Werwolf und ist nicht länger einer der Helden. Dann darf dieser Spieler diesen Spuk im Buch des Verräters (Traitor's Tome) nachlesen und alles tun was unter „Was du als erstes tun musst“ steht.
        \bitem Keine der Werwolf Eigenschaften kann auf null fallen, es sei denn der Angriff wird mit einem speziellen Gegenstand ausgeführt.Jeder schaden von anderen Quellen wird halbiert (aufrunden).
        \bitem Weder du noch der Hund kann Gegenstände tragen oder Fahrstühle benutzen.
        \bitem Wenn der Verräter gewinnt und ein weiterer Werwolf hat einen Helden getötet, dann gewinnt der Spieler der diesen Werwolf gespielt hat ebenfalls.
    \end{itemize}
}

\outro{
 Du gleitest durch das Haus und nach draußen in den Garten, den beruhigenden Geschmack von Blut auf der Zunge  genießend. Ein Schnalzer mit deinem Schwanz und du überspringst die Moos bewachsene Mauer, so dass du dahinter in der Auffahrt landest. Schon verspürst du, nicht mehr als eine Meile weit entfernt, den süßlich nach Eschenholz duftenden Geruch von weiteren Menschen.
Heute Nacht hat die Jagd erst begonnen.
}