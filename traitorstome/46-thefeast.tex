
\traitor{46}{The Feast}{Das Festmahl}{}

\introduction{
Während du durch die Flure läufst, riechst du das Aroma eines köstlichen Festessens. Du bist zu einem Festmahl eingeladen. Als du versuchst herauszufinden, was du als Hauptgang serviert bekommst, siehst du plötzlich die schreckliche Wahrheit. Menschliche Körperteile liegen überall herum, angenagt, halb zerkocht, auf dem Tisch auf einer Silberplatte ein menschlicher Kopf. Deine Freunde im alten House on the Hill haben dein Lieblingsessen zubereitet.
Du bist ein zivilisierter Kannibale und die Missgeburten im Haus sind bereit dir bei der Vorbereitung zum nächsten Gang zu helfen: die Opfer hast du selber hier ins Haus gebracht.

}

\Trightnow{

    \begin{itemize}
    \bitem Dein Charakter ist weiter im Spiel, wurde aber zum Verräter.
    \bitem Falls das Esszimmer(DiningRoom) noch nicht entdeckt ist, suche sie aus dem Raumstapel heraus und lege sie ins Erdgeschoss (ground floor). Dann mische den Stapel.
    \bitem Lege eine der Anzahl der Mitspieler entsprechende Menge pinken Kannibalen-marker (cannibal freaks) ins Esszimmer.
    \end{itemize}
}

\whatyouknowabouttheheros{
Die Opfer die die Kannibalen gefangen haben sind gerade aus der Dachkammer (attic) entkommen. Die Helden werden versuchen, sie zu retten.

}

\Tyouwinwhen{
entweder alle Opfer gefressen wurden oder alle Helden tot sind. Selbst wenn ein einziges Opfer durch die Eingangstür entkommt, kannst du nur noch gewinnen indem du alle Helden tötest.

}

\newpage



\monster{Kannibalen (cannibal freaks)}{2}{4}{4}{}



\specialattackrules{

    \begin{itemize}
    \bitem Wird ein Kannibale durch einen Angriff besiegt stirbt er.
    \bitem Opfer können nicht angreifen.
    \bitem Machst Du oder ein Kannibale einen Angriff gegen ein Opfer, stirbt das Opfer und der Angreifer kann es zu einem Festmahl zubereiten.
    \bitem Ein Opfer nimmt keinen Schaden wenn es den Angriff abgewehrt hat
    \bitem Du oder die Kannibalen beinflussen die Bewegungen der Opfer nicht (und andersherum)
    \end{itemize}
}
\hauntsection{Festmahl}

  \begin{itemize}
    \bitem Wenn ein Opfer getötet wurde, wird es zum Körper (victim-marker umdrehen). Wenn ein Forscher getötet wird, lege die Figur auf die Seite. Bist du zu Beginn deines Zuges in einem Raum mit einem Körper und/oder toten Forscher, kannst du oder ein Kannibale diesen Körper aufessen. Es darf dafür aber kein lebender Forscher im gleichen Raum zugegen sein.
    \bitem Du brauchst eine ganze Spielrunde, um einen Körper oder toten Forscher zu verspeisen (keine andere Aktion ist erlaubt) und erhöhst anschließend deine 4 Fähigkeiten um je 1 Punkt. Ein Kannibale kann seine Werte auf diese Weise auch erhöhen. Notiere die neuen Werte auf einem Stück Papier.
    \bitem Sobald ein Opfer oder toter Forscher gegessen wurde entferne seinen Körper aus dem Spiel.
    \end{itemize}

\Toutro{
Du hältst den Kopf deines Opfers hoch und singst die letzte rituelle Phrase: „Durch Fleisch und Knochen und Blut bin ich nicht mehr aus Fleisch und Knochen und Blut.“ Eine große Kraft fließt durch deinen Körper und du fühlst wie die Sterblichkeit aus deinen Körperzellen entweicht. Die Kannibalen verbeugen sich und huldigen dir.
}