
\haunttitle{A Friend for the Ages}{Ein Freund für die Ewigkeit}

\introduction{

Es ist nun dreihundert Jahre her als Dir ein Portrait von Dir von einem guten guten Freund gegeben wurde.Es kennt alle Deine stärken und schwächen, dein Alter und deine Verletzungen und ja sogar Deine Moral –
Nun Du bist äusserst Böse....
Es war all die Jahre Dein Talismann gegen all den Schaden die sie Dir zufügen wollten und es gab einige schöne Momente in denen Du Ihnen allen schaden zufügen konntest.Aber jetzt ist es nicht mehr sicher. Du bist Dir sicher sie sind hinter Ihm her.Sie wollen es für sich selbst. Du musst es schützen … um jeden Preis.

}

\Trightnow{

    \begin{itemize}
        \bitem Dein Charakter verbleibt im Spiel und wird zum Verräter.
        \bitem Wenn Deine Charaktereigenschaften unterhalb des Startwerts sind, setze Sie auf den Startwert.

        \bitem Das folgende so oft wiederholen wie Spieler im Spiel sind:
Prüfe Deine Charakterwerte anhand der folgenden Angaben.
Ein Charakterwert auf Startwert zählt als 0. Ein Charakterwert eine Position über dem Startwert zählt als 1 usw.
Der niedrigste Charakterwert wird nun um +1 erhöht. Sind mehrere auf dem gleichen Wert suche Dir einen aus.

    \end{itemize}
}

\whatyouknowabouttheheros{
Sie wollen Dich töten indem sie das Portrait übermalen
}

\Tyouwinwhen{
Du alle Helden getötet hast , oder mindestens drei Farbtoken zerstört hast.
}

\hauntsection{Farbtoken zerstören}
Die Helden platzieren Farbeimer im Haus. Diese können aufgehoben, fallengelassen, getauscht und gestohlen werden wie jeder andere Gegenstand(aber sie können nicht vom Hund transportiert werden).
Jeder Held kann nur einen Farbeimer auf einmal transportieren.


\hauntsection{Was Du tun kannst}
Da es Dein Haus ist kannst Du am ende jeder Runde ein Raumteil aus dem Stapel nehmen und es beliebig Platzieren.
Danach misch den Stapel neu.

\hauntsection{Dein Portrait}
Du darfst Dein Portrait nicht ansehen.
Immer wenn Du Du die Galerie betrittst oder dort Startest musst Du einen Gesundheitswurf von 4+ schaffen.
Wenn der Wurf misslingt erhälst Du einen Mentalen Schadenspunkt.(Unabhängig von deiner Immunität weiter unten)

\specialattackrules{

    \begin{itemize}
        \bitem Du nimmst keinen Schaden durch normale Angriffe. Deine Werte können nicht durch Ereignisse, Raumeigenschaften oder sonstigem Schaden verringert werden.
(Sonderregel Galerie oben!)

        \bitem Wenn Du einen Gegenstand bekommst oder verlierst ändere Deine Werte wie auf der Karte beschrieben.
Ausnahme: Das Blutige Dolch – Du nimmst keinen Schaden wenn Du ihn verlierst!

        \bitem Ein Held kann Gegenstände von Dir stehlen während er einen physischen Angriff ausführt.

    \end{itemize}
}

\Toutro{
Der letzte von Ihnen zuckt in Verzweiflung aber Du weist sie können Dir nichts anhaben.
Dein Portrait beschützt Dich genauso wie Du es beschützt.Eventuell zerbrechen Sie alle an Deiner ewigen Jugend.
Das müsst Sie halt ertragen … Für IMMER !

}