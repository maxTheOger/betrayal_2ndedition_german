
\traitor{28}{Ring of King Solomon}{Der Ring von König Solomon}{}

\introduction{
Wenn du als Kind auf dem Boden deines Schlafzimmers Pentagramme gefunden hast, dann hast du auch den Grund für die nächtlichen blut-getränkten Träume über Ängste, Schreie und widerlich süßen Terror gefunden. Die Stimmen haben dir dann gesagt wie du dich auf die Ankunft vorzubereiten hast. Zunächst hast du versprochen Vorbereitungen nur zu treffen wenn diese Träume aufhören. Später, als du wach und alleine warst, Nacht für Nacht nur begleitet von einer eintönigen, blassen Erinnerung an Freunde und Unterhaltung, hast du schließlich geschworen die Vorbereitungen nur zu vollenden wenn diese Träume wiederkehren.
Nun steht die Ankunft kurz bevor. Du bist erschöpft vom Träumen.
Die Hölle ist angekommen.


}

\Trightnow{

    \begin{itemize}
        \bitem Dein Charakter ist immer noch im Spiel, nun ist er aber der Verräter.
        \bitem Suche dir den Raum aus in dem das Portal für die Hölle steht. Es muss ein leerer Raum mit einem Event-Symbol sein der sich mindestens 4 Felder vom nächsten Abenteurer entfernt befindet. (Wenn in dieser Entfernung kein entsprechender Raum existiert, suche einen aus der möglichst weit von einem Abenteurer entfernt ist.)
        \bitem Lege in diesen Raum den Marker DEMON LORD und so viele Marker DEMON wie es insgesamt Helden gibt (in Reihenfolge, angefangen bei DEMON 1) maximal jedoch 4
    \end{itemize}
}

\whatyouknowabouttheheros{
Sie werden versuchen die Dämonen zu töten wo es nur möglich ist.
}

\Tyouwinwhen{
...alle Helden tot sind.
}

\hauntsection{Dies müssen die Dämonen in ihrem Zug tun}
Dämonen hassen alle lebenden Kreaturen. Ein Dämon bewegt sich immer mit voller Geschwindigkeit in Richtung des nächsten Abenteurers den er attackieren kann. Er führt einen Angriff während seines Zuges durch, wenn er kann.

\newpage

\monster{Daemon 1}{3}{5}{5}{}
\monster{Daemon 2}{4}{4}{4}{}
\monster{Daemon 3}{5}{3}{3}{}
\monster{Daemon 4}{6}{2}{2}{}
\monster{Daemon Lord}{2}{7}{7}{}


\specialattackrules{

    \begin{itemize}
        \bitem Wenn ein Dämon den Träger des Ringes besiegt kann er ihn stehlen aber nicht einsetzen. Wenn ein Dämon den Ring hat kann er ihn an einen anderen Dämon weitergeben oder ihn ablegen. Ein Abenteurer der diesen Dämon jedoch mit mehr als 2 besiegt kann dadurch den Ring wieder zurückbekommen.
        \bitem Der Revolver,Dynamite kann nicht gegen den DEMON LORD eingesetzt werden.
    \end{itemize}
}

\Toutro{
Das Tor zur Hölle ist offen. Du wischst dir das Blut mit den Händen aus den Augen. Der bitter-süßliche Duft in der Luft ist wie in deinem Traum. Die zusammen gewickelten Körper deiner ehemaligen Gefährten formen einen Thron aus Fleisch für den Höllenfürsten. Das Schreien hat gerade erst begonnen.
So, wie du es schon geträumt hast.
}