
\traitor{7}{Carnivorous Ivy}{Fleischfressender Efeu}

\introduction{
Du mochtest Pflanzen schon immer. Azaleen, Astern und Efeu – oh, ja Efeu. Du kannst fühlen wie das Efeu an den Wänden des Anwesens entlang wächst und versucht durch die Fenster zu schlüpfen. Du weißt, dass du ihnen helfen musst zu wachsen und größer zu werden. Was brauchen Pflanzen um zu gedeihen? Licht, Wasser... und Dünger. Was würde denn einen guten Dünger abgeben? Ah, natürlich! Deine Freunde!

}

\Trightnow{

    \begin{itemize}
        \bitem Dein Charakter ist noch im Spiel, ist nun aber der Verräter.
        \bitem Falls Du das Buch bei Dir trägst musst Du es ablegen. Du kannst es ab jetzt nicht mehr aufnehmen!
        \bitem Lege dir doppelt so viele Paare von orangenen Wurzel- und grünen Spitzenscheiben (Root/Tip-Tokens) beiseite wie es Spieler gibt (maximal 10 Paare). Jedes Paar bildet eine Kriechpflanze.
        \bitem Lege die Kriechpflanzen (Paar aus Wurzel- und Spitzenscheibe) in die folgenden Räume: Eingangshalle (Entrance Hall), Balkon (Balcony), Schlafzimmer (Bedroom), Kapelle (Chapel), Wintergarten (Conservatory), Esszimmer (Dinig Room), Gärten (Gardens), Großer Treppenaufgang (Grand Staircase), Friedhof (Graveyard), Herrenzimmer (Master Bedroom), Innenhof (Patio)  und den Turm (Tower). Du kannst nicht mehr als eine Kriechpflanze in einen Raum legen.
        \bitem Wenn mehr mögliche Räume im Spiel sind als Kriechpflanzen, dann kannst du dir aussuchen in welche Räume die Pflanzen kommen. Wenn du mehr Kriechpflanzen als Räume zur Verfügung hast, dann kannst du die restlichen Pflanzen in die entsprechenden Räume platzieren wenn sie entdeckt werden.
    \end{itemize}
}

\whatyouknowabouttheheros{
Sie haben irgendwas vor... etwas, das deinen geliebten Efeu zerstören kann.
Die Helden können einen speziellen Gegenstand erschaffen, indem sie die Regeln im Überlebenshandbuch befolgen.
}

\Tyouwinwhen{
...alle Helden tot sind oder der spezielle Gegenstand vernichtet wurde.

Um den Gegenstand zu vernichten musst du ihn zuerst von dem Helden, der ihn hat, stehlen (siehe Spezialattacken in den Regeln). Beende dann deinen Zug am Abgrund (Chasm), im Heizungskeller (Furnace Room) oder am Unterirdischen See (Underground Lake), um den Gegenstand zu zerstören.
}


\newpage



\monster{Kriechpflanzen}{2}{5}{3}{}

  \begin{itemize}
        \bitem Eine Kriechpflanze kann wachsen. Dazu kannst du nur ihre Spitzen bewegen; die Wurzel hingegen bleibt stets in dem Raum in dem sie gepflanzt wurde.
        \bitem Eine Spitze, die im Mystischen Aufzug steht, legt den Aufzug still.
        \bitem Wurzeln behindern nicht die Bewegung der Helden, Spitzen schon.
    \end{itemize}

\specialattackrules{

    \begin{itemize}
        \item Wurzeln können sich nicht bewegen, nicht angreifen und nicht angegriffen werden, Spitzen schon.
        \bitem Wenn eine Spitze einen Helden im Kampf besiegt so nimmt dieser keinen Schaden. Er oder sie wird von der Pflanze geschnappt und verliert alle seine Gegenstände; diese Fallen auf den Boden und verbleiben in dem Raum. Die Spitze beendet dann ihren Zug.
        \bitem Andere Spitzen können einen gefangenen Spieler nicht angreifen, du jedoch schon.
        \bitem Wenn eine Spitze ihren Zug mit einem geschnappten Helden beginnt, bewegt sie sich 2 Felder auf ihre Wurzel zu anstatt sich normal zu bewegen. Sie kann dabei jeden beliebigen Weg nehmen. Spitzen mit Helden können nicht angreifen.
        \bitem Zu Beginn eines Zuges wird jeder Held, der sich geschnappt im selben Raum wie die zur Spitze gehörige Wurzel befindet, getötet und zu Mulch verarbeitet. Entferne die Pflanze, Wurzel und Spitze.
        \bitem Die Glocke hat keine Auswirkung auf die ergriffenen Helden, das Spirit Board hat keinen effekt auf die Spitzen.

    \end{itemize}
}

\Toutro{
Du liegst im Herrenzimmer und betrachtest wie der Efeu an der Decke über dir entlang wächst und über die Bettlaken gleitet. Das Haus ist nun so friedlich. Bald musst du wieder neue „Freunde“ finden um deinen geliebten Efeu zu füttern.

Du hattest ja schon immer einen grünen Daumen.
}
