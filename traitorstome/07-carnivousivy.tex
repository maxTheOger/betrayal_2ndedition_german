
\haunttitle{Carnivorous Ivy}{Fleischfressender Efeu}

\introduction{
Du mochtest schon immer Pflanzen. Philodendrons, Azaleen, Astern und Efeu – oh, ja Efeu. Du kannst fühlen wie das Efeu an den Wänden des Anwesens entlang wächst und versucht durch die Fenster zu schlüpfen. Du weißt, dass du ihnen helfen musst zu wachsen und größer zu werden. Was brauchen Pflanzen um zu gedeihen? Licht, Wasser...und Dünger. Was würde denn einen guten Dünger abgeben? Ah, natürlich! Deine Freunde!

}

\Trightnow{

    \begin{itemize}
        \bitem Dein Charakter ist noch im Spiel, ist nun aber der Verräter.
        \bitem Falls Du die Buchkarte bei Dir trägst musst Du sie ablegen. Du kannst sie (auch später) nicht aufnehmen!
        \bitem Lege dir doppelt so viele Paare von orangenen Wurzel- und grünen Spitzenscheiben (Root/Tip-Tokens) beiseite wie es Spieler gibt (maximal 10 Paare). Jedes Paar bildet eine Kriechpflanze.
        \bitem Lege die Kriechpflanzen (Paar aus Wurzel- und Spitzenscheibe) in die folgenden Räume: Eingangshalle (Entrance Hall), Balkon (Balcony), Schlafzimmer (Bedroom), Kapelle (Chapel), Wintergarten (Conservatory), Esszimmer (Dinig Room), Großer Treppenaufgang (Grand Staircase), Friedhof (Graveyard), Innenhof (Patio), Herrenzimmer (Master Bedroom) und den Turm (Tower). Du kannst nicht mehr als eine Kriechpflanze in einen Raum legen.
        \bitem Wenn mehr mögliche Räume im Spiel sind als Kriechpflanzen, dann kannst du dir aussuchen in welche Räume die Pflanzen kommen. Wenn du mehr Kriechpflanzen als Räume zur Verfügung hast, dann kannst du die restlichen Pflanzen in neu aufgedeckte Räume platzieren.
    \end{itemize}
}

\whatyouknowabouttheheros{
Sie haben irgendwas vor...etwas, dass deinen geliebten Efeu zerstören kann.
Die Helden können einen speziellen Gegenstand erschaffen wie im Survival Book beschrieben.

}

\Tyouwinwhen{
...alle Helden tot sind oder der spezielle Gegenstand vernichtet wurde. Um diesen Gegenstand zu vernichten, stiehl ihn von dem Helden der Ihn besitzt dann beende deinen Zug am Abgrund (Chasm), Heizungskeller (Furnace Room) oder dem Unterirdischen See (Underground Lake) um den Gegenstand zu vernichten.
}



\monster{Kriechpflanzen}{2}{5}{3}{}

  \begin{itemize}
        \bitem Wenn die Kriechpflanzen wachsen, kannst du nur ihre Spitzen bewegen, die Wurzel bleibt in ihrem Ausgangsraum. Nur die Spitzen der Kriechpflanzen können angreifen oder angegriffen werden.
        \bitem Eine Spitze die in dem Aufzug steht, legt den Aufzug still.
        \bitem Wurzeln behindern nicht die Bewegung der Helden, dies können nur die Spitzen.
    \end{itemize}

\specialattackrules{

    \begin{itemize}
        \bitem Wurzeln können nicht angreifen oder angegriffen werden, aber Spitzen können.
        \bitem Wenn eine Kriechpflanzenspitze einen Abenteurer besiegt, dann erleidet dieser keinen Schaden.
        Dieser Charakter ist dann gefangen und verliert alle seine Gegenstände an den Fußboden.
        Die spitze beendet dann seinen Zug.
        \bitem Andere Spitzen können einen gefangenen Spieler nicht angreifen, Du jedoch schon.
        \bitem Wenn eine Spitze seinen Zug beginnt einen Helden zu greifen bewegt er sich 2 felder entgegen seiner Wurzelscheibe. Eine Spitze kann jeden Weg nutzen um zu seiner Wurzelscheibe zurück zu kommen.
        \bitem Spitzen die Helden transportieren können nicht angreifen.
        \bitem Am Anfang des Zuges einer Kriechpflanze wird jeder ergriffene Held getötet und zu Dünger verarbeitet sobald dieser im selben Raum wie die zugehörige Wurzel ist. Ist ein Held von einer Kriechpflanze getötet worden, entferne die Kriechpflanze und den Held aus dem Spiel.
        \bitem Die Glocke hat keine Auswirkung auf die ergriffenen Helden, das Spirit Board hat keinen effekt auf die Spitzen.

    \end{itemize}
}

\Toutro{
Du liegst im Herrenzimmer und betrachtest wie der Efeu an der Decke über dir entlang wächst und unter dein Bettdecke gleitet. Das Haus ist nun so friedlich. Bald musst du wieder neue „Freunde“ finden um deinen geliebten Efeu zu füttern. Schön, du hattest schon immer einen grünen Daumen.
}