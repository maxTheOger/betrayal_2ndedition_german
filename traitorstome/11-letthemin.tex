
\haunttitle{Let them in}{Lasst sie herein}

\introduction{
Die anderen bezeichnen Deinen neuen Freund als verrückt. Du dachtest zuerst dasselbe. Aber seine klagenden Worte und seine verstümmelten Befehle sind plötzlich auf fruchtbaren Boden gefallen und Du VERSTEHST. Der Nebel! Der Nebel den Du außerhalb des Fensters siehst. Dinge leben in dem Nebel und Du weißt das jetzt. Und sie möchten hinein. Eine Welle der Erkenntnis schwappt über Dich als Du erkennst, was sie innerhalb tun wollen. Der Verrückte schreit „Auf geht’s, öffne weit die Fenster“ Du gehorchst.

}

\Trightnow{

    \begin{itemize}
        \bitem Dein Charakter verbleibt im Spiel und wird der Verräter.

        \bitem Du verlierst alle Boni´s der Verrückten/Madman Karte. Lege sie beiseite.
        \bitem Lege ein gelbes Monster Token(der Verrückte) in den Raum mit dir.
        \bitem Lege ein pinkes Gespenste Token mit dem gesicht nach unten in die Eingangshalle(Entrance Hall) sowie in alle Räume mit einem Fenster nach draussen. Diese Gespenster sind noch außerhalb des Hauses und erwarten, daß Du oder der Verrückte sie hereinlassen.
        Wenn ein Fenster zu einer Wand zeigt wird dort KEIN Token platziert.
    \end{itemize}
}
Es gibt 6 Möglichkeiten, wo die Gespenster hereinkommen können:
Die Große Treppe (auch wenn dort kein Fenster ist) und die Eingangshalle (Tür), dazu noch Master Bedroom, Bedroom, Chapel und Dining Room.

\whatyouknowabouttheheros{
Wenn Sie erfolgreich einen Exorzismus durchführen verbannen sie die Gespenster und gewinnen.
}

\Tyouwinwhen{
alle Helden tot sind.
}

\hauntsection{Wie Du die Gespenster hereinläßt}
  \begin{itemize}
        \bitem Du mußt die Fenster und Tür öffnen, um die Gespenster hereinzulassen. Sowohl Du selbst als auch der Verrückte können das tun. Das Öffnen verbraucht 1 Bewegungspunkt.
        \bitem Sobald die Tür oder ein Fenster geöffnet sind drehe den Gespenstermarker dort um. Gespenster können sich in der selben Runde noch bewegen und angreifen, in der sie umgedreht werden.
        \bitem Wird ein Fenster Raum erst später entdeckt lege dann einen umgedrehten Gespenster-Marker hinein.
        \bitem Wir ein Fenster blockiert indem sich schon ein Gespenster Token befindet muss das Gespenster Token entfernt werden.
        \bitem Noch nicht umgedrehte Gespenster Token beinflussen andere Spieler nicht.
        \bitem Die Glocke oder das Spirit Board beinflusst die Geister nicht.

    \end{itemize}

\monster{Gespenster / Specters}{4}{}{6}{}

\monster{Der Verrückte / Madman}{7}{7}{7}

\hauntsection{Was der verrückte in seiner Runde tun muss
}
Er bewegt sich auf dem kürzest möglichen Weg zum nächsten noch umgedrehten Gespenster-Marker. Sind alle Marker umgedreht aber noch nicht alle Fenster Räume entdeckt erforscht er weiter das Haus um sie zu finden. Er ignoriert hierbei alle Raum-Inhalte. Erst wenn alle Fenster-Räume entdeckt und alle Gespenster hereingelassen wurden, kann er die Helden angreifen.

\specialattackrules{

    \begin{itemize}
        \bitem Gespenster machen immer Gesundheitsangriffe.
        \bitem Wenn der Besitzer des Ringes ein Gespenst bei einem Gesundheitsangriff besiegt ist dieses tot.

        \bitem Der Verrückte darf keine Helden angreifen, solange noch nicht alle Räume entdeckt sind und alle Gespenster aktiviert wurden. Er darf sich aber verteidigen, wenn er angegriffen wird.

    \end{itemize}
}

\Toutro{
Nicht viel später, als Du von Deinen Händen das Blut abwäschst und die Schreie aus Deinem Ohr kratzt erkennst Du was es war, was die Dinge im Nebel wollten. Gute Sache, sie hereinzulassen.
}