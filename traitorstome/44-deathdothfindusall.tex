
\traitor{44}{Death Doth Find Us All}{Der Tod findet uns Alle}{}

\introduction{
Warum merken die anderen bloß nicht, dass dieses Haus lebt? Es ist ein lebendes, mächtiges Wesen. Und wie jedes Lebewesen muss es essen, um zu überleben. Vor langer Zeit hast du einen Handel mit dem alten House on the Hill geschlossen. Alle zehn Jahre bringst du eine neue Gruppe von Abenteurern in das Haus. Sobald sie ankommen beginnt das Haus, ihnen das Leben auszusaugen. Deine Freunde altern sehr schnell bis sie zu Staub zerfallen. Als Gegenleistung erhältst du zehn weitere Jahre in Jugend und Tatkraft.
Du siehst wie die Falten kommen und die Glieder zu zittern beginnen. Es hat angefangen.

}

\Trightnow{

    \begin{itemize}
        \bitem Dein Charakter ist weiter im Spiel, wurde aber zum Verräter.
        \bitem Alle Helden altern sofort um zehn Jahre, auch der mit dem Medaillon. Schaue auf der Tabelle unten nach und sage den Helden wie sie ihre Fähigkeiten verändern müssen.
    \end{itemize}
}

\whatyouknowabouttheheros{
Sie altern schnell, aber sie haben ein Medaillon gefunden, das den Alterungsprozess irgendwie aufhält.

}

\Tyouwinwhen{
alle Helden tot sind.
}
\newpage

\hauntsection{Am Ende deines eigenen Zuges}
Jeder Held wirft einen Würfel und altert um so viele Jahrzehnte. Wenn ein Held eines neues Lebensjahrzehnt betritt, treten die folgenden Veränderungen ein:

\vspace{1cm}
\begin{tabular}{rp{7cm}}
Jahrzehnt  & Auswirkungen \\
20er & +1 Might/Stärke und
+1 Geschwindigkeit/Speed \\
30er & +1 Sanity/Gesundheit und
+1 Knowledge/Wissen \\
40er & -1 Geschwindigkeit/Speed und
+1 Sanity/Gesundheit \\
50er & -1 für einen beliebigen physischen Wert und
-1 für einen beliebigen Mentalen Wert \\
60er & -1 Might/Stärke und
-1 Geschwindigkeit/Speed und
-1 für einen beliebigen Mentalen Wert \\
70er+ & -1 von allen Deinen Werten \\
\end{tabular}
\vspace{1cm}


Die Auswirkungen sind kumulativ, addieren sich also wenn man von den 30er in die 50er wechselt.


\hauntsection{Den Tot bezwingen}
Immer wenn ein Held aus irgendeinem Grund stirbt würfelst Du sofort 3 Würfel und addierst das Ergebnis zu Deinen einzelnen oder kombinierten Werten.

\hauntsection{Das Medaillon}
Der Verräter kann das Medaillon nicht aufnehmen, stehlen oder tauschen.

\Toutro{
Asche zu Asche, Staub zu Staub. Der Tod findet uns alle… nun ja, fast alle. Du fühlst dich sehr gut, danke. Du verlässt das Haus, schließt die Tür hinter dir und sagst Adieu… bis in zehn Jahren.
}