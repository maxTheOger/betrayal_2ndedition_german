
\traitor{8}{Wail of the Banshee}{Das Klagen der Todesfee}

\introduction{
Als erstes hört ihr nur ein schwaches Geräusch außerhalb des Raumes als ob etwas die Wände hochklettert oder seine Krallen daran wetzt. Einige Sekunden später erhascht ihr einen Blick auf eine zerfetzte silbrige Robe, die kurz durch euer Sichtfeld huscht. Ihr dreht euch um, um zur Tür zu laufen als ihr bemerkt, dass etwas den Raum hinter euch betritt.
Die Kreatur kreischt. Das Geräusch peitscht durch den Raum und ihr spürt einen schrecklichen Schmerz, der sich in euer Herz bohrt. Der Tod ist nah aber nicht für dich.
Das Hexenbrett schützt dich vor der tödlichen Stimme deines Lieblings. Wenn du nur mit ihr alleine sein könntest. Du weißt, dass sie zustimmen würde mit dir zusammen zu sein...für ewig.

}

\Trightnow{

    \begin{itemize}
        \bitem Dein Charakter ist immer noch im Spiel, nun ist er aber der Verräter.
        \bitem Lege die runde Todesfeescheibe (Banshee-Token) in den Raum mit dem Verräter.
    \end{itemize}
}

\whatyouknowabouttheheros{
Sie planen deine Todesfee zum Schweigen zu bringen.
}

\Tyouwinwhen{
...alle Helden tot sind.
}

\monster{Todesfee}{8}{}{}{}

Die Todesfee bewegt sich nach ihrem eigenem Willen. Immer wenn sich die Todesfee bewegt würfle mit 2 Würfeln um zu bestimmen wie sie sich bewegt.



\rolls
\roll{0}{ Lege die Todesfee in irgendeinen Raum deiner Wahl, welches Maximal 8 Felder von ihrem Ursprungsfeld entfernt ist.
   Die Fee durchstreift keine Räume in diesem Zug. }
\roll{1}{Such dir aus in welchen Raum sie zuerst geht. Von da an bewegt sie sich dann nur noch links wenn möglich. }
\roll{2}{Such dir aus in welchen Raum sie zuerst geht. Von da an bewegt sie sich nur noch gerade aus, wenn möglich.
   Kann sie nur links oder recht abbiegen, bestimme zufällig in welche Richtung sie geht, von da an geht sie dann wieder geradeaus weiter, wenn möglich. }
\roll{3}{Such dir aus in welchen Raum sie zuerst geht. Von da an bewegt sie sich nur noch nach rechts wenn möglich. }
\roll{4}{ Du kannst in diesem Zug die Bewegung kontrollieren, aber ihr Heulen kann dieses mal nur einen Helden treffen. }
\erolls
\newpage

Wenn die Todesfee einen Raum mit nur einem Ausgang betritt, dann dreht sie sich um und verlässt ihn. Die Todesfee wird in ihrer Bewegung nicht durch Helden beeinflusst. Ihrerseits kann sie auch die Helden nicht in ihrer Bewegung hindern. Wie jedes Monster kann die Todesfee keine neuen Räume aufdecken.

Wenn die Todesfee das Obergeschoss(Upper Landing), den eigestürzten Raum (Collapsed Room), die Galerie (Gallery), das Foyer (falls die Treppen in den Keller im Spiel sind) oder einen Raum mit Geheimtreppe ,Geheimpassage oder Wandschalter Token entscheidest Du wohin sie gehen soll.Wenn Du sie bewegst entscheidest Du auch in welche Richtung sie schaut.

Die Todesfee kann keinen Aufzug benutzen!


Wenn die Todesfee einen Raum mit einem Helden durchschreitet oder im selben Raum stehen bleibt, beginnt sie zu klagen. Jeder Spieler, der das Klagen vernimmt muss einen Verstandwurf (Sanity) machen:

\rolls
\roll{6+}{Würfel einen Würfel du nimm soviel mentalen Schaden. }
\roll{3-5}{Würfel 2 Würfel und nimm soviel mentalen Schaden . }
\roll{0-2}{Würfel 4 Würfel und nimm soviel mentalen Schaden. }
\erolls

Du bist immun gegen Das Klagen der Todesfee solange du das Hexenbrett (Spirit Board) bei dir hast.
Solltest du es verlieren wirst du genauso getroffen.


\specialattackrules{
Die Todesfee kann nicht angegriffen werden.
}

\Toutro{
Die silbernen Haare umhüllen dich als du in ihre Eiskalten Augen blickst. Jetzt gibt es zwei von euch und gemeinsam stimmt ihr euer Jagdlied an. Zusammen. Für ewig.
}