
\traitor{23}{Tentacled Horror}{Tentakelhorror}

\introduction{
Seillange muskulöse Gewebe sind plötzlich überall zu sehen. Zackige, Horn bewehrte Saugnäpfe übersäen die knochenlosen Arme, pulsierend und schmatzend, wie körperlose Zähne. Die langen Dinger winden sich um dein Bein und ziehen sich fest zusammen. Die Saugnäpfe schneiden und sägen an deinem Bein herum, trennen es fast ab. Blut spritzt in alle Richtungen. Die Arme ziehen sich zusammen und du wirst polternd durch das Haus gezogen als ob jemand einen Fisch an einer Angel einholt. Du wirst schreiend in den letzten Raum gezogen… und die Tentakel such im Haus nach dem nächsten Opfer…

}

\Trightnow{

    \begin{itemize}
        \bitem Dein Forscher ist tot. Entferne die Figur aus dem Spiel und lasse alle Gegenstände fallen.
        \bitem Lege paarweise orangene Wurzel und grüne Spitze-Marker (root, tip) beiseite, und zwar so viele Paare wie Spieler teilnehmen. Jedes dieser Paare repräsentiert ein Tentakel.
        \bitem Du kannst eine Wurzel in die folgenden Räume legen (auch mehr als eine pro Raum): Ofenraum (furnace room), Konservatorium, Orgelraum, Unterirdischer See (underground lake), Garten und Abgrund (chasm). Gibt es weniger passende Räume als spieler durchsuche den Stapel und lege weitere Räume aus. Danach wird der Stapel neu gemischt.Dann lege eine Spitze in jeden Raum, in dem sich eine Wurzel befindet.
        \bitem Nimm die Turn/Damage-Karte mit einem Marker bei der „1“. Damit erfasst du die Zeit.
    \end{itemize}
}

\whatyouknowabouttheheros{
Sie suchen nach dem Kopf des Tentakel-Monsters um es zu töten. Das musst du verhindern.
}

\Tyouwinwhen{
alle Helden tot sind.
}

\monster{Kopf (Head)}{}{6}{}{}

Der Kopf der Kreatur kann die Helden angreifen. Er nimmt keinen Schaden wenn er in einem Kampf geschlagen wird. Die Helden erzählen Dir was dann passiert.




\hauntsection{Während deines Zuges musst du Folgendes tun}

Am Ende deines Spielzuges bewege den Marker um 1 Feld auf der Turn/Damage-Karte weiter.
\newpage

\monster{Suckers  @ Runde 1-2}{2}{4}{7}{}

\vspace{1cm}
\begin{tabular}{cccc}
Runde &  \speed & \might & \sanity \\
1-2 & 2 & 4 & 7 \\
3-4 & 3 & 5 & 7 \\
5-7 & 3 & 7 & 7\\
8+ & 4 & 8 & 8 \\
\end{tabular}
\\[1cm]

Die Tentakel werden immer stärker je länger die Kreatur lebt.
Wenn eine Spitze den Aufzug betritt ist dieser Funktionsunfähig bis sie wieder raus geht.
Wurzeln halten die Bewegung der Helden nicht auf, Spitzen jedoch schon.


\hauntsection{}


\specialattackrules{

\begin{itemize}
    \bitem Wurzeln können sich nicht bewegen, Spitzen schon. Wenn eine Spitze einen Helden im Kampf (physisch) schlägt verliert dieser keine Punkte.Er wird dann gepackt und verliert alle Gegenstände.Die Spitze endet seinen Zug dann.
    \bitem Ein gepackter Held kann nicht von anderen Spitzen angegriffen werden.
    \bitem Wenn eine Spitze einen Zug macht den Helden zu packen bewegt dieser sich 2 Felder in Richtung seiner Wurzel anstelle eines normalen Zugs. Eine Spitze kann jeden Weg zurück zu seiner Wurzel nehmen.
    \bitem Wenn eine Spitze beim Angriff geschlagen wird ist sie betäubt und fällt zu boden.Es lässt andere Spieler fallen. Danach wird die Spitze wieder in den Raum der Wurzel gelegt.
    \bitem Die Glocke(Bell) hat keinen Effekt auf gepackte Helden. Das Spirit Board keinen Effekt auf Spitzen.
    \end{itemize}
}

\Toutro{
Das Monster dass sich im Zentrum des Hauses aufhält hat das Mark aus den Knochen derer gesogen die es an sich heranziehen konnte. Ein undefinierbares Etwas, etwas WAS NICHT SEIN KANN grunzt zufrieden und freut sich über die Säfte und Bissen der Überreste der Forscher. Wieder stärker geworden, sendet es seine Tentakel weiter aus als zuvor. Es streckt seine suchenden Tentakel bereits aus der Einganstür ins Freie.
}