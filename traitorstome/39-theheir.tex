
\traitor{39}{The Heir}{Der Erbe}{}

\introduction{
Sie sind nun wirklich gekommen, wie du es schon immer gewusst hast … die einzig möglichen Erben des mittelalterlichen Vermögens und der Macht der königlichen Familie Romanescu. Deine Familie hat bislang immer die Kontrolle über das Vermögen gehabt, doch wenn der wahre Erbe auf dem Thron in diesem Haus sitzt und dabei den Ring und den Speer hält, werden die Romanescu’s es wieder an sich reißen. Das darf nicht passieren. Heute Nacht werden deine versteckten Meuchelmörder den Erben töten und die Macht deiner Familie erhalten.
Der Erbe muss sterben.

}

\Trightnow{

    \begin{itemize}
        \bitem Dein Charakter ist immer noch im Spiel, nun ist er aber der Verräter.
        \bitem Wenn der „Statuary Corridor“ noch nicht im spiel ist, durchsuche den Stapel danach und lege die Karte an eine Stelle im Haus die es für die Helden schwer macht sie zu erreichen.Danach mische den Stapel erneut
        \bitem Befestige an dem TURN/DAMAGE TRACK einen Plastik-Clip an der 0. Er dient als Rundenzähler.
        \bitem Du hast im Haus so viele versteckte Meuchelmörder wie Spieler am Spiel teilnehmen. Benenne sie und schreibe ihre Namen auf ein Blatt Papier.
        \bitem Schreibe nun geheim die Räume auf in denen sich die Meuchelmörder befinden und zwar nach folgenden Kriterien:
          \begin{itemize}
            \bitem Jeder Meuchelmörder muss in einem Raum platziert werden der sich bereits im Spiel befindet.
            \bitem Du kannst nicht mehr als 1 Meuchelmörder in einen Raum setzen.
            \bitem Du darfst keinen Meuchelmörder in einem Raum platzieren in dem sich im Augenblick eine Figur befindet und auch nicht in dem „Statuary Corridor“            \end{itemize}
    \end{itemize}
}

\whatyouknowabouttheheros{
Einer von ihnen ist der Erbe, aber du weißt nicht wer.
}

\Tyouwinwhen{
... der Erbe tot ist.
}

\newpage

\monster{Assasine / Meuchelmörder}{2}{}{}{}

\hauntsection{Benutzung der Meuchelmörder}

  \begin{itemize}
        \bitem Wenn ein Abenteurer einen Raum mit einem Meuchelmörder betritt, kannst du dich entscheiden ihn erscheinen zu lassen. Wenn du das tust führst du sofort einen Angriff mit diesem Meuchelmörder durch (auch wenn du nicht am Zug bist). Du kannst irgendeinen Abenteurer in diesem Raum angreifen.
        \bitem Alle Angriffe von Meuchelmördern sind lautlos. Würfele mit 2 Würfeln und der Gegener nimmt soviel Physischen Schaden. Bei einer Attacke eines Meuchelmörders darf sich sein Gegner nicht verteidigen.
        \bitem Nach einer Attacke nimmt der Meuchelmörder Gift und stirbt. Streiche diesen Meuchelmörder dann einfach auf deinem Blatt durch.
        \bitem Wenn der Gegner stirbt frag ob das der Erbe war. Der Spieler muss ehrlich antworten.
    \end{itemize}

\hauntsection{Dies musst du in deinem Zug tun}
      \begin{itemize}
            \bitem Am Ende deines Zuges schiebst du den Clip auf dem TURN/DAMAGE TRACK um eine Position weiter.
            \bitem Am Ende des 3. Zuges betreten neue Meuchelmörder (so viele wie Mitspieler) das Haus. Notiere dir wieder ihre Namen und in welchem Raum sie sich verstecken. Du kannst sie in jeden unbetretenen Raum platzieren.
            \bitem Am Ende des 6. Zuges platzierst du noch einmal neue Meuchelmörder im Haus. Verfahre dabei genau so wie am Ende von Runde 3.
        \end{itemize}


\Toutro{
Der Erbe ist tot. Der von deiner Familie gestohlene Reichtum ist sicher. Als du über den leblosen Körper gehst, hebst du eine Papier-Krone neben ihm auf. Als die Krone vom Blut des Erben benetzt wird, schließt du leise die Tür.
}