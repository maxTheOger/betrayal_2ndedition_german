
\haunttitle{}{}

\introduction{
Die Wand vor dir fängt an sich zu bewegen und schiebt sich zur Seite, wobei ein altertümlicher Sarkophag aufgedeckt wird, sein gebogene, staubiger Deckel ist mit Hieroglyphen beschriftet. Die Hieroglyphen zeichnen sich glühend ab. Eine krächzende Stimme meldet sich in deinem Verstand und sagt: "Viele Jahre bevor es dich gab, verlor ich meine Braut. Meine Tränen sind getrocknet, aber meine Liebe ist noch immer so kräftig wie die Strahlen der Sonne. Jetzt ist meine Liebe wiedergeboren. Du darfst nicht zulassen, dass sterbliche Hände mich von der erneuten Vereinigung mit ihr abhalten...“
Als die Stimme leiser wird, lächelst du. Jetzt ist alles so klar. Deine alten Freunde müssen sterben, damit die Liebenden wieder vereint sein können. Als der Sargdeckel den Sarkophag öffnet, drehst du dich, um deinen neuen Freund zu begrüßen... deinen neuen Gott.

}

\rightnow{

    \begin{itemize}
        \bitem Dein Charakter bleibt im Spiel und ist nun der Verräter.
        \bitem Lege das Mumienplättchen (Mummy token) und den Sarkophag (Sarcophagus token) in den Auslöserraum (der Raum in dem der Spuk ausgelöst wurde).
        \bitem Der Abenteurer der das Mädchen (Girl) hat, verliert es. (Dieser Abenteurer verliert alle Bonuspunkte, die er durch die Mädchenkarte hatte.) Lege das Mädchenplättchen (Girl token) in irgendeinen Raum auf der selben Etage mit dem Auslöserraum, der mindestens 5 Felder von der Mumie entfernt liegt. Sollte es keinen Raum geben der 5 Felder entfernt ist, lege sie auf der Etage soweit weg wie möglich.
        \bitem Lege die Mädchen-Karte beiseite. Wenn der Abenteurer eines Spielers den Raum erreicht in dem das Mädchenplättchen liegt, dann bekommt dieser Spieler die Karte.
        \bitem Wenn du das Mädchen bekommen hast, kannst du es der Mumie geben, während ihr zusammen im selben Raum seid. Die Mumie wird damit zum Verwalter des Mädchens.
    \end{itemize}
}

\whatyouknowabouttheheros{
Sie versuchen die Mumie zurück ins Reich der Toten zu verbannen.
}

\youwhinwhen{
...die Mumie der Verwalter des Mädchens ist, den Ring (Ring) oder das Heilige Symbol (Holy Symbol) besitzt und damit in den Raum mit dem Sarkophag zurückkehrt.
}

\hauntsection{Die Mumie muss folgendes während ihres Zuges tun}

Einmal während des Monsterzuges, muss die Mumie einen Angriff auf einen Abenteurer, der im gleichen Raum wie sie ist, ausüben (falls dies möglich ist).

\monster{Die Mumie}{3}{8}{5}{}
Würfelt die Mumie eine 0 oder 1 für die Bewegung kann sie eine Geheime Passage zu jeden Punkt des Hauses nutzen.


Die Mumie kann Gegenstände tragen und als Verwalter des Mädchens agieren, aber das Mädchen und deren Gegenstände beeinflussen ihre Eigenschaften nicht.

\specialattackrules{

    \begin{itemize}
        \bitem Die Mumie unternimmt Kraftangriffe (Might), aber fügt Geschwindigkeitsschaden (Speed) zu, solange bis die Geschwindigkeit ihres Gegners auf der niedrigsten Nummer ist (diese Angriffe können diese Eigenschaft nicht bis zu Totenkopf-Symbol herabsetzen). Danach erhält der Gegner stattdessen Kraftschaden (Might).
        \bitem Wenn die Mumie 2 oder mehr Punkte Schade mit ihrem Angriff zufügt, kann sie statt den Schaden zuzufügen Gegenstände von ihrem Gegner stehlen. Die Mumie kann mit diesem speziellen Angriff auch das Mädchen von ihrem Verwalter an sich nehmen.
    \end{itemize}
}

\outro{
Das Mädchen sitzt zusammengekauert in der Ecke, schreit zu dir um Hilfe. Die Mumie kreuzt den Raum und liest sie mit seinen, durch Leinen bekleideten Armen auf. Langsam behutsam haucht die Mumie in den jammernden Mund des Mädchens. Ihre Schreie werden weicher und ihre Tränen glühen bernsteinfarben, als sie ihre Augen hebt.
`` Mein Priester, mein Geliebter... jetzt sind wir wieder zusammen,'' schmachtet das zarte Mädchen, ``und bald wird die Welt uns anbeten. Ihr Fleisch wird unser sein, um zu brennen, an ihren Seelen werden wir uns ergötzen...''
}