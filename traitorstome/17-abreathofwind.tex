
\haunttitle{A Breath of Wind}{Der Windzug}

\introduction{
Ein zittern und ein Echo eines qualvollen Lachens zieht durch das Haus. Ein zeichen dafür das Deine Begleiter Deinen alten Freund gestört haben … Der Poltergeist.
Können sie ihn nicht einfach in ruhe lassen nach all dem was er durchgemacht hat?
Nun, Du musst nur dafür sorgen das die Störenfriede beseitigt werden …

}

\Trightnow{

    \begin{itemize}
        \bitem Dein Charakter verbleibt im Spiel und wird zum Verräter.
        \bitem Platziere den pinken Marker in den Raum wo der Fluch begann.
        \bitem Lege die Runden/Schadens-Leiste neben das Spielfeld, setze einen Marker auf  4.
    \end{itemize}
}

\whatyouknowabouttheheros{
Sie versuchen einen Exorzismus durchzuführen.
}

\Tyouwinwhen{
alle Helden tot sind.
}

\hauntsection{Hilf dem Poltergeist}
Du kannst dem Poltergeist helfen indem Du Deine Mitspieler angreifst oder indem Du einige Gegestände findest(oder stiehlst).


\monster{Der Poltergeist}{3}{X}{4}{}
Entgegen der normalen Monster kann der Poltergeist jeden Gegenstand aufnehmen, ablegen, stehlen oder tauschen.
Aber er kann alle Eigenschaften der Gegenstände ignorieren.

  \begin{itemize}
        \bitem Der Poltergeist startet mit Macht 4.
Jedesmal wenn er einen Gegenstand aufnimmt, verschiebe die Leiste um einen Punkt nach oben.
Am ende jeder Monster-Runde muss der Poltergeist eine Gegenstandskarte ziehen(und den Zähler eine Position nach oben setzen) falls er in einem der folgenden Räumen seinen Zug beendet:
Junk Room (Rumpelkammer), Storeroom (Vorratsraum), Attic (Dachspeicher),  Library (Bibliothek),  Wine Cellar (Wein Keller), Research Lab (Forschungslabor) oder das Operation Lab (Operationssaal).

        \bitem Jedesmal wenn der Poltergeist einen Gegenstand verliert, ziehe den Marker um eine Position zurück.

    \end{itemize}


\specialattackrules{

    \begin{itemize}
        \bitem Der Poltergeist kann einen separaten Macht-Angriff gegen jeden Helden den es erreicht durchführen.
(aber es kann keinen Helden mehr als einmal angreifen)
Wenn er Angreift zählt der Aktuelle Machtwert der Skala.(Max8).
Er kann wählen ein Gegenstand zu stehlen auch wenn es dadurch 2 Schadenspunkte mehr in kauf nimmt.
Der Poltergeist nimmt keinen Schaden wenn der Held gewinnt.

        \bitem Der Poltergeist ist imun gegen Macht-Angriffe und kann nicht durch einen Revolver verletzt werden.
Wenn Dynamit in dem Raum gezündet wird wo der Poltergeist sich aufhält, verliert er alle Gegenstände die er trägt und wird aus dem Spiel entfernt. Er kann sich jedoch wieder neu Formieren.(s.u.)

    \end{itemize}
}

\hauntsection{Neu Formieren des Poltergeists}
Bei jedem start der Monster-Runde kann der Geist sich in jedem Raum formieren der ein Omen-Symbol beherbergt.
Wenn der Geist das tut, verliert er alle Gegenstände und die Runde/Schadensleiste wird auf 3 gesetzt, im Junk-Room(Rumpelkammer) wird die Leiste jedoch auf 4 gesetzt.
Der Geist taucht dann in dem Raum Deiner Wahl auf und macht seine nächsten Züge.

\Toutro{
Eine Wolke aller Gegenstände dreht sich im Raum als der letzte Störenfried zu Boden fällt.
Immerhin ist jetzt wieder alles Friedlich und Du kannst Dich hinsetzten und ein Stilles Gespräch mit Deinem alten Freund führen...
}