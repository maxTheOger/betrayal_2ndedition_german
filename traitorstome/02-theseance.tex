
\haunttitle{The Séance}{Die Séance}

\introduction{
Eine kalte Hand legt sich über das Haus auf und Nebel bedeckt den Boden. Eine Stimme ertönt: „Ich brauche Ruhe … lasst meine Seele ruhen … oder ihr werdet Sterben“

Als die Worte verstummen, bewegt sich das Ouijabrett im Rhythmus deines Herzschlags. Du betrachtest es, als der Nebel auf ihm die Worte formt: TÖTE SIE ALLE
}

\Trightnow{

    \begin{itemize}
        \bitem Dein Charakter bleibt im Spiel und ist nun der Verräter.
        \bitem Lege einen kleinen pinken Monsterchip beiseite, der den Geist darstellt.
        \bitem  Lege einen dreieckigen \chip{Knowledge roll}{Wissenswurf} und einen dreieckigen \chip{Sanity roll}{Gesundheitswurf} neben das Spielfeld.
        \bitem Wenn das Pentagramm-Zimmer noch nicht im Spiel ist, suche es aus dem Stapel heraus und lege es im Keller mindestens 5 Räume von dir entfernt oder so weit wie möglich von dir entfernt an. Mische danach den Stapel.
    \end{itemize}
}

\whatyouknowabouttheheros{
Sie versuchen den Geist vor dir zu beschwören. Gelingt es, kontrollieren sie den Geist und müssen eine Aufgabe lösen. Misslingt es, versuchen sie den Geist zu zerstören.

}

\Tyouwinwhen{
alle Helden tot sind, egal wer den Geist zuerst beschwor.
}

\hauntsection{Wie der Geist beschworen wird}
Es findet ein Wettrennen zwischen dir und den Helden statt, den Geist zuerst zu beschwören. Um ihn zu beschwören, musst Du eine Seance abhalten.

  \begin{itemize}
        \bitem Mit Hilfe deines Ouijabretts (Spirit Board) musst einen \knowroll\ und einen \sanityroll\ von je 5+ bestehen. Welchen du zuerst machst ist egal, pro Runde hast du aber nur einen Versuch. Bestehst du, nehme den entsprechenden Chip. Sobald du beide Chips hast, ist der Geist beschworen.
        \bitem Hast Du den Geist vor den Helden beschworen, lege den \chipe{Ghost} neben deine Figur. Waren die Helden schneller, erklären sie dir, was passieren wird.
    \end{itemize}


\hauntsection{Wenn Du den Geist beschworen hast …}

Jetzt laut vorlesen: „Ich werde mich an den Lebenden rächen!“

\monster{Der Geist}{4}{}{6}{}

  \begin{itemize}
        \bitem Du kontrollierst den Geist, wenn du ihn zuerst beschwörst oder die Helden ihre Aufgabe nicht bestehen. Stirbt dein Held, kontrollierst du immer noch den Geist.

        \bitem Der Geist muß sich in jeder Runde in Richtung eines Helden bewegen und wenn möglich auch angreifen.

        \bitem Am Ende deines ersten Zuges, während dem du den Geist kontrollierst, beginnt das Haus zu einzustürzen. Der erste einbrechende Raum ist der Dachboden (Attic). (Ist der Dachboden nicht im Haus, nehme einen anderen unbesetzten Raum im ersten Stock.) Danach stürzt nach jedem Zug jedes Helden ein weiterer Raum ein. Lasse den jeweiligen Helden nach seinem Zug den nun einstützenden Raum bestimmen.

        \bitem Wenn ein Raum zerstört wird drehe die Platte auf ihre Rückseite. Die Zerstörung breitet sich immer nur auf benachbarte Räume aus (Türen sind hier nicht notwendig). Stürzt ein besetzter Raum ein,  sterben alle Abenteurer in ihm.

        \bitem Ist die ganze obere Etage zerstört, geht es im Erdgeschoss weiter, beginnend mit der großen Treppe (Grand Staircase). (Markiere die Räume der großen 3er-Raumkarte mit fünfeckigen \chipse{Item} als eingestürzt.) Ist das Erdgeschoss weg, fange im Keller mit dem \emph{Basement Landing} an.

        \bitem Wenn der Aufzug (Mystic elevator) eine Etage ohne offene und nicht eingestürzte Türen anfahren soll, bewegt er sich nicht.

        \bitem Nur der Geist kann sich durch zerstörte Räume bewegen, ausserdem durch Wände ohne Türen, jedoch nicht durch Decken und Böden !
    \end{itemize}

\specialattackrules{

    \begin{itemize}
        \bitem Angriffe sind erst nach erfolgreicher Séance möglich. Wenn die Helden den Geist kontrollieren, erklären Sie dir die Angriffsregeln.
        \bitem Der Geist macht \sanity-Attacken, die mentalen Schaden zufügen. Nur der Träger des Ringes oder Helden im Pentagram-Raum können den Geist angreifen (mit \sanity-Attacken).
        \bitem Erfolglose Angriffe des Geistes schaden ihm nicht.
    \end{itemize}
}

o\utro{
Ein Nebel durchströmt das Haus vom Keller bis zum Dach. Du gleitest durch ihn, still wie der Geist an deiner Seite. Dein Herzschlag wird langsamer und verstummt. Stille. Von nun an gibt es zwei Seelen die diesen Ort heimsuchen. Zusammen und für immer!
}