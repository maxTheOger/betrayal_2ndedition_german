
\traitor{42}{Comes the Hero}{Es kommt der Held}{}

\introduction{
Unsterblichkeit ist etwas, für das manche Leute alles geben würden. Deshalb wunderst du dich, dass deine Unsterblichkeit so billig zu haben war. Alles was du nun tun musst ist, ein kleines Tor zu Hölle zu öffnen und dein Meister wird dir dieses Geschenk für immer überlassen.
Obwohl du unsterblich bist, solltest du keine Zeit vergeuden. Besser du wartest nicht zu lange, bevor du das Tor des Schreckens öffnest.

}

\Trightnow{

    \begin{itemize}
    \bitem Dein Charakter ist weiter im Spiel, wurde aber zum Verräter.
    \bitem Erhöhe alle deine Fähigkeiten die augenblicklich unter dem Startwert liegen auf den Startwert
        \bitem Solltest Du noch keine Waffe(Revolver, Axt oder Dolch) besitzen, durchsuche den Stapel und
         nimm die erste Waffe die Du findest. Danach mische den Stapel erneut.
    \end{itemize}
}

\whatyouknowabouttheheros{
Sie werden versuchen zu verhindern, dass du das Tor zur Hölle öffnest.
}

\Tyouwinwhen{
du das Tor zur Hölle geöffnet hast.
}
\newpage
\hauntsection{Wie das Tor zur Hölle geöffnet wird}
Gehe folgendermaßen vor:

\begin{enumerate}
    \item Du musst einen der Helden mit deiner Waffe opfern (töten) und den Leichnam in die Katakomben, den Abgrund (chasm) oder die Pentagrammraum bringen. Einen Leichnam nimmst du genau so auf wie einen anderen Gegenstand und trägst ihn dann. Während du ihn trägst zählt jeder Raum den du betrittst 2 Zugpunkte. Der Hund kann keine Leiche tragen.
    \item Ist der Leichnam in einem der o.g. Räume kannst du einen Gesundheits- oder Wissenswurf von 4+ versuchen um das Tor zu öffnen. Diesen Wurf kannst du einmal pro Runde machen.

\end{enumerate}


\hauntsection{Auswirkungen der Unsterblichkeit}

  \begin{itemize}
    \bitem Deine Fähigkeiten gehen weder hoch noch runter, auch nicht, wenn du Gegenstände aufnimmst/verlierst ausser die Helden finden einen Weg sie zu beinflussen.
    \bitem Die Helden können keine Gegenstände von Dir stehlen.
    \bitem Die Forscher beeinflussen deine Bewegung durch die Räume nicht.
    \end{itemize}

\specialattackrules{

    \begin{itemize}
    \bitem Du kannst weder angegriffen werden noch Schaden erleiden, weder durch die Auswirkungen von Karten noch von Räumen.
    \bitem Du kannst normal angreifen, erleidest aber keinen Schaden wenn du besiegt wirst.
    \end{itemize}
}

\Toutro{
Als das Tor seinen Weg durch das Haus in deine Welt findet kannst du schon die Schreie der Verdammten und noch zu Verdammenden hören. Verdammung und Terror, Tod und Zerstörung… du bist gegen Alles das immun. Freue dich.
}