
\haunttitle{Here there be Dragons}{Hier gibt es Drachen}

\introduction{
Du schaust auf das Kinderbild, das Du auf dem Flur findest. Deine Finger berühren das Bild des feuerspeienden Drachen. „Ich wünschte, ich hätte einen Drachen“ grübelst Du. Unglaublicherweise bricht die Eingangstür auf und ein riesiger Drache bricht herein, tobend und feuerspeiend. Du mußt träumen. Du lächelst. Das sind die Art von Träumen die Du liebst. An ihrer Reaktion erkennst Du, daß Deine Begleiter immer noch ungläubig auf das schauen, was sie sehen. Gereizt entscheidest Du, daß der Traum noch ein bißchen andauern soll.
Der beste Weg dazu ist der, die Ungläubigen wegzuschaffen. „Friß sie, Drache, friß sie alle!“.

}

\Trightnow{

    \begin{itemize}
        \bitem Dein Charakter verbleibt im Spiel und wird zum Verräter.
        \bitem Setze den runden Drachen-Marker in die Eingangshalle.
        \bitem Lege den sechseckigen Schild/Shield-Marker in den Chasm- oder den Crypt-Raum.
        \bitem Legen einen anders Nummerierten sechseckigen Antike Rüstung/ Antique Armor-Marker in die Katakomben/Catacombs oder des unterirdischen Sees/ Underground Lake.
        Wenn diese Räume fehlen lege die Gegenstände in den Raum, wo der Spuk begann..
        \bitem Lege die Runden/Schadensleiste neben das Spielfeld aber markiere noch nichts.
        Du brauchst diese Leiste um Schäden zu vermerken.

        \bitem
    \end{itemize}
}

\whatyouknowabouttheheros{
Sie versuchen, den Drachen zu töten.
}

\Tyouwinwhen{
 alle Helden tot sind.
}

\hauntsection{}


\monster{Der Drache}{3}{8}{6}{}

Benutze die Schadensleiste um zu markieren, wieviel Schaden die Helden dem Drachen angetan haben. Wenn der Wert der Schäden der Anzahl an Spielern entspricht, ist der Drache tot. Der Schaden beinflusst nicht seine Attribute.
Der Drache ist imun gegen Geschwindigkeits-Attacken(Revolver, Dynamit....).Er kann nur mit Gesundheits-Angriffen von dem Ring-Besitzer schaden nehmen.


\hauntsection{}


\specialattackrules{
Der Drachen kann in seiner Runde 2x angreifen:
durch Feuer-Spucken und durch Beißen.
Er muß nicht beide machen, wenn er es tut ist die Reihenfolge egal.

    \begin{itemize}
        \bitem Feuer-Spucken: Jeder Forscher (incl. Dir) im gleichen oder benachbarten (Tür-verbunden) Raum des Drachens (wenn er Feuer spuckt) muß einen erfolgreichen Geschwindigkeits-Wurf ausführen:

        Im gleichen Raum:
        \rolls
        \roll{4+ }{ Kein Schaden}
        \roll{0-3}{physischer Schaden von 4 Würfeln. }
        \erolls

        Im benachbarten Raum:
        \rolls
        \roll{4+ }{ Kein Schaden}
        \roll{0-3}{physischer Schaden von 2 Würfeln. }
        \erolls
        \bitem Beißen: Dies ist ein Macht Angriff
        \bitem Wann immer ein Drache besiegt wurde erhält er 2 Schaden von der Summe abgezogen.
Der Revolver kann ihm nichts anhaben.
    \end{itemize}
}

\Toutro{
Du fühlst Dich zuerst ein bißchen schlecht, als der Drache Deinen ersten Freund frißt, all das Blut und die austretenden Innereien. Und Du fühlst Dich auch nicht besser als der Drache einen Deiner anderen Freunde in eine menschliche Fackel verwandelt, wie sie schreien und sich krümmen.
Gut, daß alles nur ein Traum ist.... Oder ?
}