
\traitor{27}{Amok Flesh}{Amok laufendes Fleisch}{}

\introduction{
Du hörst den hellen Ton brechenden Glases. So. Dein Clone ist aus seinem Behälter freigelassen worden. Selbst jetzt kannst du hören wie er sich vergrößert. Du wünschst dir zum tausendsten Mal, dass du nicht so ein instabiles Protoplasma für deinen Versuch gewählt hast. Aber du hast leider. Du seufzt.
Zeit, deine Freunde zusammen zu treiben. Du kannst sie nicht alle davor bewahren sich unglücklich zu machen indem sie vor deinem missratenen, wachsenden, alles verzehrenden Fleischklops wegrennen.

}

\Trightnow{

    \begin{itemize}
    \bitem Dein Charakter ist noch im Spiel, wurde aber zum Verräter.
    \bitem Lege mindestens 20 orangene Blob-Marker bereit. Du benutzt sie um den Blob der sich über das ganze Haus ausbreitet zu markieren.
    \bitem Der Spieler mit der Kristallkugel legt diese ab. Sie ist aus dem Spiel.
    \end{itemize}
}

\whatyouknowabouttheheros{
Sie versuchen, deinen wertvollen Blob zu zerstören.
}

\Tyouwinwhen{
Alle Helden entweder tot sind oder zu Blobpersonen wurden.
}

\hauntsection{Der Blob}
  \begin{itemize}
    \bitem In der ersten Monsterrunde befindet sich der Blob in dem Raum, wo die Kristallkugel war und jedem angrenzenden Raum mit Verbindungstür. Lege Blob-marker in diese Räume. In jeder weiteren Monsterrunde breitet der Blob sich in alle angrenzenden Räume mit Verbindungstür aus. Auch dort legst du Blob-marker hin. Bei seiner Ausbreitung nutzt der Blob alle Zugmöglichkeiten aus (Türen, Treppen, Geheimgänge, die Kohlenrutsche, den eingefallenen Raum, geheime Treppen, usw.). Der Blob kann sich die Kohlenrutsche und den eingefallenen Raum aufwärts ausbreiten aber nicht die Mystic Slide benutzen(
    \bitem Sobald der Blob den Fahrstuhl erreicht ist dieser Ausser Funktion.
    \newpage
    \bitem Nachdem der Blog in alle umliegende Räume vorgedrungen ist Würfel 1 würfel. Bei einer 2 erweitere den Blob ein zweites mal.(Falls keine Marker mehr da sind nimm sie aus der mitte wo niemand mehr hin kommt.
    \bitem Blob-marker zählen nicht als individuelle Monster, deshalb können sie nicht angreifen und von Dingen beeinflusst werden, die andere Monster beeinflussen.
    \bitem Jeder, der sich in einem Raum mit Blob-marker befindet wird sofort zur Blobperson (einschließlich dem Verräter) und legt alle Gegenstände und Omenkarten ab. Lege einen Blob-marker auf die Charakterkarte des entsprechenden Charakters um anzuzeigen, dass er eine Blobperson ist.
    \bitem Der Spieler der diesem Forscher kontrolliert hat, kontrolliert jetzt die Blobperson (sein neues Ziel ist es, dir zu helfen zu gewinnen).
    \bitem Der Blob wird von Glocke (bell) und Séancebrett (spirit board) nicht beeinflusst.
\monster{Blob-Personen}{2}{}{}{}
    \bitem Blobpersonen haben die Geschwindigkeit 2.
    \bitem Eine Blobperson kann nicht angreifen, angegriffen werden, Karten ziehen und Räume entdecken. Sie zieht während ihres Spielzuges und kann durch Räume mit Blob-markern hindurch ziehen.
    \bitem Zu Beginn des Monsterzuges lege einen Blob-marker in jeden Raum mit einer Blobperson. Der Blob breitet sich von diesem Raum nicht weiter aus solange er nicht mit dem grossen Blob verbunden ist.
    \end{itemize}



\Toutro{
Deine Clone füllt nun das ganze Haus. Deine Freunde, die Fliegen an der Wand, die herumfliegenden Motten – alle sind sie eins mit dem Blob. Es bleibt nur noch eines zu tun. Du lässt dich mit ausgebreiteten Armen zurückfallen, landest in der weichen Umarmung des wabbelnden Fleisches. Deine Haut beginnt zu schmelzen und deine Organe werden eins mit der riesigen Masse. Der explosive, brennende Schmerz verstummt als sogar deine Gedanken aufgesaugt werden.
Nur der Blob bleibt übrig. Er beginnt die Grenzen des Hauses zu testen.
}