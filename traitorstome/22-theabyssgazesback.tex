
\traitor{22}{The Abyss Gazes Back}{Der Abgrund starrt dich an}

\introduction{
Mit einem brutalen Tritt machst du ein weiteres Loch in den Boden. Dieses Mal, anstatt Staub und morscher Balken, hast du gefunden was du gesucht hast. Den Abgrund. Flammen. Ein Vorhof der Hölle Kichernd reibst du dir deine Hände.
Falls der knarrende Boden rund um die sich öffnende Grube nicht schon alle im Haus Anwesenden herbeieilen lässt, schreist du: „Hey, ihr alle! Wir werden alle zur HÖLLE fahren!“ Ein flackernder Lichtschein fällt auf die Wände und grauer Nebel füllt den Raum. Ein Teil des Hauses zerbröselt und fällt, fällt in einen brennenden See aus Feuer.

}

\Trightnow{

    \begin{itemize}
        \bitem Dein Charakter ist weiter im Spiel, wurde aber zum Verräter.
        \bitem Wähle einen leeren Raum im Untergeschoß (basement) der ein Omen- oder Ereignissymbol trägt (Rabe, bzw. Spirale). Hier beginnt der Abgrund. Erzähle das allen Spielern.
        \bitem Gibt es keinen Raum im Basement durchsuche den Stapel nach einem passenden Raum und lege ihn an. Mische den Stapel dann neu.
        \bitem Nimm die Turn/Damage-Karte mit einer Markierung an der „0“. Du misst damit die Zeit.
    \end{itemize}
}

\whatyouknowabouttheheros{
Sie versuchen einen Exorzismus durchzuführen der das Haus davon abbringen soll, weiter in den Abgrund gesogen zu werden.

}

\Tyouwinwhen{
alle Helden tot sind.
}

\hauntsection{Während deines Zuges musst du folgendes tun}

  \begin{itemize}
        \bitem Am Ende deines ersten Spielzuges bricht der Raum zusammen in dem der Abgrund gestartet ist. Drehe die Raumkarte um.
        \bitem Am Ende jedes deiner Spielzüge setze den Marker auf der Turn/Damage-Karte 1 Feld weiter.
        \bitem Das Haus stürzt in Deinem Zug immer weiter ein auch wenn Du schon tot bist.
        \bitem Wenn Du eine Ereignisskarte ziehen musst während Du im Basement bist kannst Du den Geheimgang oder die Geheime Treppe aus dem Stapel suchen. Danach wird der Stapel neu gemischt.
    \end{itemize}

\newpage

\hauntsection{Jeder Spieler tut während seines Spielzuges folgendes}

  \begin{itemize}
    \bitem Am Ende seines Spielzuges muss jeder Spieler einen Teil des Hauses zerstören, beginnend in der 2. Runde nach dem Start des Verrats. Der Abgrund verschlingt immer einen Raum der einem bereits verschlungenen benachbart ist. Die Räume müssen keine Verbindungstür haben. Der Abgrund vergrößert sich nach folgender Regel:

        \textbf{Runde 2} – jeder Spieler dreht 1 Raum um

        \textbf{Runde 3} – man wirft 2 Würfel und dreht so viele Räume um wie Augen fielen

        \textbf{Runde 4} – man wirft 3 Würfel und dreht so viele Räume um wie Augen fielen

        \textbf{Runde 5} und folgende – man wirft 4 Würfel und dreht so viele Räume um wie Augen fielen

    \bitem Hat der Abgrund eine ganze Etage verschlungen geht es eine Etage höher mit einem unbesetzten Zimmer mit einem unentdeckten Tür nach Wahl weiter.
    \bitem Das Startteil mit den 3 Räumen zählt in diesem Fall wie 3 separate Räume. Stürzt eines davon ein, markiert man das mit einem sechseckigen Marker.
    \bitem Ist ein Spieler (einschließlich Verräter) in einem Raum der gerade zusammenbricht, muss er einen Geschwindigkeitswurf von 4+ versuchen um zu entkommen.
    \bitem Ist der Wurf erfolgreich, springt man in einen benachbarten Raum, vorausgesetzt zwischen beiden Räumen ist eine Tür. Misslingt der Wurf oder gibt es keinen benachbarten Raum den man durch eine Tür erreichen kann, stirbt der Held.
    \bitem Wenn der Aufzug oder eine Eventkarte jemand in einen zerstörten Raum schickt stirbt er.
    \end{itemize}



\Toutro{
Die Hölle heißt dich willkommen. Als das Fleisch von den Knochen deiner Freunde verbrennt und verdampft, kannst du dir ein Lächeln nicht verkneifen. Aber deine Haut löst sich auch in dem alles verzehrenden Feuer auf. Ist es das, was du wolltest?
}