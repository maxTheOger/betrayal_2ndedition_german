
\traitor{41}{Invisible Traitor}{Der unsichtbare Verräter}{}

\introduction{
Während du diesen alten, staubigen Ort durchstreifst siehst du ein Wort, das in den Ring den du trägst eingraviert wurde. Als du dieses Wort laut vor dich hin sagst zuckt ein spitzer Schmerz durch dein linkes Auge bis ins Gehirn. Schmerz überflutet dich, vor Todesangst zerkratzt du deine Haut um ihn zu stoppen.
Dann hört der Schmerz plötzlich auf und du verlierst das Bewusstsein. Du vermisst deinen Körper. Nein, warte, du bist unsichtbar… ein unsichtbarer Jäger, bereit zu töten.
Ja, das ist es, töten! Irgendwas hat dich davon immer abgehalten aber jetzt bist du frei, zu tun was du willst. Du fühlst dich wie ein Kind zu Weihnachten. Zeit, deine Freunde, äh, Geschenke, zu öffnen.

}

\Trightnow{

    \begin{itemize}
    \bitem Dein Charakter ist weiter im Spiel, wurde aber zum Verräter.
    \bitem Nimm ein Stück Papier. Suche Dir einen Raum im Haus aus und schreibe den Raumnamen darauf.Du befindest Dich in diesem Raum.Dann entferne deine Spielfigur aus dem Haus.
    \end{itemize}
}

\whatyouknowabouttheheros{
ist, dass sie versuchen dich zu finden und zu töten.
}

\Tyouwinwhen{
alle Helden tot sind.
}

\hauntsection{Wie du ziehst}
  \begin{itemize}
    \bitem Anstatt mit deiner Figur durch die Räume zu ziehen schreibst du den Raum auf, in dem du deinen Zug beendest (benutze deine normale Geschwindigkeit und die normalen Regeln für das Ziehen). Musst du einen Raum verlassen in dem sich ein Held befindet, so zählst du diesen Raum nicht als Extra-Schritt (wie es Monstern tun).
    \bitem Du kannst immer noch neue Räume erforschen, aber in dem Moment sehen die Helden wo du bist (wenn du den neuen Raum auslegst).
    \end{itemize}

\newpage

\hauntsection{Gegenstände stehlen}

Einmal während des Zuges kannst du einen Geschwindigkeitswurf (Speed) machen und versuchen einem Gegner einen Gegenstand zu stehlen. Der Gegner muss im gleichen Raum sein. Das dies kein Angriff ist, kann der Gegner sich dagegen nicht verteidigen. Die Ergebnisse:

\rolls
\roll{0}{ Versuch misslungen. Du musst den Mitspielern sagen wen du bestehlen wolltest.}
\roll{1-3}{Versuch misslungen. Du musst den Mitspielern aber nicht sagen von wem du versucht hast etwas zu stehlen.}
\roll{4+}{Du nimmst den Gegenstand}
\erolls

\specialattackrules{

    \begin{itemize}
    \bitem Der Ring erlaubt dir keine weiteren Gesundheitsangriffe (Sanity).
    \bitem Außer Angriffe mit einem Gegenstand, deine Angriffe kommen alle aus dem Hinterhalt.
  Anstatt normal anzugreifen wirf so viele Würfel wie die hälfte Spieler im Spiel (aufrunden).
  Der Spieler den du angegriffen hast erhält genau so viele Schadenspunkte (physisch). Deine Gegner können sich nicht dagegen verteidigen.
    \bitem Falls jemand weiß (oder annimmt) wo du dich befindest und dich angreift, so läuft dieser Angriff nach den normalen Regeln ab.
    \bitem Du kannst in der gleichen Runde Anfreifen und gegenstände Stehlen.
   \end{itemize}
}

\Toutro{
Deine Freunde liegen zerstört im Haus herum. Du blickst in dein Gesicht, es ist voller Blut. Du wischst das Blut ab und dein Gesicht verschwindet. Das war lustig.
Vielleicht versuchst du es beim nächsten Mal mit Brandstiftung. Du wolltest schon immer mal sehen wie es ist, wenn jemand brennt.
}