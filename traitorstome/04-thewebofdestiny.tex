
\traitor{4}{The Web of Destiny}{Das Netz des Schicksals}

\introduction{
Dein Verstand erschaudert als du eine dunkle Präsenz spürst, die in deinen Geist hineinkriecht und sich in deinem Schädel einnistet. Du bist zwar du selbst aber gleichzeitig bist du \emph{sie} – die Spinne. Und schon zappelt ein kleines, ungeschicktes, sterbliches Wesen in deinem neuen Netz. Sein zappeln schickt ein Zittern durch dein Netz und über deinen Spinnenbauch. Du spürst wie dein Spinnen-Ich Eier in den Körper des kreischenden Opfers injiziert. Bald wwerden seine Anstrengungen aufhören und deine Babys werden aus dem zuckenden Leib quellen. Aber deine Instinkte sagen dir, dass Andere deine Brut zu zerstören versuchen werden... sofern du sie nicht zuerst zerstörst.

}

\Trightnow{

    \begin{itemize}
        \bitem Wenn es 5 oder 6 Spieler gibt, bleibt dein Charakter im Spiel und ist nun der Verräter.
        \bitem Wenn es 3 oder 4 Spieler gibt, wurde dein Charakter von der Spinne gefressen. Lege all deine Gegenstände auf den Boden und entferne deinen Charakter aus dem Spiel.
        \bitem Lege die Spinnenscheibe (Spider-Token) in den Raum, in dem sich der Spieler mit der Bisskarte (Bite) aufhält.
        \bitem Stelle den Rundenzähler mit dem Plastikclip auf 1. Du wirst ihn benutzen, um die Zeit zu verfolgen.
    \end{itemize}
}

\whatyouknowabouttheheros{
Der Spukauslöser ist in einem klebrigen Netz gefangen und mit Spinneneiern infiziert. Dieser Charakter kann sich nicht bewegen.
}

\Tyouwinwhen{
...wenn der deine Eier schlüpfen (in Zug 9) oder alle Helden tot sind.
}

\hauntsection{ Das musst du während deines Zuges machen}
Am Ende jeder deiner Züge erhöhe den Rundenanzeiger um ein Feld.

\newpage



\hauntsection{Die Spinne muss dies während ihres Zuges machen}
Die Spinne muss sich auf einen Helden zu bewegen, der nicht der Spukauslöser ist und diesen wenn möglich angreifen. Die Spinne und der Verräter können den Spukauslöser nicht angreifen solange er noch mit den Spinneneiern infiziert ist.

\monster{Die Spinne}{}{}{}{}

\begin{tightcenter}\begin{tabular}{cccc}
\textbf{Runde} & \textbf{Speed} & \textbf{Might} & \textbf{Sanity} \\
1 & 0 & 2 & 5  \\
2 & 1 & 2 & 5  \\
3 & 2 & 4 & 5  \\
4 & 4 & 4 & 5  \\
5 & 5 & 5 & 5  \\
6 & 6 & 7 & 5  \\
7+ & 6 & 8 & 5  \\
\end{tabular}
\end{tightcenter}

\specialattackrules{
Wenn die Spinne angreift, kann sie einmal pro Angriff jeden Würfel, der auf 0 gelandet ist, neu werfen. (Zum Beispiel: DU hast 4 Würfel für die Spinne geworfen und zwei von ihnen zeigen keine Augen. Dann können diese Würfel neu geworfen werden, aber nur einmal.)
}

\Toutro{
...deine Brut macht ein Festmahl aus diesen leckeren, leckeren Menschen.
}
