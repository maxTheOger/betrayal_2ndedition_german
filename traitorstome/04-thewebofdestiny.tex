
\traitor{4}{The Web of Destiny}{Das Netz des Schicksals}

\introduction{
Dein Verstand erschaudert als du eine dunkle Präsenz spürst, die in deinen Geist hineinkriecht und sich in deinem Schädel einnistet. Du bist zwar du selbst aber gleichzeitig bist du ES – die Spinne. Und schon zappelt ein kleines ungeschicktes sterbliches Wesen in deinem neuen Netz. Du spürst wie dein Spinnen-Ich Eier in den Bauch des kreischenden Opfers injezierst. Bald werden seine Krämpfe aufhören und deine Babys werden aus dem
zuckendem Leib quellen. Plötzlich laufen weitere Erschütterungen durch die Fäden deines Netz bis in deinen Verstand. Deine Instinkte verraten, dass weitere Kreaturen anwesend sind, die deine Brut vernichten wollen, wenn du sie nicht zuerst vernichtest.

}

\Trightnow{

    \begin{itemize}
        \bitem Wenn es 5 oder 6 Spieler gibt, bleibt dein Charakter im Spiel und ist nun der Verräter.
        \bitem Wenn es 3 oder 4 Spieler gibt, wurde dein Charakter von der Spinne gefressen. Leg all deine Gegenstände auf den Boden und entferne deinen Charakter aus dem Spiel.
        \bitem Leg die Spinnenscheibe (Spider-Token) in den selben Raum wo sich der Spieler mit der Bisskarte (Bite) aufhält.
        \bitem Stell den Rundenzähler, für die anderen Spieler verdeckt, auf 1.
    \end{itemize}
}

\whatyouknowabouttheheros{
Der Spukauslöser ist in einem klebrigen Netz gefangen und mit Spinneneiern infiziert. Dieser Charakter kann sich nicht bewegen.
}

\Tyouwinwhen{
...entweder wenn der Rundenanzeiger auf 9 gestiegen ist oder alle Helden tot sind.
}

\hauntsection{ Das musst du während deines Zuges machen}
Am Ende jeder deiner Züge erhöhe den Rundenanzeiger um ein Feld.

\newpage



\hauntsection{Die Spinne muss dies während ihres Zuges machen}
Die Spinne muss sich auf einen Helden zu bewegen, der nicht der Spukauslöser ist und diesen wenn möglich angreifen. Die Spinne kann den Spukauslöser nicht angreifen solange er noch mit den Spinneneiern infiziert ist.

\monster{Die Spinne}{}{}{}{}

\begin{tightcenter}\begin{tabular}{cccc}
\textbf{Runde} & \textbf{Speed} & \textbf{Might} & \textbf{Sanity} \\
1 & 0 & 2 & 5  \\
2 & 1 & 2 & 5  \\
3 & 2 & 4 & 5  \\
4 & 4 & 4 & 5  \\
5 & 5 & 5 & 5  \\
6 & 6 & 7 & 5  \\
7+ & 6 & 8 & 5  \\
\end{tabular}
\end{tightcenter}

\specialattackrules{
Jedes mal wenn die Spinne bei einem Angriff blanke Würfel dabei hat, kann sie diese Würfel einmalig erneut werfen.
}

\Toutro{
...wirst du ein paar richtig schöne Feiertage haben mit diesen leckeren, leckeren Menschen.
}