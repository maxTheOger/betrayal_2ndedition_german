
\traitor{45}{Tick, tick, tick}{Tick, tick, tick}{}

\introduction{
Du kicherst unkontrolliert als du deine Bastelei anschaust. Du hast eine Bombe um jeden Abenteurer gebunden. Sie können dir jetzt zwar nicht mehr helfen, respektieren dich und dienen dir aber stattdessen.
}

\Trightnow{

    \begin{itemize}
    \bitem Dein Charakter ist weiter im Spiel, wurde aber zum Verräter.
    \bitem Wenn du die Dynamit-Karte nicht hast, nimm sie dem Forscher weg der sie hat oder suche sie dir aus dem Kartenstapel (item) heraus. Mische den Kartenstapel danach.
    \bitem Nimm die Turn/Damage-Karte mit einem Marker bei der „0“. Du brauchst sie, um die Zeit zu messen.
    \end{itemize}
}

\whatyouknowabouttheheros{
Du hast jedem von ihnen eine Zeitbombe umgebunden. Sie versuchen, die Bombe zu entschärfen.

}

\Tyouwinwhen{
entweder alle Spieler tot sind oder du die Große Bombe fertig stellst.

}

\hauntsection{Die Große Bombe }
Du stellst diese fertig in Runde 12. Die Große Bombe explodiert dann und alle sterben.


\hauntsection{Das machst du mit der Turn/Damage-Karte}
Am Ende deines Zuges rücke den Marker um 1 Feld vor. Die dann eingestellte Zahl addiere zur Zahl der Spieler die noch leben. Wirf mit genauso vielen Würfeln (maximal 8). Ist das Ergebnis 8+  explodiert der nächste Spieler zu deiner Linken. Der Forscher explodiert und mit ihm alles und jedermann im gleichen Raum. Gegenstände und andere Dinge die er bei sich trägt sind verloren, die entsprechenden Karten sind aus dem Spiel. Explosionen können auch dann noch stattfinden, wenn der Verräter bereits tot ist.
\newpage

\specialattackrules{

    \begin{itemize}
    \bitem Du kannst aus dem Raum in dem du zum Verräter wurdest nicht heraus. Du bist vollauf damit beschäftigt, die Große Bombe zu bauen. Bist Du im Aufzug bewegt er sich nicht mehr
    \bitem Du hast einen Annäherungsschalter um die Bomben zu zünden. Beginnend in deiner nächsten Runde explodiert jeder Spieler der in einen Raum zieht, der an den angrenzt in dem du dich befindest. Alle Spieler im gleichen Raum explodieren ebenfalls und mit ihnen alle Gegenstände und Omen-Karten. Du wirst dadurch nicht verletzt.
    \bitem Hat ein Held seine Bombe entschärft kann er nicht mehr explodieren.Enweder musst Du warten bis er in deinem Raum ist um ihn anzugreifen, oder auf den Big Bang warten.
    \bitem Du kannst das Dynamit in dem Raum benutzen in dem du dich befindest anstatt damit in einem angrenzenden Raum anzugreifen. Du wirst vom Dynamit nicht verletzt.
    \end{itemize}
}

\Toutro{
Die Bombe ist wunderbar. Ein wahres Meisterwerk. Der Gipfel deines Könnens und deiner kranken Gedanken. Es wird Zeit der Welt von deinem Ruhm zu erzählen…
}