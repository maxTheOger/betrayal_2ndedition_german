
\traitor{16}{The Phantoms Embrace}{Die Umarmung des Phantoms}

\introduction{
Sie denken, daß sie Dich kennen. Sie denken, Du tust alles was sie Dir sagen, aber sie liegen falsch. Sie versuchen, das Mädchen von Dir wegzunehmen, und das ist ihr größter Fehler. Jetzt ist sie sicher vor ihnen. Du läßt ein Phantom erscheinen um sie zu beschützen sie versteckt zu halten innerhalb des Hauses. Wenn sie kommen um sie zu holen hast Du eine kleine Überraschung für sie vorbereitet. Bald werden Deine alten Freunde nicht mehr in der Lage sein, Dich weiter herumzukommandieren.
}

\Trightnow{

    \begin{itemize}
        \bitem Dein Charakter verbleibt im Spiel und wird zum Verräter.
        \bitem Der Forscher, der das Mädchen besitzt verliert dieses (sie flieht). Lege die Mädchen-Karte und den magenta Marker neben das Spielfeld, der bisherige Besitzer muß seine Charaktereigenschaften aktualisieren.

        \bitem Lege den pinken Phantom-Marker neben das Spielfeld.
        \bitem Lege 20 unterscheidbare Marker neben das Spielfeld (Ratten- oder Blob-Marker sind geeignet)
        \bitem Lege die Runden/Schadens-Leiste neben das Spielfeld, setze einen Marker auf 0.
Die Leiste brauchst Du um die Zeit zu markieren.
    \end{itemize}
}

\whatyouknowabouttheheros{
Sie versuchen das Mädchen und sich selbst zu retten
}

\Tyouwinwhen{
entweder alle Helden tot sind oder Du das Haus in die Luft sprengst indem sich noch Helden befinden.
}

\hauntsection{Du musst folgendes in Deiner Runde tun}
Am Ende jeder Deiner Runden setzt Du den Zeit-Marker um 1 Feld weiter.
Dann würfelst Du mit der gleichen Menge an Würfeln, wie die Zeitleiste anzeigt.
Das Haus explodiert wenn Du die folgende Zahl oder höher wirfst:

\vspace{1cm}
\begin{tabular}{cc}
Zahl der Spieler &      Haus Explodiert bei \\
3 &  8+ \\
4 &  7+ \\
5 &  6+ \\
6 &  5+ \\
\end{tabular}

\newpage


\monster{Das Phantom}{}{6}{5}{}

  \begin{itemize}
        \bitem Das Phantom erscheint immer, wenn die Helden einen Keller-Raum mit einem Ereignis- oder Omen-Symbol neu entdecken. Wenn dieser Raum erkundet wurde lege den Phantom- und den Mädchen-Marker in diesen Raum.
Lege auch einen der 20 verschiedenen Marker dort hinein.
        \bitem Wenn das Phantom erscheint kann ein Held dieses angreifen. Wenn das Phantom besiegt wurde ist es getötet und der Forscher bekommt das Mädchen – im anderen Fall flieht das Phantom mit dem Mädchen weiter, beide Marker werden wieder beiseite gelegt. Sie erscheinen wieder, wenn ein Forscher einen anderen Keller-Raum mit Ereignis- oder Omen- Symbol neu entdeckt.
        \bitem Die Helden können das Mädchen nicht per Spezial-Angriff stehlen.

        \bitem Wenn der gesamte Keller entdeckt wurde und das Phantom immer noch lebt bestimme vor jeder Monster-Runde einen Keller-Raum, wo beide in dieser Runde erscheinen werden.

        \bitem Das Phantom besucht keinen Raum im Keller 2x solange noch nicht überall die verschiedenen Marker liegen.
    \end{itemize}

\specialattackrules{
Das Phantom kann selbst nicht angreifen, aber es kann sich verteidigen. Bei einer erfolgreichen Verteidigung flieht es.

}

\Toutro{
 Tick, Tack, Tick, Tack, BOOOOOOM.
}