
\traitor{36}{Better with Friends}{Besser mit Freunden}{}

\introduction{
Vor vielen Jahren bist du in dem unterirdischen Sumpf unter dem Haus ertrunken. Seit dieser Zeit existierst du alleine als Geist in deinem feuchten Grab. Dann hast du das Medaillon gefunden. Seine Macht ermöglichte es dir in die sterbliche Form zurückzukehren und andere Personen zu finden, die diese zwielichtige Existenz mit dir teilen.
Du hast deine Freunde, die du hierher gebracht hast, sorgfältig ausgewählt. Nun ist es für sie Zeit dich im Tode zu begleiten. Schon hörst du, wie das Wasser das Basement füllt. Du darfst sie nicht entkommen lassen. Der Tod ist besser mit Freuden.

}

\Trightnow{

    \begin{itemize}
        \bitem Dein Charakter ist weiter im Spiel, wurde aber zum Verräter.
        \bitem Lege den sechseckigen Ruderboot-Marker (rowboat) in die Dachkammer. Ist die Dachkammer nicht im Spiel, lege den Marker hinein, sobald sie entdeckt wird.
        \bitem Lege 4 dreieckige Stärkewurf-Marker bereit.
        \bitem Nimm die Turn/Damage-Karte mit einem Marker auf der „0“. Damit hältst du die ablaufende Zeit fest.
    \end{itemize}
}

\whatyouknowabouttheheros{
Das Haus und seine Umgebung versinken in dem unterirdischen Sumpf. Die Helden werden versuchen, vorher zu entkommen.

}

\Tyouwinwhen{
entweder die hälfte aller Helden tot sind oder das Ruderboot funktionsunfähig ist.

}

\hauntsection{Während deines Spielzuges musst du}
Den Marker auf der Turn/Damage-Karte am Ende deiner Runde um 1 Feld weiter setzen.

\newpage
\hauntsection{Das Steigen des Wassers}

Abhängig von der Spielrunde steht das Wasser im Haus bis zu einer bestimmten Höhe. Sobald du den Marker der Turn/Damage-Karte ein Feld weiter setzt, informiere die Helden welche Stockwerke des Hauses überflutet sind und ob die Überflutung vollständig oder nur teilweise ist.

\vspace{1cm}
\begin{tabular}{rp{7cm}}
Runde & Wasserstand \\
1 & Basement teilweise überflutet \\
2 & Basement vollständig überflutet \\
3 & Wie 2 + Erdgeschoß teilweise überflutet \\
4 & 2 + Erdgeschoß vollständig überflutet \\
5 & Wie 4 + Obergeschoss teilweise überflutet \\
6 & Haus vollständig überflutet für den Rest des Spiels \\
\end{tabular}
\vspace{1cm}

Das Wasser beeinflusst alle Spieler aber nicht dich. Die Auswirkungen des Wassers stehen im Buch der Helden (secrets of survival).

\specialattackrules{
Du kannst einmal während deines Zuges das Ruderboot angreifen, aber nur mit einem Stärke-Angriff. Du kannst einen Stärkewurf von 3+ versuchen um das Boot zu beschädigen. Jeder Schaden am Ruderboot wird durch einen Stärkewurf-Marker auf deiner Charakterkarte angezeigt. Der 4. erfolgreiche Angriff zerstört das Boot; entferne es aus dem Haus, wenn das passiert.

}

\Toutro{
Die leblosen Körper deiner Freunde treiben neben dir her und stoßen in dem trüben Wasser gelegentlich aneinander. Nach einer Weile öffnen sie die Augen und du führst sie hinab in das kalte, dunkle Wasser. Du führst sie heim.
}