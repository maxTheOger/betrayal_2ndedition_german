
\traitor{40}{Buried Alive}{Lebendig begraben}{}

\introduction{
Das Séance-Brett bewegt sich hin und zurück, hin und zurück über die Buchstaben, ohne dass jemand seine Hand zu Hilfe nimmt. Alle starren erschreckt darauf und lesen LEBENDIG BEGRABEN. Dummes Brett. Es zeigt gerade deinen Gegnern dass sie jemanden vermissen- das Opfer das du lebendig begraben hast als du das Haus betratest.
Wenn du jetzt nichts machst finden deine Gegner wahrscheinlich das Opfer. Das wäre schlimm! Es ist schön jemanden zu quälen, aber wenn das Opfer nicht stirbt ist es wie … ja, wie ein Essen ohne Nachtisch. Und du liebst den Nachtisch.

}

\Trightnow{

    \begin{itemize}
    \bitem Dein Charakter ist weiter im Spiel, wurde aber zum Verräter
    \bitem Markiere die Turn/Damage-Karte mit einem Plastikmarker an der „0“. Du benutzt diese Karte als Uhr.
    \bitem Sind weniger als 5 Räume(Basement Landing eingeschlossen) im Basement entdeckt zieh weitere Basement Karten und lege sie beliebig aus bis Du 5 Räume liegen hast.
    \bitem Wähle einen Raum im Basement und schreibe ihn auf. Dort hast du die Person begraben.
    \end{itemize}
}

\whatyouknowabouttheheros{
ist, dass sie versuchen die begrabene Person zu retten.

}

\Tyouwinwhen{
die begrabene Person tot ist.

}
\newpage

\hauntsection{Während deines Zuges tust du folgendes}

  \begin{itemize}
    \bitem Am Ende jeder Runde bewege den Marker auf der Turn/Damage-Karte ein Feld weiter. Dann wirf so viele Würfel wie die Karte aktuell Punkte zeigt. Das Ergebnis gibt an wie viel physischen Schaden die Person erleidet.
    \bitem Halte fest, wie viel Schaden die begrabene Person jede Runde erleidet. Sie kann 12 Schadenspunkte absorbieren bevor sie stirbt.
\end{itemize}


\hauntsection{Das Spirit Board (Séance-Brett)}
  \begin{itemize}
    \bitem Das Spirit Board hilft den Gegnern. Derjenige der es gerade besitzt kann es niemand anderem geben oder fallen lassen.
    \bitem Du kannst einen speziellen Angriff machen, um das Spirit Board von den Gegnern zu stehlen (wenn du einen Stärkeangriff (Might) mit einem Ergebnis von mindestens 2+ Schadenspunkten machst, kannst du, anstatt dem Gegner den Schaden zuzufügen das Spirit Board stehlen). Das Spirit Board ist sofort zerstört und kommt aus dem Spiel.
\end{itemize}


\Toutro{
Ahh, ja. Deine Gegner waren leider zu langsam. Irgendwie hat das Ganze dich hungrig gemacht. Zeit für das Abendessen… mit Nachtisch.
}