
\haunttitle{Ghost Bride}{Geisterbraut}

\introduction{
Eine Erscheinung in einem weißen Spitzenkleid schimmert von deinen Augen. „Du hast mich alle diese langen Jahre allein gelassen“ sagt eine Frauenstimme, „Aber ich habe gewartet. Auf dich. Auf unsere Hochzeit.“ Der Geist gleitet vorwärts zu einem deiner Kameraden und sagt: „Wenn du erst einmal TOT BIST WIE ICH, können wir für IMMER zusammen sein.“
Der Geist verschwindet aber der schwache Klang einer Orgel, die einen Hochzeitsmarsch spielt durchstreift das Haus. Du hast Tränen in den Augen. Du hasst es, wenn Liebe zurückgewiesen wird. Du wirst diese Hochzeit erleben… … ob der Bräutigam will oder nicht.

}

\Trightnow{

    \begin{itemize}
        \bitem Dein Charakter ist weiter im Spiel, wurde aber zum Verräter.
        \bitem  Lege den magenta Braut-Marker in deinen Raum.
        \bitem Lege den Sarg-marker (Sarcophagus) in die Krypta (Crypt), entweder sofort oder sobald der Raum entdeckt wird.
        \bitem Die Geisterbraut wählt einen Bräutigam unter den anderen Spielern. Sie erwählt denjenigen der den Ring bei sich hat, es sei denn, dieser Charakter ist weiblich. In diesem Fall wählt sie den männlichen Abenteurer der am Ältesten ist. Danach sagt sie Laut wer der Bräutigam wird.
      (Sind nur weibliche Spieler im Spiel, nimm einen Männlichen Charakter aus der Box und stell seine Figur in die Eingangshalle. Diese kann sich nicht bewegen, aber kann angegriffen werden.)
        \bitem Nimm die Turn/Damage-Karte mit einem Plastik-Clip auf der „0“. Du benutzt diese Karte um die Zeit zu messen, nachdem die Hochzeit stattgefunden hat.
        \bitem Wenn die Kapelle (chapel) noch nicht entdeckt ist, suche sie aus dem Raum-Stapel heraus und lege sie so schwer wie möglich(für die Helden) zu erreichen hin.
       Dann mische den entsprechenden Stapel neu.
    \end{itemize}
}

\whatyouknowabouttheheros{
Sie werden alles daran setzen die Hochzeit zu verhindern.
}

\Tyouwinwhen{
du die Geisterbraut mit ihrem Bräutigam verheiratest.
}

\monster{Die Geisterbraut (@ 3-4 Spieler)}{4}{}{6}{}
\monster{Die Geisterbraut (@ 5-6 Spieler)}{5}{}{7}{}

Die Braut kann durch Wände ziehen (aber nicht durch Decken und Böden) und erleidet keinen physischen Schaden ausser durch den Ring. Wurde Sie besiegt erleidet sie keinen Schaden.



\specialattackrules{

    \begin{itemize}
        \bitem Die Braut macht Gesundheits-Angriffe (sanity) durch die sie allen Spielern, außer dem erwählten Bräutigam mentalen Schaden zufügt.
        \bitem Wenn Sie den Bräutigam angreift gelten folgende Regeln:
   1-2 Schadenspunkte = -1 might ; 3-4 Schadenspunkte = -2 might ; 5+ Schaden = -3 might
    \end{itemize}
}

\hauntsection{Wie man die Braut verheiratet}
      \begin{itemize}
            \bitem Zuerst tötet man den erwählten Bräutigam, der damit zu einem Geist unter deiner Kontrolle wird. Er lässt dabei alle Gegenstände fallen.
            \bitem Dann stelle Braut und Bräutigam in die Kapelle (chapel).
            \bitem Beginne mit der Hochzeit und starte auf der Turn/Damage-Karte.
            \bitem Am Ende deiner Runde setze den Turn/Damage-Marker 1 Feld weiter.
  Nach 3 Runden ist die Hochzeit vollendet.

        \end{itemize}


\Toutro{
Du musst bei Hochzeiten immer weinen.
}