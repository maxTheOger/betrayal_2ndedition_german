
\haunttitle{Perchance to Dream}{Vielleicht nur ein Traum}

\introduction{
Du hast immer aus diesen Augen geschaut, gefangen in einem Körper aus Fleisch. Aber was ist das? Dein Körper hat einen neuen Ort betreten, einen sehr, sehr bösen Ort. Einen Ort, in dem du - wie du mit einem Schauer der Vorfreude entdeckst - ungeahnte Macht besitzt.
Das ist deine Chance, alles zu ändern. Alles was du machst ist flüstern.

\normalfont{Ein Gedanke keimt in deinem Bewusstsein, flüstert dir zu, eine kleine Pause zu machen, dich hinzusetzen und für eine Zeit zu verweilen. `Warum lege ich mich nicht hin und mache ein kleines Nickerchen? Es ist so ein hübsches Schlafzimmer.`}
\itshape

Die Augen deines fleischlichen Körpers schließen sich und zum ersten Mal öffnen sich die Augen deines Unterbewusstseins. Es ist Zeit, deinen Träumen freien Lauf zu lassen. Lass das Schreien beginnen.
}

\Trightnow{
    \begin{itemize}
        \bitem Kippe deine Figur in dem Raum, in dem sie sich befindet, um. Dein Körper schläft. Du kannst dich nicht bewegen oder Aktionen ausführen. Lasse alle deine Gegenstände fallen. Lege den Hund, das Mädchen und den Verrückten (Dog, Girl, Madman)beiseite, wenn du sie besitzt und verstelle entsprechend deine Eigenschaften. Du kannst dadurch nicht sterben. Setze den Marker nötigenfalls auf 1 Stufe über Totenkopf.
        \bitem Setze so viele pinke \chips{Nightmare}{Albtraum} wie Spieler mitspielen in den Raum, in dem du schläfst.
        \bitem Zähle unauffällig die Anzahl an Fluchträumen im Haus. Dieses sind alle Räume mit nach außen gehenden Fenstern und Türen: Grand Staircase, Master Bedroom, Bedroom, Chapel, Dining Room, Conservatory, Entrance Hall, Gardens, Graveyard, Patio, Tower und Balcony. Wenn ein Raum so liegt, dass sein Fenster falsch ist, dass heißt an eine Wand angrenzt, zählt er trotzdem als Fluchtraum.

        Gibt es weniger Fluchträume als Spieler, füge aus dem Stapel die nötige Zahl Räume hinzu und mische den Stapel neu.

        Schreibe die Anzahl der Fluchträume auf, aber verberge sie vor den anderen Spielern.
        \bitem Lege so viele Chips wie es Fluchträume gibt beiseite. (Es gibt keine speziellen Chips dafür, nimm irgendwelche anderen, zB die Item-Chips).
    \end{itemize}
}

\whatyouknowabouttheheros{
Sie versuchen deinen physischen Körper wieder aufzuwecken.
}

\Tyouwinwhen{
… so viele Alpträume dem Haus entfliehen können, wie Du auf deinem Zettel vermerkt hast. Wenn das passiert, zeigst du den Helden deine notierte Zahl.
}
\hauntsection{Wie Alpträume entkommen}
  \begin{itemize}
        \bitem Ein Albtraum in einem Flucht-Raum entkommt mit einer weiteren Bewegung.
        \bitem Ein Fluchtzimmer kann nur 1x benutzt werden, nach einer erfolgreichen Flucht aus einem Raum wird dort zur Markierung ein Fluchtchip platziert. Wenn neue Flucht-Räume entdeckt werden kannst Du sie nutzen, es erhöht jedoch nicht die nötige Zahl auf deinem Zettel.
        \bitem Wann immer ein Alptraum getötet wird oder geflohen ist, kannst Du einen neuen Alptraum erschaffen und in den Raum stellen, wo Du schläfst. (Verwende die Chips erneut.)
    \end{itemize}


\monster{Die Alpträume}{5}{4}{4}{}

\specialattackrules{

    \begin{itemize}
        \bitem Die Alpträume können Forscher mit ihrem \might-Wert angreifen. Sie verursachen aber immer mentalen Schaden anstelle physischen Schadens.
        \bitem Wenn ein Alptraum durch eine Heldenattacke besiegt wurde, ist er tot und nicht nur betäubt. Wird der Alptraum besiegt während er selbst angreift, ist er nur betäubt.
    \end{itemize}
}

\Toutro{
Du schaust zum ersten Mal von außerhalb in die plötzlich geöffneten Augen deines Körpers. Entsetzt erkennt dein fleischlicher Körper dich, das Unterbewußte, das solange in ihm verschlossen war. Er versucht seinen grauenhaften Horror auszuschreien, aber der Schrei erstickt unter einem Schwarm lebendiger Alpträume.
}