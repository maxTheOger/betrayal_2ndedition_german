
\haunttitle{House of the Living Dead}{Haus der lebenden Toten}

\introduction{
Müde lehnst du dich mit dem Rücken gegen die Wand. Irgendetwas hinter dieser Wand macht ein Geräusch. Tick. Tick-tick. Tick. Was ist das? Vielleicht Ratten oder ein Insekt? Du erinnerst dich daran, dass sich einmal Wespen bei dir zu Hause in der Wand breit gemacht haben. Verfluchtes Ungeziefer! Du bückst dich runter um durch ein großes Loch in der Holzverkleidung zu spähen. Eine aschgraue Hand schießt plötzlich hervor, packt dich im Nacken und zieht dich durch die Wand. Du stirbst bevor du überhaupt schreien kannst.
Blinzel. Blinzel Schnüff. „Mmmh, Hunger. Muss Fleisch fressen. Töten. Fressen. FRESSEN!“

}

\Trightnow{

    \begin{itemize}
        \bitem Dein Charakter ist tot. Leg all deine Gegenstände im Zimmer ab und ersetze deine Figur durch den runden ZOMBIE LORD - Marker
        \bitem Platziere so viele rote ZOMBIE – Marker wie Spieler teilnehmen in den folgenden Räumen (in der Reihenfolge): CRYPT, GRAVEYARD, ENTRANCE HALL, UNDERGROUND LAKE, GARDENS, CHAPEL,CONSERVATORY und PENTAGRAM CHAMBER.
        \bitem Sind mehr Marker übrig als entdeckte Räume leg die Zombies in der Reihenfolge in die vorhandenen Räume.
        \bitem Danach legst Du erneut einen Zombie in die Räume die bereits einen(oder mehrere) Zombies haben
        \bitem Befestige am Turn/Damage Track einen Plastick-Clip bei der 0. Du benutzt ihn um Schadenspunkte anzuzeigen.
    \end{itemize}
}

\whatyouknowabouttheheros{
Sie kämpfen gegen dich. Meide angriffslustige Abenteurer die Waffen besitzen, wenn möglich.
}

\Tyouwinwhen{
...alle Helden tot sind.
}



\monster{Zombies}{2}{5}{2}{}
\hauntsection{Besondere Kampfregeln}
  \begin{itemize}
        \bitem Wenn ein Zombie mit einer Waffe in einer MIGHT-Attacke besiegt wird, stirbt der Zombie. Jede andere Attacke betäubt den Zombie nur. Dynamit töten den Zombie ebenfalls.
        \bitem Wenn ein Abenteurer stirbt, wird er automatisch zum Zombie und bekommt auch die Eigenschaftswerte eines Zombies.
        \bitem Der Abenteurer muss dann das Verräter Buch lesen und gegen die anderen Kämpfen. Wenn der Verräter gewonnen hat und der Spielerzombie einen anderen Spieler getötet hat gewinnt der Spieler ebenfalls.
        \bitem Zombies können den Aufzug nicht benutzen, aber der Zombielord kann.
    \end{itemize}

\monster{Zombie Lord}{3}{7}{2}{}

\hauntsection{Besondere Kampfregeln}
  \begin{itemize}
    \bitem Du (als Zombie Lord) kannst nur von einem Abenteurer verletzt werden der das Medaillon bei sich trägt. Anstatt betäubt zu werden, kannst du bis zu 7 Schäden vertragen bevor du stirbst. Benutze den TURN/DAMAGE TRACK um die erhaltenen Schäden zu markieren.
    \bitem Der Zombielord kann (genau wie normale Monster) keine Räume entdecken oder Gegenstände aufnehmen.
  \end{itemize}

\Toutro{
Alle sind tot. Schnüff. Immer noch Hunger. Schnüff, schnüff. Es kommt dir der Gedanke in dein totes, verdammtes Bewusstsein, dass auch das Fleisch von Zombies Fleisch ist. Wenn auch nicht so lecker.
Du schreitest vor. „Mmmh, Hunger. Muss Zombie-Fleisch fressen. Töten. Fressen. FRESSEN!“
}