
\traitor{29}{Frankenstein’s Legacy}{Frankenstein's Vermächtnis}{}

\introduction{
Du schlägst das Buch auf, und liest zufällig diese Passage:
"...um den Sinn des Lebens zu untersuchen, müssen wir als erstes nach dem Tod greifen... ein Kirchfriedhof ist lediglich ein Behälter für Körper denen das Leben entzogen wurde und jetzt Futter für Würmer. Ich alleine habe dieses erstaunliche Geheimnis entdeckt: dass das was einst tot war, wiederbelebt werden kann!"
Und dann:
"Die Knochen sind aus dem Leichenhaus eingesammelt worden... die Haut wurde von Männern eingeheimst, die nicht länger als 3 Tage tot sind...die Bestandteile, groß und klein, wurden zusammengenäht, um den Anschein eines menschlichen Grundgestells zu erzeugen. Es erwartet nur genügend Spannung um zu laufen, noch einmal und es LEBT!"
Du bemerkst, mit einem wachsenden Gefühl der Aufregung, dass du über ein Labor-Notizbuch gestolpert bist, welches ausführlich die Wiederbelebung toten Gewebes beschreibt. Und du kannst dieses Experiment beenden. Du musst. Im Namen der Wissenschaft.

}

\Trightnow{

    \begin{itemize}
        \bitem Dein Charakter ist weiterhin im Spiel, aber hat sich in den Verräter verwandelt.
        \bitem Lege das runde Frankenstein's Monster-Plättchen entweder in das Forschungslabor (Research Laboratory) oder ins Operationslabor (Operating Laboratory). Wenn keiner dieser Räume im Haus ist, durchsuche den Stapel mit den Raumkarten danach und lege ihn mit dem Frankenstein's Monster-Plättchen darauf in die obere Etage. Danach mische den Stapel neu.
    \end{itemize}
}

\whatyouknowabouttheheros{
Sie denken dein Monster ist eine Scheußlichkeit und wollen versuchen es zu töten. Sie wissen ebenfalls, dass das Monster durch Feuer verwundbar ist.

}

\Tyouwinwhen{
... alle Helden tot sind.
}
\newpage


\monster{Frankenstein’s Monster}{3}{8}{}{}

Frankenstein's Monster muss immer mit voller Geschwindigkeit zu dem nächsten Abenteurer laufen, den es angreifen kann. Es macht, wenn möglich, einen Angriff pro Zug.




\specialattackrules{

    \begin{itemize}
        \bitem Frankenstein's Monster bekommt 2 Punkte auf sein Angriffsergebnis gut geschrieben. (nicht auf seine Verteidigung)
        \bitem Frankenstein's Monster kann nicht durch den Revolver,Dynamit beschädigt werden.
        \bitem Wenn du mit einem Kraftangriff mehr als 2 Punkte Schaden zufügst, kannst du stattdessen deinem Gegner eine Fackel stehlen. Wenn du das erfolgreich machst, zerstörtst du die Fackel.
    \end{itemize}
}

\Toutro{
Als du die Haut deiner früheren Freunde einheimst, passt du auf nicht zuviel zu zerreißen, du schneidest gerade Streifen. Du denkst an dein rosiges Schicksal. Beim Wiederbeleben des Körpers, welchen du in dem Labor gefunden hast (und aus dessen Taten du später Nutzen ziehen wirst) befindest du dich in einem Überschuss aus Haut, Organen, Zähnen und Knochen. Mit dem was du jetzt hast, kannst das Experiment das in dem Buch beschrieben wurde ganz von neuem beginnen. Ist Wissenschaft nicht wundervoll?
}