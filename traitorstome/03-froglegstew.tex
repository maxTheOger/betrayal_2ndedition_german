
\haunttitle{Frog-Leg Stew}{Froschschenkeleintopf}

\introduction{
Aus den Schatten dringt eine Stimme zu dir durch: “Du wirst mir helfen, das wirst du doch mein dickes kleines Gänschen? Schreckliche Leute sind überall in meinem Haus...und dann wagten sie auch noch mein schönes Zauberbuch zu stehlen! Hilf mir sie zu bestrafen mein Liebling! Sei ein braves kleines Püppchen!“

}

\rightnow{

    \begin{itemize}
        \bitem Dein Charakter ist immer noch im Spiel, ist nun aber der Verräter.
        \bitem Leg dir die rote Katzenscheibe (Cat-Token) bei Seite.
        \bitem Leg die vier orangene Froschscheiben (Frog-Token) bei Seite.
        \bitem Lege die runde Hexenscheibe (Witch-Token) in die Eingangshalle (Entrance Hall)
        \bitem Lege jetzt oder wenn die Räume später entdeckt werden eine Wurzelscheibe (Root-Token) in den Wintergarten (Conservatory), eine in die Speisekammer (Larder), eine in die Küche (Kitchen). Verrate nicht in welche Räume die Wurzelscheiben gelegt werden solange sie nicht aufgedeckt wurden. Dein Charakter kann keine Wurzeln einsammeln.
    \end{itemize}
}

\whatyouknowabouttheheros{
Sie haben das Zauberbuch (Book-Karte) der Hexe. Du solltest versuchen es von ihnen zu bekommen, denn damit könnten sie die Unverwundbarkeit der Hexe durchbrechen.

}

\Tyouwinwhen{
...alle Helden tot sind oder sie in Frösche verwandelt wurden.
}

\hauntsection{Wenn der erste Abenteurer in einen Frosch verwandelt wurde, dann lege die Katzenscheibe in den Raum, in dem der Spuk begann. Die Katze bewegt sich während der Monsterziehphase zum nächst gelegenen Frosch.
Wenn die Katze im selben Raum wie ein Frosch ist, dann greift sie diesen mit einem Kraftwurf an. Gewinnt die Katze, frisst sie den Frosch. Die Katze ignoriert Frösche, die getragen werden.
}


\monster{Die Katze}{3}{3}{2}{}

\monster{Die Hexe}{4}{3}{6}{}

\hauntsection{Zaubersprüche der Hexe:}
Die Hexe kann immer nur einen Zauber pro Runde sprechen. Sie kann nicht normal angreifen.
Wenn die Hexe das Buch von den Helden bekommt kann sie jedoch bei jedem Zug die ersten beiden Zaubersprüche nutzen.
  \begin{itemize}
        \bitem \emph{Froschhaut:} Die Hexe kann diesen Zauber auf einen Helden im selben Raum sprechen. Die Hexe und der Held machen einen Verstandswurf (Sanity). Hat die Hexe einen höheren Wert gewürfelt als der Held, dann verwandelt sich dieser in einen Frosch und er verliert alle seine Gegenstände. Tausche den spieler gegen ein Frosch Token aus.Die Kraft (Might) und das Wissen (Knowlege) werden auf sein niedrigsten Wert eingestellt. Ein Frosch kann nicht angreifen, keine Karten ziehen und keine Räume aufdecken. Ein anderer Abenteurer kann den Frosch aufnehmen und wie ein Gegenstand tragen. Weder ein Spieler noch die Hexe kann den Frosch angreifen. Der Frosch dient nur der Katze als Snack.
        \bitem \emph{Feuerodem:} Die Hexe kann diesen Zauber auf einen Helden in Ihrer Sichtlinie (eine ununterbrochene gerade Linie von offenen Türen) oder im selben Raum sprechen.
Der Zauber verursacht am Helden zwei Würfel physischen (physical) Schaden.
        \bitem \emph{Rabenflügel:} Diesen Zauber kann die Hexe auf sich selber anwenden um sich in jeden Raum in dem Haus zu teleportieren. Sie kann diesen Zauber auch auf dich anwenden, wenn du mit ihr in einem Raum bist.
    \end{itemize}

\specialattackrules{
Die Hexe ist zur Zeit unverwundbar: Sie kann nicht angegriffen werden. Sie kann nicht selber normal angreifen aber sie kann jede Runde einen Zauber sprechen. Sie kann keine Gegenstände aufnehmen.

}

\outro{
„Ein kleines Fröschchen, zwei kleine Fröschchen...Mach weiter mein Püppchen, verfüttere deine bösen Froschfreunde an dieses süße Kätzchen. Da ist ja auch schon mein geliebtes Küken.“
}