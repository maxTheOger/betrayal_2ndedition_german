
\traitor{26}{Pay the Piper}{Zahl die Zeche}{}

\introduction{
Du hörst das leise Knuspern und Kratzen die ganze Zeit. Deine Freunde scheinen es nicht zu bemerken. Aber Du tust es. Ratten im Gemäuer. Zuerst ignorierst Du die Ratten. Aber dann hörst Du ihre flüsternden kleinen Stimmen in Deinem Kopf. Sie flüstern Worte, die Freundschaft versprechen, Vertrauen, Hingabe. Alles für Dich. Warum? Weil Du ihr Freund bist, ihre Sippe. Du bist eine Werratte mit spezieller Kraft und Möglichkeiten. Das Knuspern und Kratzen
Deiner Rattenverwandschaft sagt Dir, was Du tun mußt.

}

\Trightnow{

    \begin{itemize}
        \bitem Dein Charakter verbleibt im Spiel und wird zum Verräter.

        \bitem Wenn einige Deiner Characterwerte unterhalb des Startwertes sind, setzte sie auf den Startwert zurück und addiere überall +1 auf.

        \bitem Setze eine Anzahl an roten Ratten-Marker neben das Spielfeld. Es müssen doppelt soviele sein wie Spieler teilnehmen. Lege 1 davon in jeden Raum mit Ereignis- Gegenstand- und Omen-Symbol. Sind noch Marker übrig legst Du in Räume Deiner Wahl einen 2. Marker usw. Sind es weniger Marker entscheidest Du, wo sie hingelegt werden.

        \bitem Lege 5 dreieckige Sanity/Gesundheits-Wurf-Marker neben das Spielfeld.

    \end{itemize}
}

\whatyouknowabouttheheros{
Sie versuchen alle Ratten im Haus zu töten was Dich daran hindert, Dein Ritual zu vollenden.


}

\Tyouwinwhen{
entweder alle Helden tot sind oder Du Dein gottloses Ratten-Ritual vollendest.

}
\newpage


\monster{Ratten}{3}{2}{1}{}



\specialattackrules{

    \begin{itemize}
        \bitem Wenn eine Ratte besiegt wird ist sie tot und nicht nur ohnmächtig.

        \bitem Eine Gruppe von Ratten im selben Raum kann gemeinsam angreifen. Sie addieren hierzu ihre Macht-Werte zu einem Gesamtwert, hierbei dürfen maximal 8 Würfel verwendet werden. Bei einem Fehlschlag einer Gruppe von Ratten erleiden sie keinen Schaden.

        \bitem Im Pentagramm-Zimmer können Dir die Helden nichts anhaben.
Weder die Ratten noch die Helden dürfen diesen Raum betreten.


    \end{itemize}
}
\hauntsection{Wie Du das Ritual vollendest}

  \begin{itemize}
        \bitem Gehe in das Pentagramm-Zimmer. Hier bist Du während Deiner Arbeit am Ritual sicher.

        \bitem Du kannst einen Gesundheitswurf von 3+ versuchen, um das Ritual auszuführen. Bei Erfolg nehme 1 Gesundheits-Marker und lege diesen auf Deine Charakter-Karte. Gleichzeitig legst Du 1 Ratten-Marker (wenn noch welche verfügbar sind) in ein benachbartes Zimmer (dieses muß nicht per Tür mit dem Pentagramm-Zimmer verbunden sein). Die Anzahl an erfolgreichen Gesundheitswürfen die Du benötigst, um das Ritual zu vollenden beträgt:

        \begin{tabular}{cc}
            Anzahl der Spieler & Benötigte Würfe \\
            3-4 & 5 \\
            5-6 & 4 \\
        \end{tabular}
    \end{itemize}

\Toutro{
Deine lieben, lieben Kleinen lecken das vergossene Blut auf und balgen sich pfeifend über den größeren Teilen. Kinder sind halt Kinder, und Kinder müssen essen.
}