

\event{A Moment Of Hope}{Ein Moment der Hoffnung}{
    Irgendetwas in diesem Raum fühlt sich seltsam richtig an. Etwas widersteht dem Bösen des Hauses.
}{
    Plaziere einen \emph{Blessing}-Chip (Segnung) in diesem Zimmer.

    Jeder Held verwendet bei jedem Charakterwurf (Might, ...) in diesem Zimmer einen zusätzlichen Würfel.
}

\event{Angry Being}{Böses Wesen}{
    Es schält sich aus dem Schleim an der Wand neben dir hervor.
}{
    Du musst einen \speedroll\ versuchen:
    \rolls
    \roll{5+}{Du kommst davon. \gainspeed{1}.}
    \roll{2-4}{\takediementaldamage{1}}
    \roll{0-1}{\takediephysicalmentaldamage{1}{1}}
    \erolls
}

\event{Bloody Vision}{Blutige Vision}{
    Die Wände in diesem Raum sind getränkt in  Blut. Es tropft von der Decke, fließt die Wände herunter, über Schränke und Möbel und auf deine Schuhe. Im nächsten Augenblick ist es fort.
}{
    Du musst einen \sanityroll\ versuchen:
    \rolls
    \roll{4+}{Du festigst deinen Geist. \gainsanity{1}.}
    \roll{2-3}{\loosesanity{1}.}
    \roll{0-1}{Wenn sich ein Entdecker oder Monster in deinem oder einem angrenzenden Zimmer befinden, musst du ihn/sie/es angreifen (wenn du kannst). Wähle denjenigen mit der niedrigsten \might.}
    \erolls
}


\event{Burning Man}{Brennender Mann}{
    Ein brennender Mann rennt durch den Raum. Seine Haut schlägt Blasen und zerfällt. Ein glutroter Schädel verbleibt, schlägt auf dem Boden auf, rollt und verschwindet.
}{
    Du musst einen \sanityroll\ versuchen:
    \rolls
    \roll{4+}{Es wird heiß am Kragen, du bist aber ansonsten okay. \gainsanity{1}.}
    \roll{2-3}{Raus, raus, du musst hier raus! Setze deinen Entdecker in die Eingangshalle.}
    \roll{0-1}{Du gehst in Flammen auf. \takediephysicaldamage{1} Dann nehme 1 Würfel mentalen Schaden als du die Flammen ausklopfst.}
    \erolls
}

\event{Closet Door}{Schranktür}{
    Die Schranktür dort ist offen... nur einen Spalt. Darin muss sich etwas befinden.
}{
    Lege den \chip{Closet}{Schrank} in dieses Zimmer.

    Einmal während seines Zuges, kann ein Entdecker zwei Würfel werfen, um den Schrank zu öffnen.

    \rolls
    \roll{4}{Ziehe eine \itemcardd .}
    \roll{2-3}{Ziehe eine \eventcard .}
    \roll{0-1}{Ziehe eine \eventcard\ und entferne den \chipe{Closet}.}
    \erolls

}

\event{Creepy Crawlies}{Gruselige Krabbler}{
    Eintausend Käfer stürzen sich aus deiner Haut, kommen unter deiner Kleidung und aus deinen Haaren hervor.
}{
    Du musst einen \sanityroll\ versuchen:
    \rolls
    \roll{5+}{Du blinzelst und sie sind weg. \gainsanity{1}.}
    \roll{1-4}{\loosesanity{1}.}
    \roll{0}{\loosesanity{2}.}
    \erolls

}


\event{Creepy Puppet}{Gruselige Puppe}{
    Du siehst eine dieser Puppen, bei denen dir die Haare zu Berge stehen. Sie springt dich mit einem kleinen Speer in den Händen an.
}{
    Der Spieler zu deiner Rechten wirft einen \mightroll\ mit 4 Würfeln für die Puppe. Du verteidigst dich normal, indem du einen \mightroll\ ausführst.

    Wenn du Schaden davonträgst, erhält der Entdecker mit dem Speer (Spear) 2 \might\ (es sei denn, du besitzt den Speer).
}


\event{Debris}{Schutt}{
    Mörtel fällt von den Wänden und der Decke.
}{
    Du musst einen \speedroll\ versuchen:
    \rolls
    \roll{3+}{Du weichst aus. \gainspeed{1}.}
    \roll{1-2}{Du bist unter Schutt vergraben. \takediephysicaldamage{1}}
    \roll{0}{Du bist unter Schutt vergraben. \takediephysicaldamage{2}}
    \erolls

    Wenn du unter Schutt begraben bist, behalte diese Karte. Du kannst nichts machen, bis du befreit wurdest. Einmal pro Zug kann ein Entdecker einen \mightroll\ versuchen, um dich zu befreien. (Du kannst diesen Wurf ebenfalls versuchen.) 4+ befreit. Nach drei erfolglosen Versuchen befreist du dich in deinem folgenden Zug automatisch und kannst normal agieren.
}

\event{Disquieting Sounds}{Beunruhigende Geräusche}{
    Das Geschrei eines Babys, einsam und verlassen.

    Ein Entsetzensschrei.

    Das Krirschen zerbrechenden Glases.

    Dann: Stille.
}{
    Würfle mit 6 Würfeln. Wenn du genauso viele oder mehr Augen würfelst, als Omen aufgedeckt wurden, erhälst du 1 \sanity.
    Falls nicht, nehme 1 \emph{Würfel} geistigen Schaden.
}

\event{Drip ... Drip ... Drip ...}{Tropf ... Tropf ... Tropf ...}{
    Ein rhythmisches Geräusch nagt an deinem Verstand.
}{
    Lege einen \chip{Drip}{Tröpfeln} in diesen Raum.

    Jeder Entdecker rollt bei jedem Charakterwurf (Might, ...) in diesem Raum mit einem Würfel weniger.
}

\event{Footsteps}{Fußabdrücke}{
    Die Dielen knarren leise. Staub steigt auf. Fußabdrücke erscheinen auf dem schmutzigen Boden. Als sie dich schließlich erreichen, sind sie plötzlich fort.
}{
    Würfle mit einem Würfel. (Befindest du dich in der Kapelle, würfle mit 2 Würfeln.)

    \rolls
    \roll{4}{Du und der nächstgelegene Entdecker erhalten 1 \might.}
    \roll{3}{Du erhälst 1 \might\ und der nächst\-ge\-le\-ge\-ne Entdecker verliert 1 \sanity.}
    \roll{2}{\loosesanity{1}.}
    \roll{1}{\loosespeed{1}.}
    \roll{0}{Jeder Entdecker verliert eine Stufe einer Charaktereigenschaft (Might, ...) seiner Wahl.}
    \erolls
}

\event{Funeral}{Beerdigung}{
    Du siehst einen offenen Sarg. Von innen.
}{
    Du musst einen \sanityroll versuchen:
    \rolls
    \roll{4+}{Du blinzelst und er ist verschwunden. \gainsanity{1}.}
    \roll{2-3}{Die Vision verstört dich. \loosesanity{1}.}
    \roll{0-1}{Du befindest dich wirklich in dem Sarg. Du nimmst 1 \sanity\ und 1 \might\ Schaden, als du dich herausgräbst. Wenn der Friedhof (Graveyard) oder die Crypta (Crypt) entdeckt wurden, versetze deine Figur in einen dieser Räume. (Du wählst aus.)}
    \erolls
}

\event{Grave Dirt}{Grabesschmutz}{
    Dieser Raum ist unter einer dicken Schicht Dreck begraben. Du hustest, als sich der Staub auf deiner Haut und in deinen Lungen absetzt.
}{
    Versuche einen \mightroll:
    \rolls
    \roll{4+}{Du schüttelst den Staub ab. \gainmight{1}.}
    \roll{0-3}{Irgendwas stimmt nicht. Behalte diese Karte. Nehme 1 phsyischen Schaden am Anfang jeder deiner Züge. Lege diese Karte aus dem Spiel, sobald einer deiner Charakterwerte steigt oder du deinen Zug auf dem Balkon, im Garten, auf dem Friedhof, im Turnraum, im Lagerrraum, auf der Veranda oder auf dem Turm beendest. (Balcony, Gardens, Graveyard, Gymnasium, Larder, Patio or Tower)}
    \erolls
}

\event{Groundkeeper}{Hausmeister}{
    Du drehst dich um und siehst einen Mann in Gärtnerkleidung. Er hebt seine Schaufel und greift an. Zentimeter vor deinem Gesicht verschwindet er, einzig matschige Fußabdrücke hinterlassend.
}{
    Versuche einen \knowroll. (Ein Entdecker im Garten verzichtet auf zwei seiner Würfel).

    \rolls
    \roll{4+}{Du findest etwas im Schlamm. Ziehe eine \itemcardd.}
    \roll{0-3}{Der Hausmeister erscheint wieder und schlägt dir die Schaufel ins Gesicht. Der Spieler zu deiner Rechten wirft einen \mightroll\ mit 4 Würfeln für den Hausmeister. Du verteidigst dich normal, indem du einen \mightroll\ ausführst. }
    \erolls
}


\event{Hanged Men}{Die Erhängten}{
    Ein Hauch kühler Luft fährt durch den Raum. Vor dir hängen drei Männer an ausgefransten Seilen. Sie starren dich mit kalten, toten Augen an. Das Trio pendelt sanft im Wind und verschwindet hinter dem Vorhang aus Staub, der von der Decke herabrieselt. Du fängst an zu husten.
}{
    Du musst einen Wurf für jede Charaktereigenschaft (Might, ...) werfen:
    \rolls
    \roll{2+}{Der entsprechende Wert bleibt unbeeinflusst.}
    \roll{0-1}{Du verlierst 1 Stufe der entsprechenden Eigenschaft.}
    \erolls

    Wenn du bei allen 4 Würfen 2+ wirfst, erhälst du einen zusätzlichen Punkt zu einem Charakterwert deiner Wahl.
}


\event{Hideous Shriek}{Scheußliches Kreischen}{
    Es beginnt mit einem Flüstern, aber endet in einem seelenzerreißenden Schrei.
}{
    Jeder Entdecker muss einen \sanityroll\ versuchen:

    \rolls
    \roll{4+}{Du widerstehst dem Geräusch.}
    \roll{1-3}{\takediementaldamage{1}}
    \roll{0}{\takediementaldamage{2}}
    \erolls

    Jedes Ergebnis betrifft nur den Entdecker, der würfelt.
}

\event{Image In The Mirror}{Bild im Spiegel}{
    (Version ohne fettgedruckten Einleitungstext.)
}{
    Es befindet sich ein alter Spiegel im Zimmer. Deine erschrockene Reflektion bewegt sich von alleine. Du erkennst dich, aber in einer anderen Zeit. Deine Reflektion schreibt auf den Spiegel:
    \vspace{\parsep}
    \begin{tightcenter}\begin{bf}DAS WIRD HELFEN\end{bf}\end{tightcenter}
    Dann reicht sie dir einen Gegenstand durch den Spiegel.

    Ziehe eine \itemcardd.
}

\event{Image In The Mirror}{Bild im Spiegel}{
    Wenn du keine \itemcards\ besitzt, gilt dieser Effekt für den nächsten Entdecker zu deiner Linken, der eine \itemcardd\ besitzt. Lege diese Karte aus dem Spiel, wenn niemand eine \itemcardd\ besitzt.
}{
    Es befindet sich ein alter Spiegel im Zimmer. Deine erschrockene Reflektion bewegt sich von alleine. Du erkennst dich, aber in einer anderen Zeit. Du musst deiner Reflektion helfen, also schreibst du auf den Spiegel:
    \vspace{\parsep}
    \begin{tightcenter}\begin{bf}DAS WIRD HELFEN\end{bf}\end{tightcenter}
    Dann reiche einen Gegenstand durch den Spiegel.

    Wähle eine deiner Gegenstandskarten (aber keine \omencard) und lege sie auf das Itemdeck. Dann mische das Deck. Erhalte 1 \know.
}

\event{It Is Meant to be}{So soll es sein}{
    Du brichst auf dem Boden zusammen, Visionen zukünftiger Ereignisse fließen durch deine Gedanken.
}{
    Wähle eine dieser beiden Optionen:
    \rolls
    \roll{$\bullet$}{Schaue dir die drei obersten Karten einer der vier Decks an, vertausche sie nach Wahl und lege sie zurück auf den Kartenstapel. Verrate niemandem dein Wissen.}
    \roll{$\bullet$}{Du kannst stattdessen auch mit vier Wür\-feln werfen und das Ergebnis aufschreiben. Bei irgendeinem zukünftigen Wurf kannst du dieses Ergebnis verwenden anstatt zu wür\-feln. Wenn diese Zahl größer als das maximal mögliche Ergebnis ist, verwende das höchst\-mög\-liche Ergebnis.}
    \erolls
}


\event{Jonah's Turn}{Jonah's Zug}{
    Zwei Jungen spielen mit einem hölzernen Kreisel. ``Willst du auch mal drehen, Jonah?'' fragt einer.

    ``Nein'', sagt Jonah, ``Ich will ihn ganz.'' Jonah nimmt den Kreisel und schlägt dem anderen Jungen ins Gesicht. Der Junge fällt. Jonah schlägt  ihn noch, als der Anblick schwindet.
}{
    Wenn ein Entdecker die Rätsel-Schachtel (Puzzle Box) hat, legt dieser sie aus dem Spiel und zieht stattdessen eine Ersatz-\itemcardd. Wenn dies passiert, erhälst du 1 \sanity. Andernfalls nehme du einen Würfel mentalen Schaden.
}

\event{Lights Out}{Lichter aus}{
    Deine Taschenlampe erlischt. Keine Sorge, jemand anderes hat Batterien.
}{
    Behalte diese Karte. Du kannst pro Zug nur ein Feld laufen bis du deinen Zug bei einem der anderern Entdecker beendest. Lege diese Karte danach aus dem Spiel. Du kannst du nun wieder normal laufen.

    Wenn du die Kerze besitzt oder deinen Zug im Ofenraum (Furnace Room) beendest, lege sie ebenfalls beiseite.
}

\event{Locked Safe}{Verschlossener Safe}{
    Hinter einem Portrait befindet sich ein Wandsafe. Natürlich verklemmt.
}{
    Lege einen \chipe{Safe} in den Raum.

    Einmal pro Runde kann ein Entdecker den Safe mithilfe eines \knowroll\ öffnen:

    \rolls
    \roll{5+}{Ziehe zwei \itemcards\ und entferne den \chipe{Safe}.}
    \roll{2-4}{\takediephysicaldamage{1} Der Safe bleibt verschlossen.}
    \roll{0-1}{\takediephysicaldamage{2} Der Safe bleibt verschlossen.}
    \erolls
}

\event{Mists From The Walls}{Nebel aus den Wänden}{
    Nebel fließt aus den Wänden heraus. Man erkennt Gesichter in dem Dunst, menschliche und ... unmenschliche.
}{
    Jeder Entdecker im Keller (Basement) muss einen \sanityroll\ versuchen:

    \rolls
    \roll{4+}{Die Gesichter sind nur Einbildungen in Licht und Schatten. Alles ist gut.}
    \roll{1-3}{\takediementaldamage{1} (Nehme einen Würfel zusätzlichen mentalen Schaden, wenn sich dein Entdecker in einem Raum mit einem Ereignissymbol befindet.)}
    \roll{0}{\takediementaldamage{1} (Nehme zwei Würfel zusätzlichen mentalen Schaden, wenn sich dein Entdecker in einem Raum mit einem Ereignissymbol befindet.)}
    \erolls

    Jedes Ergebnis betrifft nur den Entdecker, der würfelt.
}

\event{Mystic Slide}{Mystische Rutsche}{
    Bist du im Keller (Basement), betrifft dieses Event den nächsten Entdecker zu deiner Linken, der sich nicht im Keller aufhält. Lege diese Karte aus dem Spiel, wenn alle Entdecker im Keller sind.

    Der Boden vor dir fällt in die Tiefe ab.
}{
    Setze den \chip{Slide}{Rutsche} in diesen Raum, dann versuche einen \mightroll\ um zu rutschen.

    \rolls
    \roll{5+}{Du kontrollierst die Rutsche. Versetze deine Figur in einen Raum deiner Wahl in irgendeinem Stock unterhalb des Ausgangsstockwerks.}
    \roll{0-4}{Ziehe Zimmerkarten bis du eine Kellerkarte ziehst. Plaziere die Karte. (Wenn keine Kellerräume mehr auf dem Stapel liegen, nehme einen bereits existierenden Kellerraum.) Du fällst in diesen Raum und nimmst einen Würfel physischen Schaden. Wenn du nicht an der Reihe bist, ziehe keine Karte für diesen Raum.}
    \erolls

    Ab jetzt kann jeder Entdecker versuchen, zu rutschen.
}


\event{Night View}{Nächtliche Aussicht}{
    Du siehst die Vision eines geisterhaften Pärchens über das Gelände laufen, leise wandelnd in ihrer Hochzeitsgaderobe.
}{
    Du musst einen \knowroll\ versuchen:
    \rolls
    \roll{5+}{Du erkennst die Geister als frühere Bewohner des Hauses. Du rufst ihre Namen, sie drehen sich zu dir um, dunkle Geheimnisse des Hauses flüsternd. \gainknow{1}.}
    \roll{0-4}{Du fährst erschrocken zurück, unfähig zuzusehen.}
    \erolls
}

\event{Phone Call}{Anruf}{
    Ein Telefon klingelt im Zimmer. Du fühlst dich verpflichtet abzunehmen.
}{
    Würfle mit zwei Würfeln. Die zuckersüße Stimme einer alten Frau sagt:
    \rolls
    \roll{4}{``Tee und Kuchen! Tee und Kuchen! Du warst immer mein Liebster.'' \gainsanity{1}.}
    \roll{3}{``Ich bin immer für dich da, mein Süßer. Ich beobachte dich ...'' \gainknow{1}.}
    \roll{1-2}{``Wir sind hier, mein Spatz. Gib uns einen Kuss!'' \takediementaldamage{1}}
    \roll{0}{``Böse kleine Kinder müssen bestraft werden!'' \takediephysicaldamage{2}}
    \erolls
}

\event{Possession}{Besessenheit}{
    Ein Schatten schält sich aus der Wand. Du verharrst starr, als der Schatten dich umrundet und dich bis ins Mark auskühlt.
}{
    Du musst eine Charaktereigenschaft auswählen und für diese einen Wurf versuchen:
    \rolls
    \roll{4+}{Du widerstehst den Verlockungen des Schattens. Erhalte 1 in einem Charakterwert deiner Wahl.}
    \roll{0-3}{Der Schatten zerrt von deiner Energie. Die ausgewählte Eigenschaft sinkt auf die Stufe direkt über dem Schädel. Wenn die Eigenschaft schon auf ihrer niedrigsten Stufe steht, erniedrige eine andere Eigenschaft. }
    \erolls
}

\event{Revolving Wall}{Drehende Wand}{
    Ein Teil der Wand schwingt herum.
}{
    Lege den \chip{Wall-Switch}{Wandschalter} an eine Wand ohne Ausgang oder an eine Ecke in diesem Raum. Wenn es keinen Raum auf der anderen Seite des Wandschalters gibt, ziehe Raumkarten, bis du eine für das entsprechende Stockwerk gefunden hast und lege sie an. (Wenn es für das Stockwerk keine Karten  mehr gibt, nehme diese Karte aus dem Spiel.) Tritt in den Raum ein.

    Einmal während des Zuges eines Entdeckers, wenn sich dieser in einem der beiden vom Wandschalter berührten Räume befindet, kann er/sie einen \knowroll\ versuchen, um den Wandschalter umzulegen:

    \rolls
    \roll{3+}{Der Entdecker findet den versteckten Hebel und geht durch den Durchgang. Dies zählt nicht als Bewegung.}
    \roll{0-2}{Der Entdecker kann den versteckten Hebel nicht finden und den Durchgang nicht passieren.}
    \erolls
}


\event{Rotten}{Verrottet}{
    Der Gestank in diesem Zimmer ist schrecklich.

    Es riecht nach Tod, wie Blut.

    Ein Schlachthausgeruch.
}{
    Du musst einen \sanityroll versuchen:
    \rolls
    \roll{5+}{Verwirrende Gerüche, mehr nicht. \gainsanity{1}.}
    \roll{2-4}{\loosemight{1}.}
    \roll{1}{Verliere 1 \might\ und 1 \speed.}
    \roll{0}{Du krümmst dich vor Übelkeit. Verliere 1 von jeder Charaktereigenschaft.}
    \erolls
}

\event{Secret Passage}{Geheimgang}{
    Ein Bereich der Wand fährt zur Seite. Dahinter erstreckt sich ein modriger Tunnel.
}{
    Lege einen \chip{Secret Passage}{Geheimgang} in diesen Raum. Würfle mit 3 Würfeln und lege den zweiten \chipe{Secret Passage} in:

    \rolls
    \roll{6}{Irgendeinen schon existierenden Raum.}
    \roll{4-5}{Einen existierenden Raum im oberen Stockwerk.}
    \roll{2-3}{Einen existierenden Raum im Erdgeschoss.}
    \roll{0-1}{Einen existierenden Kellerraum.}
    \erolls

    Du kannst nun den Geheimgang nutzen, auch wenn du keine Bewegung mehr übrig hast.

    Die Benutzung des Geheimgangs zählt als eine Bewegung. Der Geheimgang selbst zählt aber nicht als Feld.

    Ab jetzt kann jeder Entdecker den Geheimgang nutzen. Ein Entdecker kann seinen Zug nicht im Geheimgang beenden.
}



\event{Secret Stairs}{Geheimtreppe}{
    Ein schreckliches Knarzen hallt umher. Du hast eine geheime Treppe entdeckt.
}{
    Lege einen \chip{Secret Stairs}{Geheimtreppe} in diesen Raum und einen zweiten \chipe{Secret Stairs} in einen anderen existierenden Raum auf einem anderen Stockwerk. Das verwenden der Geheimtreppe zählt als Bewegung. Die Treppe selbst zählt jedoch nicht als Feld.

    Du kannst den Treppen jetzt folgen, auch wenn du keine Bewegung mehr übrig hast. Wenn du ihnen jetzt folgst, ziehe eine \eventcard\ im nächsten Raum.
}

\event{Shrieking Wind}{Kreischender Wind}{
    Ein Wind zieht durchs Haus, ein langsames Crescendo hin zu einem kreischenden Heulen.
}{
    Jeder Spieler im Garten, auf dem Friedhof, in der Patio, auf dem Turm, auf dem Balkon (Garden, Graveyard, Patio, Tower or Balcony) oder einem Raum mit einem nach draußen führenden Fenster muss einen \mightroll\ versuchen:

    \rolls
    \roll{5+}{Du bleibst auf den Füßen.}
    \roll{3-4}{Der Wind wirft dich um. \takediephysicaldamage{1}}
    \roll{1-2}{Der Wind kühlt deine Seele aus. \takediementaldamage{1}}
    \roll{0}{Der Wind wirft dich derb um. \takediephysicaldamage{1}. Lege einen deiner Gegenstände (wenn du welche hast) in die Eingangshalle (Entrance Hall).}
    \erolls

    Jedes Ergebnis betrifft nur den Entdecker, der würfelt.
}


\event{Silence}{Stille}{
    Im Kellergewölbe wird plötzlich alles Still. Auch das Geräusch des eigenen Atems verstummt.
}{
    Jeder Entdecker im Keller (Basement) muss einen \sanityroll\ versuchen:

    \rolls
    \roll{4+}{Du wartest ruhig bis dein Hörsinn zurückkehrt.}
    \roll{1-3}{Du schreist einen tonlosen Schrei. \takediementaldamage{1} }
    \roll{0}{Du rastest aus. \takediementaldamage{2}}
    \erolls

    Jedes Ergebnis betrifft nur den Entdecker, der würfelt.
}

\event{Skeletons}{Gerippe}{
    Mutter und Kind, sich immer noch umarmend.
}{
    Lege einen \chip{Skeleton}{Skelett} in diesen Raum. Nehme einen Würfel mentalen Schaden.

    Einmal während seines Zuges kann ein Entdecker einen \sanityroll versuchen, um die Skelette zu durchsuchen.

    \rolls
    \roll{5+}{Ziehe eine \itemcardd. Entferne den \chipe{Skeleton}.}
    \roll{0-4}{Du stocherst herum, findest jedoch nichts. \takediementaldamage{1}}
    \erolls

    Jedes Ergebnis betrifft nur den Entdecker, der würfelt.
}

\event{Smoke}{Rauch}{
    Rauch wabert um dich herum. Du hustest während du dir Tränen aus den Augen wischt.
}{
    Lege den \chip{Smoke}{Rauch} in den Raum. Der Rauch blockiert die Sichtlinie von angrenzenden Räumen. Ein Entdecker würfelt mit zwei Würfeln weniger (aber mindestens mit einem Würfel) bei allen Charakterwürfen, während er sich in diesem Zimmer aufhält.
}


\event{Something Hidden}{Etwas Verstecktes}{
    Dieser Raum ist sonderbar, aber weshalb? Es liegt dir fast auf der Zunge.
}{
    Wenn du herausfinden willst, was seltsam ist, versuche einen \knowroll:
    \rolls
    \roll{4+}{Eine Sektion der Wand schiebt sich seitwärts, eine Nische preisgebend. Ziehe eine \itemcardd.}
    \roll{0-3}{Du kannst es einfach nicht herausfinden, was dich ein wenig verrückt macht. \loosesanity{1}.}
    \erolls
}

\event{Something Slimy}{Etwas Schleimiges}{
    Was ist das an deiner Ferse?

    Ein Insekt? Eine Tentakel?

    Eine um sich greifende tote Hand?
}{
    Du musst einen \speedroll\ versuchen:

    \rolls
       \roll{4+}{Du kommst frei. \gainspeed{1}.}
       \roll{1-3}{\loosemight{1}.}
       \roll{0}{Verliere 1 \might\ und 1 \speed.}
       \erolls
}

\event{Spider}{Spinne}{
    Eine faustgroße Spinne landet auf deiner Schulter ... und klettert in deine Haare.
}{
    Du musst einen \speedroll\ versuchen um sie wegzuwischen oder einen \sanityroll\ um still zu halten.

    \rolls
    \roll{4+}{Sie ist weg. Erhalte 1 in der Eigenschaft, die du für den Wurf benutzt hast.}
    \roll{1-3}{Sie beißt dich. \takediephysicaldamage{1}}
    \roll{0}{Sie beißt ein Stück Fleisch aus dir heraus. \takediephysicaldamage{2}}
    \erolls
}

\event{The Beckoning}{Der Lockruf}{
    Draußen.

    Du musst nach draußen gehen.

    In die Freiheit fliegen!
}{
    Jeder Entdecker im Garten, auf dem Friedhof, in der Patio, auf dem Turm, auf dem Balkon (Garden, Graveyard, Patio, Tower or Balcony) oder einem Raum mit einem nach draußen führenden Fenster muss einen \sanityroll\ versuchen:

    \rolls
    \roll{3+}{Du trittst vom Sims zurück.}
    \roll{0-2}{Du springst in die Patio. (Wenn sie sich nicht im Haus befindet, durchsuche den Zimmerstapel danach, lege sie an und mische die Karten.) Setze deine Figur in die Patio und erleide einen Würfel physischen Schaden.}
    \erolls

    Jedes Ergebnis betrifft nur den Entdecker, der würfelt.
}

\event{The Lost One}{Die Verlorene}{
    Eine Frau in Kleidern aus der Zeit des Bürgerkriegs winkt dich herbei. Du fällst in Trance.
}{
    Du musst einen \knowroll\ versuchen. Wenn du mehr als 5 würfelst, entkommst du der Trance und erhälst 1 \know. Andernfalls würfle mit drei Würfeln um zu erkennen, wohin dich die Verlorene führt.

    \rolls
    \roll{6}{Versetze deinen Entdecker in die Eingangshalle (Entrance Hall).}
    \roll{4-5}{Versetze deine Figur an die Treppe im ersten Stock (Upper Landing).}
    \roll{2-3}{Ziehe Zimmerkarten, bis du eine Karte für den ersten Stock findest.}
    \roll{0-1}{Ziehe Zimmerkarten, bis du einen Kellerraum findest.}
    \erolls

    Wenn du für dieses Ereignis eine Zimmerkarte ziehen musstest, lege diese Karte an und setze deinen Entdecker hinein. Findest du keine Zimmerkarte für das gewürfelte Stockwerk, versetze deine Figur in die Eingangshalle.
}


\event{The Voice}{Die Stimme}{
    ``Ich bin unter dem Boden, begraben unter dem Boden ...''

    Die Stimme wispert und verschwindet.
}{
    Du musst einen \knowroll\ versuchen:
    \rolls
    \roll{4+}{Du findest etwas unter dem Boden. Ziehe eine \itemcardd.}
    \roll{0-3}{Du gräbst und suchst nach der Stimme, aber vergebens.}
    \erolls
}

\event{The Walls}{Die Wände}{
    Dieser Raum ist warm.

    Wände aus Fleisch pulsieren in einem beständigen Herzschlag. Dein eigenes Herz schägt im Takt mit dem Haus. Du wirst in die Wände gezogen ... und bricht anderswo wieder heraus.
}{
    Du musst eine Zimmerkarte ziehen und anlegen. Versetze deinen Entdecker in den neuen Raum.
}

\event{Webs}{Spinnenweben}{
    Beiläufig hebst du deinen Arm um einige Spinnenweben beiseite zu wischen ... aber sie wollen sich nicht wegschieben lassen. Sie kleben.
}{
    Du musst einen \mightroll\ versuchen:
    \rolls
    \roll{4+}{Du kommst frei. \gainmight{1}\ und nehme diese Karte aus dem Spiel.}
    \roll{0-3}{Du steckst fest. Behalte diese Karte.}
    \erolls

    Steckst du fest, kannst du nichts machen, bis du befreit bist. Einmal pro Zug kann ein Entdecker einen \mightroll\ versuchen, um dich zu befreien. (Du kannst diesen Wurf ebenfalls versuchen.) 4+ befreit, aber du erhälst keinen \emph{Might}-Bonus. Jeder, der es nicht schafft dich zu befreien, kann sich für den Rest seines Zuges nicht bewegen. Nach drei erfolglosen Versuchen, befreist du dich in deinem folgenden Zug automatisch und kannst normal agieren.

    Nehme diese Karte aus dem Spiel, sobald du dich befreit hast.
}


\event{What The ... ?}{Was zum ... ?}{
    Als du den Weg zurückblickst, den du gekommen bist, siehst du ... nichts.

    Nur dichter Nebel und Dunst wo eben noch ein Raum war.
}{
    Nehme die Zimmerkachel, auf der du stehst (nachdem du alles andere beiseite geräumt hast) und setze sie an anderer Stelle im selben Stockwerk wieder an, sodass die Tür an eine bisher unentdeckte Tür anschließt (und setze alles wieder darauf was du weggeräumt hast). Wenn es keinen unentdeckten Durchgang auf diesem Stockwerk gibt, versetze das Zimmer in ein anderes Stockwerk.
}

\event{Whoops!}{Hoppla!}{
    Du fühlst einen Körper an deinen Füßen. Bevor du einen Schritt zurücktreten kannst, wirst du umgestoßen. Eine kichernde Stimme rennt davon.
}{
    Nehme alle deine \itemcards\ (keine \omencards) und mische sie. Der Spieler zu deiner Rechten zieht zufällig eine davon und nimmt sie aus dem Spiel. Dann lege deine \itemcards\ wieder offen hin.
}