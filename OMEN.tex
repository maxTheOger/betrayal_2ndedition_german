
\omen{Bite}{Biss}{
    Ein Knurren, der süße Duft von Tod. Schmerz. Dunkelheit. Fort.
}{
    \emph{Etwas} beißt dich, als du diese Karte ziehst. Der Spieler zu deiner Rechten wirft einen \mightroll\ mit 4 Würfeln für das mysteriöse \emph{Etwas}, bevor es in der Dunkelheit verschwindet. Du verteidigst dich normal, indem du einen \mightroll\ ausführst.

    \omencantbedts
}

\omen{Book}{Buch}{
    Ein Tagebuch oder Laborprotokolle? Ein antikes Manuskript oder moderner Bestseller?
}{
    Erhalte sofort 2 \know.

    Verliere 2 \know, wenn du das Buch verlierst.
}

\omen{Crystal Ball}{Kristallkugel}{
    Trübe Bilder erscheinen im Inneren des Glases.
}{
    Ist der \haunt\ offenbart, kannst du mittels eines \knowroll s während deines Zuges in die Kristallkugel schauen:

        \rolls
        \roll{4+}{Du siehst die Wahrheit. Durchsuche das Gegenstandsdeck (Items) oder das Ereignisdeck (Events) und wähle eine Karte aus. Mische die Karten und lege deine Karte oben auf.}
        \roll{1-3}{Du wendest deinen Blick ab. Verliere 1 \sanity.}
        \roll{0}{Du starrst direkt in die Hölle. Verliere 2 \sanity.}
        \erolls
}

\omen{Dog}{Hund}{
    BEGLEITER

    Dieser räudige Hund scheint freundlich. Du hoffst, dass er das auch wirklich ist.
}{
    Erhalte sofort 1 \might\ und 1 \sanity.

    Verliere 1 \might\ und 1 \sanity, wenn du die Aufsicht über den Hund abgibst.

    Verwende einen kleinen Monsterspielstein, der den Hund darstellt und lege ihn in deinen Raum. (Verwende eine Monsterfarbe die noch nicht auf dem Spielfeld liegt.) Einmal während deines Zuges kann der Hund zu einem maximal 6 Felder entfernten, schon entdeckten Raum laufen, um danach direkt wieder zurückzukehren. Er kann einen Gegenstand aufnehmen, tragen und/oder fallenlassen, bevor er zurückkehrt.

    Der Hund wird von Gegnern nicht verlangsamt. Er kann keine Einbahnstraßen nehmen oder Räume, die einen Wurf erfordern durchqueren. Er kann keine Gegenstände tragen, die die Bewegung verlangsamen.

    \omencantbedts
}

\omen{Girl}{Mädchen}{
    BEGLEITER

    Ein Mädchen.

    Gefesselt.

    Alleine.

    Du befreist sie!
}{
    Erhalte sofort 1 \sanity\ und 1 \know.

    Verliere 1 \sanity\ und 1 \know, wenn du die Mädchen-Karte verlierst.

    \omencantbedts
}

\omen{Holy Symbol}{Heiliges Symbol}{
    Ein Symbol der Ruhe in einer rastlosen Welt.
}{
    Erhalte sofort 2 \sanity.

    Verliere 2 \sanity, wenn du das Heilige Symbol verlierst.
}

\omen{Madman}{Ein Verrückter}{
    BEGLEITER

    Ein schäumender Irrer.
}{
    Erhalte sofort 2 \might\ und verliere 1 \sanity.

    Verliere 2 \might\ und erhalte 1 \sanity, wenn du diese Karte verlierst.

    \omencantbedts
}

\omen{Mask}{Maske}{
    Eine düstere Maske, um deine Absichten zu verschleiern.
}{
    Einmal pro Zug kannst du mittels eines \sanityroll s versuchen, die Maske zu benutzen.

    \rolls
    \roll{4+}{Du kannst die Maske auf- oder absetzen.

    Wenn du die Maske aufziehst, erhalte 2 \know\ und verliere 2 \sanity.

    Wenn du die Maske absetzt, erhalte 2 \sanity\ und verliere 2 \know.}
    \roll{0-3}{Du kannst die Maske in dieser Runde nicht verwenden.}
    \erolls
}

\omen{Medallion}{Medallion}{
    Ein Medallion mit einem eingravierten Pentagram.
}{
    Du bist immun gegenüber den Effekten der Pentagramm-Kammer, der Crypta und des Friedhofs (Pentagam Chamber, Crypt, Graveyard).
}

\omen{Ring}{Ring}{
    Ein abgenutzter Ring mit einer unleserlichen Inschrift.
}{
    Wenn du einen Gegner angreifst, der einen \emph{Sanity}-Wert besitzt, kannst du ihn mit einem \sanityroll\ anstelle eines \mightroll s angreifen. (Dein Gegner verteidigt sich dann mit einem \sanityroll\ und der Schaden wirkt statt physisch mental.)
}

\omen{Skull}{Schädel}{
    Ein rissiger Schädel mit fehlenden Zähnen.
}{
    Wenn du mentalen Schaden erleidest, kannst du ihn stattdessen komplett in physischen Schaden umwandeln.
}

\omen{Spear}{Speer}{
    WAFFE

    Eine vor Macht pulsierende Waffe.
}{
    Du kannst zwei zusätzliche Würfel (bis maximal 8 Würfel) werfen, wenn du eine \might\ Attacke mit dieser Waffe ausführst.

    \nootherweapon
}

\omen{Spirit Board}{Ouijabrett}{
    Eine Tafel mit Buchstaben und Zahlen, um mit den Toten zu sprechen.
}{
    Bevor du dich während deines Zuges bewegst, darfst du unter die oberste Karte des Zimmerkartenstapels schauen.

    Benutzt du das Ouijabrett, nachdem der \haunt\ offenbart wurde, kann der Verräter beliebige Monster ein Feld weiter in deine Richtung bewegen. (Wenn du der Verräter bist, musst du die Monster nicht bewegen.) Wenn es keinen Verräter gibt, rücken alle Monster 1 Feld zu dir.
}