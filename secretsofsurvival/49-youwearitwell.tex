
\survival{49}{You Wear it Well}{Er steht dir gut}{+}

\introduction{
Du hörst einen deiner Freunde durch das Haus rennen, lachend und gleichzeitig schluchzend. Gerade als du in Richtung deines Freundes laufen willst, fühlst du einen kraftvollen Wind durch deine Seele ziehen und sie verlässt deinen Körper.
Du kannst die Wände des Hauses noch sehen, aber sie sind schemenhaft und verzerrt. Ein silbernes Seil verbindet dich mit deinem leblosen Körper. Alles was du willst ist, wieder in dein eigenes Fleisch hinein zu kommen. Aber wie? Wie?

}

\rightnow{

    \begin{itemize}
    \bitem Lege deine Spielfigur auf die Seite, sie ist bewusstlos. Lege einen blauen Seele-marker (soul)  in den Raum mit deinem Körper. Der marker ist deine entleibte Seele.
    \bitem Lege so viele Wissenswurf-marker bereit wie Spieler am Spiel teilnehmen. Lege ebenso viele Gesundheitswurfmarker bereit.
    \bitem Falls es noch keine Möglichkeit gibt vom Keller zum Erdgeschoss zu kommen, suche das passende Treppenteil aus dem Stapel, leg sie an und misch den Stapel erneut.
    \end{itemize}
}

\whatyouknowaboutthebadguys{
Der Verräter arbeitet mit einem mächtigen Astralgeist zusammen. Wenn dieser einen leblosen Körper einnehmen kann wird er unsterblich.

}

\youwinwhen{
du den Astralgeist zerstörst. Ist das passiert, kehrst du in deinen physischen Körper zurück.

}

\hauntsection{Wie man den Astralgeist zerstört}

Der Forscher macht Wissens- und Gesundheitswurf-Angriffe gegen den Astralgeist. Bei jedem erfolgreichen Angriff legt man einen entsprechenden Wurf-marker auf die Charakterkarte des Forschers (Angreifers). Haben alle Forscher zusammen so viele von diesen markern gesammelt wie Spieler am Spiel teilnehmen, ist der Astralgeist zerstört.
\newpage
\hauntsection{Seelen}
Als entleibte Seele:
  \begin{itemize}
    \bitem behältst du alle deine Eigenschaften.
    \bitem kannst du durch Wände ziehen nicht aber durch Decken und Böden.
    \bitem Du kannst frei durch den Coal Chute, Collapsed Room und Gallery ziehen.
    \bitem Du wirst nicht durch Raumtexte beinflusst.
    \bitem kannst du deine Gegenstände benutzen, sie aber nicht mit anderen tauschen oder an andere abgeben.
    \bitem kannst du keine neuen Gegenstände nehmen.
    \bitem Wird deine Seele zerstört werden die Gegenstände auch zerstört.
    \bitem kannst du keine neuen Räume erforschen.
    \bitem kannst du bei Angriff oder Verteidigung nur Ge-sundheits- oder Wissenswürfe machen.
    \bitem kannst du die Schädelkarte (skull) nicht benutzen.
    \bitem erleidest du für jeden physischen Schaden stattdessen mentalen Schaden in gleicher Höhe.
    \end{itemize}

\specialattackrules{

    \begin{itemize}
        \bitem Verlierst Du den Angriff gegen den Astralgeist nimmst Du keinen Schaden.
    \bitem Machst du einen psychischen Angriff gegen den Verräter und gewinnst, so ist dieser nur betäubt, nimmt aber keinen Schaden.
    \bitem Du kannst dich gegen den Verräter nicht verteidigen, wenn dieser deinen entseelten Körper angreift. Wenn er das tut, erleidest du 2 Würfel mentalen Schaden.
   \end{itemize}
}

\outro{
Dein Freund schickt einen vernichtenden psychischen Angriff auf den Astralgeist. Die Luft flimmert und du löst dich auf.
Als du wach wirst fühlt sich dein Körper ganz eigenartig an, als ob du auf einer Party etwas zu viel getrunken hast und den Mantel von jemand anders an hast. Aber letztendlich ist es DEIN Körper. Du bist sicher, bald steht er dir wieder gut.
}