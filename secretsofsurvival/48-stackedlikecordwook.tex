
\survival{48}{Stacked Like Cordwood}{Der Serienkiller}{+}

\introduction{
Als du auf dem Weg zum Haus warst, hat einer deiner Freunden von einer Serie grausamer Morde erzählt, die hier vor 5 Jahren passiert sind. Eine Gruppe von Teenagern hatte die Sicherheit ihres Sommercamps verlassen um das alte Haus on the Hill zu erkunden. Sie hatten auf amouröse Abenteuer gehofft… bis ein Verrückter sie, einen nach dem anderen, abschlachtete. Dein Freund erzählt, dass der Mörder nicht getötet werden konnte und dass sein entstellter Körper immer wieder zurückkehrt um weiter zu töten.
Jetzt hat jemand von euch das Bild des Verrückten in der Halle hängen sehen. Wie man aus dem Portrait entnehmen kann, könnte er ein Verwandter eines deiner Gefährten sein… und zwar von dem, der diese Geschichte erzählt hat. Du spähst durch einen Türspalt und siehst deinen verräterischen Gefährten in einem Raum mit Toten, die wie ein Klafter Holz gestapelt sind (eigentlicher Name des Szenarios).
Und dann hörst du jemanden… oder etwas… es kommt durch die Eingangstür.

}

\rightnow{
Packe so viele dreieckige Stärke-Wurf-Marker(Might Roll) und Wissens-Wurf-Marker(Knowledge Roll)beiseite wie Spieler im Spiel sind.

}

\whatyouknowaboutthebadguys{
Der Verräter arbeitet mit dem Roten Jack (so wird der Verrückte genannt), einem übernatürlichen Serienkiller, zusammen. Jack kann nur mit einer verzauberten Waffe die irgendwo im Haus versteckt ist dauerhaft getötet werden. Bezwingst du ihn auf andere Weise, kehrt er stets stärker zurück als er vorher war.

}

\youwinwhen{
du die verzauberte Waffe gefunden hast und den Roten Jack damit getötet.
}
\newpage
\hauntsection{Wie man die Verzauberte Waffe findet}

\begin{itemize}
    \bitem Du weißt dass es eine verzauberte Waffe im Haus gibt. Es ist entweder die Axt, der Speer, der Blutdolch (blood dagger) oder der Opferdolch(Sacrificial Dagger).
   Du triffst die Wahl, welcher dieser 4 Waffen es ist.
    \bitem Wenn die Helden die verzauberte Waffe noch nicht haben, müssen sie danach suchen. Sie befindet sich in der Bücherei (library), der Kapelle (chapel), dem Grabgewölbe (vault, muss offen sein) oder der Dachkammer (attic). Einmal während deines Zuges kannst du einen Wissenswurf von 3+ in einem dieser Räume versuchen, um die Waffe zu finden. Hast du Erfolg, suche dir die entsprechende Waffe aus dem Kartenstapel und mische die Karte dann neu.
    \bitem Danach muss der Forscher herausfinden, wie er die verzauberte Waffe gegen den Roten Jack verwenden kann. Der Forscher, der die Waffe hat, kann einen Stärke- oder Wissenswurf von 5+ versuchen um die Waffe zu untersuchen. Bei jedem erfolgreichen Versuch lege einen Wissens- oder Stärkewurf-marker auf die Charakterkarte dieses Spielers. Haben alle Spieler zusammen so viele von diesen markern gesammelt, wie Spieler am Spiel teilnehmen, kann die Waffe zum Töten vom Roten Jack verwendet werden.
    \bitem Wenn danach jemand Jack mit der verzauberten Waffe besiegt, ist dieser tot.

\end{itemize}

\hauntsection{Am anfang deines Zuges machst du folgendes}
Jack erzeugt eine Aura von Furcht. Zu Beginn deines Zuges (wenn du mit Jack im gleichen Raum bist) musst du einen erfolgreichen Gesundheitswurf von 3+ machen oder du verlierst je 1 Punkt physische und psychische Fähigkeit.


\outro{
Der Rote Jack ist von der Waffe durchbohrt, schiebt dich aber zurück, dieses schreckliche Lächeln auf dem Gesicht. Gerade als seine Hände sich um deinen Hals schließen wollen verschwindet Jack und die Waffe fällt zu Boden.
Als du das Haus verlässt, schaust du noch einmal sein Bild an der Wand an. Der Killer ist weg… für dieses Mal.
}