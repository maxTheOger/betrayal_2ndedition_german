
\haunttitle{The Abyss Gazes Back}{Der Abgrund starrt dich an}

\introduction{
Das Haus erzittert und stöhnt. Eine Hitzewelle streift dich. Ein weiteres Erzittern und dann der Ton splitternden Holzes und zusammenstürzenden Betons. Einer deiner Freunde ruft: „Passt alle auf. Wir werden alle zur HÖLLE fahren!“
Ein flackerndes Licht erhellt die Wände und grauer Nebel strömt in die Räume. Ein Teil des Hauses stürzt zusammen und hinab in einen See aus Feuer. Du kriechst in Sicherheit und fragst dich verzweifelt, wie du das Haus vor dem Zusammensturz retten kannst. Du möchtest nicht mit zur Hölle fahren.
}

\rightnow{
Lege so viele dreieckige Gesundheitswurf-Markern (sanitary roll) bereit, wie Spieler mitspielen.
Das gleiche mache mit den dreieckigen Wissenswurf-Markern (knowledge roll).

}

\whatyouknowaboutthebadguys{
Der Verräter begrüßt den Abgrund… und möchte sicher sein, dass alle Anwesenden den Ritt zur Hölle mitmachen.
}

\youwhinwhen{
du einen erfolgreichen Exorzismus durchführst um das Haus am weiteren Zusammen- brechen zu hindern.
}

\hauntsection{Wie man den Exorzismus durchführt}
Du musst einen Exorzismus durchführen um den Abgrund daran zu hindern das Haus völlig aufzusaugen. Das verlangt eine Reihe von erfolgreichen Exorzismus-Würfen, genauso viele wie Spieler teilnehmen. Jeder Wurf benötigt einen speziellen Raum oder Gegenstand und einen Gesundheits- oder Wissenswurf. Pro Runde kann man einen Exorzismus versuchen:
  \begin{itemize}
        \bitem Du kannst einen Gesundheitswurf von 5+ versuchen während du in der Kapelle, Krypta oder Pentagrammkammer bist oder während du das Heilige (holy) Symbol oder den Ring bei dir trägst.
        \bitem Du kannst einen Wissenswurf von 5+ versuchen während du in der Bücherei (library) oder dem Forschungslabor (research lab.) bist oder während du das Buch (book) oder die Kristallkugel (crystal ball) bei dir trägst.
    \end{itemize}
Bei jedem erfolgreichen Exorzismus legst du einen Gesundheits- oder Wissensmarker (je nachdem welche Fähigkeit du gewählt hast) in den Raum oder die Gegenstandskarte, den/die du für den Exorzismus verwendet hast.

Hat man einen Raum/einen Gegenstand erfolgreich für einen Exorzismus eingesetzt, kann niemand diesen Raum/Gegenstand mehr dazu benutzen. Wenn die Spieler so viele Marker platziert haben wie Spieler am Spiel teilnehmen ist der Zerfall des Hauses gestoppt.Wurde der Raum mit einem Marker zerstört, zählt er trotzdem.


\hauntsection{Während deines Zuges musst du}
Am Ende deines Zuges wird die der Verräter sagen, dass du einige Räume im Haus umdrehen musst. Diese Räume sind zusammengestürzt und nun Teil des Abgrundes.

\hauntsection{Wie du mit dem Abgrund umgehst}

    \begin{itemize}
        \bitem  Der Verräter hält auf der Turn/Damage-Karte die abgelaufene Zeit fest.
        \bitem Trägst du das Heilige Symbol bei dir und bist du in einem Raum an den ein zerstörter Raum angrenzt, kannst du das Heilige Symbol opfern anstatt Raumteile umzudrehen(es muss eine Tür zwischen den beiden Räumen geben). Wenn du das tust, lege das Heilige Symbol ab und du brauchst in keine Räume umzudrehen. Das Fortschreiten des Markers auf der Turn/Damage-Karte wird jedoch nicht gestoppt oder verzögert.
        \bitem Bist du in einem Raum der gerade zusammenbricht, musst du einen Geschwindigkeitswurf (speed) von 4+ versuchen um gerade noch herauszukommen. Hast du Erfolg springst du in einen erforschten Nachbarraum, zu dem es eine Verbindungstür geben muss. Misslingt der Wurf oder gibt es keinen solchen Raum (benachbart mit Verbindungstür) wirst du in den Abgrund gezogen und stirbst.
        \bitem Eingangshalle, Foyer und Große Treppe zählen als separate Räume. Benutze sechseckige Marker um anzuzeigen, welcher dieser Räume schon vom Abgrund verschlungen wurde.
        \bitem Wenn Dich der Aufzug oder eine Eventkarte in einen zerstörten Raum schickt stirbst Du.
    \end{itemize}


\outro{
Die letzte Beschwörung ist vorbei. Der Exorzismus ist beendet. Du wartest, hoffnungsvoll, betend, dass es außer dir noch Überlebende gibt….
Das Haus hört auf zu Zittern. Der graue Nebel zieht sich zurück. Die rote Glut ist erloschen. Du seufzt. Die Hölle hat dich heute nicht bekommen.
}