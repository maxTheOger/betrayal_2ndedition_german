
\survival{23}{Tentacled Horror}{Tentakelhorror}

\introduction{
(Für die Guten)
Seillange muskulöse Gewebe sind plötzlich überall zu sehen. Zackige, Horn bewehrte Saugnäpfe übersäen die knochenlosen Arme, pulsierend und schmatzend, wie körperlose Zähne. Die langen Dinger winden sich um das Bein eines Gefährten und ziehen sich fest zusammen. Die Saugnäpfe schneiden und sägen an seinem Bein herum, trennen es fast ab. Blut spritzt in alle Richtungen. Die Arme ziehen sich zusammen und dein Freund wird polternd durch das Haus gezogen als ob jemand einen Fisch an einer Angel einholt. Die Schreie deines Freundes werden schwächer je weiter er weggezogen wird, dann hören sie plötzlich auf.
Dann kommt das Tentakel zurück.
}

\rightnow{
Packe 3 dreieckige Stärke-Wurf-Marker beiseite
}

\whatyouknowaboutthebadguys{
Die Tentakel suchen nach dir. Jedes diese Tentakel wird durch einen Wurzel- und einen Spitze-Counter repräsentiert. Die Spitze eines Tentakels zieht dich in Richtung der dazugehörigen Wurzel. Startest du deine Runde in einem Raum mit einer Wurzel wirst du in der nächsten Spielrunde des Verräters getötet, falls du nicht gerettet wirst. Die Tentakel werden mit der Zeit immer stärker.
}

\youwhinwhen{
du das Tentakelmonster zerstörst.
}

\hauntsection{Wie du das Tentakelmonster zerstörst}

Finde den Kopf des Monsters und zerstöre ihn auf folgende Weise:

    \begin{itemize}
        \bitem Anstatt die Kristallkugel (crystal ball) zu benutzen, um einen Raum oder eine Karte zu finden, kannst du mit 4 Würfeln werfen Bei einem Ergebnis von 4+ würfele erneut um so festlegen wo der Kopf des Monsters sich befindet:

            \rolls
            \roll{Erg.}{Raum}
            \roll{0}{Speisekammer (larder) }
            \roll{1}{Küche (kitchen) }
            \roll{2}{Orgelraum (organ room) }
            \roll{3}{ Abgrund (chasm) }
            \roll{4-5}{Unterirdischer See (underground lake) }
            \roll{6}{Gewächshaus (conservatory) }
            \roll{7}{Krypta}
            \roll{8}{Ofenraum (furnace room) }
            \erolls



            Missling der Wurf befolge die negativen Auswirkungen auf der Karte.
        \bitem Die Kristallkugel zerspringt nachdem man sie benutzt hat um das Monster zu finden. Lege die Karte ab.
\newpage
        \bitem  Ist der Raum noch unentdeckt durchsuche den Stapel und gib die Karte dem Verräter. Er legt sie dann an.
        \bitem Packe ein kleinen Monster Marker in den Raum der den Kopf des Monsters repräsentiert.
        \bitem Ziehe zu dem Raum mit dem Kopf des Monsters und mache einen Angriff dagegen mit Dynamit oder dem Speer. Du musst nicht würfeln. Du tötest das Monster automatisch wenn du es mit einem der beiden Gegenstände angreifst (das Dynamit fügt dir selber keinen Schaden zu).
    \end{itemize}


\specialattackrules{

    \begin{itemize}
        \bitem Nur Spitzen können Angegriffen werden, wurzeln nicht.
        \bitem  Wenn du eine Tentakel-Spitze besiegst ist das Tentakel bewusstlos und zieht sich zu seiner Wurzel zurück. Lege die Spitze in den gleichen Raum wie die Wurzel.
        \bitem Besiegst du eine Spitze die einen Forscher umklammert hat wird der Forsche im gleichen Raum freigelassen. Die Spitze zieht sich ebenfalls zur Wurzel zurück.
        \bitem Wurzeln beeinflussen deine Bewegung nicht, Spitzen schon.
        \bitem Der Kopf kann die Helden angreifen.Wurde der Kopf besiegt lege ein dreieckigen Marker in den Raum.Sobald alle 3 Marker im Raum liegen ist die Kreatur besiegt.
    \end{itemize}
}

\hauntsection{Du musst während deines Zuges folgendes tun}
Wenn eine Tentakel-Spitze dich ergreift musst du sie zu Beginn deiner nächsten Runde angreifen. Besiegst du sie, lässt sie dich fallen und zieht sich zurück. Für dich zählen aber bis zum Ende deiner Runde alle Räume die du betrittst 2 Schritte. Verlierst du, erleidest du keinen Schaden, aber dein Zug ist beendet

\outro{
Die mit Saugnäpfen übersäten Tentakel schwingen von und zurück in ihrem Todeskampf, sie reißen einen Teil der Decke herunter, dann eine Wand ein. Ein Schrei, der weit unter deiner Hörschwelle beginnt steigert sich bis zur Unerträglichkeit. Die Kreatur gibt ihren letzten Schrei von sich und das ganze Haus beginnt zu beben.
Letztendlich ist das WAS ES NICHT GEBEN DARF nicht mehr da. Du hoffst dass du ihm höchstens noch einmal in deinen Träumen begegnest.
}