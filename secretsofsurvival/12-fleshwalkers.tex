
\haunttitle{Fleshwalkers}{Untote}

\introduction{
In der Kristallkugel erscheint das Bild des Raumes in dem du stehst. In diesem Raum steht auch eine Kristallkugel die wiederum das Bild eines Raumes mit einer Kristallkugel zeigt usw., so als ob du in einen Brunnen mit identischen Bildern fällst. Der große Spiegel explodiert und die Scherben knallen gegen die Wand. Die Wand hinter dem Spiegel ist dunkel geschwärzt. Die Kristallkugel beruhigt sich wieder. Ein neues Bild formt sich darin. Dieses mal eins von der Eingangshalle des Hauses, wo Menschen das Haus betreten. Sie sehen vertraut aus … einer der Neuankömmlinge schaut flüchtig hoch und blickt durch die Kristallkugel direkt in deine Augen. Als du ihn erkennst wirst du kreidebleich. Es sind DEINE Augen. Es ist ein Doppelgänger von dir … dein böser Zwilling.
}

\rightnow{

    \begin{itemize}
        \bitem Legt so viele EVIL TWIN – Marker in die Eingangshalle wie es Mitspieler gibt. Nehmt unterschiedliche Farben; jeder gehört zu einem speziellen Abenteurer.
        \bitem Der Spieler links vom Spuk-Auslöser beginnt.
    \end{itemize}
}

\whatyouknowaboutthebadguys{
Dieser Spuk hat keinen Verräter. Die bösen Jungs sind einfach wie ihr, nur halt böse. Und sie wollen euch alle töten.
}

\youwhinwhen{
... einer Eurer Helden überlebt und alle bösen Zwillinge getötet wurden.
}

\hauntsection{Was die bösen Zwillinge in ihrem Zug machen müssen}

  \begin{itemize}
        \bitem Die bösen Zwillinge machen ihren Monster-Zug nach dem Zug des Spuk-Auslösers
        \bitem Ein böser Zwilling bewegt sich immer auf kürzestem Weg in Richtung seines guten Zwillings. Wenn der böse Zwilling seinen Zug in einem Raum beendet in dem irgendein Abenteurer steht, attackiert er ihn.
Wenn mehrere Abenteurer im gleichen Raum stehen attackiert er seinen Zwilling zuerst(falls möglich) sonst wird per Zufall entschieden.
        \bitem Wenn ein Abenteurer getötet wurde, übernimmt dessen Spieler seinen bösen Zwilling und versucht die anderen Abenteurer zu attackieren.
    \end{itemize}

\hauntsection{Evil Twins}

  \begin{itemize}
        \bitem Jeder böser Zwilling besitzt dieselben Eigenschaftswerte wie sein guter Gegenpart zu Beginn des Spuks. Die Eigenschaften der bösen Zwillinge können sich aber nicht ändern.
        \bitem Böse Zwillinge können keine Gegenstände tragen oder Gefährten haben.
    \end{itemize}

\specialattackrules{

    \begin{itemize}
        \bitem Wenn du deinen bösen Zwilling attackierst oder dich gegen ihn verteidigst ohne im Besitz der Kristallkugel zu sein, verlierst du auf ALLE Eigenschaften 1 Punkt, unabhängig davon wer den Kampf gewinnt. Wenn du den Kampf verlierst, erhältst du zusätzlich noch den normalen Schaden. Wenn du deinen bösen Zwilling besiegst, betäubst du ihn nur.
        \bitem Wenn du deinen bösen Zwilling besiegst während du die Kristallkugel besitzt, tötest du ihn.
        \bitem Wenn du den bösen Zwilling eines anderen Abenteurers besiegst, betäubst du ihn nur. Es sei denn du besitzt die Krsitallkugel UND der Abenteurer des bösen Zwillings ist bereits tot.
        \bitem Der Held mit der Kristallkugel/Crystal Ball kann betäubte Zwillinge angreifen. Der Zwilling kontert mit seinen normalen Würfeln, nimmt aber keinen Schaden wenn er höher Würfelt als sein Angreifer.
        \bitem Du kannst die Kristallkugel/Crystal Ball von einem anderen Spieler im gleichen Raum übernehmen, wenn er sie abgeben möchte.
    \end{itemize}
}

\outro{
Schaudernd schaust du zurück. Dein Körper liegt tot im Flur. „Nicht MEIN Körper“, sagst du dir selbst. Sondern der deines Doppelgängers. Du hast ihn getötet. Er hatte vor dich zu ersetzen, richtig? Entweder du oder dein Doppelgänger, richtig?
Richtig?
}