
\survival{29}{Frankenstein’s Legacy}{Frankenstein's Vermächtnis}{}

\introduction{
Dein Abenteurergefährte ergießt sich über die vergilbten Seiten des Buches, eurer Umgebung keine Beachtung schenkend. Dein Freund beginnt etwas über Tod, Körper und Wiederbelebung von totem Fleisch zu murmeln. Was für ein Wahnsinn!
Dann sieht dein Freund auf, ein furchtbares Licht von Begeisterung in Augen und Mimik. "Zum Labor" sagt er. "Heute Nacht, werde ich den Traum von der nekrotischen Wiederbelebung realisieren. Gliedmaßen ehemals tot werden zucken, ein Gehirn ehemals leblos wird erwachen und ein Körper ehemals todgeweiht wird auferstehen. Heute Nacht wird der Tod laufen lernen!"

}

\rightnow{
Lege 5 sechseckige Fackelplättchen (Torch token) beiseite.
}

\whatyouknowaboutthebadguys{
Euer verräterischer Kamerad hat Frankenstein's Monster wiederbelebt. Um seine Stärke zu testen, hat der Verräter ihm befohlen euch alle zu töten. Das Monster ist sehr wiederstandsfähig, deshalb solltet ihr euer Bestes tun ihm auszuweichen. Glücklicherweise hat es eine Schwäche: Feuer.

}

\youwinwhen{
...das Monster tot ist
}

\newpage

\hauntsection{Wie kann das Monster getötet werden}
Es gibt zwei Möglichkeiten Frankenstein's Monster zu töten:


\begin{itemize}
    \bitem Tod durch Feuer: Gehe zum Kohlenkeller (Charred Room), Heizungskeller (Furnace Room), Pentagrammraum oder in die Küche, um eine Fackel zu entzünden. Wenn du das getan hast, lege ein Fackelplättchen auf die Charakterkarte deines Abenteurers. Es können während des Spiels unbegrenzt Fackeln gefunden werden, aber jeder Abenteurer darf gleichzeitig nur eine tragen. Wenn du dich in dem Raum mit dem Monster befindest oder in einem durch eine Tür angrenzenden Raum, kann du einen Geschwindigkeitsangriff versuchen, um die Fackel auf das Monster zu werfen. Wenn du das Monster besiegst, bekommt es einen Treffer mit der Fackel und du verlierst die Fackel. Wenn es dich besiegt, verlierst du die Fackel. Das Monster ist getötet, sobald es mit so vielen Fackel getroffen wurde, wie Spieler mitspielen.
    \bitem Tod durch Absturz: Das Monster ist nicht sehr helle. Es muss immer zu dem Abenteurer laufen, der ihm am nächsten steht, um ihn anzugreifen. Locke es zum Turm (Tower) oder zum Abgrund (Chasm). In einem von diesen Räumen kannst du einen Kraftwürfelwurf (Might roll) von 6+ versuchen, um das Monster in den Tod zu stürzen. Du kannst einmal pro Zug diesen Würfelwurf versuchen.
\end{itemize}

\outro{
Ziiisch! Du hältst eine andere Seite des Buches in die Flamme der Kerze. Diese Notizen sind eine wahre Scheußlichkeit. Du hoffst, die Zerstörung des Buches wird sicherstellen, dass das Geheimnis der Wiederbelebung für alle Zeit verborgen bleibt.
Ziiisch! Da verbrennt die Einleitung. Die nächsten paar Seiten sind gefüllt mit Formularen und Tabellen, Diagrammen und Zahlen....
Ja. Das alles sorgt für ein grausiges Gefühl. Ein plötzliches Licht bricht sich über dir, Ein Licht so brillant und wundersam, doch so einfach, wodurch du von den Möglichkeiten des Buches geblendet wirst. Wie überraschend, dass es dir alleine vorbehalten sein sollte, so ein erstaunliches Geheimnis zu entdecken.
Du verbrennst dir deine Finger, ein wenig, beim aus schlagen der brennende Seite.
}