
\survival{34}{Guillotines}{Guillotinen}{}

\introduction{
Alles dreht sich in Deinem Kopf und Du verlierst das Bewusstsein.Du erwachst.Wie lang war ich Bewusstlos ? Als du langsam aufstehst hörst Du die eine Bandaufzeichnung.
„Hallo!Ihr kennt mich nicht aber ich kenne euch! Ich möchte ein Spiel mit euch spielen. Die Tür zu diesem Schloss wird sich in einer Stunde öffnen. Doch niemand von Euch wird dann mehr am Leben sein,  es sei denn Ihr seit gut im finden von Dingen. Wie ihr sicher schon bemerkt habt hat jeder von euch ein Gerät aus Stahl um euren Hals. In seinem inneren sind Klingen mit einer Starken Feder gespannt worden.Sobald der Countdown abläuft wird der Träger einen Kopf kürzer sein … Die Schlüssel wurden in der Villa versteckt. Pro Gerät werden zwei Schlüssel benötigt um es zu entfernen. Sobald Ihr euer Gerät entfernt habt könnt Ihr gehen. Natürlich sind einige Schlüssel an sehr unzugänglichen Stellen versteckt. Wie dem auch sei... Lasset die Spiele beginnen!“
}

\hauntsection{Was der Verräter hört:}

{\itshape „All diese Leute waren Augenzeugen bei dem Verkehrsunfall Deiner Mutter. Ihr Auto fing Feuer aber niemand von Ihnen half Deiner bewusstlosen Mutter aus dem Auto bevor der Benzintank explodierte.Jetzt hast Du die Chance für Deine Rache. Das Gerät um Deinen Hals funktioniert nicht, aber die anderen Wissen das nicht. Werden die anderen Genau so engagiert sein Dir zu helfen wie Sie Deiner Mutter geholfen haben ? }

\newpage

\rightnow{

    \begin{itemize}
        \bitem Plaziere sechseckige Gegenstands-Marker(Schlüssel) in folgende Räume(oder wenn Sie entdeckt werden) Attic, Catacombs, Chasm, Collapsed Room, Crypt, Furnace Room, Gallery, Junk Room, Operating Lab, Pentagram Chamber, Tower und Vault.
        \bitem Nimm  so viele rote Monster Marker wie Spieler im Spiel sind. Mische die Marker durch, teile sie mit dem Kopf nach Unten aus.(Damit die Nummer nicht zu lesen ist)
Der Spieler mit der Nummer 1 ist der Verräter.
        \bitem Teile so viele sechseckige Gegenstands-Marker (nummeriert von 1 aufwärts) wie Spieler im Spiel sind aus. Die Spieler müssen die Marker Nummer sichtbar Neben sich legen.
        \bitem Lege den Rundenanzeiger mit einem Clip auf den Tisch und Markiere die 0 wenn 1-4 Spieler im Spiel sind. Bei 5-6 Spielern starte bei 1. Die Anzeige markiert die Zeit.
    \end{itemize}
}

\whatyouknowaboutthebadguys{
Dieser Fluch nutzt die „Versteckter Verräter“ Regeln.(Regelbuch) Jeder liest diese Seite aber einer ist der Verräter.
Jemand hat Dir ein tickendes, mörderisches Halsband umgelegt.Und das nicht genug. Scheinbar denkt einer Deiner Freunde auch noch das Du aussiehst ohne Kopf.

\paragraph{Verräter:} Der Verräter muss sich zu erkennen geben wenn er Durch das Halsband getötet würde.Davor kann er Schlüssel sammeln und tauschen um Ihr vertrauen zu gewinnen.

}

\youwinwhen{
Die Helden gewinnen wenn alle Halsbänder entfernt wurden und mindestens die Hälfte (aufrunden) noch Leben. Wenn mehr als die Hälfte gestorben sind hat der Verräter gewonnen.

Ein Halsband kann entfernt werden nachdem es aufgeschlossen wurde oder es explodiert.
Ein Halsband zählt auch als entfernt wenn sein Träger durch andere Aktionen getötet wurde oder wenn es sich auf dem enthüllten oder toten Verräter befindet.
}

\vspace{2cm}
(Dieses Szenario besteht aus zwei Seiten…)
\newpage

\hauntsection{Wie man das Halsband entfernt:}
  \begin{itemize}
        \bitem Du kannst(musst aber nicht) ein Schlüssel Marker nehmen sobald Du alle nötigen Raumanforderungen (siehe Tabelle unten) erledigt hast. Die Seitenleiste zeigt Dir den fortschritt des Timers an der an Deinem Halsband befestigt ist.

        \bitem Nachdem Du einen Schlüssel gefunden hast kannst Du Dich nicht mehr bewegen, Du kannst Ihn aber benutzen oder einem anderen Spieler geben.Jeder Schlüssel kann nur einmal benutzt werden.

        \bitem Wenn Du mindestens zwei Schlüssel bei Dir hast kannst Du jederzeit während Deines Zugs bekanntgeben das Du Dein Halsband(oder das eines Mitspielers im gleichen Raum) entfernst.
        Bitte beachten: Ein Spieler kann ein Schlüssel nicht im gleichen Zug eintauschen und benutzen!

        \bitem Die Schlüssel können nicht benutzt werden um das Halsband eines Toten zu entfernen.
    \end{itemize}

\paragraph{Crypt, Furnace Room} Nimm den Schlüssel nachdem Du durch den Raum schaden genommen hast.
\paragraph{Gallery} Du musst zum Ballroom herabspringen(der bereits im Haus sein muss) um den Schlüssel zu bekommen.
\paragraph{Vault} Der erste Spieler der seinen Zug in diesem Raum beendet nachdem der Tresor geöffnet wurde bekommt den Schlüssel.
\paragraph{Catacombs,Chasm, Tower} Du musst erfolgreich Würfeln um den Raum zu durchqueren bevor Du den Schlüssel aufnehmen kannst.Schlägt der Wurf fehl kannst Du es in der nächsten Runde erneut versuchen. Klappt es landest Du am anderen Ende des Raums und hast den Schlüssel. Du musst den gleichen Vorgang wiederholen wenn Du zurück Durch den Raum willst.
\paragraph{Attic, Junk Room, Pentagram Chamber} Um den Schlüssel zu bekommen musst Du erst würfeln wie es nötig wäre den Raum zu verlassen. Schlägt der Wurf fehl nimmst Du Schaden wie beschrieben. Erfolgreich oder nicht Du musst erneut würfeln um den Raum zu verlassen.
\paragraph{Operating Laboratory} Eine Röntgenaufnahme zeigt das der Schlüssel in Dir ist! Du musst mit jedem Deiner vier Werte einen Wurf von 3+ schaffen und 2 Würfel physischen Schaden um den Schlüssel zu bekommen. Schlägt einer der Würfe fehl bekommst Du den Schlüssel nicht dennoch aber den Schaden.

\newpage

\hauntsection{Was Du tun musst:}

am Ende der Runde des Spielers der den Fluch ausgelöst hat wird der Rundenzeiger um eins erhöht.
Dann muss jeder Spieler dessen Gegenstands-Marker-Zahl niedriger oder gleich der Rundenzeiger ist 3 Würfel werfen. Ist das Ergebnis kleiner als die aktuelle Rundezahl stirbt der/die Spieler sofort(Das Halsband wurde ausgelöst)

\paragraph{Beispiel:} In einem 4 Spieler Spiel haben Professor Longfellow und Zoe Ingstrom die Gegenstands-Marker-Zahlen 1 und 2. Nach Zwei Runden müssen beide Spieler mit 3 Würfeln würfeln.Bei einem Ergebnis von 0 oder 1 stirbt er/sie.

\hauntsection{Wenn die Helden gewinnen:}
{\itshape Als das letzte Halsband entfernt wurde öffnet sich die Eingangstür des Schlosses und Ihr fühlt einen kalten Luftzug durch die Eingangshalle strömen. Aber wer hat das alles getan und warum ? Wenn genügend Leute so eine Art Folter mögen gibt es vielleicht eine Fortsetzung … Oder Fünf … }

\hauntsection{Wenn der Verräter gewinnt:}
{\itshape Als Du dir die Kopflosen Körper der anderen Leute anschaust die Deine Mutter zum Sterben verurteilt haben fühlst Du dich so als wenn Du gerade eine wichtige Lektion im des Lebens gelernt hast. Wer auch immer dir das Beigebracht hat muss eine Art tiefgründiger Moral Apostel gewesen sein und Du willst in seine Fußstapfen treten. Entweder das oder Ihr seit einfach beide Bekloppt. Wer weis das schon ?}