
\survival{44}{Death Doth Find Us All}{Der Tod findet uns Alle}{+}

\introduction{
Schon seit du diesen Ort betreten hast, hast du ein seltsames Gefühl. Zuerst dachtest du, es wäre nur Einbildung, aber jetzt bist du da nicht mehr so sicher. Du fühlst die müde, klapprig, fast antik.Du wischst den Staub von einem Spiegel ab und siehst in dein Gesicht. Falten durchziehen deine Haut, weit tiefer als alle die du jemals gesehen hast. Dein Haar ist ergraut und deine Schultern hängen herab. Du alterst schneller als dieses schreckliche alte Haus. Besser, du findest eine Möglichkeit das Altern zu stoppen bevor du deine Jugend verlierst… oder sogar dein Leben.

}

\rightnow{

    \begin{itemize}
        \bitem Lege 5 dreieckige Gesundheitswurf- und 5 dreieckige Wissenswurf-marker an die Seite (sanity, knowledge).
        \bitem Lege beliebige sechseckige marker als „Altersmarker“ bereit.
        \bitem Jeder Held (nicht der Verräter) legt einen Altersmarker auf seine Charakterkarte. Das gilt auch für den Träger des Medaillons.
        \bitem Das Startalter deines Charakters ist das, was auf der Charakterkarte angegeben ist. Für jeden Altersmarker kommen 10 Jahre dazu. Der Verräter wird dir sagen, welche Auswirkungen das Altern hat.
    \end{itemize}
}

\whatyouknowaboutthebadguys{
Der Verräter altert nicht. Im Gegenteil, er scheint jünger als je zuvor.
}

\youwinwhen{
du den übernatürlichen Alterungsprozess stoppen kannst.
}

\hauntsection{Während deines Spielzuges musst du}
für jede 10 Jahre die du alterst einen Altersmarker auf deine Charakterkarte legen.

\newpage

\hauntsection{Wie man den Alterungsprozess stoppt}
Die Helden müssen das Ritual der Wiederverjüngung durchlaufen.
  \begin{itemize}
        \bitem Das Ritual der Wiederverjüngung verlangt so viele erfolgreiche Ritual-Würfe, wie Spieler am Spiel teilnehmen. Jeder Wurf muss in einem bestimmten Raum gemacht werden und erfordert einen Gesundheits- oder Wissenswurf (sanity, knowledge) von 5+. Jeder Spieler kann während seines Zuges nur einen Wiederverjüngungswurf machen.
        \bitem Man kann den Wurf nur machen, wenn man das Medaillon besitzt und in einem der folgenden Räume ist: Katakomben, verkohlter Raum (charred room), Krypta, Galerie, Küche (kitchen), Pentagrammraum oder Turm (tower).
        \bitem Jedes Mal wenn ein Wiederverjüngungswurf erfolgreich war legst du einen Gesundheits- oder Wissensmarker (je nachdem welche Fähigkeit du benutzt hast) in den Raum in dem du gewürfelt hast.
        \bitem Räume in denen bereits ein erfolgreicher Wurf getätigt wurde, können für das Ritual nicht ein zweites Mal benutzt werden.

    \end{itemize}


\hauntsection{Das Medaillon}

  \begin{itemize}
        \bitem Wer immer das Medaillon trägt, er zieht immer ein Jahrzehnt von dem Alterungsprozess ab, der in der Spielrunde des Verräter stattfindet.
        \bitem Jedes Mal wenn ein Forscher stirbt, altert der Träger des Medaillons um 10 Jahre.
    \end{itemize}

\outro{
Der Zauberspruch ist gesprochen. Alle sind gespannt. Einige Minuten lang sagt niemand ein Wort. Ihr starrt euch nur gegenseitig an und wartet, ob eure Körper weiter altern und verwelken. Dann hustet jemand und ein schwaches Lachen ist zu hören, dann noch eins. Bald darauf lacht ihr alle und weint hysterisch. Ihr seid am Leben, gewiss, aber ein Teil eures Lebens ist für immer verloren.
}