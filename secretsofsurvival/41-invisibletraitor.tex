
\survival{41}{Invisible Traitor}{Der unsichtbare Verräter}{}

\introduction{
Einer deiner Mitspieler flüstert ein seltsames Wort… und schreit dann laut. Du hast so einen Schrei noch von keinem lebenden Wesen gehört. Du kommst deinem Freund zu Hilfe, aber als du ihn erreichst bist du alleine. Dann hörst du ein Schlurfen und ein hartes, kalten Lachen.
Du hast das schreckliche Gefühl dass einer deiner Freunde sich in einen Feind verwandelt hat.

}



\whatyouknowaboutthebadguys{
Der Verräter wurde unsichtbar und hat das Ziel, euch alle zu töten.
}

\youwinwhen{
der Verräter tot ist.
}

\hauntsection{Kampf ins Leere}

  \begin{itemize}
        \bitem Wenn der Held von dem Verräter angegriffen wurde und Überlebt hat kann dieser einen Wissenswurf (Knowledge) am ende der Verräterrunde versuchen:

\rolls
\roll{0-2}{Nicht ist passiert}
\roll{3-4}{Der Held weiß, dass der Verräter noch im Raum ist oder, wenn er den Raum verlassen
hat, welchen Ausgang er genommen hat.}
\roll{5+}{Der Held hat die Bewegung des Verräters geahnt und weiß, in welchem Raum dieser
sich jetzt befindet. }
\erolls

        \bitem Einmal während deines Zuges kannst Du einen Angriff gegen den unsichtbaren Verräter machen. Ist der Verräter im gleichen Raum, kannst du normal angreifen. Ist der Verräter woanders geht dein Angriff daneben und du kannst in derselben Runde keinen weiteren Angriff machen.
        \bitem Benutzt Du das Dynamit oder den Revolver in dem Raum wo der Verräter ist gelten die normalen Angriffsregeln.
    \end{itemize}

\newpage

\hauntsection{Schädel und Séance-Brett (Skull und Spirit Board)}

  \begin{itemize}
        \bitem Hast du die Schädel-Karte zu Beginn deines Zuges, kannst du einen Gesundheits-Wurf (Sanitary) von 4+ versuchen um den Verräter zu finden. Bei erfolgreichem Wurf sagt dir der Verräter in welcher Etage er sich befindet.
        \bitem Hast du die Séance-Brett-Karte zu Beginn deines Zuges, kannst du einen Wissenswurf (Knowledge) von 4+ versuchen um den Verräter zu finden. Bei erfolgreichem Wurf sagt dir der Verräter welches Symbol (falls vorhanden) in dem Raum ist in dem er sich gerade aufhält.

        \bitem
    \end{itemize}


\outro{
Der Verräter liegt tot auf dem Boden, der Körper den das Böse verlassen hat ist jetzt wieder sichtbar. Jetzt wo er da liegt sieht er gar nicht mehr so gefährlich aus.
Du weißt nicht warum sich dein Freund gegen dich gewandt hat. Du hoffst, dass das, was immer diese Verwandlung in ihm hervorgerufen hat, sich nicht eines anderen Kameraden bemächtigt, möglicherweise in einer noch entsetzlicheren Form…
}