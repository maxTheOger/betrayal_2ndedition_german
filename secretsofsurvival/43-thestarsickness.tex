
\survival{43}{The Star-Sickness}{Die Sternen Krankheit}{}

\introduction{
Neben dem Speer liegt eine Notiz die besagt:
„Nachdem ich den Meteorit berührt habe spüre ich wie er mich verändert.Ich habe Ihn mit einem Mikroskop untersucht und eine DNA gefunden die nicht von dieser Welt stammt.Viel schlimmer ich höre immer wieder Stimmen … Stimmen der Kreaturen in meinem Blut. Sie sagen mir das sie meinen Körper viel besser nutzen können als ich es jemals könnte.“
Ich habe ein Antivirus hergestellt, aber mir lief die Zeit davon. In kürze werden mich die Kreaturen übernommen haben.Doch bevor das passiert werde ich mich in meinen Speer stürzen und hoffe das die Kreaturen dabei auch vernichtet werden. Wenn ich nicht erfolgreich bin hilft diese Aufzeichnung vielleicht anderen die nach mir kommen.Ich bete das die Kreaturen mit mir sterben werden. Die verlauf der Krankheit ist folgender:
Als erstes wird dich ein Fieber packen, wie das einer Grippe, weil Dein Körper gegen sie kämpft.
Danach fühlst Du dich ziemlich verstopft.
Als zweites übernehmen die Sporen den Verstand des Wirtes.
Kannst Du Deinen Freunden vertrauen ?
Und als drittes … HORROR.
}

\rightnow{

    \begin{itemize}
        \bitem Nimm einen Satz Monster-marker der gleichen Farbe.Nummeriert von 1 bis zu der doppelten Anzahl der Spieler und mische sie. Jeder Spieler nimmt zwei, schaut sie sich an, zeigt sie aber nicht den Mitspielern! Der Spieler der Marker Nummer 1 hat ist der Verräter. Mehrere Verräter können über den Spielverlauf auftauchen. Du bleibst der Verräter auch wenn Du den Marker verlierst. Lege sie verdeckt und gemischt neben Deine Charakter Karte.
        \bitem Wenn das Research Lab noch nicht im Spiel ist sucht der Fluch-Auslöser danach im Stapel und legt es passend an. Danach wird der Stapel neu gemischt.
        \bitem  Lege die Rundenanzeige mit einem Plastik-Clip auf 1 neben das Spiel. Das ist die Zeitanzeige.
        \bitem Lege einen Satz dreieckige Wissens-Wurf-Marker und einen Satz sechseckige Gegenstands-Marker (das Serum) beiseite.

    \end{itemize}
}
\newpage
\whatyouknowaboutthebadguys{
Der Fluch benutzt die „Verstecker Verräter“ Regeln(Regelbuch). Alle lesen diese Seiten, aber einer von euch ist der Verräter.

Der Verräter ist mit einem Ausserirdischen Virus infiziert und versucht den rest von euch in Aliens zu verwandeln.

Derjenige der den Verräter-Marker(Monster Marker #1) hat darf ihn zu keiner Zeit jemandem zeigen!

}

\youwinwhen{
Ihr das Serum hergestellt habt und alle (noch lebenden) Helden gemimpft wurden.

Wenn alle Spieler tot sind oder Verräter oder Aliens wurden dann gewinnen der oder die Verräter.
In dem Fall verliert der der als letztes zum Verräter wurde.

}

\hauntsection{Was Ihr in jeder Runde tun müsst:}
Als erster Zug in jeder Runde müsst Ihr einen Raum mit einem(oder mehreren) anderen Spieler/n betreten.Du musst einen Deiner verdeckten Monster Marker einem Mitspieler in dem Raum geben(bei mehreren Such Dir einen Aus) und er gibt Dir einen.Es muss ein Spieler sein mit dem du vorher noch nichts getauscht hast.(falls möglich) Du kannst pro Runde nur einen Austausch durchführen.

Falls Du den Monster Marker #1 bekommst wirst Du auch zum Verräter, auch wenn Du den Marker später austauscht.

\hauntsection{Der Fluch-Auslöser muss folgendes tun:}
am ende seiner Runde erhöht er die Zeitanzeige um 1 Punkt.
Dann würfel so viele Würfel wie die Zeitleiste anzeigt.
Das Ergebnis ist ein Might-Angriff gegen jeden Spieler im Haus.
Jeder Spieler wiedersteht den angriff mit Might.
Verräter sind nicht immun ! Die Alien Krankheit greift jeden Körper an.
Es gibt kein oberes Limit für die Anzahl der Würfel für diesen Angriff.

\vspace{2cm}

Dieses Szenario geht auf der nächsten Seite weiter …


\newpage

\hauntsection{Wie man das Serum erzeugt:}

  \begin{itemize}
    \bitem Einmal pro Runde wenn Du im Research Lab bist kannst Du einen Wissens-Wurf mit 5+ versuchen.Addiere für jeden Mitspieler im Raum 1 hinzu und nimm einen extra Würfel wenn das Buch mit im Raum ist. War der Wurf erfolgreich packe ein Wissens-Wurf-Marker in den Raum.
    \bitem Die Spieler können auch im Coservatory und dem Garten nach Zustaten für das Serum suchen.
Am ende Deines Zugs kannst Du einen Wissens-Wurf von 3+ versuchen. Bei Erfolg packe einen Wissens-Wurf-Marker auf Deine Charakter Karte. Behandel diesen Marker als Gegenstand. Es kann abgelegt, getauscht und gestohlen werden. Du kannst jede Runde nach Zutaten suchen und mehr als einen Marker gleichzeitig tragen.
    \bitem Es ist möglich mehr Wissens-Wurf-Marker zu nehmen als Spieler im Spiel sind.
    \bitem Um das Serum herzustellen müssen so viele Wissens-Wurf-Marker im Research Lab sein wie Spieler im Spiel.Mindestens ein Marker muss von gesammelten Zutaten stammen.Der Spieler der die letzte Zutat bringt, bekommt das Serum
    \bitem Wenn Du das Serum hast kannst Du es Dir oder jedem beliebigen Freiwilligen anderen injizieren.Dafür wird 1 Bewegungspunkt verwendet.Einmal geimpft müssen die Spieler nicht mehr am Rundenanfang die Monster-Marker austauschen und die Spieler werden nicht mehr durch den Virus angegriffen.
    \bitem Wenn die Spielerwerte der geimpften unterhalb des Startwerts sind, werden sie auf den Startwert angehoben.
    \end{itemize}

\specialattackrules{
In diesem Fluch kann jeder jeden Angreifen.

Wenn ein Verräter aus irgendeinem Grund stirbt (ausser durch impfen) sagt Dieser Spieler das er ein Verräter ist und das er sich in ein Alien verwandelt. Dieser Spieler überspringt seine nächste Runde und ist die darauffolgende Runde ein Alien. Er lässt sämtliche Gegenstände fallen.

}

\newpage
\monster{Aliens}{4}{6}{4}{}

    \begin{itemize}
        \bitem Aliens werden nicht durch die Krankheit beinflusst und können keinen Gegestände oder Marker nehmen oder tauschen.
        \bitem Aliens werden für Ihre Bewegungen und Schäden wie Monster behandelt.
        \bitem Wenn Du das Serum hast kannst Du es einem unfreiwilligen Spieler oder einem Alien im gleichen Raum injizieren wenn Du ihn im physischen Kampf mit einem Geschwindigkeitswurf besiegst.
Hast Du erfolg tötest Du den Verräter oder das Alien.
        \bitem Ein Verräter dem Du das Serum injizierst stirbt sofort und wird nicht zu einem Alien.
        \bitem Bist Du in dem Raum in dem sich ein Verwandelnder Verräter befindet kannst Du Ihn für einen Bewegungspunkt impfen. Das tötet ihn sofort.
    \end{itemize}


\hauntsection{Wenn die Helden gewinnen:}
{\itshape Deine Infizierten Freunde sehen genauso aus wie immer aber sie waren irgendwie … anders.
Nachdem Du das Haus verlässt weist Du das Du Dir jeden den Du triffst genau anschauen musst … Um nach Anzeichen zu suchen …
}

\hauntsection{Wenn die Verräter gewinnen:}
{\itshape Du hast rausgefunden das es Billionen Lebensformen wie diese auf dem Planeten gibt.Es wird zeit in ein Ballungsgebiet zu gehen und mit der Fortpflanzung zu beginnen!}