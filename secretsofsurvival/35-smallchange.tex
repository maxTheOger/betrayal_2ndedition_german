
\survival{35}{Small Change}{Kleine Veränderung}{}

\introduction{
Ein paar Katzen streifen durchs Haus. Sie schauen eingeschnappt auf Dein Eindringen in ihr Revier, aber andererseits lassen Sie dich in Ruhe. Nun ja, wenn die Katzen hier die schlimmsten Geschöpfe sind hast Du nicht viel zu befürchten. Das Geräusch eines zerbrechendes Glases ruft Dich aus Deinen Gedanken heraus. Du drehst Dich um und siehst eine zerbrochene Viole auf dem Boden. Eine silberne Flüssigkeit dringt heraus und bildet verdampfend einen Nebel, der Dich umhüllt. Du schüttelst Dich einen Moment, und als Du wieder klar denken kannst findest Du Dich wieder indem Du einen Stuhl anstarrst, der anscheinend Kilometer über Dir aufragt. Das ist ja die Sicht einer Maus! Da hörst Du ein Geräusch außerhalb des Raumes ... MIAU ...

}

\rightnow{
Lege einen sechseckigen Spielzeug-Flugzeug-Marker (Toy Airplane) neben das Spielfeld.
}

\whatyouknowaboutthebadguys{
Der Verräter hat Dich schrumpfen lassen und läßt seine Katzen ins Haus. Sie wollen dich fressen.

}

\youwinwhen{
mindestens die Hälfte der Helden (aufrunden) mit dem Spielzeug-Flugzeug durch den Balcony, Gardens, Graveyard, Patio, Tower oder ein nach draußen zeigendes Fenster fliehen kann.

}

\hauntsection{Klein sein …}
  \begin{itemize}
       \itemsep-5pt

        \bitem Alle Gegenstände und Omen sind ebenfalls geschrumpft und funktionieren normal.
        \bitem Du kannst Keine weiteren Karten ziehen, das bedeutet Dein Zug Endet sobald Du einen Raum mit einem Symbol neu entdeckst.Du kannst das Haus nur im Spielzeug-Flugzeug verlassen.
        \bitem Jede Tür zählt jetzt als 1 Feld, eine Bewegung in einen benachbarten Raum kostet nun 2 Bewegungspunkte.
        \bitem Du darfst auf dem Türfeld stehen bleiben.
        \bitem Den Mystischen Fahrstuhl oder den eingestürzten Raum darfst Du nicht mehr benutzen.
        \bitem Du wirst nicht durch die Gallerie den Fitnessraum oder den Safe beinflusst.
        \bitem Du kannst Treppen benutzen, allerdings musst Du dafür einen Machtwurf von 3+ schaffen um hoch oder runter zu gehen.Schlägt der Wurf fehl kannst Du es in der nächsten Runde erneut versuchen.
    \end{itemize}

\hauntsection{Benutzung des Flugzeuges:}
  \begin{itemize}
       \itemsep-5pt
    \bitem Das Flugzeug kann sich in folgenden Räumen befinden: Bedroom, Master-Bedroom, Storeroom, Attic oder Game Room. Einmal während Deines Zuges kannst Du mit einem erfolgreichen Wissens-Wurf von 3+ in einem dieser Räume das Flugzeug finden. Bei Erfolg, lege den Toy-Airplane Marker in diesen Raum.
    \bitem Einmal während Deines Zuges kannst Du mit einem erfolgreichen Wissens-Wurf von 4+ das Flugzeug starten es wird jedoch bis zur nächsten Runde auf dem Boden bleiben. In dieser Zeit können die anderen Helden das Flugzeug betreten aber jeder Held in ihm kann währenddessen angegriffen werden.
    \bitem Das Flugzeug hat eine Geschwindigkeit von 5, wie für Dich zählen Türen jeweils 1 Bewegungspunkt extra.
     Wenn das Flugzeug fliegt können Helden darin nicht angegriffen werden.
    \bitem Du darfst das Haus nicht verlassen, wenn du nicht alle noch lebenden Helden einsammelst.
    \bitem Das Einsammeln eines Helden kostet 1 Bewegungspunkt. Wenn Du das tust muß der Held im Flugzeug mit dem höchsten Geschwindigkeits-Wert einen Wurf ausführen:
    \vspace{-10pt}
    \rolls
    \roll{4+}{Bei 4+ ist der Held erfolgreich aufgenommen. }
    \roll{2-3}{Bei 2-3 ist der Versuch fehlgeschlagen, es darf aber gegen einen weiteren Bewegungspunkt nochmals versucht }
    \roll{0-1}{Bei 0-1 gibt es eine Bruchlandung und das Flugzeug muß neu gestartet werden.}
    \erolls

    \bitem Es kostet 1 Zug um einen Raum mit einer nach Aussen Zeigenden Ecke zu verlassen.
    \bitem Fliegende Helden können angreifen oder angegriffen werden aber nur mit dem Revolver, Dynamit und dem Ring.
    \bitem Fliegende Helden müssen nicht würfeln um Treppen zu steigen.Sie können auch rauf und runter fliegen im Collapsed Room und der Gallery und sie kommen durch den Chasm ohne würfeln zu müssen.
    \bitem Der Verräter darf das Flugzeug nicht benutzen.
    \end{itemize}

\hauntsection{Wenn die Katze Dich fängt}

  \begin{itemize}
           \itemsep-5pt

        \bitem Wenn Dich eine Katze fängt hast Du eine Chance ihr zu entkommen. In Deiner nächsten Runde suchst Du Dir eine Eigenschaft aus. Du und die Katze würfeln in dieser Eigenschaft und sehen, wer eine höhere Zahl würfelt.
        \bitem Würfelst Du höher gelingt Dir die Flucht und führst Deinen Zug normal aus. Ansonsten verbleibst Du in den Fängen der Katze und Dein Zug ist beendet. Wenn vor deinem nächsten Zug ein anderer Held die Katze besiegt läßt sie Dich fallen.
        \bitem Helden die gerade von der Katze geschnappt wurden können nicht vom Flugzeug mitgenommen werden.
    \end{itemize}

\outro{
Das kleine Flugzeug spuckt und schüttelt sich bevor es rauchend durch das Fenster fliegt. Du hörst ein enttäuschtes Jaulen von den Katzen hinter Dir. Du bist geflohen! Nun ist alles was noch zu tun ist, einen Weg zu finden, wieder auf normale Größe anzuwachsen, bevor irgendein Raubvogel entscheidet aus Dir eine Mahlzeit zu machen.
}