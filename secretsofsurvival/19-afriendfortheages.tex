
\haunttitle{A Friend for the Ages}{Ein Freund für die Ewigkeit}

\introduction{
Legenden besagen dieses Haus beherbergt eine alte Böse macht.Die Gallerie ist ausgefüllt mit vielen Wertvollen sehr alten Gemälden die nach und nach von Staub bedeckt wurden.Als Du durch die Gemälde schaust kommt Dir eines seltsam vertraut vor.Du erkennst das Gesicht einer Deiner Freunde wieder, jedoch vom alter Getrübt und übersäht mit tödlichen Wunden.Unterhalb des Gemäldes ist eine verstaube Wittmung:
„Ein Freund für die Ewigkeit. Der Tod wird Dich verschonen während dieses Gemälde ewig währt.“
}

\rightnow{
Nimm zwei sechseckige Gegenstandsmarker (das sind die Farbeimer) plus so viele wie Spieler im Spiel sind.
Ausserdem nimm so viele dreieckige Wissenswurfmarker wie Spieler im Spiel sind.
Plaziere die Farbeimer in folgenden Räumen:
Attic, Abandoned Room, Collapsed Room, Patio, Statuary Corridor, Storeroom und der Wine Cellar.
Sind mehr von den Räumen vorhanden als Farbeimer, plaziere die Farbeimer in den Räumen die am weitesten von den Helden entfernt sind.
Hast Du mehr Farbeimer als Räume, lege die restlichen Farbeimer zur seite und Platziere sie erst dann in den Räumen wenn diese aufgedeckt wurden.
}

\whatyouknowaboutthebadguys{
Der Verräter wird von einem mysterieusen Gemälde beschützt das alle Schäden und Krankheiten von Ihm nimmt.
Er wird es unter allen Umständen beschützen.

}

\youwhinwhen{
Das Gemälde mit Farbe übermalt wurde oder der Verräter tot ist.
}

\hauntsection{Wie man das Gemälde übermalt}
Farbeimer können aufgenommen, abgelegt, getauscht und gestolen werden.
Ein Hund kann ihn nicht tragen.
Jeder Held kann nur einen Eimer auf einmal tragen.
Wenn Du in der Galerie bist und einen Farbeimer trägst kannst Du einen Wissenswurf von 4+ versuchen um das Gemälde zu übermalen. Bist Du erfolgreich leg den Farbeimer beiseite und lege ein Wissenswurfmarker in den Raum.
Ein Held kann das pro Runde nur einmal tun.
Wenn so viele Wissenswurfmarker im Raum sind wie Helden zu beginn des Fluchs wurde der Bann gebrochen und das Gemälde Übermalt.

\specialattackrules{
Der Verräter nimmt keinen Schaden durch normale Angriffe.
Wenn Du den Verräter durch einen physischen Angriff um 2 besiegst kannst Du Gegenstände von Ihm stehlen.


Ausnahme:


Wenn ein Held das Amulett der Zeitalter trägt während er angreift nimmt der Verräter normalen Schaden.
}

\outro{
Als Du den letzten Strich machst fühlst Du wie die Farbe dichter wird und nichts mehr von dem Gemälde zu sehen ist und die Macht verfliegt.Du blickst zu  Deinem verräterischer Begleiter. Seine Haare wachsen schnell und werden weiss, sein Gesicht bekommt falten und knöchrig während sein Körper nach und nach zusammensackt. Einen Moment später ist nichts mehr übrig als ein kleiner Haufen Staub.Als Du dich wieder zum Gemälde umdrehst wunderst Du dich.
Wo hast Du dieses Gesicht schon einmal gesehen ???
}