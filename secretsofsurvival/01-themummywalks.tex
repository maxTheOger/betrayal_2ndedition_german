
\haunttitle{The Mummy Walks}{Die wandernde Mumie}

\introduction{
Staubschwaden ziehen in den Raum und ein Schatten legt sich über dein Herz. Du hörst einen deiner Freude
schreien, ein Geräusch aus Vergnügen und Entsetzen. Eine kalte, klamme Stimme lässt deinen Verstand
erschaudern. ``Ich verlor meine Braut, viele Jahre bevor du denken kannst. Meine Tränen sind verstaubt,
aber meine Liebe ist immer noch so kräftig wie die Sonne. Jetzt ist meine Liebe für mich wiedergeboren. Es
gibt nichts mehr was uns beide trennen kann... und wenn du dich gegen mich stellst, werde ich deine Seele
aus deinem Körper reißen und sie gänzlich verschlingen.''
}

\rightnow{

    \begin{itemize}
        \bitem Lege 2 Wissenswurfplättchen (Knowledge Roll tokens) beiseite.
        \bitem Der Abenteurer mit dem Mädchen (Girl) verliert es. Dieser Abenteurer verliert alle Bonuspunkte von
der Mädchenkarte (Girl card) und legt sie bei Seite. Der Verräter legt dann das Mädchenplättchen
(Girl token) in einen anderen Raum.
        \bitem Wenn ein Charakter eines Spielers den Raum mit dem Mädchenplättchen betritt, dann erhält dieser
Spieler die Mädchenkarte.
    \end{itemize}
}

\whatyouknowaboutthebadguys{
Der Verräter versucht die Mumie mit dem Mädchen zu verheiraten.
}

\youwhinwhen{
...die Mumie ins Reich des Todes verbannt wurde, bevor sie das Mädchen heiraten konnte.
}

\hauntsection{Wie wird die Mumie verbannt}

Wenn die Buchkarte noch nicht im Spiel ist durchsucht der Held der als nächstes in einem Raum mit dem
Omen Symbol läuft nach der Karte und nimmt sie.Danach wird der Stapel neu gemischt.
Du musst den wahren Namen der Mumie in dem Buch finden und sagen. Um dies zu schaffen musst du
folgende Schritte in dieser Reihenfolge erledigen. Du kannst pro Zug nur einen Schritt erledigen.
Um den wahren Namen der Mumie heraus zu bekommen, kannst du versuchen einen Wissenswürfelwurf
(Knowledge roll) von 6+ in den folgenden Räumen zu bestehen (auf die folgenden Art und Weisen):

    \begin{itemize}
        \bitem der Raum mit dem Sarkophag (untersuche die Hieroglyphen)
        \bitem das XXX Labor (Research Labor) (überfliege die Notizen des Archäologie Teams), oder
        \bitem die Bücherei (Library) (erforsche die Geschichte der Mumie).
    \end{itemize}

Wenn du Erfolg hattest, nehme dir eine Wissenswurfscheibe.
    \begin{itemize}
        \bitem An einem Zug nach dem du Namen erforscht hast, während du das Buch hattest, kannst du einen Wissenswürfelwurf von 6+ versuchen, um den Namen der Mumie herauszufinden. Wenn du Erfolg
hattest, nehme dir ein Wissenswurfplättchen.
        \bitem Sobald du zwei von diesen Plättchen hast, bringe das Buch in den Raum in dem sich die Mumie befindet. Jeder Held kann während du dort bist, versuchen einen Gesundheitswürfelwurf (Sanity roll) zu bestehen, um einen Bann zusprechen der sie für immer vertreibt.
    \end{itemize}

    Die Mumie ist imun gegen Geschwindigkeitsangriffe(Revolver,Dynamit..)

\outro{
Ein heißer trockener Wind flüstert durch den Raum, als du den altertümlichen Wälzer zuknallst. Die Mumie setzt das Schlurfen in deine Richtung fort, ihre Augen sind tote Höhlen der Verzweiflung. Gerade als ihre Hände deine Kehle umklammern, beginnen die Umwicklungen der Mumie zu bröckeln. Die Kreatur stöhnt immer mehr und mehr über ihren Körper, der zusammengedrückt und mit dem heißen Wind hinfort geweht wird. "Meine Braut... meine einzige Liebe... nicht... mehr...."
Als der letzte Rest der Mumie verschwunden ist, hört der Wind auf. Du bist allein.
}

\hauntsection{FAQ}
Was passiert, wenn der Verräter das Buch hat? Dann müssen es die anderen ihm abjagen.
Muss der gleiche Abenteurer die beiden Wissenswürfe machen? Nein
Wenn Kraft und Geschwindigkeit beide auf der untersten Stufe sind, kann die Mumie dann auch töten? Ja..