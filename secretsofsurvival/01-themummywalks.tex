
\haunttitle{The Mummy Walks}{Die wandernde Mumie}

\introduction{
Staubschwaden ziehen in den Raum und ein Schatten legt sich über dein Herz. Du hörst einen deiner Freude
schreien, ein Geräusch aus Vergnügen und Entsetzen. Eine kalte, klamme Stimme lässt deinen Verstand
erschaudern. ``Ich verlor meine Braut, viele Jahre bevor du denken kannst. Meine Tränen sind verstaubt,
aber meine Liebe ist immer noch so kräftig wie die Sonne. Jetzt ist meine Liebe für mich wiedergeboren. Es
gibt nichts mehr was uns beide trennen kann... und wenn du dich gegen mich stellst, werde ich deine Seele
aus deinem Körper reißen und sie gänzlich verschlingen.''
}

\rightnow{

    \begin{itemize}
        \bitem Lege 2 dreieckige \chips{Knowledge Roll}{\knowroll} beiseite.
        \bitem Der Verräter verliert das Mädlchen (Girl) und die von ihr verliehenen Boni. Stattdessen legt er einen kleinen rosanen Monsterchip (der das Mädchen darstellt) in irgendeinen Raum im selben Stockwerk, in dem der Spuk offenbart wurde, jedoch mindestens 5 Felder von der Mumie entfernt. Gibt es keinen Raum, der mindestens 5 Felder entfernt ist, plaziert er den Chip so weit wie möglich entfernt.
        \bitem Wenn ein Entdecker den Raum mit dem Mädchenchip betritt, dann erhält dieser Spieler die Mädchenkarte.
    \end{itemize}
}

\whatyouknowaboutthebadguys{
Der Verräter versucht die Mumie mit dem Mädchen zu verheiraten.
}

\youwhinwhen{
...die Mumie zurück ins Reich des Todes verbannt wurde, bevor sie das Mädchen heiraten konnte.
}

\hauntsection{Wie die Mumie verbannt wird}

  \begin{itemize}
        \bitem Wenn die Buchkarte (Book) noch nicht im Spiel ist, durchsucht der Held, der als nächstes einen Raum mit Omensymbol entdeckt, den Omenstapel nach dem Buch und nimmt sie. Danach wird der Stapel neu gemischt.
        \bitem Du musst den wahren Namen der Mumie in dem Buch finden und aussprechen. Um dies zu schaffen, musst du folgende Schritte in dieser Reihenfolge erledigen. Jeder Held kann in seinem Zug nur einen der Schritte erledigen.

        \begin{enumerate}
            \item Um den wahren Namen der Mumie heraus zu bekommen, kannst du versuchen einen \knowroll\ von 6+ in den folgenden Räumen zu bestehen:

            \begin{itemize}
                \bitem Untersuche die Hieroglyphen im Raum mit dem Sarkophag oder
                \bitem überfliege die Notizen des Archäologie Teams im Forschungslabor (Research Laboratory) oder
                \bitem erforsche die Geschichte der Mumie in der Bücherei (Library).
            \end{itemize}
            Wenn du Erfolg hattest, nehme dir einen \chipe{Knowledge Roll}.

            \item Wurde der Name entdeckt, kann der Held, der das Buch besitzt, (frühstens im darauf folgenden Zug) einen \knowroll\ von 6+ versuchen, um den Namen der Mumie nachzuschlagen und den Spruch zu lernen, der zu ihrer Verbannung führt. Wenn du Erfolg hattest, nehme dir einen \chipe{Knowledge Roll}.


            \item Sobald die Helden zwei von diesen Plättchen haben, muss ein Held das Buch in den Raum mit der Mumie bringen. Jeder Held, der mit der Mumie und dem Buch im selben Raum ist, kann die Mumie durch Aussprechen des Zauberspruchs verbannen, indem er  die Mumie mittels einer \sanity-Attacke besiegt.
        \end{enumerate}

    \bitem Die Mumie ist imun gegen Geschwindigkeitsangriffe (Revolver, Dynamit..)

    \end{itemize}

\outro{
Ein heißer trockener Wind flüstert durch den Raum, als du den altertümlichen Wälzer zuknallst. Die Mumie setzt das Schlurfen in deine Richtung fort, ihre Augen sind tote Höhlen der Verzweiflung. Gerade als ihre Hände deine Kehle umklammern, beginnen die Umwicklungen der Mumie zu bröckeln. Die Kreatur stöhnt immer mehr und mehr über ihren Körper, der zusammengedrückt und mit dem heißen Wind hinfort geweht wird. "Meine Braut... meine einzige Liebe... nicht... mehr...."
Als der letzte Rest der Mumie verschwunden ist, hört der Wind auf. Du bist allein.
}

\hauntsection{FAQ}

\begin{itemize}
    \bitem Was passiert, wenn der Verräter das Buch hat? Dann müssen es die anderen ihm abjagen.
    \bitem Muss der gleiche Abenteurer die beiden Wissenswürfe machen? Nein.
    \bitem Wenn Kraft und Geschwindigkeit beide auf der untersten Stufe sind, kann die Mumie dann auch töten? Ja.
\end{itemize}