
\haunttitle{Voodoo}{Voodoo}

\introduction{
Du brichst ein Tagebuch auf. Anstatt der Beschreibung der Tage enthält jede Seite ein eingeklebtes Foto einer Voodoopuppe, wie sie alle Welt kennt. Bei allen Bildern ist das Gesicht entfernt worden, jedes trägt ein großes rotes „X“. Verrückt. Aber warte, die letzten paar Bilder haben noch ein Gesicht. Dieses ist das Gesicht eines deiner Freunde! Und dieses sieht aus wie dein Gesicht…
}


\whatyouknowaboutthebadguys{
Der Verräter hat Voodoopuppen im ganzen Haus versteckt. Jede davon ist mit einem Helden verbunden. Die Puppen befinden sich an gefährlichen Orten. Mit dem Verrinnen der Zeit (Turn/Damage-Karte) wird der Effekt dieser Voodoopuppen immer stärker.

}

\youwhinwhen{
du alle Puppen zerstört hast und mindestens die hälfte der Helden die begonnen haben überleben.
}

\hauntsection{Allgemeines}
In diesem Fluch müssen die Helden nicht in Räumen anhalten die ein Omen, Ereignis oder Gegenstandssymbol haben.Du musst nur eine Karte ziehen wenn Dein Zug in dem Raum endet.Du musst auch eine Karte ziehen wenn Du in einem Unentdeckten Raum nach der Puppe suchst.
Wenn ein Held stirbt, wird seine Puppe ebenfalls zerstört.


\hauntsection{Wie man die Puppen zerstört}

Zunächst musst du den Hinweisen des Verräters folgen um festzustellen, in welchen Räumen sich die Puppen befinden. Dann musst du die Puppen in den Räumen finden und zerstören (der Verräter muss auf Nachfrage alle bereits gegebenen Hinweise noch einmal wiederholen).

  \begin{itemize}
        \bitem Bewege dich zu einem Raum von dem du denkst es gäbe dort eine Puppe. Der Verräter gibt Hinweise die es dir erlauben herauszufinden, wo die Voodoopuppen sind.
        \bitem Du kannst einen Wissens-Wurf von 2+ versuchen um nach einer Puppe zu suchen. Du kannst pro Spielrunde in einem Raum suchen (du findest die Puppen erst, wenn du in den richtigen Räumen bist).
        \bitem Hast du eine Puppe gefunden kannst du sie automatisch in deinem nächsten Spielzug vernichten. Du kannst jede Puppe finden aber nur deine Eigene zerstören.
    \end{itemize}

\outro{
Du schnappst dir die Puppe, schaust in ihre Knopfaugen. Augen, trotz ihrer Leblosigkeit, die dich kalt anschauen. Nein! Du wirfst die Puppe wieder und wieder auf den Boden, die Knopfaugen zerspringen, die Nähte platzen, zuletzt reißt auch der Stoff und es bleibt nur ein kleiner Müllhaufen übrig. Die Puppe ist zerstört.
Warte, was hast du getan? Das war nicht das Gescheiteste was man mit einer Voodoopuppe tun kann. Du fühlst dich nicht gut… aber es hätte schlimmer kommen können.
}