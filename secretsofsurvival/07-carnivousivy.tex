
\survival{7}{Carnivorous Ivy}{Fleischfressender Efeu}

\introduction{
Ein trockenes rasselndes Geräusch hallt durch das Haus. Als erstes dachtet ihr riesige Klapperschlange würden kommen um euch zu jagen, doch dann seht ihr Zweig und Äste von Efeu gegen die Fenster drücken. Als ihr die Fenster zu knacken und zu brechen begannen, wusstet ihr das die Pflanzen kommen...um euch zu hohlen.
}

\rightnow{
Legt euch die sechseckige Pflanzensprayscheibe (Plant Spray Token) beiseite.
}

\whatyouknowaboutthebadguys{
Kriechpflanzen versuchen euch einzuwickeln. Jedes Wurzel/Spitzen-Paar (Root/Tip-Tokens)
Bilden eine Kriechpflanze. Die Pflanzen suchen nach Dünger und sie bevorzugen den Dünger den man aus Leichen gewinnt.
}

\youwhinwhen{
...das Pflanzenspray auf eine Anzahl von Kriechpflanzen in Höhe der Spielerzahl verwendet. Wenn das geschehen ist werden die restlichen Kriechpflanzen sich zurückzieh und ihr seit sicher.
}

\hauntsection{Wie man das Pflanzenspray herstellt}
Ihr müsst Pflanzenspray herstellen um die Kriechpflanzen zu vernichten. Um das machen zu können, müsst ihr das Buch (Book) in das Forschungslabor (Research Laboratory) oder die Küche (Kitchen) bringen. Einmal pro Runde könnt ihr in einem dieser Räume ein Wissenswurf (Knowlege) 5+ machen um das Pflanzenspray herzustellen. Der erfolgreiche Charakter nimmt dann die Spritzpistolenscheibe an sich. Ihr könnt nur ein einziges mal Pflanzenspray herstellen, wenn es zerstört werden sollte könnt ihr kein weiteres Spray herstellen.
\newpage

\specialattackrules{
Besondere Kampfregeln
    \begin{itemize}
        \bitem Ihr könnt eine Kriechpflanze (Wurzel/Spitzen-Paar) automatisch töten, wenn ihr die Spritzpistole in einen Raum mit der Wurzel oder mit der Spitze bringt und dort die Pflanze einsprüht statt sie anzugreifen.
        \bitem Wurzeln können nicht angreifen oder mit normalen Angriffen attackiert werden, nur mit dem Pflanzenspray.
        \bitem Spitzen können normal angegriffen werden. Wenn ihr sie besiegt ist sie betäubt (stuned) und lässt alles Fallen was sie trägt.
        \bitem Wurzeln beeinträchtigen euch nicht in eurer Bewegung, nur die Spitzen tun dies.
    \end{itemize}
}

\hauntsection{Das müsst ihr während eures Zuges machen}

Wenn Ihr von einer Pflanze eingewickelt wurdet, könnt Ihr die Spitze normal angreifen. Ihr nehmt normal Schaden wenn sie euch besiegen solltet. Wenn ihr die Spitze besiegt, dann ist sie betäubt und lässt euch frei. Wenn ihr frei seit, dann könnt ihr euch noch bewegen und den Rest eures Zuges nutzen. Solltet ihr scheitern, endet auch euer Zug.


\outro{
...die Zweig und Äste winden und kratzen über den herumliegenden Müll. Zerdeppern Vasen, schmeißen Bilder den Wänden und ruinieren die übriggebliebende Einrichtung. Für einige Sekunden fühlt es sich an als ob das Efeu das ganz Haus durchrütteln wollten, doch dann ziehen sie sich in den Boden zurück und schrumpfen zur normalen Größe. Das einzige Geräusch, dass ihr noch hört ist ein leises wimmern. Aber wer weint da? Oh, ihr seit das.
}