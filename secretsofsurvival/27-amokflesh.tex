
\survival{27}{Amok Flesh}{Amok laufendes Fleisch}{+}

\introduction{
Dein Freund murmelt etwas von einem „kontaminierten Versuchsclone“ als du ihm die schmucke Kristallkugel zeigst. Hat er es erkannt? Du hältst die Kugel vor deine Augen und siehst in sie hinein. Irgendetwas ist dort in der Mitte der Kugel eingeschlossen. Babyrosa, wie eine Amöbe.
Die Kristallkugel pulsiert in deiner Hand. Verblüfft lässt du sie fallen. Sie bekommt Risse und zerbricht wie ein fallengelassenes Ei. Ein Fleischklumpen (Blob) liegt nun zwischen den ganzen Splittern, wie ein wabbelndes Eiweiß…
…ein Eiweiß das sich bewegt und sich vergrößert, seine Größe alle paar Sekunden verdoppelt! Du fällst fast, als du zurückweichst. Der Blob schwappt vorwärts, als ob er von deinem Sturz profitieren wollte. Wenn du dich nicht gefangen hättest, wäre er über dich hinweg gerollt.
Wenn du keine Möglichkeit findest den Blob zu stoppen, was wird passieren? Wird er nie aufhören zu wachsen?

}

\rightnow{

    \begin{itemize}
        \bitem Der Spieler mit der Kristallkugel-Karte legt diese beiseite. Wer auch immer im gleichen Raum ist sollte schnellstens verschwinden, denn hier wird der Blob zu wachsen beginnen.
        \bitem Lege so viele Wissenswurf-marker bereit, wie Spieler teilnehmen. Lege ebenso viele Gesundheitswurf-marker bereit.
    \end{itemize}
}

\whatyouknowaboutthebadguys{
Der Blob breitet sich aus. Beendest du deinen Zug in einem Raum mit einem Blob-marker, wirst du in eine Blobperson verwandelt (dein neues Spielziel ist es, dem Verräter zu helfen zu gewinnen).

}

\youwinwhen{
du den Blob zerstört hast.

}

\hauntsection{Wie man den Blob zerstört}

\begin{itemize}
    \bitem Einmal während seines Zuges kann der Forscher einen Wissenswurf von 3+ versuchen, um den Blob zu untersuchen. Dafür muss man in einem dem Blob benachbarten Raum sein, der eine Verbindungstür zum Blobraum besitzt. Bei jedem erfolgreichen Wurf lege einen Wissenswurf-marker auf deine Charakterkarte.
    \newpage
    \bitem Den schwachen Punkt des Blob zu finden erfordert so viele erfolgreiche Würfe wie Spieler am Spiel teilnehmen. Sobald der letzte erforderliche Wurf erfolgreich war, lege alle Wissenswurf-marker wieder beiseite.
    \bitem Hast du die Schwäche des Blob gefunden brauchst du nur noch die chemische Formel mit der du ihn töten kannst. Dazu brauchst du genau so viele Zutaten wie Spieler teilnehmen. Jetzt kann man einmal pro Spielrunde einen Wissenswurf von 3+ versuchen um in beliebigen Räumen nach den Zutaten zu suchen. Für jeden erfolgreichen Wurf lege einen Wissenswurf-marker auf deine Charakterkarte und einen Gesundheitswurf-marker in den Raum wo du den Wurf gemacht hast. Im gleichen Raum kann kein weiterer Wurf gemacht werden.
    \bitem Jedes Mal wenn man eine Zutat gefunden hat kann man, indem man auf einen Zugpunkt verzichtet, diese Zutat auf den Blob werfen (aus einem angrenzenden Raum mit Verbindungstür). Dazu legt man den Wissenswurf-marker von seiner Charakterkarte auf den Blob. Hat die Anzahl der Zutaten auf dem Blob die Anzahl Mitspieler erreicht, ist der Blob zerstört.
\end{itemize}

\outro{
Du fasst den Becher fest an, ein herumgewickeltes Paraffinpapier verhindert, dass die grüne Brühe darin verschüttet wird. Du hoffst, dass du den richtigen Enzym fressenden Extrakt hergestellt hast. Es muss einfach richtig sein. Du wirst keine zweite Chance bekommen.
Das grummeln und die Verdauungsgeräusche des Blob dringen aus dem Nachbarraum. Mit einem Stoßgebet wirfst du den Becher auf diese wabbelnde Fleischmasse. Der Blob nimmt den Becher sofort auf.
Das Haus beginnt zu wackeln wie bei einem Erdbeben. Der Blob wehrt sich, stößt kleine Rauchwölkchen aus als er sich zuckend selber verdaut.
Alles was übrig bleibt sind kleine Lachen übel riechender Flüssigkeit, Fetzen von Kleidungsstücke, Knochenreste, ein paar Zähne und Streifen von halb zerfallener Haut.
}