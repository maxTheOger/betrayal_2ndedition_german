
\survival{40}{Buried Alive}{Lebendig begraben}{}

\introduction{
Das Séance-Brett bewegt sich hin und zurück, hin und zurück über die Buchstaben, ohne dass jemand seine Hand zu Hilfe nimmt. Alle starren erschreckt darauf und lesen LEBENDIG BEGRABEN.
Du schaust dich um und dir fällt auf dass du einen deiner Freunde vermisst seit du das Haus betreten hast. Irgendein Zauber muss dich davon abgehalten haben, dich daran zu erinnern. Wenn dein Freund wirklich lebendig begraben wurde, musst du ihn JETZT finden.

}

\rightnow{
Lege so viele dreieckige Stärkewurf-marker (Might Roll) an die Seite wie Spieler teilnehmen.
}

\whatyouknowaboutthebadguys{
Der Verräter hat einen deiner Freunde in einem Raum im Basement begraben (er schreibt diese Information für sich auf). Du weißt nicht welcher Raum der Grabraum ist, aber du weißt, es ist ein Raum der bereits offen
lag als die Verwandlung geschah.

}

\youwinwhen{
du den Eingegrabenen rettest bevor er stirbt. Das Séance-Brett wird dir helfen.
}

\hauntsection{Wie du deinen Freund rettest}
  \begin{itemize}
        \bitem • Jedes Mal wenn du einen Raum betrittst kannst du suchen. Du kannst einen Wissens-Wurf (Knowledge) von 3+ versuchen um den Raum als den Grabraum zu identifizieren. Du kannst einen Raum nur einmal pro Runde durchsuchen, kannst aber mehrere Räume pro Runde durchsuchen.
       Der Verräter muss die Wahrheit sagen ob das der Grabraum ist oder nicht.

        \bitem • Wenn du den Grabraum gefunden hast, kannst du einen Stärke-Wurf (Might) von 4+ versuchen um deinen Freund auszugraben. Um den Freund zu retten brauchen die Spieler so viele erfolgreiche Stärke-Würfe wie Spieler teilnehmen. Jeder Spieler kann einen Versuch machen wenn er an der Reihe ist und sich im Grabraum aufhält. Für jeden erfolgreichen Stärke-Wurf kommt ein Stärkewurf-marker in den Raum.

    \end{itemize}
\newpage
\hauntsection{Das Séance-Brett}

    \begin{itemize}
        \bitem Sobald der Fluch begonnen hat kannst du das Séance-Brett nicht mehr dazu benutzen, dir den obersten Raum des Raum-Stapels anzusehen. Stattdessen benutzt du es, um deinen Freund zu finden.
        \bitem Der Abenteurer mit dem Séance-Brett kann es nicht freiwillig abgeben, tauschen oder fallen lassen. Stirbt dieser Abenteurer bleibt das Brett und alle seine Besitztümer in dem Raum liegen.
       Jeder andere Abenteurer kann das Brett und die anderen Gegenstände aufheben.
        \bitem Der Abenteurer mit dem Brett kann dieses einmal während seines Zuges benutzen, und zwar solange, bis der Grabraum gefunden wurde. Sobald das passiert, lege das Séance-Brett ab.
        \bitem Wenn du das Brett benutzt um den Grabraum zu finden kannst du keine andere Aktion in dieser Runde machen. Versuche einen Gesundheitswurf (Sanitary) um den Raum zu finden:
    \end{itemize}

    \rolls
    \roll{0-2}{Kein Effekt}
    \roll{3-4}{Bewege jeden Helden 3 Felder}
    \roll{5-6}{Heile die begrabene Person.Würfel mit 2 Würfeln und der verräter zieht das Ergebnis von dem Schaden des vergrabenen ab.}
    \roll{7+}{Der Verräter muss den Standort der vergrabenen Person preisgeben}
    \erolls

\outro{
Du gräbst hektisch und befreist letztlich deinen Freund. Die Fingernägel deiner Gefährten sind zerbrochen – vom Kratzen an dem hölzernen Sarg. Blut rinnt über deine Hände. Dein Freund hat keinen Mucks von sich gegeben seit er befreit wurde. Langsam hilfst du deinem Freund das Haus zu verlassen in dem Bewusstsein, dass er wieder so wird wie er vorher war… eines Tages.
}