
\survival{38}{Hellbeasts}{Höllenviecher}{+}

\introduction{
Aus den Augenwinkeln siehst du einen roten Streifen an dir vorbeifliegen. Du drehst dich um, aber er ist verschwunden. Dann siehst du ein ähnlich seltsames Ding von der anderen Seite heran fliegen. Und noch eines. Und noch eines.
Du drehst dich um und siehst eine Fledermaus deren Körper und Flügel aus Feuer zu sein scheinen. Aber die Fledermaus hat keine Schmerzen oder stirbt. Sie umfliegt dich, die Hitze ihrer Flügel versengen dein Haar. Einer deiner Freunde lacht schadenfroh, die anderen schreien vor Schreck. Das kann nicht gut gehen.

}


\whatyouknowaboutthebadguys{
Der Verräter kommandiert einen Schwarm von Feuerfledermäusen (repräsentiert durch Fledermaus-marker (bat)) und möchte euch alle tot sehen. Die Fledermäuse können dich nicht angreifen aber sie verbrennen dich, wenn du mit ihnen im gleichen Raum bleibst.

}

\youwinwhen{
du einen erfolgreichen Exorzismus praktizierst, mit dem du die Fledermäuse aus dem Haus vertreibst.

}

\hauntsection{Wie man den Exorzismus durchführt}
Du musst diesen Exorzismus durchführen, bevor die Fledermäuse euch alle töten. Dazu muss man eine Anzahl erfolgreicher Exorzismus-Würfe machen, und zwar so viele wie Spieler am Spiel teilnehmen. Jeder dieser Würfe verlangt nach einem bestimmten Raum oder Gegenstand, und jedes verlangt einen Gesundheits- oder Wissenswurf.
Du kannst nur einen Exorzismus-Wurf pro Runde machen. Das geht folgendermaßen:

  \begin{itemize}
    \bitem Du kannst einen Gesundheitswurf von 5+ versuchen, während du in der Kapelle, Krypta oder Pentagramm Kammer bist, oder während du das Heilige Symbol oder den Ring trägst.
    \bitem Du kannst einen Wissenswurf von 5+ versuchen, während du in der Bücherei (library) oder dem Forschungslabor (research lab) bist, oder während du das Buch oder die Kristallkugel besitzt.
    \end{itemize}

\newpage Bei jedem erfolgreichen Exorzismuswurf lege einen Gesundheits- oder Wissenswurf-marker in den Raum oder auf die Gegenstandskarte, den/die du für diesen Teil des Exorzismus benutzt hast.
Haben die Helden einen Raum oder Gegenstand für einen erfolgreichen Exorzismus benutzt, so können sie den gleichen Raum oder gleichen Gegenstand nicht noch einmal benutzen.
Haben die Spieler so viele Marker platziert wie Spieler am Spiel teilnehmen, so sind die Fledermäuse verbannt (besiegt).


\specialattackrules{

    \begin{itemize}
        \bitem Feldermäuse können nicht angreifen oder angegriffen werden.
        \bitem Fledermaus-marker beeinflussen deine Bewegung nicht.
        \bitem Der Verräter sagt dir wie viel Schaden du erleidest wenn du in einem Raum mit einer Feldermaus stehen bleibst.
    \end{itemize}
}

\outro{
Die Fledermäuse sind weg, zurückgekehrt in die Hölle, aus der sie geboren wurden. Das Haus glimmt immer noch an einigen Stellen und der Geruch verbrannten Fleisches verknotet deinen Magen. Du stolperst aus dem Haus und schwörst, nie wieder hierher zurück zu kehren. Falls diese Höllenviecher den Weg zurück in diese Welt finden, möchtest du nicht da sein um sie noch einmal zu sehen.
}