
\haunttitle{I Was a Teenage Lycanthrope}{Ich war ein Teenagerwerwolf}
Lycanthrope = Werwolf, Wandlung in einen Werwolf

\introduction{
Ein Schrei zerreißt die Stille in dem herrschaftlichen Anwesen, wird lauter und immer schrecklicher bis du dir sicher bist du wirst auch schreien, wenn das nicht aufhört. Gerade als du dir sicher bist, dass du dich nicht mehr zusammenreißen kannst, wird der Schrei zittriger und steigert sich noch, dann verwandelt er sich in ein Geheul aus blanker Wut. Dein Schatten beginnt zu zittern, als du bemerkst, dass du überflutest wirst vom Licht des Vollmondes.
}

\rightnow{
Lege soviel rote Werwolfscheiben (Wolf tokens) wie es Spieler gibt und eine sechseckige Silberkugelscheibe (Silver Bullets token) beiseite.
}

\whatyouknowaboutthebadguys{
Der Verräter ist ein Werwolf der immer mächtiger und mächtiger wird. Ein Werwolf kann andere mit dem Fluch der Lycanthropy (Verwandlung in einen Werwolf) belegen, so dass diese auch zum Werwölfen werden.

}

\youwhinwhen{
...alle Werwölfe getötet wurden.
}

\hauntsection{Wie können Werwölfe getötet werden}
Du musst den Revolver finden und eine Silberkugel herstellen. Um dies zu schaffen, musst du folgende Schritte erledigen. Pro Zug kann jeder Schritt nur einmal versucht werden.
  \begin{itemize}
        \bitem Wenn du den Revolver noch nicht hast, kannst du ihn im Dachspeicher (Attic), Spielzimmer (Game Room), Rumpelkammer (Junk Room), Haupt-Schlafzimmer (Master Bedroom) oder Gewölbe (Vault) finden. Du kannst einen Wissens-Würfelwurf (Knowledge roll) von 5+ versuchen, um ihn in einem von diesen Räumen zu finden. Wenn du erfolgreich warst, suche in dem Gegenstand (Item)-Stapel nach der Revolverkarte und nimm sie an dich. Danach mische den Stapel neu. Du kannst mehrfach in dem gleichen Raum suchen, aber nicht mehr als einmal pro Zug.
        \bitem Gehe zum Versuchslaboratorium (Research Laboratory) oder zum Heizkeller (Furnace Room). Dort kannst du einen Wissens-Würfelwurf von 5+ versuchen, um eine Silberkugel herzustellen. Ein Entdecker kann daran arbeiten die Kugel herzustellen, während ein anderer den Revolver sucht. (Diese Aufgaben dürfen in beliebiger Reihenfolge erledigt werden.)
        \bitem Der Held der die Silberkugel hergestellt hat, muss sie dem Charakter mit dem Revolver geben (oder umgekehrt).
        \bitem Sobald ein Entdecker eine Silberkugel hat, kann er sie mit dem Revolver benutzen, um den Werwolf oder den Hund zu töten (siehe unten).
    \end{itemize}

\hauntsection{Das musst du während deines Zuges machen}

Wenn du von einem Werwolf oder dem Hund angegriffen wurdest und Schaden erlitten hast, lege eine Werwolfscheibe auf deine Charakterkarte. Zu Beginn jedes deiner folgenden Züge, musst du einen Gesundheits(Sanity)-Würfelwurf von 4+ bestehen, um dem Fluch des Werwolfs zu widerstehen. Schlägt dein Würfelwurf fehl, wirst du zum Werwolf und bist nicht länger ein Held. (Dann musst du diesen Spuk im Buch des Verräters (Traitor's Tome) nachlesen und alles ausführen was unter „Was du jetzt tun musst“ steht.)
Helden die gebissen wurden aber nicht zum Werwolf wurden gewinnen trotzdem wenn die Werwölfe getötet wurden.... Jedenfalls bis zum nächsten Vollmond ...

\specialattackrules{
    Wenn ein Charakter, der die Silberkugeln hat, den Revolver benutzt, um den Werwolf zu töten, dann stirbt dieser. (Dem Revolver gehen nie die Kugeln aus.)
}

\outro{
Wolken jagen über den Vollmond, verdecken sein Licht. Das Haus wird allmählich dunkel und ruhig, als du über dem arg in Mitleidenschaft gezogenen Körper deines toten Freundes stehst. Du musstest es tun, um zu überleben... aber kannst du mit dem Wissen was du getan hast leben?
}