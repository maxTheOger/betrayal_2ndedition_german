
\traitor{31}{Airborne}{Ab in die Luft}{}

\introduction{
Ihr vernehmt seltsame Geräusche, ein Zischen und Pfeifen… Einen Moment später knackt und knarzt es als ob das Haus vom Boden abhebt und  in die Lüfte getragen wird. Aber das wäre ja verrückt… Ihr lauft zu dem Fenster und stellt fest, dass Ihr nicht verrückt seid... Ein Vogel, der die Größe einer 747 hat, trägt das Haus in seinem Schnabel, vermutlich um seine Höllenbrut damit zu füttern. Ihr seht wie sich der Boden weiter und weiter entfernt. Ihr müsst entkommen... Schnell... Aber ihr benötigt etwas, um den Fall zu überleben. Du errinerst dich, dass einer deiner Freunde über ein paar Fallschirme gestolpert ist. Ihr müsst sie nur wiederfinden. Es sind vielleicht nicht genügend für jeden da, aber sicherlich werden deine Freunde verstehen, dass Du in jedem Fall einen benötigst.
}

\Trightnow{

    \begin{itemize}
        \bitem Lege halb so viele (abrunden) sechseckige Gegenstandsmarker zur Seite. Diese stellen die Fallschirme dar.
        \bitem Entferne alle Keller-Teile. Sind Spieler dort, packe sie in den Aufzug und stelle den Aufzug an ein Erdgeschossteil. Wenn der Aufzug noch nicht im Spiel ist, durchsuche den Stapel und mische ihn danach durch.
    \end{itemize}
}

\whatyouknowabouttheheros{
Dieser Fluch hat keine Verräter, nur Helden. Dennoch können nicht alle Überleben….
}

\youwinwhen{
... Du das Haus mit dem Fallschirm verlässt. Spieler die keinen Fallschirm finden, werden getötet.
}

\hauntsection{Wie man einen Fallschirm findet}

  \begin{itemize}
        \bitem Spieler können das Haus noch weiter erforschen, jedoch nicht mehr in den Keller gehen. Ziehst Du eine Keller-Karte, lege sie beiseite und ziehe eine neue Karte, so lange, bis Du eine passende Karte gefunden hast.
        \bitem Einige Fallschirme sind im Haus versteckt. Du kannst sie suchen indem Du in Räumen mit einem Omen-Symbol einen Wissens- oder Geschwindigkeitswurf von 4+ machst. Hast Du Erfolg, leg einen Fallschirm auf deine Charakterkarte. Danach ist Deine Runde beendet.
        In dem Raum kann kein weiterer Fallschirm gefunden werden.
        Du kannst immer nur einen Fallschirm tragen.
    \end{itemize}

\newpage

\specialattackrules{

    \begin{itemize}
        \bitem Du kannst einen Fallschirm von einem anderen Spieler stehlen, wenn Du mit Macht- oder Wissensattacken arbeitest. Du kannst den Fallschirm nehmen, wenn Du mit einem Auge mehr gewinnst. Der Verlierer nimmt sonst keinen Schaden. Die Runde des Angreifenden endet jedoch (bei Erfolg) danach.
        \bitem Du kannst andere Spieler angreifen und ihnen Schaden zufügen anstatt einen Fallschirm zu stehlen. Stirbt der Spieler, lässt er den Fallschirm fallen und jeder andere Spieler kann ihn aufnehmen.

        \bitem Die Spieler bremsen sich gegenseitig (genau wie Monster)
    \end{itemize}
}

\hauntsection{Das Haus verlassen}
Wenn Du einen Fallschirm hast, kannst Du das Haus verlassen. Gehe dazu zur Eingangshalle,
Balcony, Tower, Coal Chute oder dem Collapsed Room. Du benötigst 1 Bewegungspunkt.
Mache dann einen Wissenswurf oder einen Gesundheitswurf über 4+. Bei Erfolg verlässt Du das Haus.

\hauntsection{Wenn du gewinnst}
{\itshape
Die Luft zieht an dir vorbei wie ein Hurricane. Es gibt einen Ruck und dein Fallschirm öffnet sich. Dann hörst Du etwas flattern und schaust hoch. Der Fallschirm hat einige Löcher vom Kampf davongetragen, auch einige Seile sind angerissen. Aber die Löcher werden nicht grösser. Jedenfalls noch nicht…
}

\hauntsection{Wenn du verlierst}
{ \itshape Deine so genannten „Freunde“ haben Dich zurückgelassen damit die Höllenbrut dich fressen kann. Aber vielleicht kannst Du ja auf einem Deiner „Freunde“ landen und überleben?! Das solltest Du vielleicht ausprobieren… }