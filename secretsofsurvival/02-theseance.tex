
\haunttitle{The Séance}{Die Séance}

\introduction{
Eine kalte Briese steigt in dem Haus auf und Nebel bedeckt den Boden Ihr spürt einen Druck auf euer Herz. Eine stimme ertönt: „Ich brauche ruhe … lasst meine Seele ruhen … oder ihr werdet Sterben“
}

\rightnow{
Lege so viele dreieckige \chips{Knowledge roll}{Wissenwurf} sowie \chipse{Sanity roll} neben das Spielfeld wie Spieler teilnehmen. Außerdem einen sechseckigen \chip{Item}{Gegenstand}, der die Leiche darstellt.
}

\whatyouknowaboutthebadguys{
Der Verräter versucht den Geist zuerst zu beschwören.
}

\youwhinwhen{
… ihr entweder den Geist besiegt, nachdem der Verräter ihn unter seine Kontrolle bekommen hat, oder indem ihr die Knochen des Geistes beerdigt, nachdem ihr ihn beschworen habt.
}

\hauntsection{Wie der Geist beschworen wird}
Es findet ein Wettrennen zwischen Euch und dem Verräter statt, den Geist zuerst zu beschwören. Um ihn zu beschwören, musst Du eine Seance abhalten.
  \begin{itemize}
        \bitem Jeder Held in der Pentagramkammer (Pentagram Chamber) kann versuchen, einen \knowroll\ oder einen \sanityroll\ von 5+ zu bestehen. Pro Runde (oder pro Zug) kann man nur einen dieser Würfe durchführen!
        \bitem Sobald ein Wurf erfolgreich war, lege einen \know- oder \sanity-Chip in den Raum, in dem der Spuk offenbart wurde. Sobald in dem Raum halb so viele Chips wie Spieler liegen (abrunden), haben die Helden den Geist beschworen.
        \bitem Wenn die Helden den Geist vor dem Verräter beschwören, kontrollieren sie den Geist. (Folge den Anweisungen im nächsten Abschnitt.) Beschwört der Verräter den Geist zuerst, kontrolliert er den Geist.
    \end{itemize}

\hauntsection{Wenn Du den Geist zuerst beschworen hast}

    Jetzt laut vorlesen: „Beerdige meine Knochen!“

    \begin{itemize}
        \bitem Setze den \chip{Ghost}{Geist} in den Raum, in dem die Séance abgehalten wurde (letzter Würfelwurf). Er bleibt dort, bis die Helden die Kontrolle über ihn verlieren.
        \bitem Stelle die Rundenzählleiste auf 1. Am Ende jedes nun folgenden Zuges des Spielers, der die Séance vollbracht hat, wird der Rundenzähler um 1 erhöht. Ihr habt Zeit bis zum Start von Runde 5, um die Knochen zu beerdigen.
        \bitem Einmal pro Zug kann ein Held auf dem Dachboden, im Schlafzimmer oder im Herrenschlafzimmer (Attic, Bedroom, Master-Bedroom) die Knochen zu suchen. Dazu muss ein Knowledge-Wurf von 5+ bestanden werden. Findet ein Held die Gebeine, legt er den Leichnam-Chip auf seine Charakterkarte.
        \bitem Bringt den Leichnam in die Crypta oder auf den Friedhof (Crypt, Graveyard). Um das passende Grab zu finden, musst du einen \knowroll\ von 5+ bestehen (maximal ein Versuch pro Zug). War der Wurf erfolgreich begräbst du die Knochen.
        \bitem Während ihr dies versuchst, kann der Geist nicht angreifen. Gelingt es euch nicht, die Knochen vor Beginn von Runde 5 zu begraben, erhält der Verräter die Kontrolle über den Geist (entsprechend der Anweisungen im Wälzer des Verräters). Das beerdigen der Knochen reicht dann nicht mehr aus, ihr müsst den Geist jetzt zerstören.
    \end{itemize}

\specialattackrules{

    \begin{itemize}
        \bitem Niemand kann angreifen, solange die Seance nicht vollendet wurde.
        \bitem Solange der Verräter den Geist kontrolliert, kannst Du ihn mit \sanity-Attacken angreifen, allerdings nur wenn Du den Ring besitzt oder in der Pentagramkammer (Pentagram Chamber) stehst. Eine erfolgreiche \sanity-Attacke zerstört den Geist.
        \bitem Wenn der Geist einen Helden angreift und besiegt wird, nimmt er keinen Schaden.
    \end{itemize}
}

\outro{
Der Nebel verschwindet und der Druck auf euerem Herzen lässt langsam nach. Eine warmer schauer durchfährt euch. Ihr habt einer Seele den Frieden gegeben.
}