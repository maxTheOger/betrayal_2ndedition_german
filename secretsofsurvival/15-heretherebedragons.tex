
\survival{15}{Here there be Dragons}{Hier gibt es Drachen}

\introduction{
Einer Deiner Begleiter bückt sich und hebt ein Blatt Papier vom Boden auf. Er murmelt etwas, das Du nicht verstehen kannst. Bevor Du Deinen Freund fragen kannst, was los ist, bricht die Eingangstür auf. Ein riesiger Drache bricht herein, tobend und feuerspuckend Dein Freund schaut mit finsterem Blick und ruft: „Friß sie, Drache, friß sie alle!“.
}


\whatyouknowaboutthebadguys{
Der Drache möchte euch alle töten. Er kann feuerspucken und beißen. Du benötigst Waffen und Rüstung, um ihn zu
töten.
}

\youwhinwhen{
du den Drachen tötest.
}

\hauntsection{Wie Du den Drachen tötest}
Du brauchst keine speziellen Gegenstände, um den Drachen zu töten, aber der Drachen ist so wiederstandsfähig, daß Du die Antike Rüstung (Marker), das Schild (Marker) und den Speer (Karte) besitzen solltest, um eine Erfolgs-Chance zu haben. Das Schild und die Rüstung liegen irgendwo im Keller.

\hauntsection{Die Antike Rüstung:}
Diese liegt im Keller, sie hat nichts mit der Gegenstands-Karte Rüstung/Armor zu tun. Sie kann nicht gestohlen werden. Du kannst nicht beide, die Antike Rüstung und die Rüstung über die Karte, anhaben.
Das Aufheben oder Übergeben an einen anderen Spieler der Antiken Rüstung verbraucht einen ganzen Zug. Du kannst dann nicht mehr tun in dieser Runde.
Wenn Du die Antike-Rüstung trägst erhältst Du beim Angriff immer 5 Punkte Physischen Schaden weniger.
Deine Bewegung reduziert sich um 1 Bewegungspunkt.(Du kannst aber immer pro Runde 1 Raum weiter gehen)
Gegen Feuer und Hitze schützt sie nicht.

\newpage

\hauntsection{Das Schild:}
Wenn Du das Schild besitzt bist Du immun gegen Feuer und Hitze. Deine Bewegung reduziert sich um 1 Bewegungspunkt. Wenn Du die Rüstung UND das Schild trägst redurziert sich Deine Bewegung um 2.
(Wie gehabt, Du kannst immer einen Raum pro Runde gehen)
Jeder andere im gleichen Zimmer wie der Schildträger ist ebenso gegen Feuer und Hitze geschützt.


\hauntsection{Der Speer (Omen-Karte):}
Zusätzlich zum Macht-Bonus der Karte erhältst Du bei einem Macht-Angriff auf den Drachen oder zur Verteidigung gegen einen Macht-Angriff des Drachens einen Bonus von 4 auf den Wurf.



\specialattackrules{
Wenn Du vom Feuer des Drachens getroffen wurdest kannst Du wählen, ob Du einen Gegenstand ablegst und dafür 2 Punkte weniger Physischen-Schaden einstecken mußt. Du kannst das mit mehreren Gegenständen machen, jedesmal reduziert sich der Schaden um 2.
Der Verräter führt Buch auf der Schadensleiste über Deine dem Drachen zugefügten Schäden und gibt bekannt, wann der Drache tot ist.
}

\outro{
Der Drache zittert und liegt dann ganz still. Ein Kräuseln von Rauch steigt aus seiner Nase. Sein Kadaver ist blutig von Deinen Versuchen, ihn zu töten, aber bei weitem nicht so arg, wie das Blutbad ausgesehen hätte, das er über Deine Freunde gebracht hätte. Nun mußt Du mit dem Verräter verhandeln, der wie ein Idiot grinsend daneben steht. Als Du auf ihn zugehst erkennt Dein alter Freund, daß Du Wiedergutmachung möchtest.
„Aber das ist doch nur ein Traum“ protestiert er.
}