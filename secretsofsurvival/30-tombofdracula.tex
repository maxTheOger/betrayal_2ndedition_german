
\survival{30}{Tomb of Dracula}{Draculas Grab}{+}

\introduction{
Eine üble Sicherheit durchströmt deinen Magen als der Sargdeckel sich zu öffnen beginnt. Zu viele Hinweise, zu viele Zufälle, zu viele Entdeckungen als dass du zweifeln könntest dass die bleiche Hand die den Deckel jetzt ganz öffnet nicht Wirklichkeit wäre. Die Hand ist grob und breit, mit viereckigen Fingern. Haare wachsen in der Handinnenfläche. Die Fingernägel sind lang und dünn und enden in einer scharfen Spitze. Das Mädchen das neben dir steht, krallt sich an deinem Arm fest.
„Wir müssen es töten“, schreist du, „töten, bevor es komplett aufwacht.“ Jetzt entblößt das Mädchen ihre Eckzähne und fällt dich an.

}



\whatyouknowaboutthebadguys{
Der Verräter und das Mädchen(Braut) ist ein Vampir, verbündet mit Dracula. Dracula ist sehr mächtig, aber er wacht nur langsam auf. Du hast eine Anzahl von Spielrunden um schnell zu handeln, bevor er ganz aufgewacht ist. Er und seine Kreaturen werden versuchen, dich zu töten… oder dich auch zu einem Vampir zu machen.

}

\youwinwhen{
Dracula und die Braut zerstört worden sind.
}

\hauntsection{Wie die Vampire zerstört werden}


\begin{itemize}
    \bitem Wenn du eine Speer-Karte (spear) benutzt und einen Vampir mit einem Stärke-Angriff (might) besiegst, haust du einen Pflock in sein Herz und tötest ihn. Andere Angriffe fügen dem Verräter schaden zu oder betäuben Dracula bzw. die Braut.
    \newpage
    \bitem Der Verräter hält die Zeit fest die vergangen ist seitdem der Verrat begonnen hat. Sofort nachdem der Verräter den Marker auf der Turn/Damage-Karte weiter geschoben hat, würfelt einer der Forscher mit so vielen Würfeln wie Spieler teilnehmen. Ist das Ergebnis niedriger als die augenblickliche Rundenzahl, beginnt die Sonne aufzugehen. .
    \bitem Vampire werden schwächer wenn der Tag erwacht. Zu Beginn jeder Spielrunde des Verräters nachdem die Sonne aufzugehen beginnt, verliert jeder Vampir 1 Punkt von allen 4 Eigenschaften (erinnere den Verräter daran, falls nötig). Fällt die Eigenschaft eines Vampirs auf das Totenkopf-Symbol wird er zerstört.
\end{itemize}

\specialattackrules{
Besiegst du einen Vampir, fügst du ihm ganz normal Schaden zu. Wenn du auch die Karte mit dem Heiligen Symbol hast, kannst du den Vampir zwingen, einen Raum von dir weg zu ziehen (durch eine Tür), und zwar um einen Raum für jeden Schadenspunkt den du erwürfelt hast.

}

\outro{
Ein Pflock durchs Herz, das Sonnenlicht – das waren deine Waffen gegen die Blut saugende Brut und seine nachtaktiven Kinder. Der Tag ist angebrochen. Die Vampire sind vernichtet worden. Die Legende von Dracula bleibt was sie ist, eine Legende.
Sie sind wirklich weg, denkst du, als du eine Wunde an deinem Hals zu reiben beginnst. Lasse besser einen Blick darauf werfen, nur für alle Fälle.
}