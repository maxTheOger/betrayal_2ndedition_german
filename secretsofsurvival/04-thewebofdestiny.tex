
\haunttitle{The Web of Destiny}{Das Netz des Schicksals}

\introduction{
Das Netz war so riesig, dass dein Verstand es einfach nicht sehen wollte. Nun bist du darin gefangen und dein Gesicht und dein Körper sind in die klebrigen Stränge gewickelt. Schon beginnen die Fäden immer härter zu werden. Wenn du da nicht schnell raus kommst wirst du wohl niemals mehr entkommen. Am Rand deines Sichtfeldes siehst du einen Schatten wie er sich von der Decke hinunter lässt. Nein, kein Schatten. Eine Spinne gleitet über das Netz. Sie bäumt sich über dich auf und spürst plötzlich wie dein Bauch Feuer fängt. Als du runter siehst, erblickst du eine großen Stachel, der in deinem Bauch steckt. Du schreist...aber wird dich auch wer hören?
}

\rightnow{

    \begin{itemize}
        \bitem Derjenige, der den Spuk aufgedeckt hat, ist in einem klebrigen Netz gefangen. Der gefangene Charakter kann sich nicht bewegen aber er kann versuchen sich aus dem Netz zu befreien in dem er es angreift. Außerdem kann der Gefangene seine Gegenstände benutzen oder tauschen.
        \bitem Falls das Medizinköfferchen (Medical Kit) noch nicht von einem Spieler aufgedeckt wurde kann einer der Helden den Gegenstandsstapel (Item-Deck) danach durchsuchen und an sich nehmen. Danach wird der Stapel neu gemicht.
        \bitem Lege so viele dreieckige Kraftwurfscheiben (Mightroll-Tokes) beiseite wie es Spieler gibt.
        \bitem Der gefangene Charakter wurde mit Spinneneiern infiziert, vielleicht werden sie Schlüpfen?
    \end{itemize}
}

\whatyouknowaboutthebadguys{
Eine schreckliche Riesenspinne ist erwacht. Sie will den Gefangenen beschützen bis ihre Eier schlüpfen.
}

\youwhinwhen{
...der Gefangene befreit wurde, die Eier zerstört und mindestens einer das Haus lebend verlässt.
}

\hauntsection{So zerstört man das Netz und die Eier}

Solange die eier noch nicht zerstört wurden kann kein Attribut des gefangenen Spielers auf 0 gehen.

  \begin{itemize}
        \bitem Du kannst das Netz zerstören wenn du Kraftangriffe ausführst. Das Netz verteidigt sich mit einer Kraft (Might) von 4. Wenn du gewinnst legst du eine Kraftwurfscheibe in den Raum, statt Schaden zu verursachen. Du nimmst keinen Schaden wenn das Netz dich besiegt. Wenn sich im Raum genauso viele Kraftwurfscheiben befinden wie Spieler im Spiel, dann ist das Netz zerstört worden. Der gefangene Charakter ist frei.
        \bitem Wenn jemand mit dem Medizinköfferchen im selben Raum ist wie der infizierte Charakter, dann kann dieser ein Wissenswurf (Knowlege) von 4+ durchführen um die Eier zu zerstören. Wenn derjenige zusätzlich die Heilsalbe (Healing Salve) besitzt können die Eier ohne würfeln zerstört werden.
    \end{itemize}


\hauntsection{So entkommt man dem Haus}

Nachdem der gefangene Charakter befreit wurde und die Eier vernichtet, können die Helden das Haus verlassen. Ihr könnt versuchen einen Wissenswurf (Schloss knacken) oder einen
Kraftwurf (Tür aufbrechen) von 6+ zu machen. Wenn einer es schafft zieht dieser eine Ereigniskarte (Event), führt die Anweisungen durch und beendet den Zug. Erst im nächsten Zug könnt ihr das Haus verlassen.
Jeder Spieler kann danach die Eingangshalle mit einem Zug verlassen.



\outro{
Ihr bürstest die Spinnenweben aus euren Augen und ihr stolpert aus dem antiken Anwesen.
Als ihr zurück blickt bemerkt ihr ein flackerndes Licht aus dem Fenster über euch. Im schwachen Licht könnt ihr eine Bewegung ausmachen, dann noch eine. Zeit zu gehen. Jetzt!
}