
\survival{36}{Better with Friends}{Besser mit Freunden}{+}

\introduction{
Das Medaillon glüht schwarz-blau und pulsiert regelmäßig in der Dunkelheit. Du fühlst wie das Haus sich bewegt, als Wasser ins Basement eindringt. Hat nicht jemand von einem Boot gesprochen das in der Dachkammer (attic) aufbewahrt wird? Ihr rennt alle schnell die Treppe hinauf, nun, alle außer dem Verräter, der euch hierher gebracht hat. Euer verräterischer Kumpel scheint es nicht eilig zu haben zu entkommen.
Das Haus sinkt! Du hast keine Zeit darüber nachzudenken warum, aber vielleicht hast du genug Zeit zu entkommen.

}


\whatyouknowaboutthebadguys{
Der Verräter hat euch hierher gebracht damit ihr sterbt. Das Haus versinkt in einem unterirdischen Sumpf und du wirst ertrinken wenn du nicht entkommen kannst.

}

\youwinwhen{
mindestens die hälfte das Haus leben verlassen konnten(aufrunden). Du kannst keinen lebenden Held im Haus zurücklassen wenn du entkommst.

}

\hauntsection{Wie man aus dem Haus entkommen kann}
  \begin{itemize}
    \bitem Ist die Dachkammer/Attic nicht entdeckt erforsche das Obergeschoss so lange bis Du sie findest.
    \bitem Das Ruderboot ist in der Dachkammer. Trage es von dort auf den Balkon oder in den Turm (lege den Ruderboot-marker (rowboat) auf deine Charakterkarte, solange du es trägst). Das Ruderboot ist sehr schwer, so dass, während du es trägst, jeder Raum zwei Schritte benötigt (nur ein Forscher kann das Boot tragen, aber es ist ein Gegenstand mit dem man handeln kann).
    \bitem Sobald alle lebenden Helden auf dem Balkon oder im Turm sind (mit dem Ruderboot) könnt ihr entkommen. Ihr könnt nicht entkommen solange ein lebender Forscher noch in einem anderen Raum ist.
    \end{itemize}
\newpage

\hauntsection{Auswirkungen der Flut}
Wenn jemand (einschließlich dem Verräter) seinen Zug in einem überfluteten Stockwerk des Hauses beginnt, treten folgende Effekte ein:

  \begin{itemize}
    \bitem \textbf{Teilweise überflutet:} Du ziehst in dieser Spielrunde 2 Schritte weniger.
    \bitem \textbf{Vollständig überflutet:} Du ziehst in dieser Spielrunde 4 Schritte weniger und erleidest 2 Punkte physischen Schaden. Der Schaden kann nicht verhindert werden.
    \bitem Egal wie weit das Haus überflutet ist, du kannst immer wenigstens 1 Raum weit ziehen.

    \end{itemize}

\hauntsection{Eindämmen der Flut}
Der Verräter benutzt die Turn/Damage-Karte um die vergangene Zeit zu messen. Während du an der Reihe bist kannst du das Medaillon in einem teilweise oder vollständig überfluteten Raum fallen lassen und so das weitere Einsinken des Hauses für 1 Spielrunde stoppen. Tust du das, wirf die entsprechende Karte ab. Während des nächsten Spielzuges des Verräters schreitet die Zeit nicht vorwärts (pass auf, dass der Verräter sich danach richtet).


\outro{
Ihr paddelt mit aller Kraft und das Boot entfernt sich vom versinkenden Haus. Du hörst deinen Freund wieder und wieder rufen: „Komm zurück! Komm zurück! Der Tod ist besser zusammen mit Freunden! Teile ihn mit mir!“ Hmmm. Das ist die Sorte Gastfreundschaft auf die du verzichten kannst.
}