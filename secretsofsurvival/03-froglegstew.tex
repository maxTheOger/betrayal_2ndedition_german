
\survival{3}{Frog-Leg Stew}{Froschschenkeleintopf}

\introduction{
Ein krächzendes, rasselndes lachen schallt durch das Haus. Ihr brecht in kalten Schweiß aus.
„Nein, nein, ihr müsst euch nicht vor mir verstecken, meine kleine unartigen Äffchen! Ihr wart sehr böse ihr kleinen Kröten, klaut einfach Omas Buch. Sehr böse. Ich befürchte Oma muss euch nun eins auf eure Nasen geben....oder etwas schlimmeres anstellen, viel schlimmeres.“
}


\whatyouknowaboutthebadguys{
Die Hexe hat einen Zauber gesprochen, der sie unverwundbar macht. Außerdem kann sie Menschen in Frösche verwandeln.
}

\youwhinwhen{
...wenn ihr die Hexe tötet.
}

\hauntsection{So tötet man die Hexe}

Ihr müsst das Hexenbuch (die Book-Karte) benutzen um den Zauber Sterbliche Hülle auf die Hexe zu sprechen. Dies wird sie verwundbar gegenüber Attacken machen. Dieser Zauber benötigt allerdings Alraunenwurzel (Root-Token).
Um nun die Hexe zu töten müsst ihr diese Anweisungen in der richtigen Reihenfolge durchführen. Ihr könnt immer nur einen Schritt der Anweisung pro Runde ausführen.

  \begin{itemize}
        \bitem Findet eine Alraunenwurzel. Falls ihr einen Raum aufdeckt, der eine Alraunenwurzel
beinhaltet, dann wird der Verräter eine dort platzieren. Einige können aber auch schon
in bereits aufgedeckten Räumen liegen.
        \bitem Wenn einer von euch in einem Raum mit einer Alraunenwurzel ist, dann kann dieser einen Wissenswurf (Knwolege) von 4+ machen um eine Wurzel auszugraben. Schafft
er es, legt er die Wurzel auf seine Charakterkarte.
        \bitem Wenn einer von euch eine Alraunenwurzel und das Zauberbuch besitzt, während er im
selben Raum wie die Hexe ist, kann dieser einen Wissenswurf von 6+ versuchen um den Zauber Sterbliche Hülle zu sprechen. Wenn er erfolgreich war kann die Hexe normal angegriffen werden.
    \end{itemize}
\newpage
\hauntsection{Frösche}

  \begin{itemize}
        \bitem Ein Abenteurer, der in ein Frosch verwandelt wurde muss all seine Gegenstände auf den Boden legen. Dann muss er die Kraft (Might) und sein Wissen auf seinen niedrigsten Wert einstellen (Nicht auf das Totenkopfsymbol). Ein Frosch kann keine Angriffe durchführen, Karten ziehen oder neue Räume aufdecken. Ein anderer Charakter, der kein Frosch ist, kann den Frosch aufsammeln und wie ein Gegenstand mit sich herum führ
        \bitem Ist einer von euch im selben Raum wie ein Frosch und er hat das Zauberbuch bei sich, kann dieser einen Wissenswurf 4+ versuchen um den Frosch wieder zurückzuverwandeln. Dabei werden die Startattribute wiederhergestellt.
    \end{itemize}

\specialattackrules{

    \begin{itemize}
        \bitem  Die Hexe ist unangreifbar solange nicht der Sterbliche Hülle –Zauber auf sie gesprochen wurde
        \bitem Wenn die Katze im gleichen Raum ist kann man sie normal Angreifen.
    \end{itemize}
}

\outro{
 Die Hexe kreischt: „Neeeeeeein! Ihr könnt das nicht tun! Mein süßes Fleisch aufhalten. Das werdet ihr büßen. Ich werde in eure Albträume kriechen und euch zum bluten bringen! Euer Gehirn wird Jucken bis ihr ein Loch in eure Schädeldecke gekratzt habt nur um mich hinaus zu lassen. Ich werde...“
Gerade als du bereit warst ihr deine Lampe über den Schädel zu hauen um endlich ihre Reibeisenstimme für immer zum schweigen zu bringen ist sie verschwunden...für dieses mal.
}