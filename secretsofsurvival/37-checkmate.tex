
\survival{37}{Checkmate}{Schachmatt}{+}

\introduction{
Endlich hast du herausgefunden was die krakelige Handschrift in dem alten Buch das du gefunden hast bedeutet: „Ich, Ebenezer Slocum, habe Mittel und Wege gefunden, den Tod leibhaftig vor mir erscheinen zu lassen. Ich bin bereit, den Tod herauszufordern, und ich werde ihn besiegen! Durch Studien habe ich meinen Geist aufs Höchste geschärft. Oh, der Tod wird heute Nacht nicht triumphieren!“
Du schaust dich um und siehst auf einem Schachbrett eine umgefallene Figur. Als du sie berührst zerfällt sie zu Staub. Es sieht so aus als ob Ebenezer doch nicht so bereit war wie er sagte.
Gegenüber vom Tisch siehst du eine dunkle Schattengestalt erscheinen. Die Gestalt geht auf einen aus eurer Gruppe zu und zeigt auf den Tisch. Ein Schachspiel aus Ebenholz und Elfenbein steht zwischen beiden. Du hoffst dass du besser bist als Ebenezer war.

}


\whatyouknowaboutthebadguys{
Der Tod fordert dich zu einem Schachspiel heraus. Ist zu Beginn des Spielzuges des Todes niemand mit ihm im gleichen Raum, habt ihr das Schachspiel und das Spiel verloren.

}

\youwinwhen{
du den Tod schachmatt setzt, indem du bei eines Wissenswurf eine höhere Zahl würfelst als er. Einmal während des Spielzuges des Todes kann ein Held, der sich im gleichen Raum befindet, diesen Wurf versuchen.

}
\newpage

\hauntsection{Wie man den Tod besiegt}
Einige Gegenstände im Haus helfen dir, eine höhere Zahl in einem Wissenswurf zu erreichen als der Tod:

  \begin{itemize}
        \bitem Die Forscher können Heilige Siegel (holy seals) aufheben. Tut ein Spieler das, kann er einen Gesundheitswurf von 4+ versuchen, um das Siegel zu zerbrechen. Man kann nur ein Siegel während seines Spielzuges zerbrechen. Jedes Mal, wenn ein Forscher ein Siegel zerbricht, würfelt der Tod beim nächsten Wissenswurf mit einem Würfel weniger. Sind nur 3 oder 4 Spieler im Spiel, wirft er sogar jeweils 2 Würfel weniger.
        \bitem Das Buch enthält Schach-Strategien. Der Forscher der es besitzt kann 1 Würfel mehr bei seinem Wissenswurf werfen (maximal 8) wenn er Schach gegen den Tod spielt.
    \end{itemize}

\specialattackrules{

    \begin{itemize}
        \bitem Der Tod kann nur im Schachspiel angegriffen oder sonst irgendwie beeinflusst werden.

        \bitem Der Tod verlangsamt die Spielerbewegungen nicht.

    \end{itemize}
}

\outro{
„Schachmatt.“
Der Tod starrt auf seinen König, dann beginnt er zu Staub zu zerfallen. Der Tod lächelt und du fühlst wie dein Haar weiß wird. „Bis zum nächsten Mal,“ sagt der Tod…
}