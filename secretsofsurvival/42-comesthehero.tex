
\survival{42}{Comes the Hero}{Es kommt der Held}{}

\introduction{
Eine Statue steht vor dir, die Hand ausgestreckt, als ob sie erwartete, dass du etwas hineinlegst. Eine Botschaft ist in das Podium eingraviert auf dem die Statue steht: „ Den Unbesiegbaren zu besiegen.“
Was könnte das heißen? Und warum zitterst du so unkontrollierbar? Irgend etwas ist gerade besonders schief gegangen und diese Statue ist deine einzige Möglichkeit, dass sich die Dinge nicht von schlecht nach schlechter entwickeln.

}

\rightnow{
Nimm den Statue-marker und lege ihn in den Raum wo der Verrat begann.
}

\whatyouknowaboutthebadguys{
Ein unbesiegbarer Verräter benutzt ein Ritual mit dem er ein Tor zur Hölle öffnen will. Du weißt nicht genau, wie das funktioniert, aber du vermutest, es geht nur über ein Menschenopfer.

}

\youwinwhen{
der Verräter tot ist.
}

\hauntsection{Wie man den Verräter tötet}


Der Verräter kann nicht angegriffen oder durch normale Handlungen verletzt werden. Das heißt, du musst unkonventionelle Maßnahmen anwenden um ihn zu stoppen.


Die Hand der Statue ist ausgestreckt und eine Plakette an ihrem Podium sagt: „Den Unbesiegbaren zu besiegen.“ Die Statue ist unbeweglich, bis ein bestimmter Gegenstand während deines Zuges in ihre Hand gelegt wird. Wenn das passiert, wird der Gegenstand abgelegt und die Statue wird lebendig:

    \begin{itemize}
        \bitem Legt man das Heilige Symbol (holy symbol) in ihre Hand wird die Statue ein Mächtiger Richter.
        \bitem Legt man die Axt in ihre Hand, wird sie ein Mächtiger Krieger.
        \bitem Legt man die Kristallkugel in ihre Hand, wird sie ein Mächtiger Seher.
        \bitem Legt man das Buch in ihre Hand, wird sie ein Mächtiger Zauberer.
    \end{itemize}

\newpage
\monster{Lebende Statue}{8}{8}{8}{8}

Ist die Statue einmal lebendig geworden, bleibt sie für den Rest des Spieles in diesem Zustand. Sage dem Verräter was sie geworden ist.

  \begin{itemize}
        \bitem Die Statue bewegt sich nicht normal. Sie wird Durch die Mentalen Fähigkeiten der Helden aktiviert. Einmal während der Runde eines Helden der im gleichen Raum sein muss, bewegt sich die Statue wenn er einen Wissens- oder Gesundheits-wurf ausführt.
      Der Held kann die Statue dann die Anzahl der Augen weiterbewegen.
        \bitem Ist die Statue im gleichen Raum wie der Verräter, greift sie diesen nicht an. Stattdessen sinkt eine der Fähigkeiten des Verräters um 1 Punkt. Der Richter senkt die Geschwindigkeit, der Krieger die Stärke, der Seher die Gesundheit und der Zauberer das Wissen.
        \bitem Wenn der Verräter die Statue angreift und besiegt, ist diese nicht bewusstlos. Stattdessen kann die Statue in der nächsten Runde keine Fähigkeit des Verräters um 1 reduzieren. Sie kann der Verräter aber immer noch folgen.
    \end{itemize}

\outro{
Die Statue kämpfte, als kämpfte sie nicht nur für dich sondern für die ganze Welt. Der Verräter liegt geschlagen am Boden und du stehst da, erstarrt vor Ehrfurcht vor den Fähigkeiten der Statue. Die Statue verwandelt sich langsam in Stein zurück. Jetzt sitzt sie da, die Faust nachdenklich unter das Kinn gestellt. Keine Worte erscheinen unten an ihrem Podium. Alles ist ruhig.
}