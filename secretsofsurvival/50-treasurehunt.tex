
\survival{50}{Treasure Hunt}{Die Schatzsuche}{}

\introduction{

Du hast schon viele Geschichten von dem versteckten Vermögen in diesem Haus gehört.Gerade mitten im Pentagramm hast Du ein eingerahmtes Dokument entdeckt:
„Ich Josiah Enders im vollbesitz meiner geistigen Kräfte hinterlasse hiermit mein gesamtes Vermögen dem der clever genug ist es zu finden.Die Hinweise sind überall über das Haus verstreut.Viel Glück und sei Vorsichtig!“
Du würdest alles tun um an diesen Schatz zu kommen!

}

\rightnow{
Dieser Fluch hat keinen Verräter, nur Helden die alle den Schatz suchen, aber es kann nur einen gewinner Geben.

Der Fluch-Auslöser nimmt 20 rote Monster-Marker, nummeriert von 1-20, dreht sie auf den Kopf und mischt sie. Danach werden die Marker mit Nummer nach unten in jedem Raum mit Omen, Event und Gegenstands Symbolen platziert.
Diese Marker sind die Hinweise zum Schatz.


Sind weniger Räume da als Marker lege die Marker beiseite und warte bis neue Räume entdeckt wurden. Warte jedoch ab bis die entsprechende Event/Omen/Gegenstandskarte gezogen und erfüllt wurde.
Sind mehr Räume da als Marker verteilt man die Marker so gleichmässig wie möglich zwischen den Etagen.

}

\youwinwhen{
Du den Schatz entdeckt hast. Nur einer kann gewinnen!
}

\hauntsection{Nach dem Schatz suchen:}

(fehlender Abschnitt?)
\newpage
\rolls
\roll{0}{Ouch ! Du hast Dich verletzt. Würfel 2 würfel physischen Schaden.}
\roll{1-3}{Nichts ist passiert, versuchs nochmal.}
\roll{4-12}{Nimm den Hinweis Marker und Schau in die zweite Seite}
\roll{13-15}{Das Haus beginnt einzustürzen. Falls dies das erste mal ist drehe ein Hausteil auf dem obersten Stock um. Jedes weitere mal ein Raum neben dem letzten.Wenn ein Raum mit Spielern einstürzt werden die Spieler getötet.Sobald die obere Etage komplett eingestürzt ist mache mit dem Erdgeschoss (Grand Staircase) weiter.Benutze sechseckige Marker um Grand Staircase,Foyer und Entrance Hall als Eingestürzt zu markieren.Sobald das gesamte Erdgeschoss eingestürzt ist, stirbt jeder im Keller.(Sollte der Fahrstuhl in einer Etage halten wo es keine nicht eingestürzten Räume mehr gibt, bewegt er sich nicht.)}
\roll{16+}{Du hast den Schatz gefunden!}
\erolls

\hauntsection{Hinweise:}
\begin{itemize}
        \bitem Die meisten Hinweise geben Dir einen Anhaltspunkt wo Du suchen musst.
        Immer wenn Du ein Hinweis Marker nimmst, schau in die Tabelle und sieh nach ob sie dir einen Bonus für die nächste Suche bietet. Einige sind Riskant und können Dich töten.Einige sind auch Fallen welche Dir keine Bonis geben sondern Dir nur Schaden zufügen.
    \bitem Du kannst in einem Raum weiter nach dem Schatz suchen auch wenn der Hinweis schon entfernt wurde.
    \bitem Die Spieler können die Hinweise Tauschen.
    \bitem Wenn Du ein Hinweis eingetauscht hast, kannst Du seine Boni´s nicht mehr nutzen bis Du sie wieder zurückgetauscht hast.
\end{itemize}

\vspace{2cm}

Dieses Szenario geht auf der nächsten Seite weiter …

\newpage

\begin{tabular}{rp{7cm}}
Hinweis & Ergebnis \\
{1} &
Eine Karte : +7 Wissen wenn Du die Catabombs durchsuchst \\
{2} &
Ein Weinflaschen Label : +7 Wissen wenn Du den Weinkeller durchsuchst \\
{3} &
Ein Lesezeichen : +7 Wissen wenn Du die Bibliothek(Library) durchsuchst \\
{4} &
Ein Zettel : +7 Wissen wenn Du das Esszimmer(Dining Room) durchsuchst \\
{5} &
Eine Karte mit Wasserflecken : +6 Wissen wenn Du den Wintergarten(Consevatory) oder den Unterirdischen Fluss(Underground Lake) durchsuchst \\
{6} &
Ein Biblisches Zitat : +6 Wissen wenn Du die Krypta(Crypt) oder Kapelle(Chapel) durchsuchst \\
{7} &
Ein Bild eines Geheimfachs : +5 Wissen wenn Du die Dachkammer(Attic), Küche(Kitchen), Lagerraum(Storeroom) oder die Speisekammer(Larder) durchsuchst \\
{8} &
Eine verschlüsselte Nachricht : +4 Wissen wenn Du irgendeinen Kellerraum durchsuchst \\
{9} &
Ein Plan des Hauses mit einem X in jedem Raum. +3 Wissen bei jeder suche \\
{10} &
Eine wissenschaftliche Formel : Addiere Dein Wissen zu dem Wurf wenn Du eines der Labore durchsuchst. \\
{11} &
Ein Musikstück : Addiere Dein Wissen zu dem Wurf wenn Du den Orgelraum(Organ Room) durchsuchst. \\
{12} &
Eine Reihe von Schachzügen : Addiere Dein Wissen zu dem Wurf wenn Du das Spielzimmer(Game Room) durchsuchst. \\
{13} &
Einige der Statuen sind Hohl, wenn Du sie bewegen kannst. Addiere Deine Stärke(Might) zu dem Wurf wenn du den Statuen Raum(Statue Corridor) durchsuchst. \\
\end{tabular}


\specialattackrules{

    \begin{itemize}
        \bitem Die Spieler können sich gegenseitig angreifen.

        \bitem Schaden im Kampf können nicht dazu führen das irgendeine Eigenschaft auf das Totenkopf-Symbol abfällt.Dennoch können Fallen oder andere Quellen tötlich sein.

    \end{itemize}
}

\newpage
\begin{tabular}{rp{7cm}}

{14} &
Ihr werdet eure Hand ins Feuer legen … Addiere Deine Gesundheit(Sanity) zu dem Wurf wenn Du den Heizungskeller(Furnace Room) durchsuchst. \\
{15} &
Hast Du schon mal dran gedacht draussen Nachzusehen? +6 Wissen wenn Du ausserhalb suchst.(Tower, Balcony oder jeden Raum mit Aussenfenster)Achtung: Würfelst Du weniger als 13 fällst Du raus und stirbst ! \\
{16} &
Wie gut bist Du beim Klettern? Addiere Dein Kraft( might) oder Gesundheitswert(sanity) zu Deinem Wurf wenn Du den Abgrund durchsuchst.Achtung: Würfelst Du weniger als 13 stürzt du hinein und stirbst ! \\
{17} &
Falle ! Ein Giftpfeil hat Dich getroffen. Von jetzt an nimmst Du 1 Schadenspunkt zu einem Wert Deiner Wahl vor dem Beginn Deiner Runde. \\
{18} &
Falle ! Die Decke stürzt ein. Du musst einen Kraft(Might) oder Geschwindigkeitswurf(speed) von 4+ machen um zu entkommen. Verlierst Du würfel mit 3 Wüfeln und nimm das Ergebnis als physischen Schaden. \\
{19} &
Falle ! Der Raum füllt sich mit Giftgas Jeder in dem Stockwerk (auch Du) muss einen Kraftwurf(Might) von 3+ versuchen. Jeder mit weniger würfelt mit 3 Würfeln und zieht das Ergebnis von einem oder mehreren seiner Werte ab. \\
{20} &
Du findest eine Puzze Box. Eine Seite hat einen Zettel, die andere eine Giftspinne. Du kannst entweder versuchen das Rätsel zu lösen oder das Item hier lassen. Würfelst Du 5+ nimm zwei unentdeckte Hinweise aus dem Haus. Hast Du weniger würfel mit 4 Würfeln und zieh das Ergebnis von einem oder mehreren Deiner Werte ab. \\

\end{tabular}



\outro{
Während Du auf den Bahamas in der Sonne liegst und einen Daiquiri schlürfst fühlst Du dich manchmal etwas schlecht wenn Du daran denkst was Du alles getan hast um hier herzukommen.... Aber nicht oft !
}