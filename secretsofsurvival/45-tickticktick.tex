
\survival{45}{Tick, tick, tick}{Tick, tick, tick}{+}

\introduction{
Zuerst hast du das Ticken gar nicht bemerkt. Dann, in einem Moment absoluter Stille, hörst du es ganz genau: tick, tick, tick, wie ein makabrer, mechanischer Herzschlag. Als du den Verrückten kichern hörst bemerkst du dass das Ticken von DIR ausgeht. Der Verräter hat dir eine Bombe umgebunden!

}

\rightnow{
Jeder bekommt ein sechseckigen Gegenstands-Marker auf seine Charakterkarte.(Die Bombe)

}

\whatyouknowaboutthebadguys{
Der Verräter ist ein Zerstörer, der jedem von euch eine Zeitbombe umgebunden hat. Du weißt nicht, wie viel Zeit dir bleibt. Der Verräter hat auch einen Zünder, der die Bombe explodieren lassen kann, aber nur wenn du in einem Raum bist der dem benachbart ist, in dem sich der Verräter aufhält (die aneinander grenzenden Räume brauchen keine gemeinsame Tür zu haben).
Als ob das nicht schon schlimm genug wäre arbeitet der Verräter auch noch an seiner großen Bombe. Du musst den Zerstörer stoppen, bevor er euch alle tötet.

}

\youwinwhen{
der Verräter tot ist und mindestens ein Held überlebt hat.
}
\newpage
\hauntsection{Wie man den Verräter stoppen kann}
Mindestens einer von euch muss seine Bombe entschärfen, sodass er den Verräter töten kann, bevor dieser die Große Bombe fertig stellen kann.


    \begin{itemize}
        \bitem Einmal während deines Zuges kannst du einen Wissenswurf (knowledge) von 7+ versuchen um die Bombe, die an dir befestigt wurde, zu entschärfen. Hast du die Karten „Verrückter“ (Madman) brauchst du einen Wissenswurf von 5+ (du erkennst das verrückte Genie in der Konstruktion der Bombe). Ergibt der Wissenswurf 2 oder weniger zündet deine Bombe und aller Forscher im gleichen Raum wie du sterben mit dir. Gegenstände und Omen Karten kommen aus dem Spiel.
        \bitem  Anstatt zu versuchen, die eigene Bombe zu entschärfen, kann man dies mit der Bombe eines Mitspielers versuchen (falls er das erlaubt). Der Wissenswurf von 7+, bzw. 5+ und ein Ergebnis von 2 und weniger haben die gleichen Auswirkungen wie vorher beschrieben.

        \bitem  Der Verräter arbeitet solange an der Großen Bombe, bis du ihn tötest.
    \end{itemize}

\outro{
Die Bauteile der Großen Bombe des Zerstörers liegen verstreut umher. Ohne das verrückte Gehirn des Verräters der sie zusammenbauen kann sind sie nur unwichtige Fragmente ohne irgendwelche Bedeutung.
}