%%%%%%%%%%%%%%%%%%%%%%%%%%%%%%%%%%%%%%%%%
% Frequently Asked Questions
% LaTeX Template
% Version 1.0 (22/7/13)
%
% This template has been downloaded from:
% http://www.LaTeXTemplates.com
%
% Original author:
% Adam Glesser (adamglesser@gmail.com)
%
% License:
% CC BY-NC-SA 3.0 (http://creativecommons.org/licenses/by-nc-sa/3.0/)
%
%%%%%%%%%%%%%%%%%%%%%%%%%%%%%%%%%%%%%%%%%

\documentclass[11pt]{article}

\usepackage[margin=1in]{geometry} % Required to make the margins smaller to fit more content on each page
\usepackage[linkcolor=blue]{hyperref} % Required to create hyperlinks to questions from elsewhere in the document
\hypersetup{pdfborder={0 0 0}, colorlinks=true, urlcolor=blue} % Specify a color for hyperlinks
\usepackage{todonotes} % Required for the boxes that questions appear in
\usepackage{tocloft} % Required to give customize the table of contents to display questions
\usepackage{microtype} % Slightly tweak font spacing for aesthetics
\usepackage{palatino} % Use the Palatino font
\usepackage[utf8]{inputenc}

\setlength\parindent{0pt} % Removes all indentation from paragraphs

% Create and define the list of questions
\newlistof{questions}{faq}{\large List of Frequently Asked Questions} % This creates a new table of contents-like environment that will output a file with extension .faq
\setlength\cftbeforefaqtitleskip{4em} % Adjusts the vertical space between the title and subtitle
\setlength\cftafterfaqtitleskip{1em} % Adjusts the vertical space between the subtitle and the first question
\setlength\cftparskip{.3em} % Adjusts the vertical space between questions in the list of questions

% Create the command used for questions
\newcommand{\question}[1] % This is what you will use to create a new question
{
\refstepcounter{questions} % Increases the questions counter, this can be referenced anywhere with \thequestions
\par\noindent % Creates a new unindented paragraph
\phantomsection % Needed for hyperref compatibility with the \addcontensline command
\addcontentsline{faq}{questions}{#1} % Adds the question to the list of questions
\todo[inline, color=green!40]{\textbf{#1}} % Uses the todonotes package to create a fancy box to put the question
\vspace{1em} % White space after the question before the start of the answer
}

% Uncomment the line below to get rid of the trailing dots in the table of contents
%\renewcommand{\cftdot}{}

% Uncomment the two lines below to get rid of the numbers in the table of contents
%\let\Contentsline\contentsline
%\renewcommand\contentsline[3]{\Contentsline{#1}{#2}{}}


\newenvironment{tightcenter}{%
  \setlength\topsep{0pt}
  \setlength\parskip{0pt}
  \begin{center}
}{%
  \end{center}
}

\newcommand{\itemcard}[4]{
\begin{tightcenter}\subsection*{\expandafter\MakeUppercase\expandafter{#1} / \expandafter\MakeUppercase\expandafter{#2}}
\begin{bf}#3\end{bf}\end{tightcenter}
{ \parskip5pt #4  }
}

\newcommand{\event}[4]{
\begin{tightcenter}\subsection*{\expandafter\MakeUppercase\expandafter{#1} / \expandafter\MakeUppercase\expandafter{#2}}
\begin{bf}#3\end{bf}\end{tightcenter}
{ \parskip5pt #4  }
}



\newcommand{\omen}[4]{
\begin{tightcenter}\subsection{\expandafter\MakeUppercase\expandafter{#1} / \expandafter\MakeUppercase\expandafter{#2}}
\begin{bf}#3\end{bf}\end{tightcenter}
{ \parskip5pt #4

Mache nun einen Spukwurf (Haunt).}
}

\newcommand{\rolls}{\begin{itemize}
\itemsep-5pt}
\newcommand{\erolls}{\end{itemize}}
\newcommand{\roll}[2]{
\item [\bf #1] #2
}


\newcommand{\sanity}{geistige Gesundheit (Sanity)}
\newcommand{\might}{Macht (Might)}
\newcommand{\speed}{Geschwindigkeit (Speed)}
\newcommand{\know}{Wissen (Knowledge)}

\newcommand{\sanityroll}{\emph{Sanity}-Wurf}
\newcommand{\mightroll}{\emph{Might}-Wurf}
\newcommand{\speedroll}{\emph{Speed}-Wurf}
\newcommand{\knowroll}{\emph{Knowledge}-Wurf}

\newcommand{\itemcards}{\emph{Item}-Karten}
\newcommand{\itemcardd}{\emph{Item}-Karte}
\newcommand{\eventcard}{\emph{Event}-Karte}
\newcommand{\omencard}{\emph{Omen}-Karte}


\newcommand{\nootherweapon}{Du kannst keine andere Waffe verwenden, während du diese benutzt.}

\newcommand{\haunt}{Spuk (Haunt)}


\newcommand{\mental}{mentale Eigenschaft}
\newcommand{\mentals}{mentale Eigenschaften}
\newcommand{\physical}{physische Eigenschaft}
\newcommand{\physicals}{physische Eigenschaften}
\newcommand{\physicalsn}{physischen Eigenschaften}

\newcommand{\discardcard}{Entferne diese Karte nach Verwendung aus dem Spiel.}
\newcommand{\discarditem}{Entferne diesen Gegenstand nach Verwendung aus dem Spiel.}


\newcommand{\omencantbedts}{Dieses Omen kann nicht fallen gelassen, gehandelt oder gestohlen werden.}


\begin{document}




\haunttitle{The Mummy Walks}{Die wandernde Mumie}

\introduction{
Staubschwaden ziehen in den Raum und ein Schatten legt sich über dein Herz. Du hörst einen deiner Freude
schreien, ein Geräusch aus Vergnügen und Entsetzen. Eine kalte, klamme Stimme lässt deinen Verstand
erschaudern. "Ich verlor meine Braut, viele Jahre bevor du denken kannst. Meine Tränen sind verstaubt,
aber meine Liebe ist immer noch so kräftig wie die Sonne. Jetzt ist meine Liebe für mich wiedergeboren. Es
gibt nichts mehr was uns beide trennen kann... und wenn du dich gegen mich stellst, werde ich deine Seele
aus deinem Körper reißen und sie gänzlich verschlingen."
}

\rightnow{

    \begin{itemize}
        \bitem Lege 2 Wissenswurfplättchen (Knowledge Roll tokens) beiseite.
        \bitem Der Abenteurer mit dem Mädchen (Girl) verliert es. Dieser Abenteurer verliert alle Bonuspunkte von
der Mädchenkarte (Girl card) und legt sie bei Seite. Der Verräter legt dann das Mädchenplättchen
(Girl token) in einen anderen Raum.
        \bitem Wenn ein Charakter eines Spielers den Raum mit dem Mädchenplättchen betritt, dann erhält dieser
Spieler die Mädchenkarte.
    \end{itemize}
}

\whatyouknowaboutthebadguys{
Der Verräter versucht die Mumie mit dem Mädchen zu verheiraten.
}

\youwhinwhen{
...die Mumie ins Reich des Todes verbannt wurde, bevor sie das Mädchen heiraten konnte.
}

\hauntsection{Wie wird die Mumie verbannt}

Wenn die Buchkarte noch nicht im Spiel ist durchsucht der Held der als nächstes in einem Raum mit dem
Omen Symbol läuft nach der Karte und nimmt sie.Danach wird der Stapel neu gemischt.
Du musst den wahren Namen der Mumie in dem Buch finden und sagen. Um dies zu schaffen musst du
folgende Schritte in dieser Reihenfolge erledigen. Du kannst pro Zug nur einen Schritt erledigen.
Um den wahren Namen der Mumie heraus zu bekommen, kannst du versuchen einen Wissenswürfelwurf
(Knowledge roll) von 6+ in den folgenden Räumen zu bestehen (auf die folgenden Art und Weisen):

    \begin{itemize}
        \bitem der Raum mit dem Sarkophag (untersuche die Hieroglyphen)
        \bitem das XXX Labor (Research Labor) (überfliege die Notizen des Archäologie Teams), oder
        \bitem die Bücherei (Library) (erforsche die Geschichte der Mumie).
    \end{itemize}

Wenn du Erfolg hattest, nehme dir eine Wissenswurfscheibe.
    \begin{itemize}
        \bitem An einem Zug nach dem du Namen erforscht hast, während du das Buch hattest, kannst du einen Wissenswürfelwurf von 6+ versuchen, um den Namen der Mumie herauszufinden. Wenn du Erfolg
hattest, nehme dir ein Wissenswurfplättchen.
        \bitem Sobald du zwei von diesen Plättchen hast, bringe das Buch in den Raum in dem sich die Mumie befindet. Jeder Held kann während du dort bist, versuchen einen Gesundheitswürfelwurf (Sanity roll) zu bestehen, um einen Bann zusprechen der sie für immer vertreibt.
    \end{itemize}

    Die Mumie ist imun gegen Geschwindigkeitsangriffe(Revolver,Dynamit..)

\outro{
Ein heißer trockener Wind flüstert durch den Raum, als du den altertümlichen Wälzer zuknallst. Die Mumie setzt das Schlurfen in deine Richtung fort, ihre Augen sind tote Höhlen der Verzweiflung. Gerade als ihre Hände deine Kehle umklammern, beginnen die Umwicklungen der Mumie zu bröckeln. Die Kreatur stöhnt immer mehr und mehr über ihren Körper, der zusammengedrückt und mit dem heißen Wind hinfort geweht wird. "Meine Braut... meine einzige Liebe... nicht... mehr...."
Als der letzte Rest der Mumie verschwunden ist, hört der Wind auf. Du bist allein.
}

\hauntsection{FAQ}
Was passiert, wenn der Verräter das Buch hat? Dann müssen es die anderen ihm abjagen.
Muss der gleiche Abenteurer die beiden Wissenswürfe machen? Nein
Wenn Kraft und Geschwindigkeit beide auf der untersten Stufe sind, kann die Mumie dann auch töten? Ja..

\twocolumn
\section{Events}



\event{A Moment Of Hope}{Ein Moment der Hoffnung}{
    Irgendetwas in diesem Raum fühlt sich seltsam richtig an. Etwas widersteht dem Bösen des Hauses.
}{
    Plaziere einen \emph{Blessing}-Chip (Segnung) in diesem Zimmer.

    Jeder Held darf bei jedem Charakterwurf (Might, ...) in diesem Zimmer einen zusätzlichen Würfel verwenden.
}

\event{Angry Being}{Böses Wesen}{
    Es kommt aus dem Schleim an der Wand neben dir hervor.
}{
    Du musst einen \speedroll versuchen:
    \rolls
    \roll{5+}{Du kommst davon. Erhalte 1 \speed .}
    \roll{2-4}{Erhalte 1 mentalen Schaden.}
    \roll{0-1}{Erhalte 1 mentalen und 1 pysischen Schaden.}
    \erolls
}

\event{Bloody Vision}{Blutige Vision}{
    Die Wände in diesem Raum sind getränkt in  Blut. Blut tropft von der Decke, fließt die Wände herunter, über Schränke und Möbel und auf deine Schuhe. Im nächsten Augenblick ist es fort.
}{
    Du musst einen \sanityroll\ versuchen:
    \rolls
    \roll{4+}{Du festigst deinen Geist. Erhalte 1 \sanity.}
    \roll{2-3}{Verliere 1 \sanity.}
    \roll{0-1}{Wenn sich ein Entdecker oder Monster in deinem oder einem angrenzenden Zimmer befinden, musst du ihn/sie/es angreifen (wenn du kannst). Wenn möglich, wähle denjenigen mit der niedrigsten \might.}
    \erolls
}


\event{Burning Man}{Brennender Mann}{
    Ein brennender Mann rennt durch den Raum. Seine Haut schlägt Blasen und zerfällt. Ein glutroter Schädel verbleibt, schlägt auf dem Boden auf, rollt und verschwindet.
}{
    Du musst einen \sanityroll\ versuchen:
    \rolls
    \roll{4+}{Du spürst Wärme unter deiner Haut, bist aber ansonsten okay. Erhalte 1 \sanity.}
    \roll{2-3}{Raus, raus, du musst hier raus! Setze deinen Entdecker in die Eingangshalle.}
    \roll{0-1}{Du gehst in Flammen auf. Nehme 1 physischen Schaden. Dann nehme 1 mentalen Schaden als du die Flammen ausklopfst.}
    \erolls
}

\event{Closet Door}{Schranktür}{
    Die Schranktür dort ist offen... nur einen Spalt. Darin muss sich etwas befinden.
}{
    Lege den Closet-Chip (Schrank) in dieses Zimmer.

    Einmal während seines Zuges, kann ein Entdecker zwei Würfel werfen um den Schrank zu öffnen.

    \rolls
    \roll{4}{Ziehe eine \itemcardd .}
    \roll{2-3}{Ziehe eine \eventcard .}
    \roll{0-1}{Ziehe eine \eventcard\ und entferne den Closet-Chip.}
    \erolls

}

\event{Creepy Crawlies}{Gruselige Krabbler}{
    Eintausend Käfer stürzen sich aus deiner Haut, kommen unter deiner Kleidung und aus deinen Haaren hervor.
}{
    Du musst einen \sanityroll\ versuchen:
    \rolls
    \roll{5+}{Du blinzelst und sie sind weg. Erhalte 1 \sanity.}
    \roll{1-4}{Verliere 1 \sanity.}
    \roll{0}{Verliere 2 \sanity.}
    \erolls

}


\event{Creepy Puppet}{Gruselige Puppe}{
    Du siehst eine dieser Puppen, bei denen dir die Haare zu Berge stehen. Sie springt dich mit einem kleinen Speer in den Händen an.
}{
    Der Spieler zu deiner Rechten wirft einen \mightroll\ mit 4 Würfeln für die Puppe. Du verteidigst dich normal, indem du einen \mightroll\ ausführst.

    Wenn du Schaden davonträgst, erhält der Entdecker mit dem Speer (Spear) 2 \might\ (es sei denn du besitzt den Speer).
}


\event{Debris}{Schutt}{
    Mörtel fällt von den Wänden und der Decke.
}{
    Du musst einen \speedroll\ versuchen:
    \rolls
    \roll{+3}{Du weichst aus. Erhalte 1 \speed.}
    \roll{1-2}{Du bist unter Schutt vergraben. Ńehme 1 \emph{Würfel} physischen Schaden.}
    \roll{0}{Du bist unter Schutt vergraben. Ńehme 2 \emph{Würfel} physischen Schaden.}
    \erolls

    Wenn du unter Schutt begraben bist, behalte diese Karte. Du kannst nichts machen bis du befreit wurdest. Einmal pro Zug kann ein Entdecker einen \mightroll\ versuchen um dich zu befreien. (Du kannst diesen Wurf ebenfalls versuchen.) 4+ befreit. Nach drei erfolglosen Versuchen, befreist du dich in deinem folgenden Zug automatisch und kannst normal agieren.
}

\event{Disquieting Sounds}{Beunruhigende Geräusche}{
    Das Geschrei eines Babys, einsam und verlassen.

    Ein Entsetzensschrei.

    Das Kracken zerbrechenden Glases.

    Dann: Stille.
}{
    Würfle mit 6 Würfeln. Wenn du genauso viele oder mehr Augen würfelst, als Omen aufgedeckt wurden, erhälst du 1 \sanity.
    Falls nicht, nehme 1 \emph{Würfel} geistigen Schaden.
}

\event{Drip ... Drip ... Drip ...}{Tropf ... Tropf ... Tropf ...}{
    Ein rhythmisches Geräusch macht dich verrückt.
}{
    Lege einen Drip-Chip (Tröpfeln) in diesen Raum.

    Jeder Entdecker rollt bei jedem Charakterwurf (Might, ...) in diesem Raum mit einem Würfel weniger.
}

\event{Footsteps}{Fußabdrücke}{
    Die Dielen knarren leise. Staub steigt auf. Fußabdrücke erscheinen auf dem schmutzigen Boden. Als sie dich schließlich erreichen, sind sie plötzlich fort.
}{
    Würfle mit einem Würfel. (Befindest du dich in der Kapelle, würfle mit 2 Würfeln.)

    \rolls
    \roll{4}{Du und der nächstgelegene Entdecker erhalten 1 \might.}
    \roll{3}{Du erhälst 1 \might\ und der nächst\-ge\-le\-ge\-ne Entdecker verliert 1 \sanity.}
    \roll{2}{Nehme 1 \sanity\ Schaden.}
    \roll{1}{Nehme 1 \speed\ Schaden.}
    \roll{0}{Jeder Entdecker verliert eine Stufe einer Charaktereigenschaft (Might, ...) seiner Wahl.}
    \erolls
}

\event{Funeral}{Beerdigung}{
    Du siehst einen offenen Sarg. Von innen.
}{
    Du musst einen \sanityroll versuchen:
    \rolls
    \roll{4+}{Du blinzelst und er ist verschwunden. Erhalte 1 \sanity.}
    \roll{2-3}{Die Vision verstört dich. Verliere 1 \sanity.}
    \roll{0-1}{Du befindest dich wirklich in dem Sarg. Du nimmst 1 \sanity\ und 1 \might\ Schaden, als du dich herausschiebst. Wenn der Friedhof (Graveyard) oder die Crypta (Crypt) entdeckt wurden, versetze deine Figur in einen dieser Räume. (Du wählst aus.)}
    \erolls
}

\event{Grave Dirt}{Grabesschmutz}{
    Dieser Raum ist unter einer dicken Schicht Dreck begraben. Du hustest, als sich der Staub auf deiner Haut und in deinen Lungen absetzt.
}{
    Versuche einen \mightroll:
    \rolls
    \roll{4+}{Du schüttelst den Staub ab. Erhalte 1 \might}
    \roll{0-3}{Irgendwas stimmt nicht. Behalte diese Karte. Nehme 1 phsyischen Schaden am Anfang jeder deiner Runden. Lege diese Karte aus dem Spiel, sobald einer deiner Charakterwerte steigt oder wenn du einen Zug auf dem Balkon, im Garten, auf dem Friedhof, in der Turnhalle, im Lagerrraum, auf der Veranda oder auf dem Turm beendest. (Balcony, Gardens, Graveyard, Gymnasium, Larder, Patio or Tower)}
    \erolls
}

\event{Groundkeeper}{Hausmeister}{
    Du drehst dich um und siehst einen Mann in Gärtnerkleidung. Er hebt seine Schaufel und greift an. Zentimeter vor deinem Gesicht verschwindet er, einzig matschige Fußabdrücke hinterlassend.
}{
    Versuche einen \knowroll. (Ein Entdecker im Garten verzichtet auf zwei seiner Würfel).

    \rolls
    \roll{4+}{Du findest etwas im Schlamm. Ziehe eine \itemcardd.}
    \roll{0-3}{Der Hausmeister erscheint wieder und schlägt dir die Schaufel ins Gesicht. Der Spieler zu deiner Rechten wirft einen \mightroll\ mit 4 Würfeln für den Hausmeister. Du verteidigst dich normal, indem du einen \mightroll\ ausführst. }
    \erolls
}


\event{Hanged Men}{Die Erhängten}{
    Ein Hauch kühler Luft fährt durch den Raum. Vor dir hängen drei Männer an ausgefransten Seilen. Sie starren dich mit kalten, toten Augen an. Das Trio pendelt sanft im Wind und verschwindet hinter dem Vorhang aus Staub, der von der Decke herabrieselt. Du fängst an zu husten.
}{
    Du musst einen Wurf für jede Charaktereigenschaft (Might, ...) werfen:
    \rolls
    \roll{2+}{Der entsprechende Wert bleibt unbeeinflusst.}
    \roll{0-1}{Du verlierst 1 Stufe der entsprechenden Eigenschaft.}
    \erolls

    Wenn du bei allen Würfen 2+ wirfst, erhälst du einen zusätzlichen Punkt zu einem Charakterwert deiner Wahl.
}


\event{Hideous Shriek}{Scheußliches Kreischen}{
    Es beginnt mit einem Flüstern, aber endet in einem seelenzerreißenden Schrei.
}{
    Jeder Entdecker muss einen \sanityroll\ versuchen:

    \rolls
    \roll{4+}{Du widerstehst dem Geräusch.}
    \roll{1-3}{Nehme 1 \emph{Würfel} mentalen Schaden.}
    \roll{0}{Nehme 2 \emph{Würfel} mentalen Schaden.}
    \erolls

    Jedes Ergebnis betrifft nur den Entdecker, der den jeweiligen Wurf warf.
}

\event{Image In The Mirror}{Bild im Spiegel}{
    (Version ohne fettgedruckten Einleitungstext.)
}{
    Es befindet sich ein alter Spiegel im Zimmer. Deine erschrockene Reflektion bewegt sich von alleine. Du erkennst dich, aber in einer anderen Zeit. Deine Reflektion schreibt auf den Spiegel:
    \vspace{\parsep}
    \begin{tightcenter}\begin{bf}DAS WIRD HELFEN\end{bf}\end{tightcenter}
    Dann reicht sie dir einen Gegenstand durch den Spiegel.

    Ziehe eine \itemcardd.
}

\event{Image In The Mirror}{Bild im Spiegel}{
    Wenn du keine \itemcards\ besitzt, gilt dieser Effekt für den nächsten Entdecker zu deiner Linken, der eine \itemcardd\ besitzt. Lege diese Karte aus dem Spiel, wenn niemand eine \itemcardd\ besitzt.
}{
    Es befindet sich ein alter Spiegel im Zimmer. Deine erschrockene Reflektion bewegt sich von alleine. Du erkennst dich, aber in einer anderen Zeit. Du musst deiner Reflektion helfen, also schreibst du auf den Spiegel:
    \vspace{\parsep}
    \begin{tightcenter}\begin{bf}DAS WIRD HELFEN\end{bf}\end{tightcenter}
    Dann reiche einen Gegenstand durch den Spiegel.

    Wähle eine deiner Gegenstandskarten (aber keine \omencard) und lege sie auf das Itemdeck. Dann mische das Deck. Erhalte 1 \know.
}

\event{It Is Meant to be}{So soll es sein}{
    Du brichst auf dem Boden zusammen, Visionen zukünftiger Ereignisse fließen durch deine Gedanken.
}{
    Wähle eine dieser beiden Optionen:
    \rolls
    \roll{$\bullet$}{Schaue dir die drei obersten Karten einer der vier Decks an, vertausche sie nach Wahl und lege sie zurück auf den Kartenstapel. Verrate niemandem dein Wissen.}
    \roll{$\bullet$}{Du kannst stattdessen auch mit vier Würfeln werfen und das Ergebnis aufschreiben. Bei irgendeinem zukünftigen Wurf kannst du dieses Ergebnis verwenden anstatt zu würfeln. Wenn diese Zahl größer als das maximal mögliche Ergebnis ist, verwende das höchstmögliche Ergebnis.}
    \erolls
}


\event{Jonah's Turn}{Jonah's Zug}{
    Zwei Jungen spielen mit einem hölzernen Kreisel. ``Willst du auch mal drehen, Jonah?'' fragt einer.

    ``Nein'', sagt Jonah, ``Ich will ihn ganz.'' Jonah nimmt den Kreisel und schlägt dem anderen Jungen ins Gesicht. Der Junge fällt. Jonah schlägt  ihn noch, als der Anblick schwindet.
}{
    Wenn ein Entdecker die Rätsel-Schachtel (Puzzle Box) hat, legt dieser sie aus dem Spiel und zieht stattdessen eine Ersatz-\itemcardd. Wenn dies passiert, erhälst du 1 \sanity. Andernfalls erhälst du einen \emph{Würfel} mentalen Schaden.
}

\event{Lights Out}{Lichter aus}{
    Deine Taschenlampe erlischt. Keine Sorge, jemand anderes hat Batterien.
}{
    Behalte diese Karte. Du kannst pro Zug nur ein Feld laufen bis du deinen Zug bei einem der anderern Entdecker beendest. Lege diese Karte danach aus dem Spiel. Du kannst du nun wieder normal laufen.

    Wenn du die Kerze besitzt oder deinen Zug im Ofenraum (Furnace Room) beendest, lege sie ebenfalls beiseite.
}

\event{Locked Safe}{Verschlossener Safe}{
    Hinter einem Portrait befindet sich ein Wandsafe. Natürlich verklemmt.
}{
    Lege einen Safe-Chip in den Raum.

    Einmal pro Runde kann ein Entdecker den Safe mithilfe eines \knowroll\ öffnen:

    \rolls
    \roll{5+}{Ziehe zwei \itemcards und entferne den Safe-Chip.}
    \roll{2-4}{Nehme einen Würfel physischen Schaden. Der Safe bleibt verschlossen.}
    \roll{0-1}{Nehme zwei Würfel physischen Schaden. Der Safe bleibt verschlossen.}
    \erolls
}

\event{Mists From The Walls}{Nebel aus den Wänden}{
    Nebel fließt aus den Wänden heraus. Man erkennt Gesichter in dem Dunst, menschliche und ... unmenschliche.
}{
    Jeder Entdecker im Keller (Basement) muss einen \sanityroll\ versuchen:

    \rolls
    \roll{4+}{Die Gesichter sind nur Einbildungen in Licht und Schatten. Alles ist gut.}
    \roll{1-3}{Nehme einen Würfel mentalen Schaden. (Nehme einen Würfel zusätzlichen Schaden, wenn sich dein Entdecker in einem Raum mit einem Ereignissymbol befindet.)}
    \roll{0}{Nehme einen Würfel mentalen schaden. (Nehme zwei Würfel zusätzlichen Schaden, wenn sich dein Entdecker in einem Raum mit einem Ereignissymbol befindet.)}
    \erolls
}

\event{Mystic Slide}{Mystische Rutsche}{
    Bist du im Keller (Basement), betrifft dieses Event den nächsten Entdecker zu deiner Linken, der sich nicht im Keller aufhält. Lege diese Karte aus dem Spiel, wenn alle Entdecker im Keller sind.

    Der Boden fällt unter dir ab.
}{
    Setze den Slide-Chip (Rutsche) in diesen Raum, dann versuche einen \mightroll\ um zu rutschen.

    \rolls
    \roll{5+}{Du kontrollierst die Rutsche. Versetze deine Figur in einen Raum deiner Wahl in irgendeinem Stock unterhalb des Ausgangsstockwerks.}
    \roll{0-4}{Ziehe Zimmerkarten bis du eine Kellerkarte ziehst. Plaziere die Karte. (Wenn keine Kellerräume mehr auf dem Stapel liegen, nehme einen bereits existierenden Kellerraum.) Du fällst in diesen Raum und nimmst einen Würfel physischen Schaden. Wenn du nicht an der Reihe bist, ziehe keine Karte für diesen Raum.}
    \erolls

    Ab jetzt kann jeder Entdecker versuchen, zu rutschen.
}


\event{Night View}{Nächtliche Aussicht}{
    Du siehst die Vision eines geisterhaften Pärchens über das Gelände laufen, leise wandelnd in ihrer Hochzeitsgaderobe.
}{
    Du musst einen \knowroll\ versuchen:
    \rolls
    \roll{5+}{Du erkennst die Geister als frühere Bewohner des Hauses. Du rufst ihre Namen, sie drehen sich zu dir um, dunkle Geheimnisse des Hauses flüsternd. Erhalte 1 \know.}
    \roll{0-4}{Du fährst erschrocken zurück, unfähig zuzusehen.}
    \erolls
}

\event{Phone Call}{Anruf}{
    Ein Telefon klingelt im Zimmer. Du fühlst dich verpflichtet abzunehmen.
}{
    Würfle mit zwei Würfeln. Die zuckersüße Stimme einer alten Frau Sagt:
    \rolls
    \roll{4}{``Tee und Kuchen! Tee und Kuchen! Du warst immer mein Liebster.'' Erhalte 1 \sanity.}
    \roll{3}{``Ich war immer für dich da, mein Süßer. Beobachten ...'' Erhalte 1 \know.}
    \roll{1-2}{``Ich bin hier, mein Honigküchlein! Gib uns einen Kuss!'' Nehme einen Würfel mentalen Schaden.}
    \roll{0}{``Böse kleine Kinder müssen bestraft werden!'' Nehme zwei Würfel physischen Schaden.}
    \erolls
}

\event{Possession}{Besessenheit}{
    Ein Schatten schält sich aus der Wand. Du verharrst starr, als der Schatten dich umrundet und dich bis ins Mark auskühlt.
}{
    Du musst eine Charaktereigenschaft auswählen und für diese einen Wurf versuchen:
    \rolls
    \roll{4+}{Du widerstehst den Verlockungen des Schattens. Erhalte 1 in einem Charakterwert deiner Wahl.}
    \roll{0-3}{Der Schatten zerrt von deiner Energie. Die ausgewählte Eigenschaft sinkt auf ihren niedrigsten Wert über dem Schädel. Wenn die Eigenschaft schon auf ihrem niedrigsten Wert liegt, erniedrige eine andere Eigenschaft. }
    \erolls
}

\event{Revolving Wall}{Drehende Wand}{
    Die Wand deht sich an eine andere Stelle.
}{
    Lege den Wall-Switch-Chip (Wandschalter) an eine Wand ohne Ausgang oder an eine Ecke in diesem Raum. Wenn es keinen Raum auf der anderen Seite des Wandschalters gibt, ziehe Raumkarten, bis du eine für das entsprechende Stockwerk gefunden hast und lege sie an. (Wenn es für das Stockwerk keine Karten  mehr gibt, nehme diese Karte aus dem Spiel.) Tritt in den Raum ein.

    Einmal während des Zuges eines Entdeckers, wenn sich dieser in einem der beiden vom Wandschalter berührten Räume befindet, kann er/sie einen \knowroll\ versuchen, um den Wandschalter umzulegen:

    \rolls
    \roll{3+}{Der Entdecker findet den versteckten Hebel und geht durch den Durchgang. Dies zählt nicht als Bewegung.}
    \roll{0-2}{Der Entdecker kann den versteckten Hebel nicht finden und den Durchgang nicht passieren.}
    \erolls
}


\event{Rotten}{Verrottet}{
    Der Gestank in diesem Zimmer ist schrecklich.

    Es riecht nach Tod, wie Blut.

    Ein Schlachthausgeruch.
}{
    Du musst einen \sanityroll versuchen:
    \rolls
    \roll{5+}{Verwirrende Gerüche, mehr nicht. Erhalte 1 \sanity.}
    \roll{2-4}{Verliere 1 \might.}
    \roll{1}{Nehme 1 \might\ und 1 \speed\ Schaden.}
    \roll{0}{Du krümmst dich vor Übelkeit. Verliere 1 von jeder Charaktereigenschaft.}
    \erolls
}

\event{Secret Passage}{Geheimgang}{
    Ein Bereich der Wand fährt zur Seite. Dahinter erstreckt sich ein modriger Tunnel.
}{
    Lege einen \emph{Secret Passage}-Chip in diesen Raum. Würfle mit 3 Würfeln und lege den zweiten \emph{Secret Passage}-Chip in:

    \rolls
    \roll{6}{Irgendeinen schon existierenden Raum.}
    \roll{4-5}{Einen existierenden Raum im oberen Stockwerk.}
    \roll{2-3}{Einen existierenden Raum im Erdgeschoss.}
    \roll{0-1}{Einen existierenden Kellerraum.}
    \erolls

    Du kannst nun den Geheimgang nutzen, auch wenn du keine Bewegung mehr übrig hast.

    Die Benutzung des Geheimgangs zählt als eine Bewegung. Der Geheimgang selbst zählt aber nicht als Feld.

    Ab jetzt kann jeder Entdecker den Geheimgang nutzen. Ein Entdecker kann seinen Zug nicht im Geheimgang beenden.
}



\event{Secret Stairs}{Geheimtreppe}{
    Ein schreckliches Knarzen hallt umher. Du hast eine geheime Treppe entdeckt.
}{
    Lege einen \emph{Secret Stairs}-Chip (Geheimtreppe) in diesen Raum und einen zweiten \emph{Secret Stairs}-Chip in einen anderen existierenden Raum auf einem anderen Stockwerk. Das verwenden der Geheimtreppe zählt als Bewegung. Die Treppe selbst zählt jedoch nicht als Feld.

    Du kannst den Treppen jetzt folgen, auch wenn du keine Bewegung mehr übrig hast. Wenn du ihnen jetzt folgst, ziehe eine \eventcard\ im nächsten Raum.
}

\event{Shrieking Wind}{Kreischender Wind}{
    Ein Wind zieht durchs Haus, ein langsames Crescendo hin zu einem kreischenden Heulen.
}{
    Jeder Spieler im Garten, auf dem Friedhof, in der Patio, auf dem Turm, auf dem Balkon (Garden, Graveyard, Patio, Tower or Balcony) oder einem Raum mit einem nach draußen führenden Fenster muss einen \mightroll\ versuchen:

    \rolls
    \roll{5+}{Du bleibst auf den Füßen.}
    \roll{3-4}{Der Wind wirft dich um. Nehme einen Würfel physischen Schaden.}
    \roll{1-2}{Der Wind kühlt deine Seele aus. Nehme einen Würfel mentalen Schaden.}
    \roll{0}{Der Wind wirft dich derb um. Nehme einen Würfel physischen Schaden. Lege einen deiner Gegenstände (wenn du welche hast) in die Eingangshalle (Entrance Hall).}
    \erolls

    Jedes Ergebnis betrifft nur den Entdecker, der würfelt.
}


\event{Silence}{Stille}{
    Im Kellergewölbe wird plötzlich alles Still. Auch das Geräusch des eigenen Atems verstummt.
}{
    Jeder Entdecker im Keller (Basement) muss einen \sanityroll\ versuchen:

    \rolls
    \roll{4+}{Du wartest ruhig bis dein Hörsinn zurückkehrt.}
    \roll{1-3}{Du schreist einen tonlosen Schrei. Nehme einen Würfel mentalen Schaden. }
    \roll{0}{Du rastest aus. Nehme zwei Würfel mentalen Schaden.}
    \erolls

    Jedes Ergebnis betrifft nur den Entdecker, der würfelt.
}

\event{Skeletons}{Gerippe}{
    Mutter und Kind, sich immer noch umarmend.
}{
    Lege einen \emph{Skeleton}-Chip (Skelett) in diesen Raum. Nehme einen Würfel mentalen Schaden.

    Einmal während seines Zuges kann ein Entdecker einen \sanityroll versuchen, um die Skelette zu durchsuchen.

    \rolls
    \roll{5+}{Ziehe eine \itemcardd. Entferne den \emph{Skeleton}-Chip.}
    \roll{0-4}{Du stocherst herum, findest jedoch nichts. Nehme einen Würfel mentalen Schaden.}
    \erolls

    Jedes Ergebnis betrifft nur den Entdecker, der würfelt.
}

\event{Smoke}{Rauch}{
    Rauch wabert um dich herum. Du hustest während du dir Tränen aus den Augen wischt.
}{
    Lege den \emph{Smoke}-Chip (Rauch) in den Raum. Der Rauch blockiert die Sichtlinie von angrenzenden Räumen. Ein Entdecker würfelt mit zwei Würfeln weniger (aber mindestens mit einem Würfel) bei allen Charakterwürfen, während er sich in diesem Zimmer aufhält.
}


\event{Something Hidden}{Etwas Verstecktes}{
    Dieser Raum ist sonderbar, aber weshalb? Es liegt dir fast auf der Zunge.
}{
    Wenn du herausfinden willst, was seltsam ist, versuche einen \knowroll:
    \rolls
    \roll{4+}{Eine Sektion der Wand schiebt sich seitwärts, eine Nische preisgebend. Ziehe eine \itemcardd.}
    \roll{0-3}{Du kannst es einfach nicht herausfinden, was dich ein wenig verrückt macht. Verliere 1 \sanity.}
    \erolls
}

\event{Something Slimy}{Etwas Schleimiges}{
    Was ist das an deiner Ferse?

    Ein Insekt? Eine Tentakel?

    Eine um sich greifende tote Hand?
}{
    Du musst einen \speedroll\ versuchen:

    \rolls
       \roll{4+}{Du kommst frei. Erhalte 1 \speed.}
       \roll{1-3}{Nehme 1 \might\ Schaden.}
       \roll{0}{Nehme 1 \might\ und 1 \speed\ Schaden.}
       \erolls
}

\event{Spider}{Spinne}{
    Eine faustgroße Spinne landet auf deiner Schulter ... und klettert in deine Haare.
}{
    Du musst einen \speedroll\ versuchen um sie wegzuwischen oder einen \sanityroll\ um still zu halten.

    \rolls
    \roll{4+}{Sie ist weg. Erhalte 1 in der Eigenschaft, die du für den Wurf benutzt hast.}
    \roll{1-3}{Sie beißt dich. Nehme einen Würfel physischen Schaden.}
    \roll{0}{Sie beißt ein Stück Fleisch aus dir heraus. Nehme 2 Würfel physischen Schaden.}
    \erolls
}

\event{The Beckoning}{Der Lockruf}{
    Draußen.

    Du musst nach draußen gehen.

    In die Freiheit fliegen!
}{
    Jeder Entdecker im Garten, auf dem Friedhof, in der Patio, auf dem Turm, auf dem Balkon (Garden, Graveyard, Patio, Tower or Balcony) oder einem Raum mit einem nach draußen führenden Fenster muss einen \sanityroll\ versuchen:

    \rolls
    \roll{3+}{Du trittst vom Sims zurück.}
    \roll{0-2}{Du springst in die Patio. (Wenn sie sich nicht im Haus befindet, durchsuche den Zimmerstapel danach, lege sie an und mische die Karten.) Setze deine Figur in die Patio und erleide einen Würfel physischen Schaden.}
    \erolls

    Jedes Ergebnis betrifft nur den Entdecker, der den jeweiligen Wurf warf.
}

\event{The Lost One}{Die Verlorene}{
    Eine Frau in Kleidern aus der Zeit des Bürgerkriegs winkt dich herbei. Du fällst in Trance.
}{
    Du musst einen \knowroll\ versuchen. Wenn du mehr als 5 würfelst, entkommst du der Trance und erhälst 1 \know. Andernfalls würfle mit drei Würfeln um zu erkennen, wohin dich die Verlorene führt.

    \rolls
    \roll{6}{Versetze deinen Entdecker in die Eingangshalle (Entrance Hall).}
    \roll{4-5}{Versetze deine Figur an die Treppe im ersten Stock (Upper Landing).}
    \roll{2-3}{Ziehe Zimmerkarten, bis du eine Karte für den ersten Stock findest.}
    \roll{0-1}{Ziehe Zimmerkarten, bis du einen Kellerraum findest.}
    \erolls

    Wenn du für dieses Ereignis eine Zimmerkarte ziehen musstest, lege diese Karte an und setze deinen Entdecker hinein. Findest du keine Zimmerkarte für das gewürfelte Stockwerk, versetze deine Figur in die Eingangshalle.
}


\event{The Voice}{Die Stimme}{
    ``Ich bin unter dem Boden, begraben unter dem Boden ...''

    Die Stimme wispert und verschwindet.
}{
    Du musst einen \knowroll\ versuchen:
    \rolls
    \roll{4+}{Du findest etwas unter dem Boden. Ziehe eine \itemcardd.}
    \roll{0-3}{Du gräbst und suchst nach der Stimme, aber vergebens.}
    \erolls
}

\event{The Walls}{Die Wände}{
    Dieser Raum ist warm.

    Wände aus Fleisch pulsieren in einem beständigen Herzschlag. Dein eigenes Herz schägt im Takt mit dem Haus. Du wirst in die Wände gezogen ... und bricht anderswo wieder heraus.
}{
    Du musst eine Zimmerkarte ziehen und anlegen. Versetze deinen Entdecker in den neuen Raum.
}

\event{Webs}{Spinnenweben}{
    Beiläufig hebst du deinen Arm um einige Spinnenweben beiseite zu wischen ... aber sie wollen sich nicht wegschieben lassen. Sie kleben.
}{
    Du musst einen \mightroll\ versuchen:
    \rolls
    \roll{4+}{Du kommst frei. Erhalte 1 \might\ und lege diese Karte aus dem Spiel.}
    \roll{0-3}{Du steckst fest. Behalte diese Karte.}
    \erolls

    Steckst du fest, kannst du nichts machen, bis du befreit bist. Einmal pro Zug kann ein Entdecker einen \mightroll\ versuchen, um dich zu befreien. (Du kannst diesen Wurf ebenfalls versuchen.) 4+ befreit, aber du erhälst keinen \emph{Might}-Bonus. Jeder, der es nicht schafft dich zu befreien, kann sich für den Rest seines Zuges nicht bewegen. Nach drei erfolglosen Versuchen kannst du dich automatisch in deinem nächsten Zug befreien und diesen Zug normal spielen.

    Lege diese Karte aus dem Spiel, sobald du dich befreit hast.
}


\event{What The ... ?}{Was zum ... ?}{
    Als du den Weg zurückblickst, den du gekommen bist, siehst du ... nichts.

    Nur dichter Nebel und Dunst wo eben noch ein Raum war.
}{
    Nehme die Zimmerkachel, auf der du stehst (nachdem du alles andere beiseite geräumt hast) und setze sie an anderer Stelle im selben Stockwerk wieder an, sodass die Tür an eine bisher unentdeckte Tür anschließt (und setze alles wieder darauf was du weggeräumt hast). Wenn es keinen unentdeckten Durchgang auf diesem Stockwerk gibt, versetze das Zimmer in ein anderes Stockwerk.
}

\event{Whoops!}{Hoppla!}{
    Du fühlst einen Körper an deinen Füßen. Bevor du einen Schritt zurücktreten kannst, wirst du umgestoßen. Eine kichernde Stimme rennt davon.
}{
    Nehme alle deine \itemcards\ (keine \omencards) und mische sie. Der Spieler zu deiner Rechten zieht zufällig eine davon und legt sie aus dem Spiel. Dann lege deine \itemcards\ wieder offen hin.
}

\pagebreak
\pagebreak

\twocolumn
\section{Omen}


\omen{Bite}{Biss}{
    Ein Knurren, der süße Duft von Tod. Schmerz. Dunkelheit. Fort.
}{
    \emph{Etwas} beißt dich, als du diese Karte ziehst. Der Spieler zu deiner Rechten wirft einen \mightroll\ mit 4 Würfeln für das mysteriöse \emph{Etwas}, bevor es in der Dunkelheit verschwindet. Du verteidigst dich normal, indem du einen \mightroll\ ausführst.

    \omencantbedts
}

\omen{Book}{Buch}{
    Ein Tagebuch oder Laborprotokolle? Ein antikes Manuskript oder moderner Bestseller?
}{
    Erhalte sofort 2 \know.

    Verliere 2 \know, wenn du das Buch verlierst.
}

\omen{Crystal Ball}{Kristallkugel}{
    Trübe Bilder erscheinen im Inneren des Glases.
}{
    Ist der \haunt\ offenbart, kannst du mittels eines \knowroll s während deines Zuges in die Kristallkugel schauen:

        \rolls
        \roll{4+}{Du siehst die Wahrheit. Durchsuche das Gegenstandsdeck (Items) oder das Ereignisdeck (Events) und wähle eine Karte aus. Mische die Karten und lege deine Karte oben auf.}
        \roll{1-3}{Du wendest deinen Blick ab. Verliere 1 \sanity.}
        \roll{0}{Du starrst direkt in die Hölle. Verliere 2 \sanity.}
        \erolls
}

\omen{Dog}{Hund}{
    BEGLEITER

    Dieser räudige Hund scheint freundlich. Du hoffst, dass er das auch wirklich ist.
}{
    Erhalte sofort 1 \might\ und 1 \sanity.

    Verliere 1 \might\ und 1 \sanity, wenn du die Aufsicht über den Hund abgibst.

    Verwende einen kleinen Monsterspielstein, der den Hund darstellt und lege ihn in deinen Raum. (Verwende eine Monsterfarbe die noch nicht auf dem Spielfeld liegt.) Einmal während deines Zuges kann der Hund zu einem maximal 6 Felder entfernten, schon entdeckten Raum laufen, um danach direkt wieder zurückzukehren. Er kann einen Gegenstand aufnehmen, tragen und/oder fallenlassen, bevor er zurückkehrt.

    Der Hund wird von Gegnern nicht verlangsamt. Er kann keine Einbahnstraßen nehmen oder Räume, die einen Wurf erfordern durchqueren. Er kann keine Gegenstände tragen, die die Bewegung verlangsamen.

    \omencantbedts
}

\omen{Girl}{Mädchen}{
    BEGLEITER

    Ein Mädchen.

    Gefesselt.

    Alleine.

    Du befreist sie!
}{
    Erhalte sofort 1 \sanity\ und 1 \know.

    Verliere 1 \sanity\ und 1 \know, wenn du die Mädchen-Karte verlierst.

    \omencantbedts
}

\omen{Holy Symbol}{Heiliges Symbol}{
    Ein Symbol der Ruhe in einer rastlosen Welt.
}{
    Erhalte sofort 2 \sanity.

    Verliere 2 \sanity, wenn du das Heilige Symbol verlierst.
}

\omen{Madman}{Ein Verrückter}{
    BEGLEITER

    Ein schäumender Irrer.
}{
    Erhalte sofort 2 \might\ und verliere 1 \sanity.

    Verliere 2 \might\ und erhalte 1 \sanity, wenn du diese Karte verlierst.

    \omencantbedts
}

\omen{Mask}{Maske}{
    Eine düstere Maske, um deine Absichten zu verschleiern.
}{
    Einmal pro Zug kannst du mittels eines \sanityroll s versuchen, die Maske zu benutzen.

    \rolls
    \roll{4+}{Du kannst die Maske auf- oder absetzen.

    Wenn du die Maske aufziehst, erhalte 2 \know\ und verliere 2 \sanity.

    Wenn du die Maske absetzt, erhalte 2 \sanity\ und verliere 2 \know.}
    \roll{0-3}{Du kannst die Maske in dieser Runde nicht verwenden.}
    \erolls
}

\omen{Medallion}{Medallion}{
    Ein Medallion mit einem eingravierten Pentagram.
}{
    Du bist immun gegenüber den Effekten der Pentagramm-Kammer, der Crypta und des Friedhofs (Pentagam Chamber, Crypt, Graveyard).
}

\omen{Ring}{Ring}{
    Ein abgenutzter Ring mit einer unleserlichen Inschrift.
}{
    Wenn du einen Gegner angreifst, der einen \emph{Sanity}-Wert besitzt, kannst du ihn mit einem \sanityroll\ anstelle eines \mightroll s angreifen. (Dein Gegner verteidigt sich dann mit einem \sanityroll\ und der Schaden wirkt statt physisch mental.)
}

\omen{Skull}{Schädel}{
    Ein rissiger Schädel mit fehlenden Zähnen.
}{
    Wenn du mentalen Schaden erleidest, kannst du ihn stattdessen komplett in physischen Schaden umwandeln.
}

\omen{Spear}{Speer}{
    WAFFE

    Eine vor Macht pulsierende Waffe.
}{
    Du kannst zwei zusätzliche Würfel (bis maximal 8 Würfel) werfen, wenn du eine \might\ Attacke mit dieser Waffe ausführst.

    \nootherweapon
}

\omen{Spirit Board}{Ouijabrett}{
    Eine Tafel mit Buchstaben und Zahlen, um mit den Toten zu sprechen.
}{
    Bevor du dich während deines Zuges bewegst, darfst du unter die oberste Karte des Zimmerkartenstapels schauen.

    Benutzt du das Ouijabrett, nachdem der \haunt\ offenbart wurde, kann der Verräter beliebige Monster ein Feld weiter in deine Richtung bewegen. (Wenn du der Verräter bist, musst du die Monster nicht bewegen.) Wenn es keinen Verräter gibt, rücken alle Monster 1 Feld zu dir.
}

\pagebreak

\section{Items / Gegenstände}



\itemcard{Lucky Stone}{Glücksstein}{
    Ein glattes, gewöhnlich aussehendes Stück Felsgestein. Du fühlst, dass er dir Glück bringen wird.
}{
    Nachdem du gewürfelt hast, kannst du an diesem Stein reiben, um einen oder mehrere deiner Würfel nocheinmal zu würfeln.

    \discarditem
}

\itemcard{Dark Dice}{Dunkle Würfel}{
    Wie stehts um dein Glück?
}{
    Einmal pro Runde kannst du 3 Würfel werfen:
    \rolls
    \roll{6}{Gehe sofort zu irgendeinem Entdecker, aber nicht zum Verräter}
    \roll{5}{Verschiebe einen anderen Entdecker aus deinem Raum in einen angrenzenden Raum.}
    \roll{4}{+1 \physical}
    \roll{3}{Bewege dich ohne Bewegungskosten sofort in einen angrenzenden Raum.}
    \roll{2}{+1 \mental}
    \roll{1}{Ziehe eine Ereigniskarte.}
    \roll{0}{Reduziere alle deine Eigenschaften auf den ersten Schritt über dem Schädel und entferne die Karte aus dem Spiel.}
    \erolls
}


\itemcard{Medical Kit}{Medizintasche}{
    Ein Arztkoffer, aber die wichtigsten Medikamente sind schon geplündert worden.
}{
    Einmal pro Zug kannst du versuchen, dich oder einen anderen Entdecker im gleichen Raum mit einem Wissenswurf zu heilen.

    \rolls
    \roll{8+}{+3 für \physicals}
    \roll{6-7}{+2 für \physicals}
    \roll{4-5}{+1 für eine \physical}
    \roll{0-3}{Es passiert nichts.}
    \erolls

    Du kannst eine Eigenschaft maximal bis zu ihrem Startwert heilen.
}

\itemcard{Idol}{Idol}{
    Vielleicht hat es dich für einen höheren Zweck auserwählt. Möglicherweise Menschenopfer.
}{
    Einmal pro Zug kannst du an dem Idol reiben, bevor du einen Charakter-, Kampf- oder Ereigniswurf würfelst, um 2 zusätzliche Würfel (bis maximal 8 Würfel) einzusetzen. Jedes mal verlierst du 1 \sanity.

}

\itemcard{Axe}{Akt}{
    WAFFE

    Sehr scharf.
}{
    Wenn du diese Waffe verwendest, kannst einen zusätzlichen Würfel (bis maximal 8 Würfel) für einen \mightroll\ benutzen.

    \nootherweapon
}

\itemcard{Adrenaline Shot}{Adrenalinespritze}{
    Eine Spritze, die eine seltsame, fluosziierende Flüssigkeit enthält.
}{
    Du kannst diesen Gegensstand einsetzen, bevor du einen Charakterwurf würfelst, um zum Wurfergebnis 4 hinzuzufügen.

    \discarditem
}

\itemcard{Angel Feather}{Engelsfeder}{
    Eine makellose Feder schwirrt auf deine Hand.
}{
    Wenn du einen Wurf jedweder Art würfeln musst, kannst du stattdessen eine beliebige Zahl von 0 bis 8 direkt als Würfelergebnis verwenden.

    \discarditem
}

\itemcard{Dynamite}{Dynamit}{
    Die Zündschnur brennt \emph{noch} nicht.
}{
    Anstatt zu Attakieren kannst du Dynamit durch eine Tür in einen benachbarten Raum werfen. Jeder Entdecker und jedes Monster mit Macht und Geschwindigkeitsmerkmalen in diesem Raum muss einen \speedroll\ versuchen:

    \rolls
    \roll{5+}{Kein Schaden}
    \roll{0-4}{Du nimmst 4 Schaden in \physicalsn}
    \erolls

    \discarditem
}

\itemcard{Pickpocket's Gloves}{Handschuhe des Diebes}{
    Sich selbst zu helfen war noch nie so einfach.
}{
    Wenn du zusammen mit einem anderen Entdecker in einem Raum bist, kannst du ihm einen Gegenstand klauen.

    \discarditem
}

\itemcard{Revolver}{Revolver}{
    WAFFE

    Eine alte, wirksam scheinende Waffe.
}{
    Du kannst den Revolver verwenden, um mit einem \speedroll\ anstelle eines \mightroll s anzugreifen. (Dein Gegner verteidigt sich dann mit \speed\ und nimmt physischen Schaden.)

    Würfle mit einem zusätzlichen Würfel bei deinem Angriffswurf.

    Mit dem Revolver kannst du im gleichen Raum oder in gerader Sichtlinie durch mehrere Türen treffen. Wenn du jemanden in einem anderen Raum angreifst, nimmst du im Falle einer Niederlage keinen Schaden.

     \nootherweapon

}

\itemcard{Rabbit's Foot}{Hasenfuß}{
    Der Hase hatte wohl kein Glück.
}{
    Einmal pro Zug kannst du \emph{einen} Würfel erneut werfen. Der zweite Wurf zählt.
}

\itemcard{Puzzle Box}{Rätsel-Schachtel}{
    Irgendwie muss man die doch öffnen können.
}{
    Einmal pro Zug kannst du versuchen die Kiste mit einem \knowroll\ zu öffnen.

    \rolls
    \roll{+6}{Du öffnest die Box. Ziehe zwei \itemcards\ und entferne die Schachtel aus dem Spiel.}
    \roll{0-5}{Du schaffst es nicht die Box zu öffnen.}
    \erolls
}

\itemcard{Blood Dagger}{Blutdolch}{
    WAFFE

    Eine fiese Waffe. Nadeln und Schläuche winden sich aus dem Griff und stechen direkt in deine Venen.
}{
    Du kannst drei zusätzliche Würfel (bis maximal 8 Würfel) werfen, wenn du eine \might\ Attacke mit dieser Waffe ausführst. Du verlierst dabei 1 \speed.

    \nootherweapon

    Du kannst diesen Dolch nicht handeln oder fallenlassen. Wenn er gestohlen wird, nimmst du 2 Schaden in \physicalsn.
}


\itemcard{Bell}{Glocke}{
    Eine Messingglocke mit schallendem Gong.
}{
    Erhalte sofort 1 \sanity.

    Verliere 1 \sanity, wenn du die Glocke verlierst.

    Ist der \haunt\ offenbart, kannst du mittels eines \sanityroll s während deines Zuges die Glocke läuten:

    \rolls
    \roll{5+}{Rücke beliebige, aber frei bewegliche Entdecker einen Raum näher zu dir.}
    \roll{0-4}{Der Verräter darf beliebig viele Monster einen Raum näher zu dir rücken. (Wenn du der Verräter bist, gilt dies nicht.) Wenn es keinen Verräter gibt, bewegen sich alle Monster in deine Richtung.}
    \erolls
}


\itemcard{Armor}{Rüstung}{
    Es ist nur eine Requisite vom Renaissancefest, aber immerhin ist sie aus Metall.
}{
    Jedes Mal (nicht nur einmal pro Runde) wenn du physischen Schaden nimmst, erhälst du einen Schaden weniger.

    Dieser Gegenstand kann nicht gestohlen werden.
}

\itemcard{Music Box}{Spieluhr}{
    Eine handgefertigte Antiquität. Sie spielt eine gespenstische Melodie, die dir nicht mehr aus dem Kopf geht.
}{
    Einmal pro Zug kannst du die Spieluhr öffnen oder schließen.

    Während die Spieluhr offen ist, muss jeder Entdecker oder jedes Monster mit \emph{Sanity}-Merkmal, dass den Raum der Spieluhr betritt oder seinen Zug darin beginnt, einen \sanityroll\ werfen. Schafft er keine 4+, ist er/es für den Rest des Zuges hypnotisiert.

    Wenn ein Entdecker oder Monster, das die Spieluhr bei sich trägt, hypnotisiert wird, lässt er/es die Spieluhr fallen. Ist die Spieluhr währenddessen offen, bleibt sie offen.

}

\itemcard{Healing Salve}{Heilsalbe}{
    Eine klebrige Paste in einer flachen Schale.
}{
    Du kannst dich oder einen anderen lebenden Entdecker im selben Raum mit dieser Salbe behandeln. Hebe entweder den \might\ oder den Geschwindigkeitswert (Speed) des geheilten Entdeckers auf seinen Startwert an.

    \discarditem
}


\itemcard{Smelling Salts}{Riechsalz}{
    Wow, das haut rein.
}{
    Du kannst dir oder einen anderen lebenden Entdecker im selben Raum das Riechsalz unter die Nase halten. Hebe den Wissenswert (Knowledge) des entsprechenden Entdeckers auf seinen Startwert an.

    \discarditem
}


\itemcard{Sacrificial Dagger}{Opferdolch}{
    Ein gewundener Sporn aus Eisen, überzogen mit mysteriösen Symbolen und triefend vor Blut.
}{
    Du kannst drei zusätzliche Würfel (bis maximal 8 Würfel) werfen, wenn du eine \might\ Attacke mit dieser Waffe ausführst, aber vorher musst du einen \knowroll werfen:

    \rolls
    \roll{6+}{Kein Effekt.}
    \roll{3-5}{Nehme einen Schaden in einer \mental.}
    \roll{0-2}{Der Dolch winded sich in deiner Hand. Nehme zwei Schaden in \physicalsn. Du kannst in diesem Zug nicht mehr angreifen.}
    \erolls
}

\itemcard{CANDLE}{Kerze}{
    Sie lässt die Schatten tanzen - wenigstens hoffst du sie macht genau das.
}{
    Wenn du eine Ereigniskarte ziehst, darfst du jeden das Ereignis betreffenden Charakterwurf (Might, ...) mit einem zusätzlichen Würfel bestreiten.

    Wenn du die Glocke (Bell), das Buch (Book) und die Kerze besitzt, erhalte +2 von jeder Eigenschaft (Might, ...). Sobald du einen dieser Gegenstände verlierst, verliere 2 von jeder Eigenschaft.
}

\itemcard{Bottle}{Flasche}{
    Ein trüber Flakon, in dem eine schwarze Flüssigkeit schwimmt.
}{
    Ist der \haunt\ offenbart, kannst du während deines Zuges aus der Flasche trinken. Würfle mit drei Würfeln:


    \rolls
    \roll{6}{Versetze deine Figur in einen beliebigen Raum.}
    \roll{5}{Erhalte 2 \might\ und 2 \speed.}
    \roll{4}{Erhalte 2 \know\ und 2 \sanity.}
    \roll{3}{Erhalte 1 \know\ und verliere 1 \might.}
    \roll{2}{Verliere 2 \know\ und 2 \sanity.}
    \roll{1}{Verliere 2 \might\ und 2 \speed.}
    \roll{0}{Verliere 2 von jeder Eigenschaft.}
    \erolls
}

\itemcard{Amulet Of The Ages}{Amulett der Zeitalter}{
    Altes Silber und eingelassene Juwelen, Inschriften von Segnungen.
}{
    Erhalte sofort 1 von jeder Eigenschaft.

    Verliere 3 von jeder Eigenschaft, wenn du das Amulett verlierst.
}



\end{document}