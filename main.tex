%%%%%%%%%%%%%%%%%%%%%%%%%%%%%%%%%%%%%%%%%
% Frequently Asked Questions
% LaTeX Template
% Version 1.0 (22/7/13)
%
% This template has been downloaded from:
% http://www.LaTeXTemplates.com
%
% Original author:
% Adam Glesser (adamglesser@gmail.com)
%
% License:
% CC BY-NC-SA 3.0 (http://creativecommons.org/licenses/by-nc-sa/3.0/)
%
%%%%%%%%%%%%%%%%%%%%%%%%%%%%%%%%%%%%%%%%%

\documentclass[10pt,a4paper,oneside,ngerman]{article}

\usepackage[ngerman]{babel}

\usepackage[margin=1in]{geometry} % Required to make the margins smaller to fit more content on each page
%\usepackage[linkcolor=blue]{hyperref} % Required to create hyperlinks to questions from elsewhere in the document
%\hypersetup{pdfborder={0 0 0}, colorlinks=true, urlcolor=blue} % Specify a color for hyperlinks
%\usepackage{todonotes} % Required for the boxes that questions appear in
%\usepackage{tocloft} % Required to give customize the table of contents to display questions
\usepackage{microtype} % Slightly tweak font spacing for aesthetics
%\usepackage{palatino} % Use the Palatino font
\usepackage[utf8]{inputenc}

\setlength\parindent{0pt} % Removes all indentation from paragraphs



\newenvironment{tightcenter}{%
  \setlength\topsep{0pt}
  \setlength\parskip{0pt}
  \begin{center}
}{%
  \end{center}
}

\newcommand{\itemcard}[4]{
\begin{tightcenter}\subsection{\expandafter\MakeUppercase\expandafter{#1} / \expandafter\MakeUppercase\expandafter{#2}}
\begin{bf}#3\end{bf}\end{tightcenter}
{ \parskip5pt #4  }
}

\newcommand{\event}[4]{
\begin{tightcenter}\subsection{\expandafter\MakeUppercase\expandafter{#1} / \expandafter\MakeUppercase\expandafter{#2}}
\begin{bf}#3\end{bf}\end{tightcenter}
{ \parskip5pt #4  }
}



\newcommand{\omen}[4]{
\begin{tightcenter}\subsection{\expandafter\MakeUppercase\expandafter{#1} / \expandafter\MakeUppercase\expandafter{#2}}
\begin{bf}#3\end{bf}\end{tightcenter}
{ \parskip5pt #4

Mache nun einen Spukwurf (Haunt).}
}

\newcommand{\rolls}{\begin{itemize}
\itemsep-5pt}
\newcommand{\erolls}{\end{itemize}}
\newcommand{\roll}[2]{
\item [\bf #1] #2
}


\newcommand{\sanity}{geistige Gesundheit (Sanity)}
\newcommand{\might}{Macht (Might)}
\newcommand{\speed}{Geschwindigkeit (Speed)}
\newcommand{\know}{Wissen (Knowledge)}

\newcommand{\sanityroll}{\emph{Sanity}-Wurf}
\newcommand{\mightroll}{\emph{Might}-Wurf}
\newcommand{\speedroll}{\emph{Speed}-Wurf}
\newcommand{\knowroll}{\emph{Knowledge}-Wurf}

\newcommand{\itemcards}{\emph{Item}-Karten}
\newcommand{\itemcardd}{\emph{Item}-Karte}
\newcommand{\eventcard}{\emph{Event}-Karte}
\newcommand{\omencard}{\emph{Omen}-Karte}
\newcommand{\omencards}{\emph{Omen}-Karten}


\newcommand{\nootherweapon}{Du kannst keine andere Waffe verwenden, währ\-end du diese benutzt.}

\newcommand{\haunt}{Spuk (Haunt)}


\newcommand{\mental}{mentale Eigenschaft}
\newcommand{\mentals}{mentale Eigenschaften}
\newcommand{\physical}{physische Eigenschaft}
\newcommand{\physicals}{physische Eigenschaften}
\newcommand{\physicalsn}{physischen Eigenschaften}

\newcommand{\discardcard}{Entferne diese Karte nach Verwendung aus dem Spiel.}
\newcommand{\discarditem}{Entferne diesen Gegenstand nach Verwendung aus dem Spiel.}


\newcommand{\omencantbedts}{Dieses Omen kann nicht fallen gelassen, gehandelt oder gestohlen werden.}


\newcommand{\room}[2]{\expandafter\MakeUppercase\expandafter{#1} / \expandafter\MakeUppercase\expandafter{#2}}

\newcommand{\iroom}[2]{\item \room{#1}{#2}}

\begin{document}

\tableofcontents


\survival{1}{The Mummy Walks}{Die wandernde Mumie}

\introduction{
Staubschwaden ziehen in den Raum und ein Schatten legt sich über dein Herz. Du hörst einen deiner Freude
schreien, ein Geräusch aus Vergnügen und Entsetzen. Eine kalte, klamme Stimme lässt deinen Verstand
erschaudern. ``Ich verlor meine Braut, viele Jahre bevor du denken kannst. Meine Tränen sind verstaubt,
aber meine Liebe ist immer noch so kräftig wie die Sonne. Jetzt ist meine Liebe für mich wiedergeboren. Es
gibt nichts mehr was uns beide trennen kann... und wenn du dich gegen mich stellst, werde ich deine Seele
aus deinem Körper reißen und sie gänzlich verschlingen.''
}

\rightnow{

    \begin{itemize}
        \bitem Lege 2 dreieckige \chips{Knowledge Roll}{\knowroll} beiseite.
        \bitem Der Verräter verliert das Mädlchen (Girl) und die von ihr verliehenen Boni. Stattdessen legt er einen kleinen rosanen Monsterchip (der das Mädchen darstellt) in irgendeinen Raum im selben Stockwerk, in dem der Spuk offenbart wurde, jedoch mindestens 5 Felder von der Mumie entfernt. Gibt es keinen Raum, der mindestens 5 Felder entfernt ist, plaziert er den Chip so weit wie möglich entfernt.
        \bitem Wenn ein Entdecker den Raum mit dem Mädchenchip betritt, dann erhält dieser Spieler die Mädchenkarte.
    \end{itemize}
}

\whatyouknowaboutthebadguys{
Der Verräter versucht die Mumie mit dem Mädchen zu verheiraten.
}

\youwhinwhen{
...die Mumie zurück ins Reich des Todes verbannt wurde, bevor sie das Mädchen heiraten konnte.
}

\hauntsection{Wie die Mumie verbannt wird}

  \begin{itemize}
        \bitem Wenn die Buchkarte (Book) noch nicht im Spiel ist, durchsucht der Held, der als nächstes einen Raum mit Omensymbol entdeckt, den Omenstapel nach dem Buch und nimmt sie. Danach wird der Stapel neu gemischt.
        \bitem Du musst den wahren Namen der Mumie in dem Buch finden und aussprechen. Um dies zu schaffen, musst du folgende Schritte in dieser Reihenfolge erledigen. Jeder Held kann in seinem Zug nur einen der Schritte erledigen.

\newpage
        \begin{enumerate}
            \item Um den wahren Namen der Mumie heraus zu bekommen, kannst du versuchen einen \knowroll\ von 6+ in den folgenden Räumen zu bestehen:

            \begin{itemize}
                \bitem Untersuche die Hieroglyphen im Raum mit dem Sarkophag oder
                \bitem überfliege die Notizen des Archäologie Teams im Forschungslabor (Research Laboratory) oder
                \bitem erforsche die Geschichte der Mumie in der Bücherei (Library).
            \end{itemize}
            Wenn du Erfolg hattest, nehme dir einen \chipe{Knowledge Roll}.

            \item Wurde der Name entdeckt, kann der Held, der das Buch besitzt, (frühstens im darauf folgenden Zug) einen \knowroll\ von 6+ versuchen, um den Namen der Mumie nachzuschlagen und den Spruch zu lernen, der zu ihrer Verbannung führt. Wenn du Erfolg hattest, nehme dir einen \chipe{Knowledge Roll}.


            \item Sobald die Helden zwei von diesen Plättchen haben, muss ein Held das Buch in den Raum mit der Mumie bringen. Jeder Held, der mit der Mumie und dem Buch im selben Raum ist, kann die Mumie durch Aussprechen des Zauberspruchs verbannen, indem er  die Mumie mittels einer \sanity-Attacke besiegt.
        \end{enumerate}

    \bitem Die Mumie ist imun gegen Geschwindigkeitsangriffe (Revolver, Dynamit..)

    \end{itemize}

\outro{
Ein heißer trockener Wind flüstert durch den Raum, als du den altertümlichen Wälzer zuknallst. Die Mumie setzt das Schlurfen in deine Richtung fort, ihre Augen sind tote Höhlen der Verzweiflung. Gerade als ihre Hände deine Kehle umklammern, beginnen die Umwicklungen der Mumie zu bröckeln. Die Kreatur stöhnt immer mehr und mehr über ihren Körper, der zusammengedrückt und mit dem heißen Wind hinfort geweht wird. "Meine Braut... meine einzige Liebe... nicht... mehr...."
Als der letzte Rest der Mumie verschwunden ist, hört der Wind auf. Du bist allein.
}

\hauntsection{FAQ}

\begin{itemize}
    \bitem Was passiert, wenn der Verräter das Buch hat? Dann müssen es die anderen ihm abjagen.
    \bitem Muss der gleiche Abenteurer die beiden Wissenswürfe machen? Nein.
    \bitem Wenn Kraft und Geschwindigkeit beide auf der untersten Stufe sind, kann die Mumie dann auch töten? Ja.
\end{itemize}

\pagebreak

%\twocolumn
\section{Rooms / Räume}


\begin{itemize}
    \parskip-3pt
    \iroom{Abandoned Room}{Stillgelegter Raum}
    \iroom{Attic}{Dachboden}
    \iroom{Balcony}{Balkon}
    \iroom{Ballroom}{Ballzimmer.}
    \iroom{Basement Landing}{Landeplatz im Keller (Kohleschütte)}
    \iroom{Bedroom}{Schlafzimmer}
    \iroom{Bloody Room}{Blutiges Zimmer}
    \iroom{Catacombs}{Katakomben}
    \iroom{Chapel}{Kapelle}
    \iroom{Charred Room}{verbrannter Raum}
    \iroom{Chasm}{Kluft}
    \iroom{Coal Chute}{Kohleschacht}
    \iroom{Collapsed Room}{Eingestürztes Zimmer}
    \iroom{Conservatory}{Wintergarten}
    \iroom{Creaky Hallway}{Knarrender Hausflur}
    \iroom{Crypt}{Crypta}
    \iroom{Dining Room}{Speisesaal}
    \iroom{Dusty Hallway}{Verstaubter Hausflur}
    \iroom{Entrance Hall}{Eingangshalle}
    \iroom{Furnace Room}{Ofenraum}
    \iroom{Foyer}{Foyer}
    \iroom{Gallery}{Gallerie}
    \iroom{Game Room}{Spielzimmer}
    \iroom{Gardens}{Gärten}
    \iroom{Grand Staircase}{Große Treppe}
    \iroom{Graveyard}{Friedhof}
    \iroom{Gymnasium}{Turnraum}
    \iroom{Junk Room}{Gerümpelzimmer}
    \iroom{Kitchen}{Küche}
    \iroom{Larder}{Lagerraum}
    \iroom{Library}{Bibliothek}
    \iroom{Master Bedroom}{Herrenschlafzimmer}
    \iroom{Mystic Elevator}{Mystischer Aufzug}
    \iroom{Operating Laboratory}{Operationssaal}
    \iroom{Organ Room}{Orgelzimmer}
    \iroom{Patio}{Patio, Innenhof}
    \iroom{Pentagram Camber}{Pentagramkammer}
    \iroom{Research Laboratory}{Forschungslabor}
    \iroom{Servants' Quarters}{Quatier des Dieners}
    \iroom{Stairs From Basement}{Kellertreppe}
    \iroom{Statuary Corridor}{Bildhauerkorridor}
    \iroom{Storeroom}{Abstellraum}
    \iroom{Tower}{Turm}
    \iroom{Underground Lake}{Untergrundsee}
    \iroom{Upper Landing}{Oberer Treppenabsatz}
    \iroom{Vault}{Tresorraum}
    \iroom{Wine Cellar}{Weinkeller}
\end{itemize}





\twocolumn
\section{Events / Ereignisse}

\input{EVENTS}

\pagebreak
\pagebreak

\twocolumn
\section{Omen}


\omen{Bite}{Biss}{
    Ein Knurren, der süße Duft von Tod. Schmerz. Dunkelheit. Fort.
}{
    \emph{Etwas} beißt dich, als du diese Karte ziehst. Der Spieler zu deiner Rechten wirft einen \mightroll\ mit 4 Würfeln für das mysteriöse \emph{Etwas}, bevor es in der Dunkelheit verschwindet. Du verteidigst dich normal, indem du einen \mightroll\ ausführst.

    \omencantbedts
}

\omen{Book}{Buch}{
    Ein Tagebuch oder Laborprotokolle? Ein antikes Manuskript oder moderner Bestseller?
}{
    Erhalte sofort 2 \know.

    Verliere 2 \know, wenn du das Buch verlierst.
}

\omen{Crystal Ball}{Kristallkugel}{
    Trübe Bilder erscheinen im Inneren des Glases.
}{
    Ist der \haunt\ offenbart, kannst du einmal pro Zug mittels eines \knowroll s in die Kristallkugel schauen:

        \rolls
        \roll{4+}{Du siehst die Wahrheit. Durchsuche das Gegenstandsdeck (Items) oder das Ereignisdeck (Events) und wähle eine Karte aus. Mische die Karten und lege deine Karte oben auf.}
        \roll{1-3}{Du wendest deinen Blick ab. Verliere 1 \sanity.}
        \roll{0}{Du starrst direkt in die Hölle. Verliere 2 \sanity.}
        \erolls
}

\omen{Dog}{Hund}{
    BEGLEITER

    Dieser räudige Hund scheint freundlich. Du hoffst, dass er das auch wirklich ist.
}{
    Erhalte sofort 1 \might\ und 1 \sanity.

    Verliere 1 \might\ und 1 \sanity, wenn du die Aufsicht über den Hund abgibst.

    Verwende einen kleinen Monstermarker, der den Hund darstellt und lege ihn in deinen Raum. (Verwende eine Monsterfarbe die noch nicht auf dem Spielfeld liegt.) Einmal während deines Zuges kann der Hund zu einem maximal 6 Felder entfernten, schon entdeckten Raum laufen, um danach direkt wieder zurückzukehren. Er kann einen Gegenstand aufnehmen, tragen und/oder fallenlassen, bevor er zurückkehrt.

    Der Hund wird von Gegnern nicht verlangsamt. Er kann keine Einbahnstraßen nehmen oder Räume durchqueren, die einen Wurf erfordern. Er kann keine Gegenstände tragen, die die Bewegung verlangsamen.

    \omencantbedts
}

\omen{Girl}{Mädchen}{
    BEGLEITER

    Ein Mädchen.

    Gefesselt.

    Alleine.

    Du befreist sie!
}{
    Erhalte sofort 1 \sanity\ und 1 \know.

    Verliere 1 \sanity\ und 1 \know, wenn du die Mädchen-Karte verlierst.

    \omencantbedts
}

\omen{Holy Symbol}{Heiliges Symbol}{
    Ein Symbol der Ruhe in einer rastlosen Welt.
}{
    Erhalte sofort 2 \sanity.

    Verliere 2 \sanity, wenn du das Heilige Symbol verlierst.
}

\omen{Madman}{Ein Verrückter}{
    BEGLEITER

    Ein schäumender Irrer.
}{
    Erhalte sofort 2 \might\ und verliere 1 \sanity.

    Verliere 2 \might\ und erhalte 1 \sanity, wenn dein Begleiter dich verlässt.

    \omencantbedts
}

\omen{Mask}{Maske}{
    Eine düstere Maske, um deine Absichten zu verschleiern.
}{
    Einmal pro Zug kannst du mittels eines \sanityroll s versuchen, die Maske zu benutzen.

    \rolls
    \roll{4+}{Du kannst die Maske auf- oder absetzen.

    Wenn du die Maske aufziehst, erhalte 2 \know\ und verliere 2 \sanity.

    Wenn du die Maske absetzt, erhalte 2 \sanity\ und verliere 2 \know.}
    \roll{0-3}{Du kannst die Maske in dieser Runde nicht verwenden.}
    \erolls
}

\omen{Medallion}{Medallion}{
    Ein Medallion mit einem eingravierten Pentagram.
}{
    Du bist immun gegenüber den Effekten der Pen\-ta\-gram-Kammer, der Crypta und des Friedhofs (Pentagam Chamber, Crypt, Graveyard).
}

\omen{Ring}{Ring}{
    Ein abgenutzter Ring mit einer unleserlichen Inschrift.
}{
    Wenn du einen Gegner angreifst, der einen \emph{Sanity}-Wert besitzt, kannst du ihn mit einem \sanityroll\ anstelle eines \mightroll s angreifen. (Dein Gegner verteidigt sich dann mit einem \sanityroll\ und der Schaden wirkt statt physisch mental.)
}

\omen{Skull}{Schädel}{
    Ein rissiger Schädel mit fehlenden Zähnen.
}{
    Wenn du mentalen Schaden erleidest, kannst du ihn stattdessen komplett in physischen Schaden umwandeln.
}

\omen{Spear}{Speer}{
    WAFFE

    Eine vor Macht pulsierende Waffe.
}{
    Du kannst zwei zusätzliche Würfel (bis maximal 8 Würfel) werfen, wenn du eine \might\ Attacke mit dieser Waffe ausführst.

    \nootherweapon
}

\omen{Spirit Board}{Ouijabrett}{
    Eine Tafel mit Buchstaben und Zahlen, um mit den Toten zu sprechen.
}{
    Bevor du dich während deines Zuges bewegst, darfst du unter die oberste Karte des Zimmerkartenstapels schauen.

    Benutzt du das Ouijabrett, nachdem der \haunt\ offenbart wurde, kann der Verräter beliebige Monster ein Feld weiter in deine Richtung bewegen. (Wenn du der Verräter bist, musst du die Monster nicht bewegen.) Wenn es keinen Verräter gibt, rücken alle Monster 1 Feld zu dir.
}

\pagebreak

\section{Items / Gegenstände}

\input{ITEMS}



\end{document}