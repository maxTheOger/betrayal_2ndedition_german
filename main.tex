%%%%%%%%%%%%%%%%%%%%%%%%%%%%%%%%%%%%%%%%%
% Frequently Asked Questions
% LaTeX Template
% Version 1.0 (22/7/13)
%
% This template has been downloaded from:
% http://www.LaTeXTemplates.com
%
% Original author:
% Adam Glesser (adamglesser@gmail.com)
%
% License:
% CC BY-NC-SA 3.0 (http://creativecommons.org/licenses/by-nc-sa/3.0/)
%
%%%%%%%%%%%%%%%%%%%%%%%%%%%%%%%%%%%%%%%%%

\documentclass[11pt]{article}

\usepackage[margin=1in]{geometry} % Required to make the margins smaller to fit more content on each page
\usepackage[linkcolor=blue]{hyperref} % Required to create hyperlinks to questions from elsewhere in the document
\hypersetup{pdfborder={0 0 0}, colorlinks=true, urlcolor=blue} % Specify a color for hyperlinks
\usepackage{todonotes} % Required for the boxes that questions appear in
\usepackage{tocloft} % Required to give customize the table of contents to display questions
\usepackage{microtype} % Slightly tweak font spacing for aesthetics
\usepackage{palatino} % Use the Palatino font
\usepackage[utf8]{inputenc}

\setlength\parindent{0pt} % Removes all indentation from paragraphs

% Create and define the list of questions
\newlistof{questions}{faq}{\large List of Frequently Asked Questions} % This creates a new table of contents-like environment that will output a file with extension .faq
\setlength\cftbeforefaqtitleskip{4em} % Adjusts the vertical space between the title and subtitle
\setlength\cftafterfaqtitleskip{1em} % Adjusts the vertical space between the subtitle and the first question
\setlength\cftparskip{.3em} % Adjusts the vertical space between questions in the list of questions

% Create the command used for questions
\newcommand{\question}[1] % This is what you will use to create a new question
{
\refstepcounter{questions} % Increases the questions counter, this can be referenced anywhere with \thequestions
\par\noindent % Creates a new unindented paragraph
\phantomsection % Needed for hyperref compatibility with the \addcontensline command
\addcontentsline{faq}{questions}{#1} % Adds the question to the list of questions
\todo[inline, color=green!40]{\textbf{#1}} % Uses the todonotes package to create a fancy box to put the question
\vspace{1em} % White space after the question before the start of the answer
}

% Uncomment the line below to get rid of the trailing dots in the table of contents
%\renewcommand{\cftdot}{}

% Uncomment the two lines below to get rid of the numbers in the table of contents
%\let\Contentsline\contentsline
%\renewcommand\contentsline[3]{\Contentsline{#1}{#2}{}}


\newenvironment{tightcenter}{%
  \setlength\topsep{0pt}
  \setlength\parskip{0pt}
  \begin{center}
}{%
  \end{center}
}

\newcommand{\itemcard}[4]{
\begin{tightcenter}\subsection*{\expandafter\MakeUppercase\expandafter{#1} / \expandafter\MakeUppercase\expandafter{#2}}
\begin{bf}#3\end{bf}\end{tightcenter}
{ \parskip5pt #4  }
}


\newcommand{\omen}[4]{
\begin{tightcenter}\subsection{\expandafter\MakeUppercase\expandafter{#1} / \expandafter\MakeUppercase\expandafter{#2}}
\begin{bf}#3\end{bf}\end{tightcenter}
{ \parskip5pt #4

Mache nun einen Spukwurf (Haunt).}
}

\newcommand{\rolls}{\begin{itemize}
\itemsep-5pt}
\newcommand{\erolls}{\end{itemize}}
\newcommand{\roll}[2]{
\item [\bf #1] #2
}


\newcommand{\sanity}{geistige Gesundheit (Sanity)}
\newcommand{\might}{Macht (Might)}
\newcommand{\speed}{Geschwindigkeit (Speed)}
\newcommand{\know}{Wissen (Knowledge)}

\newcommand{\sanityroll}{\emph{Sanity}-Wurf}
\newcommand{\mightroll}{\emph{Might}-Wurf}
\newcommand{\speedroll}{\emph{Speed}-Wurf}
\newcommand{\knowroll}{\emph{Knowledge}-Wurf}

\newcommand{\itemcards}{\emph{Item}-Karten}


\newcommand{\nootherweapon}{Du kannst keine andere Waffe verwenden, während du diese benutzt.}

\newcommand{\haunt}{Spuk (Haunt)}


\newcommand{\mental}{mentale Eigenschaft}
\newcommand{\mentals}{mentale Eigenschaften}
\newcommand{\physical}{physische Eigenschaft}
\newcommand{\physicals}{physische Eigenschaften}
\newcommand{\physicalsn}{physischen Eigenschaften}

\newcommand{\discardcard}{Entferne diese Karte nach Verwendung aus dem Spiel.}
\newcommand{\discarditem}{Entferne diesen Gegenstand nach Verwendung aus dem Spiel.}


\newcommand{\omencantbedts}{Dieses Omen kann nicht fallen gelassen, gehandelt oder gestohlen werden.}


\begin{document}




\haunttitle{The Mummy Walks}{Die wandernde Mumie}

\introduction{
Staubschwaden ziehen in den Raum und ein Schatten legt sich über dein Herz. Du hörst einen deiner Freude
schreien, ein Geräusch aus Vergnügen und Entsetzen. Eine kalte, klamme Stimme lässt deinen Verstand
erschaudern. "Ich verlor meine Braut, viele Jahre bevor du denken kannst. Meine Tränen sind verstaubt,
aber meine Liebe ist immer noch so kräftig wie die Sonne. Jetzt ist meine Liebe für mich wiedergeboren. Es
gibt nichts mehr was uns beide trennen kann... und wenn du dich gegen mich stellst, werde ich deine Seele
aus deinem Körper reißen und sie gänzlich verschlingen."
}

\rightnow{

    \begin{itemize}
        \bitem Lege 2 Wissenswurfplättchen (Knowledge Roll tokens) beiseite.
        \bitem Der Abenteurer mit dem Mädchen (Girl) verliert es. Dieser Abenteurer verliert alle Bonuspunkte von
der Mädchenkarte (Girl card) und legt sie bei Seite. Der Verräter legt dann das Mädchenplättchen
(Girl token) in einen anderen Raum.
        \bitem Wenn ein Charakter eines Spielers den Raum mit dem Mädchenplättchen betritt, dann erhält dieser
Spieler die Mädchenkarte.
    \end{itemize}
}

\whatyouknowaboutthebadguys{
Der Verräter versucht die Mumie mit dem Mädchen zu verheiraten.
}

\youwhinwhen{
...die Mumie ins Reich des Todes verbannt wurde, bevor sie das Mädchen heiraten konnte.
}

\hauntsection{Wie wird die Mumie verbannt}

Wenn die Buchkarte noch nicht im Spiel ist durchsucht der Held der als nächstes in einem Raum mit dem
Omen Symbol läuft nach der Karte und nimmt sie.Danach wird der Stapel neu gemischt.
Du musst den wahren Namen der Mumie in dem Buch finden und sagen. Um dies zu schaffen musst du
folgende Schritte in dieser Reihenfolge erledigen. Du kannst pro Zug nur einen Schritt erledigen.
Um den wahren Namen der Mumie heraus zu bekommen, kannst du versuchen einen Wissenswürfelwurf
(Knowledge roll) von 6+ in den folgenden Räumen zu bestehen (auf die folgenden Art und Weisen):

    \begin{itemize}
        \bitem der Raum mit dem Sarkophag (untersuche die Hieroglyphen)
        \bitem das XXX Labor (Research Labor) (überfliege die Notizen des Archäologie Teams), oder
        \bitem die Bücherei (Library) (erforsche die Geschichte der Mumie).
    \end{itemize}

Wenn du Erfolg hattest, nehme dir eine Wissenswurfscheibe.
    \begin{itemize}
        \bitem An einem Zug nach dem du Namen erforscht hast, während du das Buch hattest, kannst du einen Wissenswürfelwurf von 6+ versuchen, um den Namen der Mumie herauszufinden. Wenn du Erfolg
hattest, nehme dir ein Wissenswurfplättchen.
        \bitem Sobald du zwei von diesen Plättchen hast, bringe das Buch in den Raum in dem sich die Mumie befindet. Jeder Held kann während du dort bist, versuchen einen Gesundheitswürfelwurf (Sanity roll) zu bestehen, um einen Bann zusprechen der sie für immer vertreibt.
    \end{itemize}

    Die Mumie ist imun gegen Geschwindigkeitsangriffe(Revolver,Dynamit..)

\outro{
Ein heißer trockener Wind flüstert durch den Raum, als du den altertümlichen Wälzer zuknallst. Die Mumie setzt das Schlurfen in deine Richtung fort, ihre Augen sind tote Höhlen der Verzweiflung. Gerade als ihre Hände deine Kehle umklammern, beginnen die Umwicklungen der Mumie zu bröckeln. Die Kreatur stöhnt immer mehr und mehr über ihren Körper, der zusammengedrückt und mit dem heißen Wind hinfort geweht wird. "Meine Braut... meine einzige Liebe... nicht... mehr...."
Als der letzte Rest der Mumie verschwunden ist, hört der Wind auf. Du bist allein.
}

\hauntsection{FAQ}
Was passiert, wenn der Verräter das Buch hat? Dann müssen es die anderen ihm abjagen.
Muss der gleiche Abenteurer die beiden Wissenswürfe machen? Nein
Wenn Kraft und Geschwindigkeit beide auf der untersten Stufe sind, kann die Mumie dann auch töten? Ja..

\twocolumn
\section{Omen}


\omen{Bite}{Biss}{
    Ein Knurren, der süße Duft von Tod. Schmerz. Dunkelheit. Fort.
}{
    \emph{Etwas} beißt dich, als du diese Karte ziehst. Der Spieler zu deiner Rechten wirft einen \mightroll\ mit 4 Würfeln für das mysteriöse \emph{Etwas}, bevor es in der Dunkelheit verschwindet. Du verteidigst dich normal, indem du einen \mightroll\ ausführst.

    \omencantbedts
}

\omen{Book}{Buch}{
    Ein Tagebuch oder Laborprotokolle? Ein antikes Manuskript oder moderner Bestseller?
}{
    Erhalte sofort 2 \know.

    Verliere 2 \know, wenn du das Buch verlierst.
}

\omen{Crystal Ball}{Kristallkugel}{
    Trübe Bilder erscheinen im Inneren des Glases.
}{
    Ist der \haunt\ offenbart, kannst du mittels eines \knowroll s während deines Zuges in die Kristallkugel schauen:

        \rolls
        \roll{4+}{Du siehst die Wahrheit. Durchsuche das Gegenstandsdeck (Items) oder das Ereignisdeck (Events) und wähle eine Karte aus. Mische die Karten und lege deine Karte oben auf.}
        \roll{1-3}{Du wendest deinen Blick ab. Verliere 1 \sanity.}
        \roll{0}{Du starrst direkt in die Hölle. Verliere 2 \sanity.}
        \erolls
}

\omen{Dog}{Hund}{
    BEGLEITER

    Dieser räudige Hund scheint freundlich. Du hoffst, dass er das auch wirklich ist.
}{
    Erhalte sofort 1 \might\ und 1 \sanity.

    Verliere 1 \might\ und 1 \sanity, wenn du die Aufsicht über den Hund abgibst.

    Verwende einen kleinen Monsterspielstein, der den Hund darstellt und lege ihn in deinen Raum. (Verwende eine Monsterfarbe die noch nicht auf dem Spielfeld liegt.) Einmal während deines Zuges kann der Hund zu einem maximal 6 Felder entfernten, schon entdeckten Raum laufen, um danach direkt wieder zurückzukehren. Er kann einen Gegenstand aufnehmen, tragen und/oder fallenlassen, bevor er zurückkehrt.

    Der Hund wird von Gegnern nicht verlangsamt. Er kann keine Einbahnstraßen nehmen oder Räume, die einen Wurf erfordern durchqueren. Er kann keine Gegenstände tragen, die die Bewegung verlangsamen.

    \omencantbedts
}

\omen{Girl}{Mädchen}{
    BEGLEITER

    Ein Mädchen.

    Gefesselt.

    Alleine.

    Du befreist sie!
}{
    Erhalte sofort 1 \sanity\ und 1 \know.

    Verliere 1 \sanity\ und 1 \know, wenn du die Mädchen-Karte verlierst.

    \omencantbedts
}

\omen{Holy Symbol}{Heiliges Symbol}{
    Ein Symbol der Ruhe in einer rastlosen Welt.
}{
    Erhalte sofort 2 \sanity.

    Verliere 2 \sanity, wenn du das Heilige Symbol verlierst.
}

\omen{Madman}{Ein Verrückter}{
    BEGLEITER

    Ein schäumender Irrer.
}{
    Erhalte sofort 2 \might\ und verliere 1 \sanity.

    Verliere 2 \might\ und erhalte 1 \sanity, wenn du diese Karte verlierst.

    \omencantbedts
}

\omen{Mask}{Maske}{
    Eine düstere Maske, um deine Absichten zu verschleiern.
}{
    Einmal pro Zug kannst du mittels eines \sanityroll s versuchen, die Maske zu benutzen.

    \rolls
    \roll{4+}{Du kannst die Maske auf- oder absetzen.

    Wenn du die Maske aufziehst, erhalte 2 \know\ und verliere 2 \sanity.

    Wenn du die Maske absetzt, erhalte 2 \sanity\ und verliere 2 \know.}
    \roll{0-3}{Du kannst die Maske in dieser Runde nicht verwenden.}
    \erolls
}

\omen{Medallion}{Medallion}{
    Ein Medallion mit einem eingravierten Pentagram.
}{
    Du bist immun gegenüber den Effekten der Pentagramm-Kammer, der Crypta und des Friedhofs (Pentagam Chamber, Crypt, Graveyard).
}

\omen{Ring}{Ring}{
    Ein abgenutzter Ring mit einer unleserlichen Inschrift.
}{
    Wenn du einen Gegner angreifst, der einen \emph{Sanity}-Wert besitzt, kannst du ihn mit einem \sanityroll\ anstelle eines \mightroll s angreifen. (Dein Gegner verteidigt sich dann mit einem \sanityroll\ und der Schaden wirkt statt physisch mental.)
}

\omen{Skull}{Schädel}{
    Ein rissiger Schädel mit fehlenden Zähnen.
}{
    Wenn du mentalen Schaden erleidest, kannst du ihn stattdessen komplett in physischen Schaden umwandeln.
}

\omen{Spear}{Speer}{
    WAFFE

    Eine vor Macht pulsierende Waffe.
}{
    Du kannst zwei zusätzliche Würfel (bis maximal 8 Würfel) werfen, wenn du eine \might\ Attacke mit dieser Waffe ausführst.

    \nootherweapon
}

\omen{Spirit Board}{Ouijabrett}{
    Eine Tafel mit Buchstaben und Zahlen, um mit den Toten zu sprechen.
}{
    Bevor du dich während deines Zuges bewegst, darfst du unter die oberste Karte des Zimmerkartenstapels schauen.

    Benutzt du das Ouijabrett, nachdem der \haunt\ offenbart wurde, kann der Verräter beliebige Monster ein Feld weiter in deine Richtung bewegen. (Wenn du der Verräter bist, musst du die Monster nicht bewegen.) Wenn es keinen Verräter gibt, rücken alle Monster 1 Feld zu dir.
}

\pagebreak

\section{Items / Gegenstände}



\itemcard{Lucky Stone}{Glücksstein}{
    Ein glattes, gewöhnlich aussehendes Stück Felsgestein. Du fühlst, dass er dir Glück bringen wird.
}{
    Nachdem du gewürfelt hast, kannst du an diesem Stein reiben, um einen oder mehrere deiner Würfel nocheinmal zu würfeln.

    \discarditem
}

\itemcard{Dark Dice}{Dunkle Würfel}{
    Wie stehts um dein Glück?
}{
    Einmal pro Runde kannst du 3 Würfel werfen:
    \rolls
    \roll{6}{Gehe sofort zu irgendeinem Entdecker, aber nicht zum Verräter}
    \roll{5}{Verschiebe einen anderen Entdecker aus deinem Raum in einen angrenzenden Raum.}
    \roll{4}{+1 \physical}
    \roll{3}{Bewege dich ohne Bewegungskosten sofort in einen angrenzenden Raum.}
    \roll{2}{+1 \mental}
    \roll{1}{Ziehe eine Ereigniskarte.}
    \roll{0}{Reduziere alle deine Eigenschaften auf den ersten Schritt über dem Schädel und entferne die Karte aus dem Spiel.}
    \erolls
}


\itemcard{Medical Kit}{Medizintasche}{
    Ein Arztkoffer, aber die wichtigsten Medikamente sind schon geplündert worden.
}{
    Einmal pro Zug kannst du versuchen, dich oder einen anderen Entdecker im gleichen Raum mit einem Wissenswurf zu heilen.

    \rolls
    \roll{8+}{+3 für \physicals}
    \roll{6-7}{+2 für \physicals}
    \roll{4-5}{+1 für eine \physical}
    \roll{0-3}{Es passiert nichts.}
    \erolls

    Du kannst eine Eigenschaft maximal bis zu ihrem Startwert heilen.
}

\itemcard{Idol}{Idol}{
    Vielleicht hat es dich für einen höheren Zweck auserwählt. Möglicherweise Menschenopfer.
}{
    Einmal pro Zug kannst du an dem Idol reiben, bevor du einen Charakter-, Kampf- oder Ereigniswurf würfelst, um 2 zusätzliche Würfel (bis maximal 8 Würfel) einzusetzen. Jedes mal verlierst du 1 \sanity.

}

\itemcard{Axe}{Akt}{
    WAFFE

    Sehr scharf.
}{
    Wenn du diese Waffe verwendest, kannst einen zusätzlichen Würfel (bis maximal 8 Würfel) für einen \mightroll\ benutzen.

    \nootherweapon
}

\itemcard{Adrenaline Shot}{Adrenalinespritze}{
    Eine Spritze, die eine seltsame, fluosziierende Flüssigkeit enthält.
}{
    Du kannst diesen Gegensstand einsetzen, bevor du einen Charakterwurf würfelst, um zum Wurfergebnis 4 hinzuzufügen.

    \discarditem
}

\itemcard{Angel Feather}{Engelsfeder}{
    Eine makellose Feder schwirrt auf deine Hand.
}{
    Wenn du einen Wurf jedweder Art würfeln musst, kannst du stattdessen eine beliebige Zahl von 0 bis 8 direkt als Würfelergebnis verwenden.

    \discarditem
}

\itemcard{Dynamite}{Dynamit}{
    Die Zündschnur brennt \emph{noch} nicht.
}{
    Anstatt zu Attakieren kannst du Dynamit durch eine Tür in einen benachbarten Raum werfen. Jeder Entdecker und jedes Monster mit Macht und Geschwindigkeitsmerkmalen in diesem Raum muss einen \speedroll\ versuchen:

    \rolls
    \roll{5+}{Kein Schaden}
    \roll{0-4}{Du nimmst 4 Schaden in \physicalsn}
    \erolls

    \discarditem
}

\itemcard{Pickpocket's Gloves}{Handschuhe des Diebes}{
    Sich selbst zu helfen war noch nie so einfach.
}{
    Wenn du zusammen mit einem anderen Entdecker in einem Raum bist, kannst du ihm einen Gegenstand klauen.

    \discarditem
}

\itemcard{Revolver}{Revolver}{
    WAFFE

    Eine alte, wirksam scheinende Waffe.
}{
    Du kannst den Revolver verwenden, um mit einem \speedroll\ anstelle eines \mightroll s anzugreifen. (Dein Gegner verteidigt sich dann mit \speed\ und nimmt physischen Schaden.)

    Würfle mit einem zusätzlichen Würfel bei deinem Angriffswurf.

    Mit dem Revolver kannst du im gleichen Raum oder in gerader Sichtlinie durch mehrere Türen treffen. Wenn du jemanden in einem anderen Raum angreifst, nimmst du im Falle einer Niederlage keinen Schaden.

     \nootherweapon

}

\itemcard{Rabbit's Foot}{Hasenfuß}{
    Der Hase hatte wohl kein Glück.
}{
    Einmal pro Zug kannst du \emph{einen} Würfel erneut werfen. Der zweite Wurf zählt.
}

\itemcard{Puzzle Box}{Rätsel-Schachtel}{
    Irgendwie muss man die doch öffnen können.
}{
    Einmal pro Zug kannst du versuchen die Kiste mit einem \knowroll\ zu öffnen.

    \rolls
    \roll{+6}{Du öffnest die Box. Ziehe zwei \itemcards\ und entferne die Schachtel aus dem Spiel.}
    \roll{0-5}{Du schaffst es nicht die Box zu öffnen.}
    \erolls
}

\itemcard{Blood Dagger}{Blutdolch}{
    WAFFE

    Eine fiese Waffe. Nadeln und Schläuche winden sich aus dem Griff und stechen direkt in deine Venen.
}{
    Du kannst drei zusätzliche Würfel (bis maximal 8 Würfel) werfen, wenn du eine \might\ Attacke mit dieser Waffe ausführst. Du verlierst dabei 1 \speed.

    \nootherweapon

    Du kannst diesen Dolch nicht handeln oder fallenlassen. Wenn er gestohlen wird, nimmst du 2 Schaden in \physicalsn.
}


\itemcard{Bell}{Glocke}{
    Eine Messingglocke mit schallendem Gong.
}{
    Erhalte sofort 1 \sanity.

    Verliere 1 \sanity, wenn du die Glocke verlierst.

    Ist der \haunt\ offenbart, kannst du mittels eines \sanityroll s während deines Zuges die Glocke läuten:

    \rolls
    \roll{5+}{Rücke beliebige, aber frei bewegliche Entdecker einen Raum näher zu dir.}
    \roll{0-4}{Der Verräter darf beliebig viele Monster einen Raum näher zu dir rücken. (Wenn du der Verräter bist, gilt dies nicht.) Wenn es keinen Verräter gibt, bewegen sich alle Monster in deine Richtung.}
    \erolls
}


\itemcard{Armor}{Rüstung}{
    Es ist nur eine Requisite vom Renaissancefest, aber immerhin ist sie aus Metall.
}{
    Jedes Mal (nicht nur einmal pro Runde) wenn du physischen Schaden nimmst, erhälst du einen Schaden weniger.

    Dieser Gegenstand kann nicht gestohlen werden.
}

\itemcard{Music Box}{Spieluhr}{
    Eine handgefertigte Antiquität. Sie spielt eine gespenstische Melodie, die dir nicht mehr aus dem Kopf geht.
}{
    Einmal pro Zug kannst du die Spieluhr öffnen oder schließen.

    Während die Spieluhr offen ist, muss jeder Entdecker oder jedes Monster mit \emph{Sanity}-Merkmal, dass den Raum der Spieluhr betritt oder seinen Zug darin beginnt, einen \sanityroll\ werfen. Schafft er keine 4+, ist er/es für den Rest des Zuges hypnotisiert.

    Wenn ein Entdecker oder Monster, das die Spieluhr bei sich trägt, hypnotisiert wird, lässt er/es die Spieluhr fallen. Ist die Spieluhr währenddessen offen, bleibt sie offen.

}

\itemcard{Healing Salve}{Heilsalbe}{
    Eine klebrige Paste in einer flachen Schale.
}{
    Du kannst dich oder einen anderen lebenden Entdecker im selben Raum mit dieser Salbe behandeln. Hebe entweder den \might\ oder den Geschwindigkeitswert (Speed) des geheilten Entdeckers auf seinen Startwert an.

    \discarditem
}


\itemcard{Smelling Salts}{Riechsalz}{
    Wow, das haut rein.
}{
    Du kannst dir oder einen anderen lebenden Entdecker im selben Raum das Riechsalz unter die Nase halten. Hebe den Wissenswert (Knowledge) des entsprechenden Entdeckers auf seinen Startwert an.

    \discarditem
}


\itemcard{Sacrificial Dagger}{Opferdolch}{
    Ein gewundener Sporn aus Eisen, überzogen mit mysteriösen Symbolen und triefend vor Blut.
}{
    Du kannst drei zusätzliche Würfel (bis maximal 8 Würfel) werfen, wenn du eine \might\ Attacke mit dieser Waffe ausführst, aber vorher musst du einen \knowroll werfen:

    \rolls
    \roll{6+}{Kein Effekt.}
    \roll{3-5}{Nehme einen Schaden in einer \mental.}
    \roll{0-2}{Der Dolch winded sich in deiner Hand. Nehme zwei Schaden in \physicalsn. Du kannst in diesem Zug nicht mehr angreifen.}
    \erolls
}

\itemcard{CANDLE}{Kerze}{
    Sie lässt die Schatten tanzen - wenigstens hoffst du sie macht genau das.
}{
    Wenn du eine Ereigniskarte ziehst, darfst du jeden das Ereignis betreffenden Charakterwurf (Might, ...) mit einem zusätzlichen Würfel bestreiten.

    Wenn du die Glocke (Bell), das Buch (Book) und die Kerze besitzt, erhalte +2 von jeder Eigenschaft (Might, ...). Sobald du einen dieser Gegenstände verlierst, verliere 2 von jeder Eigenschaft.
}

\itemcard{Bottle}{Flasche}{
    Ein trüber Flakon, in dem eine schwarze Flüssigkeit schwimmt.
}{
    Ist der \haunt\ offenbart, kannst du während deines Zuges aus der Flasche trinken. Würfle mit drei Würfeln:


    \rolls
    \roll{6}{Versetze deine Figur in einen beliebigen Raum.}
    \roll{5}{Erhalte 2 \might\ und 2 \speed.}
    \roll{4}{Erhalte 2 \know\ und 2 \sanity.}
    \roll{3}{Erhalte 1 \know\ und verliere 1 \might.}
    \roll{2}{Verliere 2 \know\ und 2 \sanity.}
    \roll{1}{Verliere 2 \might\ und 2 \speed.}
    \roll{0}{Verliere 2 von jeder Eigenschaft.}
    \erolls
}

\itemcard{Amulet Of The Ages}{Amulett der Zeitalter}{
    Altes Silber und eingelassene Juwelen, Inschriften von Segnungen.
}{
    Erhalte sofort 1 von jeder Eigenschaft.

    Verliere 3 von jeder Eigenschaft, wenn du das Amulett verlierst.
}



\end{document}