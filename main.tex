%%%%%%%%%%%%%%%%%%%%%%%%%%%%%%%%%%%%%%%%%
% Frequently Asked Questions
% LaTeX Template
% Version 1.0 (22/7/13)
%
% This template has been downloaded from:
% http://www.LaTeXTemplates.com
%
% Original author:
% Adam Glesser (adamglesser@gmail.com)
%
% License:
% CC BY-NC-SA 3.0 (http://creativecommons.org/licenses/by-nc-sa/3.0/)
%
%%%%%%%%%%%%%%%%%%%%%%%%%%%%%%%%%%%%%%%%%

\documentclass[10pt,a4paper,oneside,ngerman]{article}

\usepackage[ngerman]{babel}

\usepackage[margin=1in]{geometry} % Required to make the margins smaller to fit more content on each page
%\usepackage[linkcolor=blue]{hyperref} % Required to create hyperlinks to questions from elsewhere in the document
%\hypersetup{pdfborder={0 0 0}, colorlinks=true, urlcolor=blue} % Specify a color for hyperlinks
%\usepackage{todonotes} % Required for the boxes that questions appear in
%\usepackage{tocloft} % Required to give customize the table of contents to display questions
\usepackage{microtype} % Slightly tweak font spacing for aesthetics
%\usepackage{palatino} % Use the Palatino font
\usepackage[utf8]{inputenc}

\setlength\parindent{0pt} % Removes all indentation from paragraphs



\newenvironment{tightcenter}{%
  \setlength\topsep{0pt}
  \setlength\parskip{0pt}
  \begin{center}
}{%
  \end{center}
}

\newcommand{\itemcard}[4]{
\begin{tightcenter}\subsection{\expandafter\MakeUppercase\expandafter{#1} / \expandafter\MakeUppercase\expandafter{#2}}
\begin{bf}#3\end{bf}\end{tightcenter}
{ \parskip5pt #4  }
}

\newcommand{\event}[4]{
\begin{tightcenter}\subsection{\expandafter\MakeUppercase\expandafter{#1} / \expandafter\MakeUppercase\expandafter{#2}}
\begin{bf}#3\end{bf}\end{tightcenter}
{ \parskip5pt #4  }
}



\newcommand{\omen}[4]{
\begin{tightcenter}\subsection{\expandafter\MakeUppercase\expandafter{#1} / \expandafter\MakeUppercase\expandafter{#2}}
\begin{bf}#3\end{bf}\end{tightcenter}
{ \parskip5pt #4

Mache nun einen Spukwurf (Haunt).}
}

\newcommand{\rolls}{\begin{itemize}
\itemsep-5pt}
\newcommand{\erolls}{\end{itemize}}
\newcommand{\roll}[2]{
\item [\bf #1] #2
}


\newcommand{\sanity}{geistige Gesundheit (Sanity)}
\newcommand{\might}{Macht (Might)}
\newcommand{\speed}{Geschwindigkeit (Speed)}
\newcommand{\know}{Wissen (Knowledge)}

\newcommand{\sanityroll}{\emph{Sanity}-Wurf}
\newcommand{\mightroll}{\emph{Might}-Wurf}
\newcommand{\speedroll}{\emph{Speed}-Wurf}
\newcommand{\knowroll}{\emph{Knowledge}-Wurf}

\newcommand{\itemcards}{\emph{Item}-Karten}
\newcommand{\itemcardd}{\emph{Item}-Karte}
\newcommand{\eventcard}{\emph{Event}-Karte}
\newcommand{\omencard}{\emph{Omen}-Karte}
\newcommand{\omencards}{\emph{Omen}-Karten}


\newcommand{\nootherweapon}{Du kannst keine andere Waffe verwenden, währ\-end du diese benutzt.}

\newcommand{\haunt}{Spuk (Haunt)}


\newcommand{\mental}{mentale Eigenschaft}
\newcommand{\mentals}{mentale Eigenschaften}
\newcommand{\physical}{physische Eigenschaft}
\newcommand{\physicals}{physische Eigenschaften}
\newcommand{\physicalsn}{physischen Eigenschaften}

\newcommand{\discardcard}{Entferne diese Karte nach Verwendung aus dem Spiel.}
\newcommand{\discarditem}{Entferne diesen Gegenstand nach Verwendung aus dem Spiel.}


\newcommand{\omencantbedts}{Dieses Omen kann nicht fallen gelassen, gehandelt oder gestohlen werden.}


\newcommand{\room}[2]{\expandafter\MakeUppercase\expandafter{#1} / \expandafter\MakeUppercase\expandafter{#2}}

\newcommand{\iroom}[2]{\item \room{#1}{#2}}

\begin{document}

\tableofcontents


\survival{1}{The Mummy Walks}{Die wandernde Mumie}

\introduction{
Staubschwaden ziehen in den Raum und ein Schatten legt sich über dein Herz. Du hörst einen deiner Freude
schreien, ein Geräusch aus Vergnügen und Entsetzen. Eine kalte, klamme Stimme lässt deinen Verstand
erschaudern. ``Ich verlor meine Braut, viele Jahre bevor du denken kannst. Meine Tränen sind verstaubt,
aber meine Liebe ist immer noch so kräftig wie die Sonne. Jetzt ist meine Liebe für mich wiedergeboren. Es
gibt nichts mehr was uns beide trennen kann... und wenn du dich gegen mich stellst, werde ich deine Seele
aus deinem Körper reißen und sie gänzlich verschlingen.''
}

\rightnow{

    \begin{itemize}
        \bitem Lege 2 dreieckige \chips{Knowledge Roll}{\knowroll} beiseite.
        \bitem Der Verräter verliert das Mädlchen (Girl) und die von ihr verliehenen Boni. Stattdessen legt er einen kleinen rosanen Monsterchip (der das Mädchen darstellt) in irgendeinen Raum im selben Stockwerk, in dem der Spuk offenbart wurde, jedoch mindestens 5 Felder von der Mumie entfernt. Gibt es keinen Raum, der mindestens 5 Felder entfernt ist, plaziert er den Chip so weit wie möglich entfernt.
        \bitem Wenn ein Entdecker den Raum mit dem Mädchenchip betritt, dann erhält dieser Spieler die Mädchenkarte.
    \end{itemize}
}

\whatyouknowaboutthebadguys{
Der Verräter versucht die Mumie mit dem Mädchen zu verheiraten.
}

\youwhinwhen{
...die Mumie zurück ins Reich des Todes verbannt wurde, bevor sie das Mädchen heiraten konnte.
}

\hauntsection{Wie die Mumie verbannt wird}

  \begin{itemize}
        \bitem Wenn die Buchkarte (Book) noch nicht im Spiel ist, durchsucht der Held, der als nächstes einen Raum mit Omensymbol entdeckt, den Omenstapel nach dem Buch und nimmt sie. Danach wird der Stapel neu gemischt.
        \bitem Du musst den wahren Namen der Mumie in dem Buch finden und aussprechen. Um dies zu schaffen, musst du folgende Schritte in dieser Reihenfolge erledigen. Jeder Held kann in seinem Zug nur einen der Schritte erledigen.

\newpage
        \begin{enumerate}
            \item Um den wahren Namen der Mumie heraus zu bekommen, kannst du versuchen einen \knowroll\ von 6+ in den folgenden Räumen zu bestehen:

            \begin{itemize}
                \bitem Untersuche die Hieroglyphen im Raum mit dem Sarkophag oder
                \bitem überfliege die Notizen des Archäologie Teams im Forschungslabor (Research Laboratory) oder
                \bitem erforsche die Geschichte der Mumie in der Bücherei (Library).
            \end{itemize}
            Wenn du Erfolg hattest, nehme dir einen \chipe{Knowledge Roll}.

            \item Wurde der Name entdeckt, kann der Held, der das Buch besitzt, (frühstens im darauf folgenden Zug) einen \knowroll\ von 6+ versuchen, um den Namen der Mumie nachzuschlagen und den Spruch zu lernen, der zu ihrer Verbannung führt. Wenn du Erfolg hattest, nehme dir einen \chipe{Knowledge Roll}.


            \item Sobald die Helden zwei von diesen Plättchen haben, muss ein Held das Buch in den Raum mit der Mumie bringen. Jeder Held, der mit der Mumie und dem Buch im selben Raum ist, kann die Mumie durch Aussprechen des Zauberspruchs verbannen, indem er  die Mumie mittels einer \sanity-Attacke besiegt.
        \end{enumerate}

    \bitem Die Mumie ist imun gegen Geschwindigkeitsangriffe (Revolver, Dynamit..)

    \end{itemize}

\outro{
Ein heißer trockener Wind flüstert durch den Raum, als du den altertümlichen Wälzer zuknallst. Die Mumie setzt das Schlurfen in deine Richtung fort, ihre Augen sind tote Höhlen der Verzweiflung. Gerade als ihre Hände deine Kehle umklammern, beginnen die Umwicklungen der Mumie zu bröckeln. Die Kreatur stöhnt immer mehr und mehr über ihren Körper, der zusammengedrückt und mit dem heißen Wind hinfort geweht wird. "Meine Braut... meine einzige Liebe... nicht... mehr...."
Als der letzte Rest der Mumie verschwunden ist, hört der Wind auf. Du bist allein.
}

\hauntsection{FAQ}

\begin{itemize}
    \bitem Was passiert, wenn der Verräter das Buch hat? Dann müssen es die anderen ihm abjagen.
    \bitem Muss der gleiche Abenteurer die beiden Wissenswürfe machen? Nein.
    \bitem Wenn Kraft und Geschwindigkeit beide auf der untersten Stufe sind, kann die Mumie dann auch töten? Ja.
\end{itemize}

\pagebreak

%\twocolumn
\section{Rooms / Räume}


\begin{itemize}
    \parskip-4pt
    \iroom{Abandoned Room}{Stillgelegter Raum}
    \iroom{Attic}{Dachboden}
    \iroom{Balcony}{Balkon}
    \iroom{Ballroom}{Ballzimmer.}
    \iroom{Basement Landing}{Landeplatz im Keller (Kohleschütte)}
    \iroom{Bedroom}{Schlafzimmer}
    \iroom{Bloody Room}{Blutiges Zimmer}
    \iroom{Catacombs}{Katakomben}
    \iroom{Chapel}{Kapelle}
    \iroom{Charred Room}{verbrannter Raum}
    \iroom{Chasm}{Kluft}
    \iroom{Coal Chute}{Kohleschacht}
    \iroom{Collapsed Room}{Eingestürztes Zimmer}
    \iroom{Conservatory}{Wintergarten}
    \iroom{Creaky Hallway}{Knarrender Hausflur}
    \iroom{Crypt}{Crypta}
    \iroom{Dining Room}{Speisesaal}
    \iroom{Dusty Hallway}{Verstaubter Hausflur}
    \iroom{Furnace Room}{Ofenraum}
    \iroom{Gallery}{Gallerie}
    \iroom{Game Room}{Spielzimmer}
    \iroom{Gardens}{Gärten}
    \iroom{Graveyard}{Friedhof}
    \iroom{Gymnasium}{Turnraum}
    \iroom{Junk Room}{Gerümpelzimmer}
    \iroom{Kitchen}{Küche}
    \iroom{Larder}{Lagerraum}
    \iroom{Library}{Bibliothek}
    \iroom{Master Bedroom}{Herrenschlafzimmer}
    \iroom{Mystic Elevator}{Mystischer Aufzug}
    \iroom{Operating Laboratory}{Operationssaal}
    \iroom{Organ Room}{Orgelzimmer}
    \iroom{Patio}{Patio, Innenhof}
    \iroom{Pentagram Camber}{Pentagramkammer}
    \iroom{Research Laboratory}{Forschungslabor.}
    \iroom{Servants' Quarters}{Quatier des Dieners}
    \iroom{Stairs From Basement}{Kellertreppe}
    \iroom{Statuary Corridor}{Bildhauerkorridor}
    \iroom{Storeroom}{Abstellraum}
    \iroom{Tower}{Turm}
    \iroom{Underground Lake}{Untergrundsee}
    \iroom{Upper Landing}{Oberer Treppenabsatz}
    \iroom{Vault}{Tresorraum}
    \iroom{Wine Cellar}{Weinkeller}
\end{itemize}



\input{rooms}

\twocolumn
\section{Events / Ereignisse}



\event{A Moment Of Hope}{Ein Moment der Hoffnung}{
    Irgendetwas in diesem Raum fühlt sich seltsam richtig an. Etwas widersteht dem Bösen des Hauses.
}{
    Plaziere einen \emph{Blessing}-Chip (Segnung) in diesem Zimmer.

    Jeder Held darf bei jedem Charakterwurf (Might, ...) in diesem Zimmer einen zusätzlichen Würfel verwenden.
}

\event{Angry Being}{Böses Wesen}{
    Es kommt aus dem Schleim an der Wand neben dir hervor.
}{
    Du musst einen \speedroll versuchen:
    \rolls
    \roll{5+}{Du kommst davon. Erhalte 1 \speed .}
    \roll{2-4}{Erhalte 1 mentalen Schaden.}
    \roll{0-1}{Erhalte 1 mentalen und 1 pysischen Schaden.}
    \erolls
}

\event{Bloody Vision}{Blutige Vision}{
    Die Wände in diesem Raum sind getränkt in  Blut. Blut tropft von der Decke, fließt die Wände herunter, über Schränke und Möbel und auf deine Schuhe. Im nächsten Augenblick ist es fort.
}{
    Du musst einen \sanityroll\ versuchen:
    \rolls
    \roll{4+}{Du festigst deinen Geist. Erhalte 1 \sanity.}
    \roll{2-3}{Verliere 1 \sanity.}
    \roll{0-1}{Wenn sich ein Entdecker oder Monster in deinem oder einem angrenzenden Zimmer befinden, musst du ihn/sie/es angreifen (wenn du kannst). Wenn möglich, wähle denjenigen mit der niedrigsten \might.}
    \erolls
}


\event{Burning Man}{Brennender Mann}{
    Ein brennender Mann rennt durch den Raum. Seine Haut schlägt Blasen und zerfällt. Ein glutroter Schädel verbleibt, schlägt auf dem Boden auf, rollt und verschwindet.
}{
    Du musst einen \sanityroll\ versuchen:
    \rolls
    \roll{4+}{Du spürst Wärme unter deiner Haut, bist aber ansonsten okay. Erhalte 1 \sanity.}
    \roll{2-3}{Raus, raus, du musst hier raus! Setze deinen Entdecker in die Eingangshalle.}
    \roll{0-1}{Du gehst in Flammen auf. Nehme 1 physischen Schaden. Dann nehme 1 mentalen Schaden als du die Flammen ausklopfst.}
    \erolls
}

\event{Closet Door}{Schranktür}{
    Die Schranktür dort ist offen... nur einen Spalt. Darin muss sich etwas befinden.
}{
    Lege den Closet-Chip (Schrank) in dieses Zimmer.

    Einmal während seines Zuges, kann ein Entdecker zwei Würfel werfen um den Schrank zu öffnen.

    \rolls
    \roll{4}{Ziehe eine \itemcardd .}
    \roll{2-3}{Ziehe eine \eventcard .}
    \roll{0-1}{Ziehe eine \eventcard\ und entferne den Closet-Chip.}
    \erolls

}

\event{Creepy Crawlies}{Gruselige Krabbler}{
    Eintausend Käfer stürzen sich aus deiner Haut, kommen unter deiner Kleidung und aus deinen Haaren hervor.
}{
    Du musst einen \sanityroll\ versuchen:
    \rolls
    \roll{5+}{Du blinzelst und sie sind weg. Erhalte 1 \sanity.}
    \roll{1-4}{Verliere 1 \sanity.}
    \roll{0}{Verliere 2 \sanity.}
    \erolls

}


\event{Creepy Puppet}{Gruselige Puppe}{
    Du siehst eine dieser Puppen, bei denen dir die Haare zu Berge stehen. Sie springt dich mit einem kleinen Speer in den Händen an.
}{
    Der Spieler zu deiner Rechten wirft einen \mightroll\ mit 4 Würfeln für die Puppe. Du verteidigst dich normal, indem du einen \mightroll\ ausführst.

    Wenn du Schaden davonträgst, erhält der Entdecker mit dem Speer (Spear) 2 \might\ (es sei denn du besitzt den Speer).
}


\event{Debris}{Schutt}{
    Mörtel fällt von den Wänden und der Decke.
}{
    Du musst einen \speedroll\ versuchen:
    \rolls
    \roll{+3}{Du weichst aus. Erhalte 1 \speed.}
    \roll{1-2}{Du bist unter Schutt vergraben. Ńehme 1 \emph{Würfel} physischen Schaden.}
    \roll{0}{Du bist unter Schutt vergraben. Ńehme 2 \emph{Würfel} physischen Schaden.}
    \erolls

    Wenn du unter Schutt begraben bist, behalte diese Karte. Du kannst nichts machen bis du befreit wurdest. Einmal pro Zug kann ein Entdecker einen \mightroll\ versuchen um dich zu befreien. (Du kannst diesen Wurf ebenfalls versuchen.) 4+ befreit. Nach drei erfolglosen Versuchen, befreist du dich in deinem folgenden Zug automatisch und kannst normal agieren.
}

\event{Disquieting Sounds}{Beunruhigende Geräusche}{
    Das Geschrei eines Babys, einsam und verlassen.

    Ein Entsetzensschrei.

    Das Kracken zerbrechenden Glases.

    Dann: Stille.
}{
    Würfle mit 6 Würfeln. Wenn du genauso viele oder mehr Augen würfelst, als Omen aufgedeckt wurden, erhälst du 1 \sanity.
    Falls nicht, nehme 1 \emph{Würfel} geistigen Schaden.
}

\event{Drip ... Drip ... Drip ...}{Tropf ... Tropf ... Tropf ...}{
    Ein rhythmisches Geräusch macht dich verrückt.
}{
    Lege einen Drip-Chip (Tröpfeln) in diesen Raum.

    Jeder Entdecker rollt bei jedem Charakterwurf (Might, ...) in diesem Raum mit einem Würfel weniger.
}

\pagebreak
\pagebreak

\twocolumn
\section{Omen}

\input{omen/crystalball}

\pagebreak

\section{Items / Gegenstände}



\itemcard{Lucky Stone}{Glücksstein}{
    Ein glattes, gewöhnlich aussehendes Stück Felsgestein. Du fühlst, dass er dir Glück bringen wird.
}{
    Nachdem du gewürfelt hast, kannst du an diesem Stein reiben, um einen oder mehrere deiner Würfel nocheinmal zu würfeln.

    \discarditem
}

\itemcard{Dark Dice}{Dunkle Würfel}{
    Wie stehts um dein Glück?
}{
    Einmal pro Runde kannst du 3 Würfel werfen:
    \rolls
    \roll{6}{Gehe sofort zu irgendeinem Entdecker, aber nicht zum Verräter}
    \roll{5}{Verschiebe einen anderen Entdecker aus deinem Raum in einen angrenzenden Raum.}
    \roll{4}{+1 \physical}
    \roll{3}{Bewege dich ohne Bewegungskosten sofort in einen angrenzenden Raum.}
    \roll{2}{+1 \mental}
    \roll{1}{Ziehe eine Ereigniskarte.}
    \roll{0}{Reduziere alle deine Eigenschaften auf den ersten Schritt über dem Schädel und entferne die Karte aus dem Spiel.}
    \erolls
}


\itemcard{Medical Kit}{Medizintasche}{
    Ein Arztkoffer, aber die wichtigsten Medikamente sind schon geplündert worden.
}{
    Einmal pro Zug kannst du versuchen, dich oder einen anderen Entdecker im gleichen Raum mit einem Wissenswurf zu heilen.

    \rolls
    \roll{8+}{+3 für \physicals}
    \roll{6-7}{+2 für \physicals}
    \roll{4-5}{+1 für eine \physical}
    \roll{0-3}{Es passiert nichts.}
    \erolls

    Du kannst eine Eigenschaft maximal bis zu ihrem Startwert heilen.
}

\itemcard{Idol}{Idol}{
    Vielleicht hat es dich für einen höheren Zweck auserwählt. Möglicherweise Menschenopfer.
}{
    Einmal pro Zug kannst du an dem Idol reiben, bevor du einen Charakter-, Kampf- oder Ereigniswurf würfelst, um 2 zusätzliche Würfel (bis maximal 8 Würfel) einzusetzen. Jedes mal verlierst du 1 \sanity.

}

\itemcard{Axe}{Akt}{
    WAFFE

    Sehr scharf.
}{
    Wenn du diese Waffe verwendest, kannst einen zusätzlichen Würfel (bis maximal 8 Würfel) für einen \mightroll\ benutzen.

    \nootherweapon
}

\itemcard{Adrenaline Shot}{Adrenalinespritze}{
    Eine Spritze, die eine seltsame, fluosziierende Flüssigkeit enthält.
}{
    Du kannst diesen Gegensstand einsetzen, bevor du einen Charakterwurf würfelst, um zum Wurfergebnis 4 hinzuzufügen.

    \discarditem
}

\itemcard{Angel Feather}{Engelsfeder}{
    Eine makellose Feder schwirrt auf deine Hand.
}{
    Wenn du einen Wurf jedweder Art würfeln musst, kannst du stattdessen eine beliebige Zahl von 0 bis 8 direkt als Würfelergebnis verwenden.

    \discarditem
}

\itemcard{Dynamite}{Dynamit}{
    Die Zündschnur brennt \emph{noch} nicht.
}{
    Anstatt zu Attakieren kannst du Dynamit durch eine Tür in einen benachbarten Raum werfen. Jeder Entdecker und jedes Monster mit Macht und Geschwindigkeitsmerkmalen in diesem Raum muss einen \speedroll\ versuchen:

    \rolls
    \roll{5+}{Kein Schaden}
    \roll{0-4}{Du nimmst 4 Schaden in \physicalsn}
    \erolls

    \discarditem
}

\itemcard{Pickpocket's Gloves}{Handschuhe des Diebes}{
    Sich selbst zu helfen war noch nie so einfach.
}{
    Wenn du zusammen mit einem anderen Entdecker in einem Raum bist, kannst du ihm einen Gegenstand klauen.

    \discarditem
}

\itemcard{Revolver}{Revolver}{
    WAFFE

    Eine alte, wirksam scheinende Waffe.
}{
    Du kannst den Revolver verwenden, um mit einem \speedroll\ anstelle eines \mightroll s anzugreifen. (Dein Gegner verteidigt sich dann mit \speed und nimmt physischen Schaden.)

    Würfle mit einem zusätzlichen Würfel bei deinem Angriffswurf.

    Mit dem Revolver kannst du im gleichen Raum oder in gerader Sichtlinie durch mehrere Türen treffen. Wenn du jemanden in einem anderen Raum angreifst, nimmst du im Falle einer Niederlage keinen Schaden.

     \nootherweapon

}

\itemcard{Rabbit's Foot}{Hasenfuß}{
    Der Hase hatte wohl kein Glück.
}{
    Einmal pro Zug kannst du \emph{einen} Würfel erneut werfen. Der zweite Wurf zählt.
}

\itemcard{Puzzle Box}{Rätsel-Schachtel}{
    Irgendwie muss man die doch öffnen können.
}{
    Einmal pro Zug kannst du versuchen die Kiste mit einem \knowroll\ zu öffnen.

    \rolls
    \roll{+6}{Du öffnest die Box. Ziehe zwei \itemcards\ und entferne die Schachtel aus dem Spiel.}
    \roll{0-5}{Du schaffst es nicht die Box zu öffnen.}
    \erolls
}

\itemcard{Blood Dagger}{Blutdolch}{
    WAFFE

    Eine fiese Waffe. Nadeln und Schläuche winden sich aus dem Griff und stechen direkt in deine Venen.
}{
    Du kannst drei zusätzliche Würfel (bis maximal 8 Würfel) werfen, wenn du eine \might Attacke mit dieser Waffe ausführst. Du verlierst dabei 1 \speed.

    \nootherweapon

    Du kannst diesen Dolch nicht handeln oder fallenlassen. Wenn er gestohlen wird, nimmst du 2 Schaden in \physicalsn.
}


\itemcard{Bell}{Glocke}{
    Eine Messingglocke mit schallendem Gong.
}{
    Erhalte sofort 1 \sanity.

    Verliere 1 \sanity, wenn du die Glocke verlierst.

    Ist der \haunt\ offenbart, kannst du mittels eines \sanityroll s während deines Zuges die Glocke läuten:

    \rolls
    \roll{5+}{Rücke beliebige, aber frei bewegliche Entdecker einen Raum näher zu dir.}
    \roll{0-4}{Der Verräter darf beliebig viele Monster einen Raum näher zu dir rücken. (Wenn du der Verräter bist, gilt dies nicht.) Wenn es keinen Verräter gibt, bewegen sich alle Monster in deine Richtung.}
    \erolls
}



\end{document}